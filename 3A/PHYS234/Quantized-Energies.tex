% Created 2024-04-17 Wed 17:46
% Intended LaTeX compiler: pdflatex
\documentclass[11pt]{article}
\usepackage[utf8]{inputenc}
\usepackage[T1]{fontenc}
\usepackage{graphicx}
\usepackage{longtable}
\usepackage{wrapfig}
\usepackage{rotating}
\usepackage[normalem]{ulem}
\usepackage{amsmath}
\usepackage{amssymb}
\usepackage{capt-of}
\usepackage{hyperref}
\usepackage{parskip,darkmode,braket}
\enabledarkmode
\author{Arnav Gupta}
\date{\today}
\title{Quantized Energies}
\hypersetup{
 pdfauthor={Arnav Gupta},
 pdftitle={Quantized Energies},
 pdfkeywords={},
 pdfsubject={},
 pdfcreator={Emacs 29.3 (Org mode 9.7)}, 
 pdflang={English}}
\begin{document}

\maketitle
\tableofcontents

\section{Spectroscopy}
\label{sec:org1e92841}
Quantized energy levels arise from particle interactions that lead to bound systems (nuclei, atoms, etc).

Lowest energy state at \(n=1\) is called the \textbf{ground state}, and levels above are the \textbf{excited states}.

For energy eigenvalues \(E_{i}\) and energy eigenstates \(\ket{E_{i}}\), the probability of a particular
energy measurement is
$$
\mathcal{P}_{E_{i}} = \left| \braket{E_{i}| \psi} \right|^{2}
$$
and the energy eigenvalue equation is
$$
\hat{H} \ket{E_{i}} = E_{i} \ket{E_{i}}
$$
\section{Energy Eigenvalue Equation}
\label{sec:orgb168b27}
The quantum mechanical Hamiltonian operator for a particle moving in 1 dimension (from classical energy of
such a particle) is
$$
\hat{H} = \frac{\hat{p}_{x}^{2}}{2m} + V(\hat{x})
$$

The wave function is a representation of the abstract quantum state so in position representation:
$$
\ket{\psi} \doteq \psi(x)
$$
For the energy eigenstates in position representation:
$$
\ket{E_{i}} \doteq \varphi_{E_{i}}(x)
$$

Similarly, the operators in position representation are:
$$
\hat{x} \doteq x, \hat{p} = -i\hslash \frac{d}{dx}
$$

The energy eigenvalue equation in position representation becomes:
$$
\left( -\frac{\hslash^{2}}{2m} \frac{d^{2}}{dx^{2}} + V(x) \right) \varphi_{E}(x) = E\varphi_{E}(x)
$$

Operator equations turn into differential equations.
\section{The Wave Function}
\label{sec:orgdc1ef92}
The wave function is a representation of a quantum state using the eigenstates of the position
operator \(\hat{x}\) as the basis states, where position is not quantized (continuous).

State in position basis cannot be represented as a column vector since it is continuous, so instead
must use quantum mechanical wave function.

Wave function is equivalent to probability amplitude (in Dirac notation) so
$$
\psi(x) = \braket{x | \psi}
$$
which gives the probablity distribution function:
$$
\mathcal{P}(x) = \left| \psi(x) \right|^{2}
$$
and so \(\mathcal{P}(x) \, dx\) is the infinitesimal probability of detecting a particle at \(x\) within an
infinitesimal region of width \(dx\).
\(\mathcal{P}(x)\) has dimensions of inverse length.

Measuring \(\mathcal{P}(x)\) does not give \(\psi(x)\), similar to the discrete case.

To find the probability that a particle is in a finite interval \(a < x < b\):
$$
\mathcal{P}_{a < x < b} = \int_{a}^{b} \left| \psi(x) \right|^{2}
$$
and so the normalization condition is also an integral over all \(x\).

Rules for translating braket formulas to wave function formulas:
\begin{enumerate}
\item Replace ket with wave function.
\item Replace bra with wave function conjugate.
\item Replace braket with integral over all space.
\item Replace operator with position representation.
\end{enumerate}

The probability that the state \(\psi(x)\) is measured to have the physical observable for which
\(\phi(x)\) is the eigenstate is
$$
\mathcal{P}_{\psi \to \varphi} = \left| \braket{\varphi | \psi} \right|^{2}
= \left| \int_{-\infty}^{\infty} \varphi^{*}(x) \psi_{x}(x) \, dx \right|^{2}
$$
For energy this becomes:
$$
\mathcal{P}_{E_{n}} = \left| \braket{E_{n} | \psi} \right|^{2}
= \left| \int_{-\infty}^{\infty} \varphi_{n}^{*}(x) \psi_{x}(x) \, dx \right|^{2}
$$
where \(\varphi_{n}(x)\) is the energy eigenstate with energy \(E_{n}\).

The expectation value of position in wave function notation is
$$
\braket{\hat{x}} = \int_{-\infty}^{\infty} x \left| \psi(x) \right|^{2} \, dx
$$

The expectation value of momentum in wave function notation is
$$
\braket{\hat{p}} = \int_{-\infty}^{\infty} \psi^{*}(x) \left(-i\hslash\frac{d}{dx}\right) \psi(x)\, dx
$$
\section{Infinite Square Well}
\label{sec:orgbefb710}
Potential energy functions that have a minimum are called \textbf{potential wells}.
Particle energy is conserved, so kinetic energy is \(T(x) = E - V(x)\).

Particles that have their motion constrained by the potential well are in \textbf{bound states}.
Particles with energies beyond potential well do not have motion constrained, so they are in
\textbf{unbound states}.

The \textbf{particle in a box} exists in a well with
$$
V(x) = \begin{cases}
        \infty & x < 0 \\
        0 & 0 < x < L \\
        \infty & x > L
\end{cases}
$$

\textbf{Infinite square well} illustrates most important features of a particle bound to a limited region of space.

Solving the infinite square well gives a \textbf{quantization condition} with \textbf{quantum number} \(n\), which is
used to label quantized states and energies.
Using these gives the energy quantization for the system:
$$
E_{n} = \frac{n^{2}\pi^{2}\hslash^{2}}{2mL^{2}}
$$
which are the allowed energies with \(n=1\) as the ground state.
Further, the energy eigenstates are:
$$
\varphi_{n}(x) = \sqrt{\frac{2}{L}} \sin \frac{n\pi x}{L}
$$

Here, the energy and the wavelength are related, as opposed to classical where energy and amplitude are.

A system exhibits properties similar to classical particles and classical waves, so there is
\textbf{wave-particle duality}.

The probability density is
$$
\mathcal{P}_{n}(x) = \frac{2}{L} \sin^{2} \frac{n\pi x}{L}
$$
Probability of finding the particle outside the well is 0, but also 0 at some places inside the well.
\section{Finite Square Well}
\label{sec:org620bd43}
The potential energy of a finite square well is
$$
V(x) = \begin{cases}
       V_{0} & x < -a \\
       0 & -a x < a \\
       V_{0} & x > a
\end{cases}
$$

In the regions outside the well, the wave function must be a decaying exponential.
Boundary conditions to solve this are:
\begin{enumerate}
\item \(\varphi_{E}(x)\) is continuous
\item \(\frac{d\varphi_{E}(x)}{dx}\) is continuous unless \(V = \infty\)
\end{enumerate}

Symmetry condition can also be used to solve, giving even and odd solutions.

Using these conditions gives transcendental equations that must be solved graphically or numerically
for each different well.

Finite well eigenstates extend into the classically forbidden region (unlike infinite well),
referred to as \textbf{barrier penetration}.
\section{Observations}
\label{sec:org6e667e5}
For higher energy states, the wave function penetrates further into the classically forbidden region,
evident in finite well states.

A given finite well energy eigenvalue lies below the corresponding infinite well energy eigenvalue.
\section{Superposition States and Time Dependence}
\label{sec:orgfa3fb86}
The Schrödinger equation in energy eigenstate basis can be used to find the time evolution.

Probability and expectation value of energy are time-independent.
\end{document}
