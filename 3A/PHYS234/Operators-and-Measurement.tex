% Created 2024-04-17 Wed 17:47
% Intended LaTeX compiler: pdflatex
\documentclass[11pt]{article}
\usepackage[utf8]{inputenc}
\usepackage[T1]{fontenc}
\usepackage{graphicx}
\usepackage{longtable}
\usepackage{wrapfig}
\usepackage{rotating}
\usepackage[normalem]{ulem}
\usepackage{amsmath}
\usepackage{amssymb}
\usepackage{capt-of}
\usepackage{hyperref}
\usepackage{parskip,darkmode}
\enabledarkmode
\author{Arnav Gupta}
\date{\today}
\title{Operators and Measurement}
\hypersetup{
 pdfauthor={Arnav Gupta},
 pdftitle={Operators and Measurement},
 pdfkeywords={},
 pdfsubject={},
 pdfcreator={Emacs 29.3 (Org mode 9.7)}, 
 pdflang={English}}
\begin{document}

\maketitle
\tableofcontents

\section{Operators, Eigenvalues, and Eigenvectors}
\label{sec:org6f6c1a8}
\textbf{Operator}: mathematical object that acts on a ket and transforms it into a new ket

\textbf{Postulate 2}: a physical observable is represented mathematically by an operator \(A\)
that acts on a ket

\textbf{Postulate 3}: the only possible result of a measurement of an observable is one of
the eigenvalues \(a_{n}\) of the corresponding operator \(A\)

An operator is always diagonal in its own basis.
Eigenvectors are unit vectors in their own basis.
\subsection{Matrix Representation of Operators}
\label{sec:org0dac4c4}
Each matrix element can be found as a bra multiplied by an operator multiplied by
a ket.
\section{New Operators}
\label{sec:org2109d70}
\subsection{Hermetian Operators}
\label{sec:orge73f16e}
An operator \(A\) is Hermetian if it is equal to its Hermetian adjoint.
\subsection{Projection Operators}
\label{sec:org4c0b7f5}
\textbf{Closure/Completeness}: the sum of the projector operators for all eigenstates is the identity operator.

When a projector operator for an eigenstate acts on a state, it produces a new ket that is aligned
along the eigenstate and has a magnitude equal to the amplitude (including the phase) for the
state to be in that eigenstate.

\textbf{Postulate 5}: after a measurement of \(A\) that yields the result \(a_{n}\), the quantum system
is in a new state that is the normalized projection of the original system ket onto the
ket or kets corresponding to the result of the measurement, giving
$$
        \ket{\psi '} = \frac{P_{n} \ket{\psi}}{\sqrt{\bra{\psi} P_{n} \ket{\psi}}}
$$

Quantum measurements cannot be made without disturbing the system (except where input and
output are the same), this leads to collapse.
\section{Measurement}
\label{sec:org73204a0}
The mean (expected value) measurement for an operator \(A\) on a state \(\ket{\psi}\) is
$$
        \left< A \right> = \bra{\psi} A \ket{\psi} = \sum_{n} a_{n} P_{a_{n}}
$$

The standard deviation is defined as:
$$
        \Delta A = \sqrt{\left< (A - \left< A \right>)^{2} \right>}
        = \sqrt{\left< A^{2} \right> - \left< A \right>^{2}}
$$
which is derived from root-mean-square.
\section{Commuting Observables}
\label{sec:org6619904}
The commutator of two operators is \([A, B] = AB - BA\).

If the commutator is 0, the operators commute.

Commuting operators share common eigenstates.

The commutation relations for the spin component operators are:
$$
[S_{x}, S_{y}] = i \hslash S_{z}, [S_{y}, S_{z}] = i \hslash S_{x}, [S_{z}, S_{x}] = i \hslash S_{y}
$$
\section{Uncertainty Principle}
\label{sec:org0b090b1}
$$
\Delta A \Delta B \ge \frac{1}{2} \left| \left< \left[A, B\right] \right> \right|
$$

One spin component can be known absolutely but not 2 or simultaneously.
\end{document}
