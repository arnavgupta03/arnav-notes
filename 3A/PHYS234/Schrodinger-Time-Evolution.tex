% Created 2024-04-17 Wed 17:47
% Intended LaTeX compiler: pdflatex
\documentclass[11pt]{article}
\usepackage[utf8]{inputenc}
\usepackage[T1]{fontenc}
\usepackage{graphicx}
\usepackage{longtable}
\usepackage{wrapfig}
\usepackage{rotating}
\usepackage[normalem]{ulem}
\usepackage{amsmath}
\usepackage{amssymb}
\usepackage{capt-of}
\usepackage{hyperref}
\usepackage{parskip,darkmode}
\enabledarkmode
\author{Arnav Gupta}
\date{\today}
\title{Schrödinger Time Evolution}
\hypersetup{
 pdfauthor={Arnav Gupta},
 pdftitle={Schrödinger Time Evolution},
 pdfkeywords={},
 pdfsubject={},
 pdfcreator={Emacs 29.3 (Org mode 9.7)}, 
 pdflang={English}}
\begin{document}

\maketitle
\tableofcontents

\section{Schrödinger Equation}
\label{sec:orgbbf4d30}
\textbf{Postulate 6}: the time evolution of a quantum system is determined by the Hamiltonian or total energy
operator \(H(t)\) through the Schrödinger equation
$$
i\hslash \frac{d}{dt} \ket{\psi(t)} = H(t) \ket{\psi(t)}
$$

The Hamiltonian is an observable so it is a Hermetian operator.
The energy eigenvalue equation is
$$
H \ket{E_{n}} = E_{n} \ket{E_{n}}
$$

The basis of eigenvectors of the Hamiltonian is the \textbf{energy basis}.

If the initial state of the system at time 0 is
$$
\ket{\psi(0)} = \sum_{n} c_{n} \ket{E_{n}}
$$
then the time evolution of this state under the action of the time-independent Hamiltonian \(H\)
is
$$
\ket{\psi(t)} = \sum_{n} c_{n} e^{-iE_{n} t/\hslash} \ket{E_{n}}
$$

For an observable \(A\) whose eigenstates consist of superpositions of energy eigenstates,
where \(\ket{a_{1}} = \alpha_{1} \ket{E_{1}} + \alpha_{2} \ket{E_{2}}\).
Then for \(\ket{psi(0)} = c_{1}\ket{E_{1}} + c_{2} \ket{E_{2}}\),
\$\$
P\textsubscript{a\textsubscript{1}} = |\(\alpha\)\textsubscript{1}|\textsuperscript{2} |c\textsubscript{1}|\textsuperscript{2} + |\(\alpha\)\textsubscript{2}|\textsuperscript{2} |c\textsubscript{2}|\textsuperscript{2}
\begin{itemize}
\item 2\text{Re}\left( \(\alpha\)\textsubscript{1} c\textsubscript{1}\textsuperscript{$\backslash$*} \(\alpha\)\textsubscript{2}\textsuperscript{$\backslash$*} c\textsubscript{2} e\textsuperscript{-i(E\textsubscript{2} - E\textsubscript{1})t/\hslash} \right)
\end{itemize}
$$
and so the probabilities are time dependent, which is determined by the difference of the energies
of the 2 states involved in the superposition.
The corresponding angular frequency of the time evolution
$$
\(\omega\)\textsubscript{21} = \frac\{E\textsubscript{2} - E\textsubscript{1}\}\{\hslash\}
\$\$
is the \textbf{Bohr frequency}.
\subsection{Recipe for Solving Time-Dependent Quantum Mechanics}
\label{sec:org432f06e}
Given a Hamiltonian \(H\) and an initial state \(\ket{\psi(0)}\), find the probability that
the eigenvalue \(a_{j}\) of the observable \(A\) is measured at time \(t\).
\begin{enumerate}
\item Diagonalize \(H\) to find the eigenvalues and eigenvectors for the energies.
\item Write \(\ket{\psi(0)}\) in terms of the energy eigenstates \(\ket{E_{n}}\).
\item Multiply each eigenstate coefficient by \(e^{-iE_{n}t/\hslash}\) to get \(\ket{\psi(t)}\).
\item Calculate the probability \(P_{a_{j}} = \left| \braket{a_{j} \mid \psi(t)} \right|^{2}\).
\end{enumerate}
\section{Spin Precession}
\label{sec:org465994a}
For the magnetic dipole
$$
\mu = g \frac{q}{2m_{e}} S
$$
the Hamiltonian is
$$
H = -g \frac{q}{2m_{e}} S \cdot B
$$
\subsection{Magnetic Field in the z-Direction}
\label{sec:orgc95d80c}
If \(B = B_{0} \hat{z}\), then \(H = \omega_{0} S_{z}\)
for \(\omega_{0} = -g\frac{q}{2m_{e}} B_{0}\).

\textbf{Spin precession}: where probabilities change for certain spins since the Hamiltonian
is defined for other spins

\textbf{Larmor precession}: the precession of the spin vector

\textbf{Larmor frequency}: the frequency of precession

\textbf{Ehrenfest's Theorem}: quantum mechanical expectation values obey classical laws
\subsection{Magnetic Field in a General Direction}
\label{sec:org46d9eb6}
\textbf{Spin flip}: where a state flips to the orthogonal state

The probability of a spin flip is
$$
P_{+ \to -} = \frac{\omega_{1}^{2}}{\omega_{0}^{2} + \omega_{1}^{2}}
\sin^{2} \left( \frac{\sqrt{\omega_{0}^{2} + \omega_{1}^{2}}}{2} t \right)
$$
which is Rabi's formula.
\end{document}
