% Created 2024-04-05 Fri 11:41
% Intended LaTeX compiler: pdflatex
\documentclass[11pt]{article}
\usepackage[utf8]{inputenc}
\usepackage[T1]{fontenc}
\usepackage{graphicx}
\usepackage{longtable}
\usepackage{wrapfig}
\usepackage{rotating}
\usepackage[normalem]{ulem}
\usepackage{amsmath}
\usepackage{amssymb}
\usepackage{capt-of}
\usepackage{hyperref}
\usepackage{parskip, darkmode}
\enabledarkmode
\author{Arnav Gupta}
\date{\today}
\title{Control Flow}
\hypersetup{
 pdfauthor={Arnav Gupta},
 pdftitle={Control Flow},
 pdfkeywords={},
 pdfsubject={},
 pdfcreator={Emacs 29.3 (Org mode 9.7)}, 
 pdflang={English}}
\begin{document}

\maketitle
\tableofcontents

\section{Program Graphs}
\label{sec:org6ee9190}
Given a program written in an imperative programing language,
its program graph is a directed graph in which
\begin{itemize}
\item nodes are statement fragments (or complete statements)
\item edges represent flow of control
\end{itemize}
\subsection{Control Flow Graphs}
\label{sec:org9589c83}
Models all executions of a method by describing control structures:
\begin{itemize}
\item nodes are statements or sequences of statements (basic blocks)
\item edges are transfers of control
\begin{itemize}
\item an edge \((s_{1}, s_{2})\) indicates that \(s_{1}\) may be
followed by \(s_{2}\) in an execution
\end{itemize}
\end{itemize}

Basic block: sequence of statements such that if the first statement is executed
all statements will be (no branches)
\begin{itemize}
\item intermediate statements not shown if there is \(\le 1\) exiting edge and \(\le 1\) entering edge
\item has one entry point and one exit point
\end{itemize}

Sometimes annotated with branch predicates and defs.
\subsubsection{The \texttt{if} Statement}
\label{sec:org9f1d856}
From source code to CFG there is an issue about branching:
\begin{itemize}
\item in a CFG, nodes corresponding to branching should not contain any assignemnts
\item \texttt{return} nodes must be distinct
\end{itemize}

Short circuiting conditions must be shown with separate branches for each condition.
\subsubsection{\texttt{do}, \texttt{while}, and \texttt{for} Loops}
\label{sec:org5240483}
Can have:
\begin{itemize}
\item dummy nodes for condition checking or initializing loops
\item nodes to implicitly increment loop
\end{itemize}

\texttt{do} loops will branch at last step, \texttt{while} loops branch at first
\subsubsection{\texttt{case} Structure}
\label{sec:org7f8d399}
Cases without breaks fall through to the next case
\subsubsection{\texttt{try-catch} Exceptions}
\label{sec:org3d0ded9}
Both \texttt{throw} and catching each type of exception must be shown separately.
\section{Coverage}
\label{sec:orge4c21f5}
\subsection{Statement Coverage}
\label{sec:org0af3557}
Has the following characteristics:
\begin{itemize}
\item achieved when all statements in a method have been executing at least once.
\begin{itemize}
\item faults cannot be discovered if the parts containing them are not executed
\end{itemize}
\item equivalent to covering all nodes in a CFG
\item executing a statement is a weak guarantee of correctness, but easy to achieve
\end{itemize}
$$
        \text{Statement Coverage} = \frac{\text{\# of executed statements}}{\text{total \# of statements}}
$$

Most used in industry with coverage target at 80-90\%.

Problems include:
\begin{itemize}
\item predicate may be tested for only one value
\item loop bodies may be iterated only once
\item not all branches (cases) are necessarily covered
\end{itemize}
\subsection{Segment Coverage}
\label{sec:org54da7ba}
Counts segments (basic blocks) rather than statements
\begin{itemize}
\item can produce drastically different numbers since it does not account for \# of statements per segment
\end{itemize}
\subsection{Branch Coverage}
\label{sec:org992b2b0}
Has the following characteristics:
\begin{itemize}
\item achieved when every branch from a node is executed at least once
\item at least one true and one false evaluation for each predicate
\item can be achieved with \(D+1\) paths in a CFG with \(D\) 2-way branching nodes and no loops
\begin{itemize}
\item less if there are loops
\end{itemize}
\end{itemize}
$$
        \text{Branch Coverage} = \frac{\text{\# of executed branches}}{\text{total \# of branches}}
$$

Problems include:
\begin{itemize}
\item short-circuit evaluation means that many predicates might not be evaluated
\item compound predicate treated as a single statement
\begin{itemize}
\item if \(n\) clauses, \(2^{n}\) combinations but only 2 tested
\end{itemize}
\item only a subset of all entry-exit paths is tested
\end{itemize}
\subsection{Condition Coverage}
\label{sec:orgf0ba0d8}
Has the following characteristics:
\begin{itemize}
\item condition coverage reports the true or false outcome of each condition
\item measures the conditions independetly of each other
\item can fail to consider short-circuit
\end{itemize}
$$
        \text{Condition Coverage} =
        \frac{\text{\# of conditions that are both T and F}}{\text{total \# of conditions}}
$$
\subsection{Condition/Decision Coverage}
\label{sec:org21ac2ad}
Also called branch and condition coverage,
it is computed by considering both branch and condition coverage measures.
\subsection{Multple Condition Coverage}
\label{sec:org4511c63}
All combinations of condition constituents in decisions are considered when calculating coverage.
\begin{itemize}
\item also implies condition and branch coverage
\end{itemize}
\subsection{Modified Condition/Decision Coverage (MC/DC)}
\label{sec:orgecea819}
\textbf{Key idea}: test important combinations of conditions and limiting testing costs
\begin{itemize}
\item extend branch and decision coverage with the requirement that each condition should affect
the decision outcome independetly
\item in other words, each condition should be evaluated one time to true and one time to false,
and this with affecting the decision's outcome
\end{itemize}
\subsection{Path Coverage}
\label{sec:orgc6a77cb}
Has the following characteristics:
\begin{itemize}
\item test case for each possible path, though some are infeasible and number could be infinite
\item key is to determine critical paths
\end{itemize}
\subsection{Loop Coverage}
\label{sec:org7610e97}
Loops are highly fault-prone
\begin{itemize}
\item every loop involves a decision to traverse the loop or not
\item can do boundary value analysis on the index variable
\end{itemize}

Has the following characteristics:
\begin{itemize}
\item minimal coverage should execute the loop body 0, 1, and 2+ times
\item single loop should have more extensive coverage with setting loop control variable to:
\begin{itemize}
\item min - 1, min, min + 1
\item typical
\item max - 1, max, max + 1
\end{itemize}
\item for nested loop, start at the innermost loop
\begin{itemize}
\item set all outer loops to min value
\item set all other loops to typical values
\item test cases for a single loop for the innermost loop
\item move up in nested loop level
\item if outermost loop done, do cases for single loop for all loops
in the nest simultanously
\end{itemize}
\end{itemize}
\end{document}
