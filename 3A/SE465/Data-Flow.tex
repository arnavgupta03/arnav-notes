% Created 2024-04-05 Fri 11:42
% Intended LaTeX compiler: pdflatex
\documentclass[11pt]{article}
\usepackage[utf8]{inputenc}
\usepackage[T1]{fontenc}
\usepackage{graphicx}
\usepackage{longtable}
\usepackage{wrapfig}
\usepackage{rotating}
\usepackage[normalem]{ulem}
\usepackage{amsmath}
\usepackage{amssymb}
\usepackage{capt-of}
\usepackage{hyperref}
\usepackage{parskip, darkmode}
\enabledarkmode
\author{Arnav Gupta}
\date{\today}
\title{Data Flow}
\hypersetup{
 pdfauthor={Arnav Gupta},
 pdftitle={Data Flow},
 pdfkeywords={},
 pdfsubject={},
 pdfcreator={Emacs 29.3 (Org mode 9.7)}, 
 pdflang={English}}
\begin{document}

\maketitle
\tableofcontents

\section{Data Flow Analysis}
\label{sec:orgfcfa461}
For CFG paths that are significant for data flow in the program:
\begin{itemize}
\item focuses on the assignment of values to variables and their use
\item done by analyzing occurrences of variables
\begin{itemize}
\item definition occurence: value bound to variable
\item use occurrence: value of variable is referred
\begin{itemize}
\item predicate use: variable used to decide whether predicate is true
\item computational use: compute a value for defining other variables or output value
\end{itemize}
\end{itemize}
\end{itemize}
\section{Definition and Uses}
\label{sec:org84873ff}
Program contain variables: defined by assigning values and using in expressions

With pointers, both the pointer variable and the data it points to
must be considered.

With arrays, index variables must be considered, though only the part of the array
used neds to be considered (same for fields in a class).
\subsection{Uses}
\label{sec:org183a7f9}
\subsubsection{c-use}
\label{sec:orgc91a5ed}
Refers to uses of a variable that occur within an expression as part
of an assignment statement, in an output statement, as a parameter
within a function call, and in subscript expressions. (computational)
\subsubsection{p-use}
\label{sec:org5af8e97}
Refers to the occurrence of a variable in an expression used as a condition
in a branch statement such as \texttt{if} or \texttt{while}. (predicate)
\subsection{Formalization}
\label{sec:orgfba7810}
Variable definition \(d(v, n)\): value is assigned to \(v\) at node \(n\)
(LHS of assignment, input statement)

Variable use
\begin{itemize}
\item c-use\((v, n)\): variable \(v\) used in a computation at node \(n\)
(RHS of assignment, argument of procedure call, output statement)
\item p-use\((v, m, n)\): variable \(v\) used in predicate from node \(m\) to \(n\)
(as part of an expression used for a decision)
\end{itemize}

Variable kill \(k(v, n)\): variable \(v\) deallocated at node \(n\)
\section{Data Flow Action}
\label{sec:org71c4f89}
\subsection{Checklist}
\label{sec:org900ccad}
For each successive pair of actions:
\begin{itemize}
\item suspicious: dd
\item likely defect: dk, kk
\item defect: ku
\item okay: du, kd, ud, uk, uu
\end{itemize}

First occurrence of action on a variable:
\begin{itemize}
\item suspicious: k, u (unless global)
\item okay: d
\end{itemize}

Last occurrence of action on a variable:
\begin{itemize}
\item suspicious: d (defined but never used after)
\item okay: k, u (but maybe deallocation forgotten)
\end{itemize}
\subsection{Data Flow Graph}
\label{sec:orgbcc2151}
Captures the flow of definitions and uses across basic blocks
in a program.
Similar to a CFG in that nodes, edges, and paths in the CFG
are preserved in the DFG.
\begin{itemize}
\item annotate nodes with def and c-use as needed, and each edge
with p-use as needed
\item label each edge with the condition which when true causes
the edge to be taken
\end{itemize}
\subsubsection{Paths and Pairs}
\label{sec:org8977cf0}
Complete path: initial node is start node, final node
is exit node

Simple path: all nodes except possibly first and last
are distinct

Loop free path: all nodes are distinct

def-clear path with respect to \(v\): any path starting from
a node at which variable \(v\) is defined and ending at a node
at which \(v\) is used, without redefining \(v\) anywhere else
along the path

du-pair with respect to \(v\): (d, u) such that
\begin{itemize}
\item d is the node where \(v\) is defined
\item u is the node ehrtr \(v\) is used
\item def-clear path with respect to \(v\) from \(d\) to \(u\)
\item also referred to as reach: def of \(v\) at d reaches the
use at u
\end{itemize}

du-path with respect to \(v\):
a path \(P = \left< n_{1}, n_{2}, \dots, n_{j}, n_{k} \right>\)
such that \(d(v, n_{1})\) and either one of the following two cases:
\begin{itemize}
\item c-use of \(v\) at node \(n_{k}\) and \(P\) is a def-clear simple path
with respect to \(v\)
\item p-use of \(v\) on edge \(n_{j}\) to \(n_{k}\) and
\(\left< n_{1}, n_{2}, \dots, n_{j} \right>\) is a def-clear
loop-free path
\end{itemize}
\section{Data Flow Coverage}
\label{sec:org6aa04d5}
\subsection{All-Definitions}
\label{sec:org8dc1ec6}
At least one def-clear path from every defining node of \(v\)
to at least one use of \(v\), be that a c-use or p-use
\subsection{All-Uses}
\label{sec:orge543e87}
At least on def-clear path form every defining node of \(v\)
to every (reachable) use of \(v\)
\subsection{All-P-Uses/Some-C-Uses}
\label{sec:org98a16d0}
At least one def-clear path from every defining node of \(v\)
to every reachable p-use of \(v\)
\begin{itemize}
\item if none, then to a c-use
\end{itemize}
\subsection{All-C-Uses/Some-P-Uses}
\label{sec:orgf10a37e}
At least one def-clear path from every defining node of \(v\)
to every reachable c-use of \(v\)
\begin{itemize}
\item if none, then to a p-use
\end{itemize}
\subsection{All-DU-Paths}
\label{sec:org5469d08}
All du paths covered
\end{document}
