% Created 2024-04-05 Fri 11:43
% Intended LaTeX compiler: pdflatex
\documentclass[11pt]{article}
\usepackage[utf8]{inputenc}
\usepackage[T1]{fontenc}
\usepackage{graphicx}
\usepackage{longtable}
\usepackage{wrapfig}
\usepackage{rotating}
\usepackage[normalem]{ulem}
\usepackage{amsmath}
\usepackage{amssymb}
\usepackage{capt-of}
\usepackage{hyperref}
\usepackage{parskip, darkmode}
\enabledarkmode
\author{Arnav Gupta}
\date{\today}
\title{Integration Testing}
\hypersetup{
 pdfauthor={Arnav Gupta},
 pdftitle={Integration Testing},
 pdfkeywords={},
 pdfsubject={},
 pdfcreator={Emacs 29.3 (Org mode 9.7)}, 
 pdflang={English}}
\begin{document}

\maketitle
\tableofcontents

\section{Integration Testing}
\label{sec:org561a1fc}
\subsection{Strategy}
\label{sec:org0c15665}
How individual components are assembled to form larger program entities:
\begin{itemize}
\item \textbf{problem 1}: test component interaction
\item \textbf{problem 2}: determine optimal order of integration
\end{itemize}

Strategy impacts:
\begin{itemize}
\item form of writing unit test cases
\item type of test tools to use
\item order of coding/testing components
\item cost of generating test cases
\item cost of locating and correcting detected defects
\end{itemize}
\subsection{Testing}
\label{sec:org8764c04}
\textbf{Objective}: ensure components work in isolation and when assembled, requires
a component dependency structure

If all components work in isolation, faults will most be interface related.
\subsubsection{Stubs}
\label{sec:orgcca9a9c}
Stubs replace called modules and can replace whole components.
They must be declared/invoked as the real module (same name, param list,
return type, modifiers)

Common functions:
\begin{itemize}
\item display/log trace message or passed params
\item return value according to test objective
\end{itemize}
\subsubsection{Drivers}
\label{sec:org25c2f0d}
Driver modules are used to call tested modules (param passing, handling
return values), typically simpler than stubs.
\subsection{Strategies}
\label{sec:orgfbd5a7a}
\textbf{Comparison criteria for strategies}: fault localization, effort needed,
degree of testing of modules achieved, possiblity for parallel development
\subsubsection{Big Bang Integration}
\label{sec:org481983b}
Non-incremental strategy to integrate all components as a whole.
Assumes components are initially tested in isolation.

Advantages
\begin{itemize}
\item convenient for small/stable systems
\end{itemize}

Disadvantages
\begin{itemize}
\item does not allow parallel development
\item fault localization difficult
\item easy to miss interface faults
\end{itemize}
\subsubsection{Top-Down Integration}
\label{sec:org7c9a970}
Incremental strategy to test high level components, then called components
until lowest-level components.
Possible to alter order to test as early as possible (critical or IO components).

Advantages:
\begin{itemize}
\item fault localization easier
\item few or no drivers needed
\item possiblity to obtain an early prototype
\item testing can be in parallel with implementation
\item different order of testing/implementation possible
\item major design flaws found first
\end{itemize}

Disadvantages:
\begin{itemize}
\item needs lots of stubs
\item potentially reusable components can be inadequately tested
\end{itemize}
\subsubsection{Bottom-Up Integration}
\label{sec:orgb72f1b7}
Incremental strategy to test low level components, then components calling
them until highest-level components.

Advantages:
\begin{itemize}
\item fault localization easier
\item no stubs needed
\item reusable components tested thoroughly
\item testing can be in parallel with implementation
\item different order of testing/implementation possible
\end{itemize}

Disadvantages:
\begin{itemize}
\item needs lots of drivers
\item high-level components tested last
\item no concept of early skeletal system
\end{itemize}
\subsubsection{Sandwich Integration}
\label{sec:org3e753f5}
Combines top-down and bottom-up by distinguishing 3 layers: logic (top), middle,
operational (bottom)
\subsubsection{Other Integration Testing Strategies}
\label{sec:orgd124167}
\begin{itemize}
\item \textbf{risk driven integration}: integrate based on criticality
\item \textbf{function/thread-based integration}: integrate components according to
threads/functions they belong to
\end{itemize}
\section{Testing Object-Oriented Systems}
\label{sec:org8e13734}
\subsection{Motivation}
\label{sec:org6b1a38e}
Object-oriented code needs more testing.
Where unit testing applies the same, OO concepts must be accounted for
such as encapsulation, inheritance, polymorphism, abstract classes, and exceptions.

Unit and integration testing especially affected since they are white-box.
\subsection{Class vs Procedure Testing}
\label{sec:org10238a3}
Procedural programming
\begin{itemize}
\item basic component is function
\item testing method is based on IO relation
\end{itemize}

OO programming
\begin{itemize}
\item basic component is class
\item objects are tested
\item correctness cannot simply be defined as an IO relation, but also object state
\end{itemize}

Methods can be tested independently, though complexity lies in method interactions.
Method behaviour is meaningless unless analyzed in relation to other operations
and joint effect on shared state.

Testing classes requires recognizing:
\begin{itemize}
\item the identification of behaviour to be observed
\item manipulation of object state without violating encapsulation
\item polymorphism and dynamic binding to test all possible invocations of interfaces
\item each exception to be tested
\end{itemize}

New fault models are required to target OO specific faults.
\subsection{OO Integration Levels}
\label{sec:org02abd75}
Classes introduce a new abstraction level leading to more unit testing methods:
\begin{itemize}
\item \textbf{basic unit testing}: the testing of a single operation (method)
of a class (intra-method testing)
\item \textbf{unit testing}: the testing of methods within a class (intra-class testing),
the integration of methods
\item \textbf{intra-class testing}: black box and white box approaches (data flow), with:
\begin{itemize}
\item each exception raised at least once
\item each interrupt forced to occur at least once
\item each attribute set and got at least once
\item testing based on states
\item done with big bang approach
\item full cycle of objects
\item complex methods tested with stubs or mocks
\end{itemize}
\end{itemize}

For integration testing, there are different scopes for interactions among
classes:
\begin{itemize}
\item \textbf{cluster integration}
\begin{itemize}
\item integration of 2+ classes through inheritance
\begin{itemize}
\item incremental test of inheritance hierarchies
\end{itemize}
\item integration of 2+ clasxes through containment
\item integration of 2+ associated classes/clusters to form a component
\item big bang, bottom-up, top-down, scenario based
\end{itemize}
\item \textbf{system/subsystem integration}
\begin{itemize}
\item integration of components into a single application
\item similar techniques as cluster integration
\item client/server integration
\end{itemize}
\end{itemize}
\subsection{Mock Objects}
\label{sec:org0692851}
Testing of OO systems requires drivers and stubs, but it is difficult to
flexibly stub dependent code without changing code under test or
maintaining a library of stub objects (which is also difficult).

Mock object:
\begin{itemize}
\item a form of stubs based on interfaces
\item easier to setup and control
\item isolates code from details that may be filled in later
\item can be refined incrementally by replacing with actual code
\item based on dependency inversion principle
\item test control mock behaviour so mock transparently replaces actual code
\end{itemize}

Creation can be done manually (more control) or using frameworks like Mockito.
\subsection{Inheritance}
\label{sec:orgc53ed34}
Makes understanding code more difficult. Test suite for a method almost never
adequate for overriding methods.

Inherited methods must be tested in the context of inheriting classes,
with accidental reuse, abstract classes, and multiple inheritance.

Interactions possible to miss:
\begin{itemize}
\item inherited methods should be retested in the context of a subclass
\begin{itemize}
\item modifying a superclass requires retesting subclasses
\item adding or modifying subclasses requires retesting methods inherited
from each ancestor superclass
\end{itemize}
\item overriding of methods
\begin{itemize}
\item overriding subclass method has to be tested but different test
sets are needed
\item reason 1: if test cases are derived from program structure
(data and control flow), the structure of the overriding method
may be different
\item reason 2: the overriding method behaviour could also be different
\end{itemize}
\item integration and polymorphism
\begin{itemize}
\item can make test sets grow exponentially
\end{itemize}
\end{itemize}

Coverage must be extended for inheritance to cover the level of
coverage in the context of each class as separate measurements.
100\% inheritance means all code fully exercised in each
appropriate context.
\subsection{Abstract Classes}
\label{sec:org6cd8627}
Cannot be instantiated but they define an interface and
behaviour that implementing classes must adhere to.
Must test abstract classes for functional compliance.

\textbf{Functional compliance}: a module's compliance with some
documented or published functional specification.
\subsubsection{Abstract Test Pattern}
\label{sec:orgfbb1014}
Provides the following:
\begin{itemize}
\item a reusable way to build a test suite across descendants
\item a test suite that can be reused for future descendants
\end{itemize}

Tests that define the \emph{functionality} of the interface belong in the
abstract test class. Test specific to an \emph{implementation} belong
in a concrete test class.

\textbf{Rule 1}
\begin{itemize}
\item write an abstract test class for every interface and
abstract class
\item an abstract test should have test cases that cannot
be overridden
\item should also have an abstract factory method for creating instances
of the class to be tested
\end{itemize}

\textbf{Rule 2}
\begin{itemize}
\item write a concrete test class for every implementation
of the interface (or abstract class)
\item the concrete test class should extend the abstract test class
and implement the factory method
\end{itemize}
\subsection{Cluster Integration}
\label{sec:org07d7ea9}
Needs a class dependency tree since the dependency structure
is not always that clear cut.
\subsubsection{Big Bang}
\label{sec:orge2a36d0}
Recommmended only when the cluster is stable, small, and the components
are tightly coupled.
\subsubsection{Bottom-Up}
\label{sec:org31f0d86}
Most widely used technique: integrate classes starting from leaves
and moving up the depdendency tree
\subsubsection{Top-Down}
\label{sec:orgd3570e5}
Widely used technique: integrate classes starting from the top and
moving down the dependency tree to leaves.
\subsubsection{Scenario-Based}
\label{sec:org45d3fdc}
Describes interaction (collaboration) of classes, represented as
interaction diagrams.

Approach:
\begin{enumerate}
\item Map collaborations onto dependency tree
\item Choose a sequence to apply collaborations
\item Develop and run test exercising collaboration
\end{enumerate}
\subsection{Integration Order Problem}
\label{sec:org77abd4f}
Issues related to integration order:
\begin{itemize}
\item not advised to perform a big-bang integration (must be done stepwise,
leading to stubs)
\item not always feasible to construct a stub that is simpler than the
code it simulates
\item stub generation cannot be automated since it requires an
understanding of the semantics of the simulated functions
\item fault potential for some stubs may be the same or higher than
real function
\item minimizing the number of stubs to be developed should result
in drastic savings
\item class dependency graphs usually do not form simple hierarchies
but complex networks
\item most OO systems have strong connectivity and cyclic interdependencies
\end{itemize}
\subsubsection{Kung et al Strategy}
\label{sec:orga7a589c}
Aims at producing a partial ordering of testing levels based on class
diagrams.

Classes under test at given level should only depend on classes
previously tested, no stub required since previously tested
classes can be included in subsequent testing levels.

Class dependency info can either come from reverse engineering or
design documentation.

Dynamic relationships between classes are not taken into account by Kung et al.

Some testing levels become infeasible because of abstract classes.
\subsubsection{Object Relation Diagram}
\label{sec:orgf02e77b}
Uses static info only.
The ORD for a program \(P\) is an edge-labelled graph where the nodes
represent the classes in \(P\) and the edges represent the
inheritance, composition, and association relations.

For any two classes:
\begin{itemize}
\item edge labelled I means subclass to superclass
\item edge labelled C means from  composed of to
\item edge labelled A means from associates with to
\end{itemize}

Identiy the effect of a class change at the class level using a CFW.
CFW(\(X\)) for a class \(X\): set of classes that could be
affected by a change to class \(X\) (should be retested)

Assuming the ORD is not modfied by the change, for adequate testing
CFW(\(X\)) has to include subclasses of \(X\), classes composed with \(X\), and
classes associated with \(X\).

Can be defined from UML (more straightforward) or source code.

CFW(\(X\)) contains all the classes such that there is a directed path from
them to \(X\) in the ORD.

Test order provides the tester with a road map for conducting integration
testing. Desirable order is the one that minimizes the number of stubs
or stubbing effort.
\begin{itemize}
\item test independent classes first, then dependent classes based on their
relationships
\item similarly a subclass would be tested after its superclasses
\end{itemize}

\textbf{Order of Acyclic ORDs}:
can generate a test order that ensures that \(X\) is tested before all classes
of CFW(\(X\)), so stubs are not required (simply a topological sort)

\textbf{Order of Cyclic ORDs}:
topological sort cannot be applied, so this requires cycle breaking

\textbf{Cluster}: maximal set of nodes that are mutually reachable through the relation
(strongly connected component)

\textbf{Cycle breaking}: identify and remove an edge from a cluster until the graph
becomes acyclic
\begin{itemize}
\item each deletion results in at least one stub
\begin{itemize}
\item cannot be a parent class
\item should not be a composition
\item association is good for deletion, every directed cycle always contains
at least one association edge
\end{itemize}
\end{itemize}
\end{document}
