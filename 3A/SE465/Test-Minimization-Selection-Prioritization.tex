% Created 2024-04-21 Sun 14:32
% Intended LaTeX compiler: pdflatex
\documentclass[11pt]{article}
\usepackage[utf8]{inputenc}
\usepackage[T1]{fontenc}
\usepackage{graphicx}
\usepackage{longtable}
\usepackage{wrapfig}
\usepackage{rotating}
\usepackage[normalem]{ulem}
\usepackage{amsmath}
\usepackage{amssymb}
\usepackage{capt-of}
\usepackage{hyperref}
\usepackage{parskip,darkmode}
\enabledarkmode
\author{Arnav Gupta}
\date{\today}
\title{Test Minimization, Selection, and Prioritization}
\hypersetup{
 pdfauthor={Arnav Gupta},
 pdftitle={Test Minimization, Selection, and Prioritization},
 pdfkeywords={},
 pdfsubject={},
 pdfcreator={Emacs 29.3 (Org mode 9.7)}, 
 pdflang={English}}
\begin{document}

\maketitle
\tableofcontents

\section{Motivation}
\label{sec:orgbb834c7}
Software testing is expensive and can have too many tests to run.
Most tests don't fail.
\section{Minimization}
\label{sec:orge268da8}
Given a test suite \(T\), a set of requirements \(\{r_{1}, \dots, r_{n}\}\)
that must be satisfied to provide the desired adequate testing of the program,
and subsets of \(T\) \(\{T_{1}, \dots, T_{n}\}\) associated with each \(r_{i}\) such
that any test case \(t_{j}\) belonging to \(T_{i}\) can be used to achieve
requirement \(r_{i}\).

The problem is to find a representative set of test cases \(t_{j}\) that will satisfy
all \(r_{i}\).

Finding the minimal set of test cases is a minimal hitting set problem.
The set cover problem is to identify the smallest sub-collection whose union is the
universe.

If \(r_{1}\) can be satisfied by only 1 test case, the test case is \textbf{essential}.
If a test case satisfies only a subset of the test requirements satisfied by another
test case, it is redundant.

\textbf{GE Heuristic}:
\begin{enumerate}
\item select all essential test cases in the test suite
\item for the remaining test requirements, use the additional greedy algorithm
\begin{enumerate}
\item select the test case that satisfies the max number of unsatisfied test requirements
\end{enumerate}
\end{enumerate}

\textbf{GRE Heuristic}:
\begin{enumerate}
\item remove all redundant test cases in the test suite, which may make some test cases essential
\item perform the GE heuristic on the reduced test suite
\end{enumerate}

\textbf{Hitting Set Heuristic}:
\begin{enumerate}
\item all test cases that occur in single element \(T_{i}\) are first included in the hitting set
and all \(T_{i}\) containing any of these elements are marked
\item all unmarked \(T_{i}\) with 2 elements are considered, the test case that occurs in the maximum
number of \(T_{i}\) of two elements is chosen and included in the hitting set
\item all unmarked \(T_{i}\) with 3 elements are considered, only test cases involved in a tie from earlier
are considered here
\item back to step 2
\end{enumerate}
\subsection{Data Flow Graph}
\label{sec:orga999da7}
To minimize tests with data flow graph, get du pairs and associated testing sets:
\begin{itemize}
\item select essential tests
\item select remaining tests that might be needed
\end{itemize}
\section{Test Selection}
\label{sec:org4c1e970}
While test case selection reduces size of a test suite, most selection techniques are
\textbf{modification-aware}, meaning selection is not only temporary but also focused on
modified parts of the program.

Test cases are selected because they are relevant to changed parts of the system under
test, which involves a white-box static analysis of the program code.
\subsection{Regression Testing}
\label{sec:org6d3452e}
Given a program \(P\), its modified version \(P'\), and the test set \(T\) used to test \(P\),
find a way using \(T\) to gain sufficient confidence in the correctness of \(P'\).

Steps for regression testing:
\begin{enumerate}
\item identify the modifications made to \(P\)
\item select \(T' \subseteq T\), the set of tests to re-execute on \(P'\), which may use the
identified modifications
\item retest \(P'\) with \(T'\), establishing the correctness of \(P'\) with respect to \(T'\)
\item if necessary, create new tests for \(P'\)
\item create a new complete test set for \(P'\), including tests from steps 2 and 4, and old
tests that are not selected, provided that they remain valid
\end{enumerate}

To reduce the time involved in re-testing, choose only tests from \(T\) that produce
different output when run on \(P'\), called modification-revealing tests.

A test \(T_{i}\) is \textbf{modification-revealing} if it produces different outputs in \(P\)
and \(P'\).

Cannot decide if a test is modification-revealing, but can identify the
necessary condition.
For a test to produce different output in \(P\) and \(P'\), it must execute some code
modified from \(P\) to \(P'\).

A test \(T_{i}\) is modification-traversing if it executes a new or modified statement
in \(P'\), or misses a statement in \(P'\) that is executed in \(P\).

Modification-revealing tests are necessarily modification-traversing, but the
other way is not necessary.
\subsection{Evaluation Criteria}
\label{sec:orga03c270}
Criteria are:
\begin{itemize}
\item inclusiveness
\begin{itemize}
\item if \(T\) contains \(n\) modification-revealing tests and \(S\) selects \(m\) of those
tests, the inclusiveness of \(S\) is \(m/n\)
\item if a method \(S\) always selects all modification-revealing tests, \(S\) is safe
\end{itemize}
\item precision
\begin{itemize}
\item measure the extent to which a selective strategy omits tests that are
non-modification-revealing
\item if \(T\) contains \(n\) non-modification-revealing tests and \(S\) omits \(m\) of
those tests, precision of \(S\) is \(m/n\)
\end{itemize}
\end{itemize}

To identify fault-revealing test cases for \(P'\), find the modification-revealing
test cases for \(P\) and \(P'\) using:
\begin{itemize}
\item \textbf{P-Correct-For-T assumption}: assume that for each test \(t \in T\), when \(P\)
was tested with \(t\), \(P\) halted and produced the correct output
\item \textbf{Obsolete-Test-Identification assumption}: assume that there exists an
effective procedure for determining, for each test \(t \in T'\), whether \(t\)
is obsolete for \(P'\)
\end{itemize}

A modification-revealing test must also be fault-revealing.

\textbf{Controlled-Regression-Testing assumption}: when \(P'\) is tested with \(t\), all
factors that might influence the output of \(P'\), except the code in \(P'\), are
kept constant with respect to their states when \(P\) was tested with \(t\)

Given that the Controlled-Regression-Testing assumption holds, a non-obsolete
test case \(t\) can thereby be modification-revealing only if it is also
modification-traversing for \(P\) and \(P'\).

The all 3 assumptions hold \(T_{fr} = T_{mr} \subseteq T_{mt} \subseteq T\).
\subsection{Slicing}
\label{sec:org4d2de7d}
Not every statement that is executed under a test case has an effect on the
program output of the test case (RIP).

\textbf{Slicing}: a program slice consisting of all statements, including conditions
in the program that might affect the value of variable V and point P

\textbf{Backward slices S(v,n)}: refer to statement fragments that contribute to
the value of \(v\) and statement \(n\), where statement \(n\) is a use node of
variable \(v\)

\textbf{Forward slices S(v,n)}: refer to all program statements that are affected
by the value of \(v\) and statement \(n\), referring to the predicate uses
and computation uses of the variable \(v\)

Slices constructed can be very large.

\textbf{Dynamic slice} is constructed with respect to the traditional static slicing
criterion together with dynamic nfo (input sequence to program during specific
execution).

Dynamic slicing criteria are:
\begin{enumerate}
\item variables to be slices (same as static slicing)
\item point of interest in the program (same as static slicing)
\item sequence of input values for which the program was executed
\end{enumerate}

2 main properties desirable in slicing algorithms are \textbf{soundness}
and \textbf{completeness}.

To be \textbf{sound} an algorithm must never delete a statement from the original
program which could have an effect upon the slicing criterion. This allows
analyzing a slice in total confidence that is contains all statements relevant
to the criteria.

To be \textbf{complete} an algorithm must remove all statements which cannot affect the
slicing criterion. Completeness is unachievable due to undecidability.

The goal of good slicing algorithms is to delete as many statements as possible
without giving up soundness. The closer an algorithm approximates completeness,
the most precise the slices it constructs will be.
\section{Prioritization}
\label{sec:orgd93780f}
Given a test suite \(T\), the set of permutations of \(T\) \(PT\), and a function from \(PT\)
to real numbers, \(f: PT \to R\).

The problem is to find \(T' \in PT\) such that for all \(T''\) such that \(T'' \in PT\) and
\(T'' \ne T'\), \(f(T') \ge f(T'')\).

Prioritization concerns ordering test cases for early maximization of some desirable
properties, such as rate of fault detection.
Seeks to find the optimal permutation of the sequence of test cases, and does not
involve selection, assuming all test cases may be executed in the order of the
permutation it produces, but that testing may be terminated at some arbitrary point
during the testing process.

Coverage is a metric often used as the prioritization criteria since early maximization of structural
coverage will also increase the change of early maximization of fault detection.
Prioritization seeks to achieve higher fault detection rate, but actual aim is to maximize early
coverage.
Can be done with random, hill climbing, genetic, greedy, etc.
\subsection{Test history}
\label{sec:org09e8661}
Can have historical failure rates for tests, after each test run, there is more info, such as
co-failure rate (given one test failed, get failure probability of queued tests).

Reorder tests based on cofailure rates.

In reality, too many test requests to globally reorder.
Conditional probability to reorder tests for a single change, and use a scoring function
to globally order tests.
Scoring function is
$$
new_sc = prev_sc + (P(t_{2} = \text{fail} \mid t_{1} = \text{fail}) - 0.5)
$$
where \(prev_sc\) is adjusting the existing score, the probability is given that the previous test failed,
the probability that the next one will fail, and reduce the score if the relationship is < 0.5.

Some tests may starve during prioritization.
The dispatch queue is used for prioritization.
If it is empty, tests from waiting queue go to dispatch queue, in original order,
otherwise if it is smaller than a threshold, tests in the head of the waiting queue will be
picked and put in the dispatch queue in a random order.
\section{Flaky Tests}
\label{sec:org1c0cd80}
A flaky tests passes and fails on the same build.
This could be fine if tests have inherent non-determinism (hardware/environmental, asynchronous,
AI based).
Could also be broken and expensive to fix.

At Facebook, they ran failed tests up to 10 times to find if flaky.
At Google, ran tests 3 times and only report fault if it fails 3 times in a row.
These can be expensive.
At Microsoft, run test 1000 times and check if below a flaky threshold.

To quantify flaky tests:
\begin{enumerate}
\item measure of the degree of test flakiness, where flake rate is flakes/runs
\item establish the flask rate baseline on stable build or release
\item how flaky are tests?
\begin{enumerate}
\item number of runs to have statically confident stable flake rate
\end{enumerate}
\end{enumerate}

Flaky test outcomes include true positive and true negative (fail means fault, pass means no fault),
but also false positive (fail but no fault) and false negative (pass but fault slips through).

Accuracy of a test is \((TP + TN)/\text{runs}\).
Flake rate of a test is \((FP + FN)/\text{runs} = 1 - \text{accuracy}\).
Can measure flake rate over time.

A stable build is one that has been successfully running in production.
A stable build is needed so that any test failure is a flaky failure, and any test pass is a good pass.

Stable accuracy is \(TN/\text{runs}\) or passes over runs.
Stable flake rate is \(FP/\text{runs}\) or fails over runs.

For 99\% confidence in flake rate, have 666 runs.
For 99.9\% confidence in flake rate, have 1083 runs.
Run a test over 1000 times on a stable build.
Use
$$
n = \frac{Z^{2}p(1-p)}{e^{2}}
$$
with \(n\) being the number of reruns, \(p\) being the pass rate, and \(Z\) being

To use flaky test failure rate to prioritize tests:
\begin{enumerate}
\item likelihood of change in stable state (binomial distribution)
\item prioritizing re-runs to find instabilities (probability of a set of runs)
\end{enumerate}

To get the likelihood of getting the set of test results from re-runs, use
\$P\textsubscript{t}(f, r, \text{flakerate}) = \binom{r}{f} * \text{flakerate}\textsuperscript{f} * (1-\text{flakerate})\textsuperscript{r-f}
with the probablity of getting \(f\) failures with \(r\) test runs.
Flake rate is calculated from a stable build and estimated by test failure rate.
Determine if the rate has changed at statistically significant levels.

Tests no longer have a binary outcome, but become unstable relative to the baseline build flake rate
(test failure rate).

Signal is when the test starts failing more than expected, which means someone likely changed code and
introduced a fault.

Noise is when the test fails at expected levels, so just normal interference (async/environmental).

To prioritize re-runs, use \(P_{t}\).
Re-run tests in order by \(P_{t}\) ascending.
Stop when the result is statistically significant or no more re-runs are available.
Will quickly find the tests that have become unstable.

An algorithm to find changes in stable test failure rate:
\begin{enumerate}
\item run each test once
\item calculate \(P_{t}\)
\item order tests by \(P_{t}\)
\item run the test with the lowest \(P_{t}\)
\item investigate tests that show highly unlikely fail rates (are unstable)
\item go to step 5 until no more test runs budgeted or high confidence reached (fail rate stable and
behaviour stable)
\end{enumerate}

Without flaky tests:
\begin{itemize}
\item indicator of something bad (bug) is test fail
\item confidence of test results is irrelevant, just run once
\item prioritization is which test is more likely to fail
\end{itemize}

With flaky tests:
\begin{itemize}
\item indicator of something bad (bug) is ratio between pass/fail changing
\item confidence of test results is achieved from running multiple times, either to reach a budget or
achieve statistical confidence
\item prioritization is which tests has the most unlikely results
\end{itemize}

After finding all new pass/fail distributions for each test, compare with history.
If it has significant deviance (using stats binomial test), this is an indicator of something bad,
otherwise don't do anything.

Tests are not binary (anomaly detection) so check change in stable flake rate.
As long as behaviour of system is stable, don't fix flaky test (still get signal from the test
by re-running and examining \(P_{t}\)).
Making a test less flaky means less noise, so failure becomes more unlikely and requires fewer re-runs.
Tradeoff cost to fix test and cost to re-run and calculate \(P_{t}\).
\end{document}
