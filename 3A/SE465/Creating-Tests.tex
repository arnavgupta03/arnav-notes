% Created 2024-04-05 Fri 11:45
% Intended LaTeX compiler: pdflatex
\documentclass[11pt]{article}
\usepackage[utf8]{inputenc}
\usepackage[T1]{fontenc}
\usepackage{graphicx}
\usepackage{longtable}
\usepackage{wrapfig}
\usepackage{rotating}
\usepackage[normalem]{ulem}
\usepackage{amsmath}
\usepackage{amssymb}
\usepackage{capt-of}
\usepackage{hyperref}
\usepackage{parskip, darkmode}
\enabledarkmode
\author{Arnav Gupta}
\date{\today}
\title{Creating Tests}
\hypersetup{
 pdfauthor={Arnav Gupta},
 pdftitle={Creating Tests},
 pdfkeywords={},
 pdfsubject={},
 pdfcreator={Emacs 29.3 (Org mode 9.7)}, 
 pdflang={English}}
\begin{document}

\maketitle
\tableofcontents

\section{Creating Test Cases}
\label{sec:org861de45}
In order to increase coverage, one needs to generate
test cases that exercise certain statements or follow
specific paths:
\begin{itemize}
\item how to make paths execute: generate input data that
satisfies all conditions on the path
\item key items: input vector, predicate, path predicate,
predicate interpretation, path predicate expression,
create test input from path predicate expression
\end{itemize}

\textbf{Input vector}: collection of all data entities
read by the routine whose values must be fixed
prior to entering the routine
\begin{itemize}
\item members include input arguments, global vars and
constants, files, network connections, timers
\end{itemize}

\textbf{Predicate}: a logical function evaluated at a
decision point

\textbf{Path predicate}: the set of predicates associated
with a path
\begin{itemize}
\item may contain local variables that
\begin{itemize}
\item play no role in selecting inputs that force
a path to execute
\item cannot be selected independently of the input
variables
\item are eliminated with symbolic execution
\end{itemize}
\end{itemize}
\subsection{Symbolic Execution}
\label{sec:org5d986d4}
Symbolically substituting
operations along a path in order to express
the predicate solely in terms of the input
vector and a constant vector.

Key idea:
\begin{itemize}
\item evaluate the program on symbolic input values
\item use an automated theorem prover to check whether
there are corresponding concrete input values that
make the program fail
\end{itemize}
\subsection{Path Predicate Expression}
\label{sec:org6ff7f3c}
Another name for an interpreted path predicate.
Has the following attributes:
\begin{itemize}
\item no local variables
\item a set of constraints in terms of the input vector
and maybe constants
\item by solving the constraints, inputs that force each
path can be generated
\item if a path predicate expression has no solution,
the path is infeasible
\end{itemize}

Can organize the set of path predicates using
a decision table.
\section{Principles of Maintainable Test Code}
\label{sec:org01109c5}
Tests should be fast. For slower tests:
\begin{itemize}
\item use mocks or stubs
\item redesign production code so slower pieces of code
can be tested separately from fast pieces
\item move slower tests to a different test suite that you
can run less often
\end{itemize}


Tests should:
\begin{itemize}
\item be cohesive: test a single behaviour of the system
\item be independent: not depend on oher tests to succeed
\item be isolated: clean up their mess and not rely on
existing data
\item have a reason to exist: find bugs or document behaviour
\item be repeatable: always give same results no matter how
many times they are executed
\item have strong assertions: assert that exercised code
behaves exactly as expected and break on any slight
change in the output
\item break if behaviour changes: TDD can help
\item have a single reason to fail: test code should help
understand what caused the bug
\item be easy to write: write good test infrastructure
that does all setup and tear down
\item be easy to read: all info should be clear enough
\item be easy to change and evolve: changing should not
be painful, less coupled to production code
\end{itemize}
\section{Test Smells}
\label{sec:orga0d6062}
Code smell indicates symptoms that of deeper problems in
source code, hindering comprehensibility and maintainability.
\subsection{Eager Test}
\label{sec:org2227ac9}
When a test method checks several methods of the object
to be tested, it is hard to read and understand so more
difficult to be used as documentation.

Solution: separate test code into test methods that only
test one method
\subsection{Lazy Test}
\label{sec:org00c1974}
Occurs when several test methods check the same method
using the same fixture.

Solution: join them together using inline method
\subsection{Mystery Guest}
\label{sec:org4400d37}
When a test uses external resources, the test is no longer
self-contained so difficult to understand tested
functionality.
Also this introduces hidden dependencies that can cause
tests to fail.

Solution: use refactoring inline resource, or if
external resources are needed, apply setup external
resource to remove hidden dependencies.
\subsubsection{Refactoring: Inline Resource}
\label{sec:orgc0ac75f}
To remove the dependency between a test and some
external resource, setup a fixture that holds
the same content as the resource that is used to run
the test.
\subsubsection{Refactoring: Setup External Resource}
\label{sec:org97624c9}
If necessary for a test to rely on external
resources, make sure the test explicitly creates
or allocates these resources before test and
releases them when done.
\subsection{Resource Optimism}
\label{sec:orgd731989}
Test code that makes optimistic assumptions can cause
non-deterministic behaviour in test outcomes (flaky tests).

Solution: use setup external resource to allocate
or initialize all resources used.
\subsection{Test Run War}
\label{sec:org619d083}
Arise when tests run differently on different devices,
caused by resource interference.

Solution: apply make resource unique to overcome
interference
\subsubsection{Refactoring: Make Resource Unique}
\label{sec:org9ecd8cb}
Can solve issue of overlapping resource names by
using unique IDs for all resources (timestamp).
\subsection{General Fixture}
\label{sec:org9b39b41}
When setup fixtures are too general and different
tests only access part of the fixture, tests run more slowly.

Solution: use setup only for the part of fixture shared
by all tests and put the rest of the fixture in the
method that uses it.
\subsection{Assertion Roulette}
\label{sec:org7543319}
Comes from having a number of assertions in a test method
that have no explanation, if one fails, it is unknown why.

Solution: add assertion explanations
\subsection{Indirect Testing}
\label{sec:org03e7f44}
A test class should test its counterpart in production code,
not other classes.

Solution: move tests to appropriate class
\subsection{For Tester Only}
\label{sec:org0335f8c}
When a production class contains methods only used by test methods.

Solution: methods can be removed or are only needed to setup tests
in which case they should be moved to test code
\subsection{Sensitive Equality}
\label{sec:org76c1d1f}
Tests may depend on irrelevant details for equality checks.

Solution: introduce equality method
\subsubsection{Refactoring: Introduce Equality Method}
\label{sec:org67e4583}
If an object structure needs to be checked for equality in
tests, add an implementation for the equals method.
\subsection{Test Code Duplication}
\label{sec:orgcabe8b9}
Test code can have duplication of responsibility.

Solution: can be removed by extracting methods
\subsection{Other Test Smells}
\label{sec:orgb0df1d0}
Other examples are redundant assertions, unknown tests (without
assertion), empty test, and duplicate assertions.
\end{document}
