% Created 2024-04-19 Fri 15:20
% Intended LaTeX compiler: pdflatex
\documentclass[11pt]{article}
\usepackage[utf8]{inputenc}
\usepackage[T1]{fontenc}
\usepackage{graphicx}
\usepackage{longtable}
\usepackage{wrapfig}
\usepackage{rotating}
\usepackage[normalem]{ulem}
\usepackage{amsmath}
\usepackage{amssymb}
\usepackage{capt-of}
\usepackage{hyperref}
\usepackage{parskip,darkmode}
\enabledarkmode
\author{Arnav Gupta}
\date{\today}
\title{Metamorphic Testing}
\hypersetup{
 pdfauthor={Arnav Gupta},
 pdftitle={Metamorphic Testing},
 pdfkeywords={},
 pdfsubject={},
 pdfcreator={Emacs 29.3 (Org mode 9.7)}, 
 pdflang={English}}
\begin{document}

\maketitle
\tableofcontents

\section{Metamorphic Testing}
\label{sec:orgc2b53b8}
Software testing assumes the availability of a test oracle, but this may not hold in practice.

For software running in production, if there is no crash, no way to know if there is a mistake.

\textbf{Metamorphic testing} is the generation of new test cases based on the input/output pairs of
previous test cases, in particular successful ones.

Assume there are errors inside the software in spite of apparently correct output produced
from the current input.
New test cases evolve from old ones that are normally selected according to some existing
selection criteria.

Given a program P and test case x, the corresponding output is P(x).
Assume x0 is a test case with output P(x0).
Metamorphic testing constructs a number of new test cases based on the input-output pair
(x0, P(x0)) and the error usually associated with P.
New test cases are designed to reveal errors that go undetected by the test case x0.
\end{document}
