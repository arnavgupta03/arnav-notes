% Created 2024-04-05 Fri 11:44
% Intended LaTeX compiler: pdflatex
\documentclass[11pt]{article}
\usepackage[utf8]{inputenc}
\usepackage[T1]{fontenc}
\usepackage{graphicx}
\usepackage{longtable}
\usepackage{wrapfig}
\usepackage{rotating}
\usepackage[normalem]{ulem}
\usepackage{amsmath}
\usepackage{amssymb}
\usepackage{capt-of}
\usepackage{hyperref}
\usepackage{parskip, darkmode}
\enabledarkmode
\author{Arnav Gupta}
\date{\today}
\title{Fuzzing}
\hypersetup{
 pdfauthor={Arnav Gupta},
 pdftitle={Fuzzing},
 pdfkeywords={},
 pdfsubject={},
 pdfcreator={Emacs 29.3 (Org mode 9.7)}, 
 pdflang={English}}
\begin{document}

\maketitle
\tableofcontents

\section{Fuzzing}
\label{sec:org4902d0a}
Set of automated testing techniques that identifies abnormal program behaviours by
evaluating how the tested program responds to various random inputs.

Basic fuzzer is purely random:
\begin{itemize}
\item easy to implement and fast, but relies on luck and has shallow program exploration.
\end{itemize}

Fuzzer categories:
\begin{itemize}
\item \textbf{generation-based}: inputs generated from scratch
\item \textbf{mutation-based}: inputs from modifying existing inputs
\item \textbf{dumb} or \textbf{smart} depending on awareness of input structure
\item \textbf{white/grey/black box} depending on awareness of program structure
\end{itemize}
\subsection{Goal}
\label{sec:org3433d56}
\begin{enumerate}
\item Find real bugs
\item Reduce the number of false positives
\begin{enumerate}
\item generate reasonable input
\end{enumerate}
\end{enumerate}
\subsection{Mutation-Based Fuzzing}
\label{sec:orge950eef}
\begin{enumerate}
\item Take existing input
\item Randomly modify it
\item Pass it to the program
\end{enumerate}
\subsubsection{Grammar-Based Fuzzing}
\label{sec:orged0b1a2}
Grammars can easily formally specify input languages.

A simple grammar fuzzer:
\begin{enumerate}
\item starts with the start symbol then keeps expanding it
\item to avoid expansion to infinite inputs, place a limit on the \# of nonterminals
\item to avoid being stuck in a situation where we cannot reduce the number of symbols
further, limit the total \# of expansion steps
\end{enumerate}

Grammar generated inputs can be seeds in mutation-based fuzzing.
\subsubsection{Guiding by Coverage}
\label{sec:org6cdb748}
Retrieve coverage of a test run and evolve inputs that are successful (where a new
path was found during test execution).
\subsection{Gerybox Fuzzing}
\label{sec:org79573f5}
\textbf{Power schedule}: distributes fuzzing time among the seeds in the population

\textbf{Objective}: maximize the time spend fuzzing the seeds that lead to higher coverage
increase in shorter time.

\textbf{Seed Energy}: likelihood with which a seed is chosen from a population
\subsubsection{Directed Gerybox Fuzzing}
\label{sec:org81f13aa}
Implement a power schedule that assigns more energy to seeds with a low distance
to the target function:
\begin{itemize}
\item build a call graph among function in a program
\item for each function executed in the test, calculate its shortest path to the
target function, then do an average distance, using the distance to calculate
the power schedule
\item can do more complex calculation by considering finer grained info
\end{itemize}
\subsection{Search-Based Fuzzing}
\label{sec:org2b29be5}
For deriving specific test inputs that achieve some objective.
Uses domain knowledge of which inputs is closest to what one is looking for.
\subsubsection{Hillclimbing Algorithm}
\label{sec:org08aef75}
\begin{enumerate}
\item Take a random starting point
\item Determine fitness value of all neighbours
\item Move to the neighbour with the best fitness value
\item If solution not found, back to step 2.
\end{enumerate}
\subsubsection{Genetic Algorithm}
\label{sec:orgf655098}
Based on the idea that solutions can be genetically encoded, based on natural
selection.
A fitness function takes the info contained in the description and evaluates
properties.

Emulates natural evolution with the following process:
\begin{itemize}
\item create an initial population of random chromosomes
\item select fit individuals for reproduction
\item generate new population through reproduction of selected individuals
(selecting parents and creating mutations)
\item continue until an optimal solution or limit reached
\end{itemize}
\begin{enumerate}
\item Challenge 1: Feeding Inputs
\label{sec:orgab242ce}
The fuzzing engine executes the fuzz target many times, so the target must:
\begin{itemize}
\item tolerate any kind of input
\item not exit on any input
\item join all threads at the end
\item be fast and as deterministic as possible
\item not modify any global state
\item be as narrow as possible
\end{itemize}
\item Challenge 2: Detecting Abnormal Behaviour
\label{sec:org07baee8}
Refers to crashes, triggerring user-provided assertion failure, hanging,
or allocating too much memory.

For early crash detection use sanitizers for address, thread, memory,
undefined behaviour, or leaks.
\item Challenge 3: Ensuring Progress
\label{sec:orgd3ef699}
Use coverage to evolve test cases that find new paths through program
execution, with Gerybox or search-based approaches.

Coverage with American Fuzzy Lop captures branch coverage by
instrumenting compiled programs which is fast.
\item Challenge 4: Coming up with Interesting Inputs
\label{sec:org5744615}
Must understand input type and structure, using model, grammar, or
protocol based fuzz.
\item Challenge 5: Speed
\label{sec:org14cbfe3}
To help with speed:
\begin{itemize}
\item initialize once and fork for other inputs
\item replace costly resources with cheaper ones
\item run many inputs on a single process
\item minimize the \# of test cases, discarding redundant ones
\item run in parallel, distributed
\end{itemize}
\end{enumerate}
\subsection{Problems with Fuzzing}
\label{sec:org91a0305}
Many false positive that are expensive.

Focus on code coverage, which is less important than reasonable inputs.

Cleaning to make random input more reasonable:
\begin{itemize}
\item \textbf{minimization}: eliminate redundant test failures through diffing
\item \textbf{triage}: finding similar outputs/stackdumps and grouping them in bug report
\end{itemize}
\end{document}
