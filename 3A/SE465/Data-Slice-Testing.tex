% Created 2024-04-05 Fri 11:43
% Intended LaTeX compiler: pdflatex
\documentclass[11pt]{article}
\usepackage[utf8]{inputenc}
\usepackage[T1]{fontenc}
\usepackage{graphicx}
\usepackage{longtable}
\usepackage{wrapfig}
\usepackage{rotating}
\usepackage[normalem]{ulem}
\usepackage{amsmath}
\usepackage{amssymb}
\usepackage{capt-of}
\usepackage{hyperref}
\usepackage{parskip, darkmode}
\enabledarkmode
\author{Arnav Gupta}
\date{\today}
\title{Data Slice Testing}
\hypersetup{
 pdfauthor={Arnav Gupta},
 pdftitle={Data Slice Testing},
 pdfkeywords={},
 pdfsubject={},
 pdfcreator={Emacs 29.3 (Org mode 9.7)}, 
 pdflang={English}}
\begin{document}

\maketitle
\tableofcontents

\section{Data Slices}
\label{sec:orgd21991d}
\subsection{Motivation}
\label{sec:orgffdb37e}
Argued that testing a class aims at finding the sequence of operations
for which a class will be in a state or has produced output that is
contradictory to its expected results.

Testing classes for all possible sequences is not usually possible,
so necessary to reduce the number of sequences and still provide
sufficient confidence.

Can have state-dependent faults: executing some operations is not
possible in certain states where they are legal
\subsection{Principles}
\label{sec:org13426cd}
Goal: reduce the number of method sequences to test

The concrete state of an object at a single instant of time is
equivalent to the aggregated state of each of its data members at that
instant.

Correctness of a class depends on:
\begin{itemize}
\item whether the data members are correctly representing the intended
state of an object
\item whether the member function are correctly manipulating the
representation of the object
\end{itemize}
\subsection{Approach}
\label{sec:org2618d27}
Data slice: portion of a class with only a single data member and
a set of member functions such that each member function can
manipulate the values associated with this data member.

Test one slice at a time, and for each slice, test possible sequences
of the methods belonging to that slice.

Code-based approach so potential problems if code is faulty
or with identifying legal sequences of methods.
\subsection{Generate MaDUM}
\label{sec:orgcc06cb6}
Minimal Data Member Usage Matrix: \(n \times m\) matrix where \(n\) is
the number of member methods, reports on the usage of data
members by methods

Different usage categories are constructors, reporters (read only),
transformers, others.

Can create call graph from matrix and identify redundancy where all
branches of data member usage.

Matrix accounts for indirect use, through called methods.
\subsection{Test Procedure}
\label{sec:orgda9ab09}
Classification of methods is used to decide the test steps to
be taken:
\begin{enumerate}
\item test reporters
\item test constructors
\item test transformers
\item test others
\end{enumerate}
\subsubsection{Test Reporters}
\label{sec:orge64fc62}
Test by:
\begin{enumerate}
\item Make sure the reporter method does not alter the state of the
data member that it reports on
\item Create a random sequence of operations except for reporters
of the data slice under test and append it with the
reporter method
\end{enumerate}
\subsubsection{Test Constructors}
\label{sec:org603293c}
Test that
\begin{itemize}
\item All data members are correctly initialized
\item All data members are initialized in correct order
\end{itemize}

Test by:
\begin{enumerate}
\item Run the constructor and append the reporter methods for
each data member
\item Verify if in correct initial state (state invariant)
\end{enumerate}
\subsubsection{Test Transformers}
\label{sec:org9385b65}
Test interactions between methods.

For each data slice:
\begin{enumerate}
\item Instantiate object under test with constructor (already tested)
\item Create sequences
\item Append the reporter methods (already tested)
\end{enumerate}

For each member function, execute all control flow paths where
data slice is manipulated.
\subsubsection{Test Others}
\label{sec:org51e886a}
Do not change the state, do not report any data member,
so should be tested as any other functions.
\subsection{Discussion}
\label{sec:orga58745a}
Slicing may not be helpful for some classes.
Too many sequences or impossible sequences are potential problems.
\section{Testing Derived Classes}
\label{sec:org0def790}
Derived classes may add facilities or modify the ones provided by the
base class.

Two extreme options for testing a derived class:
\begin{itemize}
\item flatten the derived class and retest all slices of the base class
in the new context
\item only test the new/redefined slices of derived class
\end{itemize}
\subsection{Bashir and Goel's Strategy}
\label{sec:orge452332}
Extend the MaDUM of the base class to generate MaDUM for the derived
class:
\begin{itemize}
\item take the MaDUM of the base class
\item add a row for each newly defined or re-defined data member of the
derived class
\item add a column for each newly defined of re-defined member
function of the derived class
\end{itemize}
\subsection{Procedure}
\label{sec:org57821a6}
\textbf{Local attributes}: similar to base class testing

\textbf{Retest inherited attributes}:
\begin{itemize}
\item if they are directly or indirectly accessed by a new or
re-defined method of the derived class
\item check the upper right quadrant of the derived MaDUM
\item check which interited attributes mandate retesting
\item once these inherited attributes are identified, the test procedure is
similar to slice testing in the base class but uses inherited and
redefined methods
\end{itemize}

An inherited data member needs to be retested if the number of entries
in its MaDUM row has increased between the base and derived MaDUMs.
\end{document}
