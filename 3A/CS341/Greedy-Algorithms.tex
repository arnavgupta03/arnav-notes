% Created 2024-04-09 Tue 17:48
% Intended LaTeX compiler: pdflatex
\documentclass[11pt]{article}
\usepackage[utf8]{inputenc}
\usepackage[T1]{fontenc}
\usepackage{graphicx}
\usepackage{longtable}
\usepackage{wrapfig}
\usepackage{rotating}
\usepackage[normalem]{ulem}
\usepackage{amsmath}
\usepackage{amssymb}
\usepackage{capt-of}
\usepackage{hyperref}
\usepackage{parskip,darkmode}
\enabledarkmode
\author{Arnav Gupta}
\date{\today}
\title{Greedy Algorithms}
\hypersetup{
 pdfauthor={Arnav Gupta},
 pdftitle={Greedy Algorithms},
 pdfkeywords={},
 pdfsubject={},
 pdfcreator={Emacs 29.3 (Org mode 9.7)}, 
 pdflang={English}}
\begin{document}

\maketitle
\tableofcontents

\section{Overview}
\label{sec:org16d4f40}
For trying to solve a combinatorial optimization problem:
\begin{itemize}
\item have a large, but finite domain \(\mathcal{D}\)
\item want to find an element \(E\) in \(\mathcal{D}\) that
minimizes/maximizes a cost function
\end{itemize}
\subsection{Strategy}
\label{sec:orgb0f3a75}
\begin{itemize}
\item build \(E\) step by step
\item don't think ahead, just try to improve as much as
possible at every step
\item simple algorithms, but usually no guarantee to get
the optimal
\item hard to prove correctness, easy to prove incorrectness
\end{itemize}
\section{Interval Scheduling}
\label{sec:orga0a2650}
\subsection{Problem}
\label{sec:orga98461a}
\begin{itemize}
\item input: \(n\) intervals \([s_{1}, f_{1}], [s_{2}, f_{2}], \dots, [s_{n}, f_{n}]\)
\item output: a maximal subset of disjoint intervals
\end{itemize}
\subsection{Algorithm}
\label{sec:org28a5c04}
\begin{enumerate}
\item Let \(S\) be the empty set
\item Sort the intervals such that \(f_{1} \le f_{2} \le \cdots \le f_{n}\)
\item For \(i\) from \(1\) to \(n\) do
\begin{enumerate}
\item if interval \(i\), \([s_{i}, f_{i}]\) has no conflicts with
intervals in \(S\)
\begin{enumerate}
\item add \(i\) to \(S\)
\end{enumerate}
\end{enumerate}
\item Return \(S\)
\end{enumerate}
\subsection{Correctness}
\label{sec:orge3c4610}
\subsubsection{The Greedy Algorithm Stays Ahead}
\label{sec:orgef4100f}
Assume \(O\) is an optimal solution, the goal is to show \(|S| = |O|\).

\begin{itemize}
\item Suppose \(i_{1}, i_{2}, \dots, i_{k}\) are the intervals in \(S\) in the
order they were added to \(S\) by the greedy algorithm
\item Similarly, the intervals in \(O\) are denoted by \(j_{1}, \dots, j_{m}\)
\begin{itemize}
\item assume that the intervals in \(O\) are ordered in the order of the
start and finish times
\end{itemize}
\item Prove that \(k = m\)
\end{itemize}
\subsubsection{Lemma}
\label{sec:org58c001b}
First consider the lemma:
For all indices \(r \le k\) we have \(f(i_{r}) \le f(j_{r})\).

By induction:
\begin{itemize}
\item for \(r = 1\), the statement is true
\item suppose \(r > 1\) and the statement is true for \(r - 1\)
\begin{itemize}
\item we show that the statement is true for \(r\)
\end{itemize}
\item by induction hypothesis, \(f(i_{r-1}) \le f(j_{r-1})\)
\item by the order on \(O\), \(f(j_{r-1}) < s(j_{r})\)
\item hence \(f(i_{r-1}) < s(j_{r})\)
\item thus, at the time the greedy algorithm chose \(i_{r}\),
the interval \(j_{r}\) was a possible choice
\item the greedy algorithm chooses an interval with the smallest finish time
\begin{itemize}
\item so \(f(i_{r}) \le f(j_{r})\)
\end{itemize}
\end{itemize}
\subsubsection{Proof}
\label{sec:org691bd48}
Theorem: The greedy algorithm returns an optimal solution

Prove by contradiction:
\begin{itemize}
\item if the output \(S\) is not optimal, then \(|S| < |O|\)
\item \(i_{k}\) is the last interval in \(S\) and \(O\) must have
an interval \(j_{k+1}\)
\item apply the previous lemma with \(r = k\) and we get \(f(i_{k}) \le f(j_{k})\)
\item we have \(f(i_{k}) \le f(j_{k}) < s(j_{k+1})\)
\item so \(j_{k+1}\) was a possible choice to add to \(S\) by the greedy algorithm
\begin{itemize}
\item this is a contradiction by how the greedy algorithm works
\end{itemize}
\end{itemize}
\section{Interval Colouring}
\label{sec:orge28b06d}
\subsection{Problem}
\label{sec:org44f9dcb}
\begin{itemize}
\item input: \(n\) intervals \([s_{1}, f_{1}], [s_{2}, f_{2}], \dots, [s_{n}, f_{n}]\)
\item output: use the minimum number of colours to colour the intervals
so that each interval gets one colour and two overlapping intervals get two
different colours
\end{itemize}
\subsection{Algorithm}
\label{sec:org0755402}
\begin{enumerate}
\item Sort the intervals by starting time: \(s_{1} \le s_{2} \le \cdots \le s_{n}\)
\item For \(i\) from 1 to \(n\) do
\begin{enumerate}
\item use the minimum available colour \(c_{i}\) to colour the interval \(i\)
(one that doesn't conflict with the colours of intervals already coloured)
\end{enumerate}
\end{enumerate}
\subsection{Correctness}
\label{sec:org7290112}
Assume the greedy algorithm uses \(k\) colours.
To prove correctness, the goal is to show that there are no other ways to solve
the problem using at most \(k-1\) colours.
\subsubsection{Proof}
\label{sec:orgb4ddd45}
Suppose interval \(\ell\) is the first interval to use the colour \(k\):
\begin{itemize}
\item interval \(\ell\) overlaps with intervals with colours \(1, \dots, k - 1\)
\item call these intervals
\([s_{i_{1}}, f_{i_{1}}], [s_{i_{2}}, f_{i_{2}}], \dots, [s_{i_{k-1}}, f_{i_{k-1}}]\)
\item for \(1 \le j \le k-1\), \(s_{i_{j}} \le s_{\ell}\)
\item all the intervals overlap with \([s_{\ell}, f_{\ell}]\)
\item since all these intervals overlap with \([s_{\ell}, f_{\ell}]\),
\(s_{\ell} \le f_{i_{j}}\) for \(1 \le j \le k-1\)
\item hence \(s_{\ell}\) is a time contained in \(k\) intervals
\item so, there is no \(k-1\) colouring
\end{itemize}
\section{Minimizing Total Completion Time}
\label{sec:orgbadfc34}
\subsection{Problem}
\label{sec:org5819e10}
\begin{itemize}
\item input: \(n\) jobs, each requiring processing time \(p_{i}\)
\item output: an ordering of the jobs such that the total completion time is minimized
\end{itemize}
\subsection{Algorithm}
\label{sec:orgbba33aa}
\begin{itemize}
\item order the jobs in non-decreasing processing times
\end{itemize}
\subsection{Correctness}
\label{sec:orge0ad702}
\begin{itemize}
\item let \(L = [e_{1}, \dots, e_{n}]\) be an optimal solution (as a permutation of
\([1, \dots, n]\))
\item suppose that \(L\) is not in non-decreasing order of processing times
\begin{itemize}
\item so there exists \(i\) such that \(t(e_{i}) > t(e_{i+1})\)
\end{itemize}
\item sum of the completion times of \(L\) is \(nt(e_{1}) + (n-1) t(e_{2}) + \cdots + t(e_{n})\)
\item contribution of \(e_{i}\) and \(e_{i+1}\) is
\((n-i+1)t(e_{i}) + (n-i)t(e_{i+1})\)
\item let \(L'\) be the permutation with \(e_{i}\) and \(e_{i+1}\) switched
\item their contribution becomes \((n-i+1)t(e_{i+1}) + (n-i)t(e_{i})\)
\item nothing else changes so \(T(L') - T(L) = t(e_{i+1}) - t(e_{i}) < 0\)
which is a \textbf{contradiction}
\end{itemize}
\end{document}
