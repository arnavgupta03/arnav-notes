% Created 2024-04-09 Tue 17:51
% Intended LaTeX compiler: pdflatex
\documentclass[11pt]{article}
\usepackage[utf8]{inputenc}
\usepackage[T1]{fontenc}
\usepackage{graphicx}
\usepackage{longtable}
\usepackage{wrapfig}
\usepackage{rotating}
\usepackage[normalem]{ulem}
\usepackage{amsmath}
\usepackage{amssymb}
\usepackage{capt-of}
\usepackage{hyperref}
\usepackage{parskip,darkmode}
\enabledarkmode
\author{Arnav Gupta}
\date{\today}
\title{Minimum Spanning Trees}
\hypersetup{
 pdfauthor={Arnav Gupta},
 pdftitle={Minimum Spanning Trees},
 pdfkeywords={},
 pdfsubject={},
 pdfcreator={Emacs 29.3 (Org mode 9.7)}, 
 pdflang={English}}
\begin{document}

\maketitle
\tableofcontents

\section{Spanning Trees}
\label{sec:org8813c3c}
For a connected graph \(G = (V, E)\), a spanning tree in \(G\) is a tree of the form \((V, A)\)
with \(A\) a subset of \(E\) (a tree with edges from \(E\) that covers all vertices).

The goal is to find a spanning tree with minimal weight.
\section{Kruskal's Algorithm}
\label{sec:orgf6e6e37}
\begin{verbatim}
GreedyMST(G):
    A = []
    sort edges by increasing weight
    for k = 1, ..., m:
        if e_k does not create a cycle in A:
            append e_k to A
\end{verbatim}
\subsection{Augmenting Sets without Cycles}
\label{sec:org7b04a7e}
\textbf{Claim}: Let \(G\) be a connected graph and let \(A\) be a subset of the edges of \(G\).
If \((V, A)\) has no cycle and \(|A| < n - 1\), then one can find an edge \(e\) not in \(A\)
such that \(A \cup \{e\}\) still has no cycle.

\textbf{Proof}:
\begin{itemize}
\item in any graph, \# vertices - \# connected components \(\le\) \# edges
\item for \((V, A)\), this gives \(n - c < n - 1\) so \(c > 1\)
\item take any edge on a path that connects two components
\end{itemize}
\subsection{Properties of the Output}
\label{sec:org2f61793}
\textbf{Claim}: If the output is \(A = [e_{1}, \dots, e_{r}]\) then \((V, A)\) is a spanning tree
\(r = n - 1\).

\textbf{Proof}:
\begin{itemize}
\item \((V, A)\) has no cycle
\item suppose \((V,A)\) is not a spanning tree, then there exists an edge \(e\) not in A such
that \((V, A \cup \{e\})\) still has no cycle
\begin{itemize}
\item for the case where \(w(e) < w(e_{1})\), this is impossible since \(e_{1}\) has the
smallest weight
\item for the case where \(w(e_{i}) < w(e) < w(e_{i+1})\), this is impossible since at
the moment we had inserted \(e_{i+1}\), we decided not to include \(e\) which means
that \(e\) created a loop with \(e_{1}, \dots, e_{i}\)
\item for the case where \(w(e_{r}) < w(e)\), this is impossible since if it was included
in \(A\) since there is no loop in \(A \cup \{e\}\)
\end{itemize}
\end{itemize}
\subsection{Exchanging Edges}
\label{sec:orgb06c5a2}
\textbf{Claim}: Let \((V, A)\) and \((V, T)\) be 2 spanning trees and let \(e\) be an edge in \(t\) but not in \(A\).
Then there is some edge \(e'\) in \(A\) but not in \(T\) such that \((V, T + e' - e)\) is still a spanning
tree. Further, \(e'\) is on the cycle that \(e\) creates in \(A\).

\textbf{Proof}:
\begin{itemize}
\item consider \(e = \{v, w\}\)
\item \((V, A + e)\) contains a cycle \(c = v, w, \dots, v\)
\item removing \(e\) from \(T\) splits \((V, T - e)\) into two connected components \(T_{1}, T_{2}\)
\item \(c\) starts in \(T_{1}\), crosses over to \(T_{2}\), so it contains another edge \(e'\) between \(T_{2}\)
and \(T_{1}\)
\item \(e'\) is in \(A\) but not in \(T\)
\item \((V, T + e' - e)\) is a spanning tree
\end{itemize}
\subsection{Correctness: Exchange Argument}
\label{sec:orga6015c6}
Let \(A\) be the output of the algorithm, \((V,T)\) be any spanning tree.
If \(T \ne A\), let \(e\) be an edge in \(T\) but not in \(A\). This means there is an edge \(e'\)
in \(A\) but not in \(T\) such that \((V, T + e' - e)\) is a spanning tree and \(e'\) is on the
cycle that \(e\) creates in \(A\).

During the algorithm, we considered \(e\) but rejected it because it created a cycle in \(A\).
All other elements in this cycle have smaller (or equal) weight, so \(w(e') \le w(e)\) and so
\(T' = T + e' - e\) has weight \(\le w(T)\) and one more common element with \(A\). This continues.
\section{Data Structures for Kruskal's Algorithm}
\label{sec:orgd6b036d}
Operations possible on disjoint sets of vertices are:
\begin{itemize}
\item \textbf{find}: identify which set contains a given vertex
\item \textbf{union}: replace 2 sets by their union
\end{itemize}
\subsection{Implementation}
\label{sec:orgc5f0f30}
\begin{verbatim}
GreedyMST_UnionFind(G):
    T = []
    U = {{v1}, ..., {vn}}
    sort edges by increasing weight
    for k in range(1, m):
        if U.find(ek.1) != U.find(ek.2):
            U.union(U.find(ek.1), U.find(ek.2))
            append ek to T
\end{verbatim}
\subsection{Linked List}
\label{sec:orgf4d7cbf}
Uses an array of linked lists for \(U\).

To do find, add an array of indices where \(X[i]\) is the set that contains \(i\).

In the worst case for this, find is \(O(1)\) but union traverses one of the linked lists,
updates the corresponding entries of \(X\), and concatenates 2 linked lists,
so union worst case is \(\Theta(n)\).

This gives Kruskal's Algorithm to be \(O(m \log(m))\) in sorting edges, \(O(m)\) for find,
\(O(n)\) for union, and overall worst case \(O(m \log(m) + n^{2})\).
\subsection{Simple Heuristics for Union}
\label{sec:org777ed7f}
\subsubsection{Modified Union}
\label{sec:org04f4c16}
Each set in \(U\) keeps track of its size and only traverse the smaller list.

Also, add a pointer to the trail of the lists to concatenate in \(O(1)\).
\subsubsection{Key Observation}
\label{sec:orgaff035d}
Worst case for 1 union is still \(\Theta(n)\) but better total time:
\begin{itemize}
\item for any given vertex \(v\), the size of the set containing \(v\) at least doubles when
we update \(X[v]\), so \(X[v]\) updated at most \(\log(n)\) times
\item so the total cost of union per vertex is \(O(\log(n))\)
\end{itemize}
\end{document}
