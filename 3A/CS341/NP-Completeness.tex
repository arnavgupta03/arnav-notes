% Created 2024-03-30 Sat 02:09
% Intended LaTeX compiler: pdflatex
\documentclass[11pt]{article}
\usepackage[utf8]{inputenc}
\usepackage[T1]{fontenc}
\usepackage{graphicx}
\usepackage{longtable}
\usepackage{wrapfig}
\usepackage{rotating}
\usepackage[normalem]{ulem}
\usepackage{amsmath}
\usepackage{amssymb}
\usepackage{capt-of}
\usepackage{hyperref}
\usepackage{parskip,darkmode}
\enabledarkmode
\author{Arnav Gupta}
\date{\today}
\title{NP-Completeness}
\hypersetup{
 pdfauthor={Arnav Gupta},
 pdftitle={NP-Completeness},
 pdfkeywords={},
 pdfsubject={},
 pdfcreator={Emacs 29.3 (Org mode 9.7)}, 
 pdflang={English}}
\begin{document}

\maketitle
\tableofcontents

\section{NP Class}
\label{sec:org5e36904}
For a problem \(X\), represent an instance of \(X\) as a binary string \(S\).

A problem \(X\) is \textbf{in NP} if there is a polynomial time verification algorithm \(ALG_{X}\) such that
the input \(S\) is a yes-instance if and only if there is a proof (certificate) \(t\) which is a binary
string of length \(\text{poly}(|S|)\) so that \(ALG_{X}(S,t)\) returns yes.

Vertex Cover, Clique, IS, HC, HP, Subset-Sum, and 3-SAT are in NP.

Not all problems are in NP.

\textbf{co-NP}: the set of decison problems whose no-instances can be certified in polynomial time

Every polynomial time solvable problem is in NP, where \textbf{P} is the set of decision problems that
can be solved in polynomial time.
$$
P \subseteq NP
$$
The most important open problme in theoretical computer science is \(P = NP\)?

The name NP comes from Non-deterministic Polynomial time, where a non-deterministic machine
has the power to correctly guess a solution.
\section{NP-Completeness}
\label{sec:org537a92c}
A problem is NP-complete if it is the hardest problem in NP.

A problem \(X \in NP\) is \textbf{NP-complete} if \(Y \le_{P} X\) for all \(Y \in NP\)

\(P = NP\) if an only if an NP-complete problem can be solved in poly-time

3-SAT is NP-Complete, proven by Cook and Levin

Consequences of this are
\begin{itemize}
\item for any NP problem \(X\), if one can prove 3-SAT \(\le_{P} X\), then \(X\) is NP-complete
\item to prove that a problem \(X \in NP\) is NP-complete, just need to find an NP-complete problem
\(Y\) and prove that \(Y \le_{P} X\)
\end{itemize}

Some NP-complete problems are
\begin{itemize}
\item 3-SAT, SAT
\item independent set, vertex cover, clique
\item (directed) Hamiltonian cycle, Hamiltonian path
\item travelling salesman
\item subset sum
\item 0/1 knapsack
\end{itemize}
\subsection{Circuit Satisfiability}
\label{sec:org9da4aff}
Circuit-SAT is defined by
\begin{itemize}
\item instance: a circuit is a DAG with labels on the vertices
\item inputs labelled by and/or/not
\item there is a marked vertex \(v\) for output
\item \textbf{problem}: is there a choice of boolean \(x_{i}\) that makes \(v\) true
\end{itemize}

Plan for the Cook-Levin Theorem is
\begin{enumerate}
\item show that Circuit-SAT is NP-complete
\item show that Circuit-SAT \(\le_{P}\) 3-SAT
\end{enumerate}

Sketch of proof that Circuit-SAT is NP-complete uses the following idea:
\begin{itemize}
\item given: instance \(S\) of \(A \in NP\)
\item want: proof (certificate) \(t\) such that \(ALG_{A}(S,t)\) is true
\item verification algorithm \(ALG_{A}(S,t)\) can be turned into a circuit with \(t\) as input
\item call Circuit-SAT to find \(t\)
\end{itemize}
\end{document}
