% Created 2024-04-09 Tue 17:55
% Intended LaTeX compiler: pdflatex
\documentclass[11pt]{article}
\usepackage[utf8]{inputenc}
\usepackage[T1]{fontenc}
\usepackage{graphicx}
\usepackage{longtable}
\usepackage{wrapfig}
\usepackage{rotating}
\usepackage[normalem]{ulem}
\usepackage{amsmath}
\usepackage{amssymb}
\usepackage{capt-of}
\usepackage{hyperref}
\usepackage{parskip,darkmode}
\enabledarkmode
\author{Arnav Gupta}
\date{\today}
\title{Dynamic Programming Part 4}
\hypersetup{
 pdfauthor={Arnav Gupta},
 pdftitle={Dynamic Programming Part 4},
 pdfkeywords={},
 pdfsubject={},
 pdfcreator={Emacs 29.3 (Org mode 9.7)}, 
 pdflang={English}}
\begin{document}

\maketitle
\tableofcontents

\section{Bellman-Ford Algorithm}
\label{sec:org3da980e}
\subsection{Outlook}
\label{sec:org6217f2f}
Source is fixed.

If no negative cycle, compute all distances \(\delta(s, v)\).

Can detect negative cycles.

Very simple pseudo-code, but slower than Dijkstra's.
\subsection{Algorithm}
\label{sec:orgf24d29d}
\subsubsection{Definition}
\label{sec:orga492952}
For \(i\) from 0 to \(n-1\) set
\(\delta_{i}(s, v)\) to be the length of the shortest path from \(s\) to \(v\) with at most
\(i\) edges.

If no such path exists \(\delta_{i}(s, v) = \infty\).
\subsubsection{Observations}
\label{sec:org75c3122}
This gives \(\delta_{0}(s, s) = 0\) and \(\delta_{0}(s, v) = \infty\) for any other \(v\).

If there is no negative cycle, \(\delta_{n-1}(s, v) = \delta(s, v)\) (shortest paths
are simple).

In any case, \(\delta(s, v) \le \delta_{i}(s, v)\) for all \(i\) and for all \(v\).
\subsubsection{Recurrence}
\label{sec:org9f95e23}
$$
\delta_{i}(s, v) = \min(\delta_{i-1}(s,v), \min_{(u, v) \in E} \delta_{i-1} (s, u) + w(u, v))
$$
\subsection{Pseudocode 1}
\label{sec:orge66aab4}
\begin{verbatim}
d0 = [0, inf, ..., inf]
parent = [s, 0, ..., 0]
for i in range(1, n-1):
    for all v in V:
        di[v] = d(i-1)[v]
        for all (u,v) in E:
            if d(i-1)[u] + w(u,v) < di[v]:
                di[v] = d(i-1)[u] + w(u,v)
                parent[v] = v
\end{verbatim}
\subsubsection{Correctness}
\label{sec:orgac50b84}
\(d_{i}[v] = \delta_{i}(s,v)\) so \(d_{n-1}[v] = \delta(s,v)\) if no negative cycles
\subsection{Pseudocode 2}
\label{sec:org412cbe2}
\textbf{Idea}: use a single array \(d\)

\begin{verbatim}
d = [0, inf, ..., inf]
parent = [s, 0, ..., 0]
for i in range(1, n-1):
    for all (u,v) in E:
        if d[u] + w(u,v) < d[v]:
            d[v] = d[u] + w(u,v)
            parent[v] = u
\end{verbatim}

\textbf{Runtime}: \(O(mn)\)
\subsubsection{Correctness}
\label{sec:orgc0416d4}
\textbf{Claim 1}: For all \(i\), after iteration \(i\), \(d[v] \le d_{i}[v]\) for all \(v\)

\textbf{Proof}: by induction:
\begin{itemize}
\item true for \(i=0\) so suppose true at index \(i-1\) and prove true at \(i\)
\item at the beginning of the loop, for all \(v\), \(d[v] \le d_{i-1}[v]\)
\item \(d[v]\) can only decrease, so this stays true throughout the loop
\item \(d[v]\) is replaced by \(\min(d[v], \min_{(u,v) \in E}(x + w(u,v)))\) where
\(x \le d_{i-1}[u]\)
\item so at the end of iteration \(i\), \(d[v] \le d_{i}[v]\)
\end{itemize}

\textbf{Relaxation}: the operation \(d[v] = \min(d[v], d[u] + w(u,v))\)

\textbf{Claim 2}: \(d[v]\) can only decrease through a relaxation, and if \(\delta(s,u) \le d[u]\)
and \(\delta(s,v) \le d[v]\) before a relaxation, then \(\delta(s,v) \le d[v]\) post-relaxation

\textbf{Proof}:
\begin{itemize}
\item obvious that \(d[v]\) can only decrease through a relaxation
\item second item: \(\delta(s,v) \le \delta(s,u) + w(u,v)\) by the triangle equality, so
\(\delta(s,v) \le d[u] + w(u.v)\), but also \(\delta(s,v) \le d[v]\)
\end{itemize}

\textbf{Consequence}: if all \(d[v]\) satisfy \(\delta(s,v) \le d[v]\) and any number of relaxations
are applied, all inequalities stay true

\textbf{Claim 3}: for \(i = 0, \dots, n-1\), after iteration \(i\),
\(\delta(s,v) \le d[v] \le \delta_{i}(s,v)\) for all \(v\)
\subsection{Summary}
\label{sec:org91a2133}
\subsubsection{No Negative Cycle}
\label{sec:org5263bf0}
At the end \(d[v] = \delta(s,v)\) for all \(v\).
In particular, for any edge \((u,v)\), \(d[v] \le d[u] + w(u,v)\) by the triangle inequality
\subsubsection{Negative Cycle}
\label{sec:org9984bf4}
Let this negative cycle be \(v_{1}, \dots, v_{k}\) where \(v_{k} = v_{1}\)

\textbf{Claim}: There must be an edge \((v_{i}, v_{i+1})\) with
\(d[v_{i+1}] > d[v_{i}] + w(v_{i}, v_{i+1})\), otherwise \(d[v_{i+1}] \le d[v_{i}] + w(v_{i}, v_{i+1})\)
for all \(i\). Then, sum and derive a contradiction.

For an extra \(O(m)\), can check the presence of a negative cycle.
\section{Floyd-Warshall Algorithm}
\label{sec:orgdb7e6a4}
\textbf{No fixed source}: computes all distances \(\delta(u,v)\)

Negative weights are fine, but no negative cycle.

Simple pseudo-code, but slower than other algorithms.

Doing Bellman-Ford from all \(u\) takes \(O(mn^{2})\).
\subsection{Subproblems for DP}
\label{sec:org1ca095a}
Bellman-Ford uses paths with fixed numbers of steps.

Floyd-Warshall restricts which vertices can be used.
\subsection{Definition}
\label{sec:org60d5d27}
For \(i\) from 0 to \(n\), set \(D_{i}(v_{j}, v_{k})\) to the length of the shortest path from
\(v_{j}\) to \(v_{k}\) with all intermediate vertices in \(v_{1}, \dots, v_{i}\).

For \(i = 0\):
\begin{itemize}
\item \(D_{0}(v_{j}, v_{j}) = 0\)
\item \(D_{0}(v_{j}, v_{k}) = w(v_{j}, v_{k})\) if there is an edge \((v_{j}, v_{k})\)
\item \(D_{0}(v_{j}, v_{k}) = \infty\) otherwise
\end{itemize}

\(D_{n}(v_{j}, v_{k}) = \delta(v_{j}, v_{k})\)
\subsection{Correctness}
\label{sec:orge49d229}
\(D_{i}(v_{j}, v_{k}) = \min(D_{i-1}(v_{j}, v_{k}), D_{i-1}(v_{j}, v_{i}) + D_{i-1}(v_{i}, v_{k}))\)

\textbf{Proof}: either the shortest path does not go through \(v_{i}\) or it does (if so, only once)
\subsection{Pseudocode}
\label{sec:orgc582ee7}
\begin{verbatim}
FloydWarshall(G):
    setup D0 above
        for i in range(1, n):
            for j in range(1, n):
                for k in range(1, n):
                    Di[vj, vk] = min(D(i-1)[vj, vk], D(i-1)[vj, vi] + D(i-1)[vi, vk])
\end{verbatim}

\textbf{Runtime and memory}: \(\Theta(n^{3})\)
\end{document}
