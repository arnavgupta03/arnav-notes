% Created 2024-04-09 Tue 17:53
% Intended LaTeX compiler: pdflatex
\documentclass[11pt]{article}
\usepackage[utf8]{inputenc}
\usepackage[T1]{fontenc}
\usepackage{graphicx}
\usepackage{longtable}
\usepackage{wrapfig}
\usepackage{rotating}
\usepackage[normalem]{ulem}
\usepackage{amsmath}
\usepackage{amssymb}
\usepackage{capt-of}
\usepackage{hyperref}
\usepackage{parskip,darkmode}
\enabledarkmode
\author{Arnav Gupta}
\date{\today}
\title{Dynamic Programming}
\hypersetup{
 pdfauthor={Arnav Gupta},
 pdftitle={Dynamic Programming},
 pdfkeywords={},
 pdfsubject={},
 pdfcreator={Emacs 29.3 (Org mode 9.7)}, 
 pdflang={English}}
\begin{document}

\maketitle
\tableofcontents

\section{Dynamic Programming}
\label{sec:orgea90bc9}
\subsection{Recipe}
\label{sec:orgaaa312a}
\subsubsection{Identify the Subproblem}
\label{sec:orgce1ada6}
Typically the computation of solutions of the subproblems will make it natural to retain the
solutions in an array:
\begin{itemize}
\item must know dimensions of the array
\item specify the precise meaning of the value in any cell of the array
\item specify where the answer will be found in the array
\end{itemize}
\subsubsection{Establish DP-recurrence}
\label{sec:org3462501}
Specify how a subproblem contributes to the solution of a larger subproblem.

How the value in a cell of the array depends on the values of other cells in the array.
\subsubsection{Set Values for the Base Case}
\label{sec:org470c63e}
Define the base values for the DP-recurrence.
\subsubsection{Specify the Order of Computation}
\label{sec:org846115c}
The algorithm must state the order of computation for the cells.
\subsubsection{Recovery of the Solution}
\label{sec:org70d5907}
Keep track of the subproblems that provided the best solutions.
Use a traceback strategy to determine the full solution.
\subsection{Key Features}
\label{sec:orgbaed5d4}
\begin{itemize}
\item solve problems through recursion
\item use a small (polynomial) number of nested subproblems
\item may have to store results for all subproblems
\item can often be turned into 1+ loops
\end{itemize}
\subsection{DP vs Divide-and-Conquer}
\label{sec:org560efae}
\begin{itemize}
\item dynamic programming usually deals with all input sizes from 1 to \(n\)
\item divide-and-conquer may not solve subproblems
\item divide-and-conquer algorithms not easy to rewrite iteratively
\end{itemize}
\section{Interval Scheduling}
\label{sec:org79253aa}
\textbf{Input}: \(n\) intervals of form \(I_{i} = [s_{i}, f_{i}]\), each interval has weight \(w_{i}\)

\textbf{Output}: a choice \(T\) of intervals that do not overlap and maximizes total weight

Greedy algorithm works in the case where all weights are 1.
\subsection{Sketch of the Algorithm}
\label{sec:orgb3f9f8d}
\textbf{Basic Idea}: choose \(I_{n}\) or not, then choose the max of
\begin{itemize}
\item \(w_{n} + O(I_{m1}, \dots, I_{ms})\) if we choose \(I_{n}\) where
\(I_{m1}, \dots, I_{ms}\) are the intervals that do not overlap with \(I_{n}\)
\item \(O(I_{1}, \dots, I_{n-1})\) if we don't choose \(I_{n}\)
\end{itemize}

\textbf{Goal}:
\begin{itemize}
\item find a way to ensure that \(I_{m1}, \dots, I_{ms}\) are of the form \(I_{1}, \dots, I_{s}\) for some
\(s < n\)
\item it then suffices to optimize over all \(I_{1}, \dots, I_{j}\) where \(1 \le j \le n\)
\end{itemize}

Assume \(I_{1}, \dots, I_{n}\) sorted by increasing end time: \(f_{i} \le f_{i + 1}\).

\textbf{Claim}: for all \(j\), the set of \(I_{k} \le I_{j}\) that do not overlap \(I_{j}\) is of the form
\(I_{1}, \dots, I_{p_{j}}\) for some \(0 \le p_{j} \le j\) where \(p_{j} = 0\) if no such interval exists

The algorithm needs all \(p_{i}\), and this can be found by comparing with finish times.
\subsection{Finding pj's}
\label{sec:org4e59da7}
Let \(A\) be a permutation of \([1, ..., n]\) such that
\(s_{A[1]} \le s_{A[2]} \le \cdots \le s_{A[n]}\).
\begin{verbatim}
FindPj(A, s1, ..., sn, f1, ..., fn):
  f0 = -infinity
  i = 1
  for k in range(0, n):
      while i <= n and fk <= s[A[i]] < f[k+1]:
          pi = k
          i++
\end{verbatim}
\textbf{Runtime}: \(O(n\log(n))\) for sorting and \(O(n)\) for loops
\subsection{Main Procedure}
\label{sec:orgfd04204}
\textbf{Definition}: M[i] is the maximal weight to get with \(I_{1}, \dots, I_{i}\)

\textbf{Recurrence}: M[0] = 0 and for \(i \ge 1\), \(M[i] = \text{max}(M[i-1], M[pi] + wi)\)

\textbf{Runtime}: \(O(n\log(n))\) for sorting twice and \(O(n)\) for finding all \(M[i]\)
\section{0/1 Knapsack Problem}
\label{sec:org3c5454d}
\textbf{Input}: items from 1 to \(n\) with weights \(w_{i}\) and values \(v_{i}\), along with a capacity \(W\)

\textbf{Output}: a subset of the items \(S\) that has total weight less than \(W\) and maximizes total value

\textbf{Basic idea}: either choose item \(n\) or not, then the optimum is the max of:
\begin{itemize}
\item \(v_{n} + O[W - w_{n}, n - 1]\) if we choose \(n\)
\item \(O[W, n-1]\) if we don't choose \(n\)
\end{itemize}

\textbf{Initial conditions}: \(O[0, i] = 0\) for any \(i\) and \(O[w, 0] = 0\) for any \(w\)

\begin{verbatim}
01KnapSack(v1, ..., vn, w1, ..., wn, W):
  initialize an array O[0..n, 0..W] with O(0, j) = 0 and O(w, 0) = 0
  for i in range(1, n):
      for w in range(1, W):
          if wi > w:
              O[w, i] = O[w, i-1]
          else:
              O[w, i] = max(vi + O[w - wi, i-1], O[w, i - 1])
\end{verbatim}
\textbf{Runtime}: \(\Theta(nW)\) which is pseudo-polynomial
\subsection{Pseudo-Polynomial Algorithms}
\label{sec:org7b0f5a1}
In our word RAM model, we assume all \(v_{i}\) and \(w_{i}\) fit in a word,
so the input size is \(\Theta(n)\) words, but the runtime also depends on the values of the inputs.

01-knapsack is NP-complete.
\end{document}
