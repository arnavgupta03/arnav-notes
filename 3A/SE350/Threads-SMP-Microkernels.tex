% Created 2024-02-13 Tue 00:03
% Intended LaTeX compiler: pdflatex
\documentclass[11pt]{article}
\usepackage[utf8]{inputenc}
\usepackage[T1]{fontenc}
\usepackage{graphicx}
\usepackage{longtable}
\usepackage{wrapfig}
\usepackage{rotating}
\usepackage[normalem]{ulem}
\usepackage{amsmath}
\usepackage{amssymb}
\usepackage{capt-of}
\usepackage{hyperref}
\author{DESKTOP-H800RKQ}
\date{\today}
\title{Threads SMP Microkernels}
\hypersetup{
 pdfauthor={DESKTOP-H800RKQ},
 pdftitle={Threads SMP Microkernels},
 pdfkeywords={},
 pdfsubject={},
 pdfcreator={Emacs 29.2 (Org mode 9.7)}, 
 pdflang={English}}
\begin{document}

\maketitle
\tableofcontents

\setlength\parindent{0pt}
\section{Processes and Threads}
\label{sec:org8b6a5e0}
Main characteristics of a process are:
\begin{itemize}
\item resource ownership
\begin{itemize}
\item space to hold process info
\item control over resources
\end{itemize}
\item scheduling/execution
\begin{itemize}
\item process follows trace through program
\item has state and priority
\end{itemize}
\end{itemize}
\subsection{Multithreading}
\label{sec:org333c881}

Multithreading: ability of the OS to support multiple,
concurrent paths of execution within a single process

In a multithreaded environment, a process has:
\begin{itemize}
\item virtual address space for process image (resource allocation)
\item protected access to resources (protection)
\end{itemize}

Threads each have:
\begin{itemize}
\item state (running, ready, etc)
\item saved thread context when not running
\item execution stack
\item static storage for local variables
\item access to the memory and resources of process (shared with other threads)
\end{itemize}

When one thread alters memory, other threads see the
results if and when they access that item.
One thread's access permissions cascade to other threads
in the same process.

Key benefits of threads:
\begin{enumerate}
\item faster to create thread in a process than a new process
\item faster to terminate a thread than a process
\item less time to switch between threads within a process
than between processes
\item communication between threads more efficient
\end{enumerate}

Threads are useful for:
\begin{enumerate}
\item splitting foreground/background work
\item asynchronous processing
\item speed of execution
\item modular program structure
\end{enumerate}

Scheduling and dispatching is done on threads, not processes.
\subsection{Thread Functionality}
\label{sec:orgd21a7cf}

Key states for a thread are running, ready, and blocked (no suspend).

Four basic thread operations associated with a change
in thread state:
\begin{enumerate}
\item \textbf{spawn}: caused by process spawn or another thread,
then placed on ready queue (register context and stack space)
\item \textbf{block}: when waiting for an event
\item \textbf{unblock}: when event occurs, moved to ready queue
\item \textbf{finish}: thread completes and is deallocated
\end{enumerate}

With multiple threads, blocked threads can wait simultaneously,
for threads within the same process and in different processes.

Must synchronize the activity of various threads so that
they do not interfere with each other.
\section{Types of Threads}
\label{sec:org74348e2}
\subsection{User-Level and Kernel-Level Threads}
\label{sec:org1b5d58c}
For pure user-level:
\begin{itemize}
\item all thread management is done by the
\end{itemize}
application and the kernel is not aware of them
\begin{itemize}
\item done using threads libraries that manage
thread control and state
\end{itemize}
\end{document}
