% Created 2024-04-11 Thu 10:45
% Intended LaTeX compiler: pdflatex
\documentclass[11pt]{article}
\usepackage[utf8]{inputenc}
\usepackage[T1]{fontenc}
\usepackage{graphicx}
\usepackage{longtable}
\usepackage{wrapfig}
\usepackage{rotating}
\usepackage[normalem]{ulem}
\usepackage{amsmath}
\usepackage{amssymb}
\usepackage{capt-of}
\usepackage{hyperref}
\usepackage{parskip,darkmode}
\enabledarkmode
\author{Arnav Gupta}
\date{\today}
\title{Concurrency: Mutual Exclusion And Synchronization}
\hypersetup{
 pdfauthor={Arnav Gupta},
 pdftitle={Concurrency: Mutual Exclusion And Synchronization},
 pdfkeywords={},
 pdfsubject={},
 pdfcreator={Emacs 29.3 (Org mode 9.7)}, 
 pdflang={English}}
\begin{document}

\maketitle
\tableofcontents

\section{Concurrency Basics}
\label{sec:org3ffc219}
Arises in 3 different contexts:
\begin{enumerate}
\item \textbf{multiple applications}: multiprogramming was invented to allow
processing time to be dynamically shared among a number of
applications
\item \textbf{structured applications}: applications can be programmed as a
set of concurrent processes
\item \textbf{operating system structure}: same structuring advantages
apply to systems programs as OSs are also a set of processes
and threads
\end{enumerate}

Key terms:
\begin{itemize}
\item \textbf{atomic operation}: action that appears to be indivisible; no
intermediate state or interruption
\item \textbf{critical section}: section of code that requires access to shared
resources, cannot be executed by 2+ processes at once
\item \textbf{deadlock}: situation in which 2+ processes cannot proceed because
each is waiting for one of the others
\item \textbf{livelock}: situation in which 2+ processes keep changing state
without doing any work
\item \textbf{mutual exclusion}: requirement that when 1 process is in a critical
section, no other process can be in a critical section accessing
the same resources
\item \textbf{race condition}: situation in which multiple threads or processes
read and write a shared data item and the final result
depends on their relative timing
\item \textbf{starvation}: situation in which a runnable process is overlooked
indefinitely by the scheduler, can be chosen but never is
\end{itemize}
\section{Mutual Exclusion: Software Approaches}
\label{sec:org08952e0}
Software approaches assume mutual exclusion at memory access level:
simultaneous access is serialized.
\subsection{Dekker's Algorithm}
\label{sec:org7a214c1}
First attempt has a \texttt{turn} variable that has the pid for the
process allowed in the critical section.
\begin{itemize}
\item guarantees mutual exclusion but has two drawbacks
\begin{enumerate}
\item processes must strictly alternate, so pace is dictated by
the slower process
\item if one process fails, the other is permanently blocked
\end{enumerate}
\item done using coroutines (pass execution control back and
forth)
\end{itemize}

Second attempt has a bool vector \texttt{flag} with each index
corresponding to a pid, where \texttt{true} means the process is in
the critical section.
\begin{itemize}
\item does not guarantee mutual exclusion
\end{itemize}

Third attempt sets flag before checking if other process
in critical section.
\begin{itemize}
\item if one process fails inside the critical section,
the other process is blocked
\item if one process fails outside the critical section,
the other process is not blocked
\item can cause deadlock if both processes set flag
before checking the other's flag
\end{itemize}

Fourth attempt has each process turn off flag to enter
critical section after turning it on.
\begin{itemize}
\item can cause livelock if each process repeatedly turns
its flags on and off
\end{itemize}

The correct solution is Dekker's algorithm:
\begin{itemize}
\item uses both bool vector \texttt{flag} and variable \texttt{turn}
\item process sets flag to \texttt{true} before checking
\item while checking, if it is not a process's turn,
it defers to the other process which will flip
its flag
\end{itemize}

\begin{verbatim}
boolean flag[2];
int turn;
void P0() {
  while (true) {
    flag[0] = true;
    while (flag[1]) {
      if (turn ==  1) {
        flag[0] = false;
        while (turn == 1);
        flag[0] = true;
      }
    }
    /* critical section */
    turn = 1;
    flag[0] = false;
    /* remainder */
  }
}
void P1() {
  while (true) {
    flag[1] = true;
    while (flag[0]) {
      if (turn ==  0) {
        flag[1] = false;
        while (turn == 0);
        flag[1] = true;
      }
    }
    /* critical section */
    turn = 0;
    flag[1] = false;
    /* remainder */
  }
}
void main() {
  flag[0] = false;
  flag[1] = false;
  turn = 1;
  parbegin(P0, P1);
}
\end{verbatim}
\subsection{Peterson's Algorithm}
\label{sec:org5cdb63c}
A simpler algorithm that uses \texttt{turn} to resolve
simultaneity conflicts.

\begin{verbatim}
boolean flag[2];
int turn;
void P0() {
  while (true) {
    flag[0] = true;
    turn = 1;
    while (flag[1] && turn == 1);
    /* critical section */
    flag[0] = false;
    /* remainder */
  }
}
void P1() {
  while (true) {
    flag[1] = true;
    turn = 0;
    while (flag[0] && turn == 0);
    /* critical section */
    turn = 0;
    flag[1] = false;
    /* remainder */
  }
}
void main() {
  flag[0] = false;
  flag[1] = false;
  parbegin(P0, P1);
}
\end{verbatim}
\section{Principles of Concurrency}
\label{sec:orgf6bcdb7}
In a multiprocessor sysmte, it is possible not only
to interleave the execution of multiple processes
but also to overlap them.

Difficulties with interleaving and overlapping:
\begin{enumerate}
\item sharing global resources is risky (read/write
race conditions)
\item difficult for OS to optimally manage resource
allocation
\item difficult to locate programming errors (less
reproducable)
\end{enumerate}

Sharing main memory permits efficient and close
interaction among processes, but can lead to problems:
\begin{itemize}
\item if one process updates a global variable and is
interrupted, another process may alter the variable
before the first process can use its value
\end{itemize}

Necessary to protect shared global variables by
controlling the code that accesses the variable.
\subsection{Operating System Concerns}
\label{sec:orgbec41be}
Concerns are:
\begin{enumerate}
\item OS must be able to keep track of processes
\item OS must allocate and deallocate resources for each
active process (processor time, memory, files,
IO devices)
\item OS must protect data and physical resources of each
process against unintended interference by other
processes
\item functioning of a process and its output must be
independent of the relative speed of execution
\end{enumerate}
\subsection{Process Interaction}
\label{sec:org4f4b492}
For processes unaware of each other:
\begin{itemize}
\item OS must be concerned about competition for resources
\item results of one process independent of the action of others
\item timing of process may be affected
\item potential problems are mutual exclusion, deadlock of
a renewable resource, starvation
\end{itemize}

For processes indirectly aware of each other:
\begin{itemize}
\item exhibit cooperation by sharing
\item results of one process may depend on information obtained
from others
\item timing of process may be affected
\item potential problems are mutual exclusion, deadlock of
a renewable resource, starvation, data coherence
\end{itemize}

For processes directly aware of each other:
\begin{itemize}
\item exhibit cooperation by communication
\item results of one process may depend on information obtained
from others
\item timing of process may be affected
\item potential problems are deadlock of
a consumable resource, starvation
\end{itemize}

If competition exists:
\begin{itemize}
\item each process should leave the state of any resource that
it uses unaffected
\item since a process must wait to use a resource, it will be
slowed down or may never get access to it
\end{itemize}

Problems that must be solved for competition are:
\begin{enumerate}
\item \textbf{Mutual exclusion}: ensuring only one program using the critical
resource in the critical section at a time
\item \textbf{Deadlock}: processes exist a loop waiting on other processes
\item \textbf{Starvation}: process is indefinitely denied access to a
resource, even without deadlock
\end{enumerate}

Processes themselves must express the requirement for mutual exclusoin,
with support from the OS to follow this.

If cooperation by sharing exists:
\begin{itemize}
\item processes must cooperate to ensure shared data is properly managed
\item control mechanisms must ensure the integrity of shared data
\end{itemize}

If cooperation by communication exists:
\begin{itemize}
\item processes must synchronize various activities through messages
\item deadlock and starvation can occur, through awaiting communication
from other processes
\end{itemize}
\subsection{Requirements for Mutual Exclusion}
\label{sec:orgd1c0458}
\begin{enumerate}
\item \textbf{Enforcement}: only one process at a time allowed in the
critical section for a resource
\item \textbf{No Interference}: a process that halts in its noncritical section
must do so without interfering with other processes
\item \textbf{No Deadlock or Starvation}
\item \textbf{No Unnecessary Delay}: a process must not be delayed access to a
critical section when there is no other process using it
\item \textbf{No Assumptions}: no assumptions made about relative process
speeds or number of processors
\item \textbf{Finite Time}: a process remains inside its critical section
for a finite time only
\end{enumerate}

Main mechanisms for Mutual Exclusion:
\begin{itemize}
\item software with Dekker's algorithm (slow)
\item special hardware instructions
\item support in the OS or a programming language
\end{itemize}
\section{Mutual Exclusion: Hardware Support}
\label{sec:orgf504b74}
\subsection{Interrupt Disabling}
\label{sec:org6c76f34}
Disabling interrupts guarantees mutual exclusion for a uniprocessor
system.

Costs of interrupt disabling:
\begin{itemize}
\item efficiency is degraded since there is limited ability to
interleave processes.
\item does not work for multiprocessor architecture, since a process
on another processor could invalidate mutual exclusion
\end{itemize}
\subsection{Special Machine Instructions}
\label{sec:orgf3be3ef}
Special instructions carry out two actions atomically
during one instruction fetch cycle.

Compare\&Swap instruction (carried out atomically)
\begin{itemize}
\item \texttt{testval} is compared to memory location
\begin{itemize}
\item if equal then a swap occurs with \texttt{newval}
\end{itemize}
\item returns either the old value or whether or not the swap occurred
\end{itemize}

\begin{verbatim}
const int n = 5;
int bolt;
void P(int i) {
  while (true) {
    while (compare_and_swap(bolt, 0, 1) == 1);
    /* critical section */
    bolt = 0;
    /* remainder */
  }
}
void main() {
  bolt = 0;
  parbegin(P(1), P(2), ..., P(n));
}
\end{verbatim}

Exchange instruction (carried out atomically)
\begin{itemize}
\item exchanges contents of register with that of memory location
\end{itemize}

\begin{verbatim}
const int n = 5;
int bolt;
void P(int i) {
  while (true) {
    int keyi = 1;
    do (exchange(&keyi, &bolt))
         while (keyi != 0);
    /* critical section */
    bolt = 0;
    /* remainder */
  }
}
void main() {
  bolt = 0;
  parbegin(P(1), P(2), ..., P(n));
}
\end{verbatim}

Advantages of machine-instruction approach:
\begin{itemize}
\item applicable to any number of processes for both uniprocessor
and multiprocessor
\item simple and easy to verify
\item can be used to support multiple critical sections, each
defined by its own variable
\end{itemize}

Disadvantages of machine-instruction approach:
\begin{itemize}
\item busy waiting: process waiting for access to a critical section
consumes processor
\item starvation possible: when a process leaves a critical section
and more than one process is waiting, some process could never
be chosen
\item deadlock possible: higher priority processes could end up
waiting on lower priority processes inside critical sections
\end{itemize}
\section{Semaphores}
\label{sec:org2d17d57}
Common concurrency mechanisms:
\begin{itemize}
\item \textbf{semaphore}: integer value used for signaling, with atomic
operations of initialize, decrement, and increment where
decrement can block a process and increment can unblock a
process
\item \textbf{binary semaphore}: semaphore that takes on only 0 and 1
\item \textbf{mutex}: similar to binary semaphore but the process that
locks a mutex must unlock it
\item \textbf{condition variable}: data type used to block a process
or thread until a condition is true
\item \textbf{monitor}: programming language construct to encapsulate
variables, critical sections, and initialization code
in one data type, where it may only be accessed inside
critical sections by one process
\item \textbf{event flags}: memory word used as a synchronization
mechanism with bits representing events that must be
waited on
\item \textbf{messages}: means for two processes to exchange info,
possibly for synchronization
\item \textbf{spinlocks}: mutual exclusion mechanism in which a
process
\end{itemize}

Two or more processes can cooperate by means of simple signals.

For a semaphore \texttt{s}:
\begin{itemize}
\item to transmit a signal, a process executes \texttt{semSignal(s)}
\item to receive a signal, a process executes \texttt{semWait(s)},
which suspends the process until a signal is transmitted
\end{itemize}

Semaphore is an integer variable with only 3 operations defined:
\begin{enumerate}
\item initialized to nonnegative integer value
\item \texttt{semWait} decrements the semaphore value, if it becomes negative
the process executing \texttt{semWait} is blocked, otherwise it
continues execution
\item \texttt{semSignal} increments the semaphore value, if it is less than
or equal to zero, a process blocked by \texttt{semWait} is
unblocked
\end{enumerate}

Consequences of semaphore definition:
\begin{enumerate}
\item no way to know before \texttt{semWait} whether it will block
\item after a process increments a semaphore and another
process wakes up, both processes run concurrently
\item when calling \texttt{semSignal}, the number of unblocked
processes may be 0 or 1
\end{enumerate}

Operations on binary semaphore are:
\begin{enumerate}
\item initialized to 0 or 1
\item \texttt{semWaitB} checks the semaphore value, if 0 then the
process executing \texttt{semWaitB} is blocked, if 1, then
the value is changed to 0 and the process continues
\item \texttt{semSignalB} checks if value is 0 (processes blocked),
if so then one blocked process is unblocked, if no
processes blocked the value is set to 1
\end{enumerate}

Mutex is a programming flag to grab and release an object:
\begin{itemize}
\item mutex locks (set to 0) when non-shareable data acquired or
processing started that cannot be performed elsewhere
\item mutex unlocks when the data is no longer needed or the
routine is finished
\end{itemize}

Semaphores use a queue to hold processes waiting on the
semaphore:
\begin{itemize}
\item strong semaphores use FIFO for release, guarantee
freedom from starvation
\item weak semaphores do not specify the order of release,
no guarantee of freedom from starvation
\end{itemize}
\subsection{Producer/Consumer Problem}
\label{sec:org394ee82}
General statement of problem:
\begin{itemize}
\item one or more producers generating some type of data
and placing into buffer
\item single consumer takes items out of buffer one at
a time
\item system is to be constrained to prevent the overlap
of buffer operations (only one agent accessing the
buffer at a time)
\end{itemize}

Order of semaphore operations can sometimes matter,
especially for \texttt{semWait} where there is potential
for deadlock.

Infinite-buffer producer/consumer problem with binary
semaphores

\begin{verbatim}
int n;
binary_semaphore s = 1, delay = 0;
void producer() {
  while (true) {
    produce();
    semWaitB(s);
    append();
    n++;
    if (n == 1) semSignalB(delay);
    semSignalB(s);
  }
}
void consumer() {
  int m;
  semWaitB(delay);
  while (true) {
    semWaitB(s);
    take();
    n--;
    m = n;
    semSignalB(s);
    consume();
    if (m == 0) semWaitB(delay);
  }
}
void main() {
  n = 0;
  parbegin(producer, consumer);
}
\end{verbatim}

Infinite-buffer producer/consumer problem with semaphores
\begin{verbatim}
semaphore n = 0, s = 1;
void producer() {
  while (true) {
    produce();
    semWait(s);
    append();
    semSignal(s);
    semSignal(n);
  }
}
void consumer() {
  while (true) {
    semWait(n);
    semWait(s);
    take();
    semSignal(s);
    consume();
  }
}
void main() {
  parbegin(producer, consumer);
}
\end{verbatim}

For a finite buffer, it can be treated as circular
with pointer values expressed module the buffer size.
Block occurs on:
\begin{itemize}
\item insert in full buffer for producer
\item remove from empty buffer for consumer
\end{itemize}
Unblock occurs on:
\begin{itemize}
\item item inserted for consumer
\item item removed for producer
\end{itemize}

Finite-buffer producer/consumer problem with semaphores
\begin{verbatim}
const int sizeofbuffer = /* buffer size */;
semaphore s = 1, n = 0, e = sizeofbuffer;
void producer() {
  while (true) {
    produce();
    semWait(e);
    semWait(s);
    append();
    semSignal(s);
    semSignal(n);
  }
}
void consumer() {
  while (true) {
    semWait(n);
    semWait(s);
    take();
    semSignal(s);
    semSignal(e);
    consume();
  }
}
void main() {
  parbegin(producer, consumer);
}
\end{verbatim}
\subsection{Implementation of Semaphores}
\label{sec:org9ef2086}
Semaphores must be implemented as atomic primitives;
only one process at a time may manipulate a
semaphore with either \texttt{semWait} or \texttt{semSignal}.

Can be done with:
\begin{itemize}
\item hardware or firmware
\item software schemes (Dekker's, Peterson's)
\item hardware-supported schemes for mutual exclusion
\begin{itemize}
\item can be done with \texttt{compare\_and\_swap} with a flag or by
inhibiting interrupts
\end{itemize}
\end{itemize}
\section{Monitors}
\label{sec:org38c8770}
May be difficult to produce a correct program using semaphores.
\subsection{Monitor with Signal}
\label{sec:org00c64c3}
Monitor: a software module consisting of one or more procedures,
an initialization sequence, and local data

Chief chracteristics are:
\begin{enumerate}
\item local data variables are only accessible only by the monitor's
procedures and not by any external procedure
\item a process enters the monitor by invoking one of its procedures
\item only one process may be executing in the monitor at a time,
any other processes that have invoked the monitor are blocked,
waiting for the monitor to become available
\end{enumerate}

Monitors support synchronization using conditional variables
contained within the monitor and accessible only within the monitor.

Condvars operated on by two functions:
\begin{itemize}
\item \texttt{cwait(c)}: suspend execution of the calling process on
condition \texttt{c}, monitor is now available for use by another process
\item \texttt{csignal(c)}: resume execution of some process blocked after a \texttt{cwait}
on the same condition:
\begin{itemize}
\item if there are multiple such processes, choose one
\item if none, do nothing
\end{itemize}
\end{itemize}
Different from semaphores as any signals that tasks aren't waiting on
are lost.

\begin{verbatim}
monitor boundedbuffer;
char buffer[N];
int nextin, nextout;
int count;
cond notfull, notempty;

void append(char x) {
  if (count == N) cwait(notfull);
  buffer[nextin] = x;
  nextin = (nextin + 1) % N;
  count++;
  csignal(notempty);
}

void take(char x) {
  if (count == 0) cwait(notempty);
  x = buffer[nextout];
  nextout = (nextout + 1) % N;
  count--;
  csignal(notfull);
}

{
  nextin = 0; nextout = 0; count = 0;
}
void producer() {
  char x;
  while (true) {
    produce();
    append();
  }
}
void consumer() {
  char x;
  while (true) {
    take();
    consume();
  }
}
void main() {
  parbegin(producer, consumer);
}
\end{verbatim}
\subsection{Alternate Models of Monitors with Notify and Broadcast}
\label{sec:org1bd068b}
Currently, a process issuing the \texttt{csignal} must either immediately
exit the monitor or be blocked on the monitor, which has 2
drawbacks:
\begin{enumerate}
\item if the process issuing the \texttt{csignal} has not finished with the
monitor, two additional process switches are required
\item process scheduling associated with a signal must be reliable,
no other process can interrupt this switching
\end{enumerate}

Bounded-buffer Monitor Code for Mesa Monitor
\begin{verbatim}
void append(char x) {
  while (count == N) cwait(notfull);
  buffer[nextin] = x;
  nextin = (nextin + 1) % N;
  count++;
  cnotify(notempty);
}

void take(char x) {
  while (count == 0) cwait(notempty);
  x = buffer[nextout];
  nextout = (nextout + 1) % N;
  count--;
  cnotify(notfull);
}
\end{verbatim}
\texttt{cnotify} differs in that the condition queue is notified, but
the signalling process continues to execute.
\texttt{cbroadcast} can also cause all processes waiting on a condition
to be placed into a ready state.

Mesa monitor less error prone and more modular than usual
Hoare monitor.
\section{Message Passing}
\label{sec:orge14721e}
When processes interact, must ensure synchronization and communication.
Can be done with message passing, which has 2 primitives:
\begin{itemize}
\item \texttt{send(destination, message);}
\item \texttt{receive(source, message);}
\end{itemize}

Sender and receiver can be blocking or nonblocking, of which there are
3 common combinations
\begin{enumerate}
\item \textbf{blocking send, blocking receive}: both sender and receiver blocked
until message is delivered, allows for tight synchronization
\item \textbf{nonblocking send, blocking receive}: sender may continue but
receiver is blocked until message arrives
\item \textbf{nonblocking send, nonblocking receive}: neither party must wait
\end{enumerate}

Nonblocking send is most natural since a sender can then send
multiple different messages.
Blocking receive is most natural since receivers generally expect
information before proceeding.
\subsection{Addressing}
\label{sec:orgcd9156f}
With direct addressing
\begin{itemize}
\item the \texttt{send} primitive includes a specific identifier
\end{itemize}
of the destination process.
\begin{itemize}
\item the \texttt{receive} primitive either:
\begin{itemize}
\item requires that the process explicitly designate a sending process
\item uses implicit addressing: the source possesses a value returned
when the receive has bene performed
\end{itemize}
\end{itemize}

With indirect addressing
\begin{itemize}
\item messages are sent to queue that holds messages, called mailbox
\item decoupling sender and receiver allows for more flexibility in
use of messages
\item relationship between senders and receivers can be:
\begin{itemize}
\item one-to-one: private communication between two processes
\item many-to-one: client-server interaction, creates port
\item one-to-many: broadcasting to set of processes
\item many-to-many: multiple server processes to provide
concurrent service to multiple clients
\end{itemize}
\item mailboxes owned by creating process (ex. port)
\end{itemize}
\subsection{Message Format}
\label{sec:org6e1a791}
Message typically divided into header and body.
Header contains:
\begin{itemize}
\item message type
\item destination ID
\item source ID
\item message length
\item control information
\end{itemize}

Body contains message contents.

Messages can be queued based on priority.
\subsection{Mutual Exclusion}
\label{sec:org199a74b}
Mutual Exclusion using Messages
\begin{verbatim}
const int n = /* number of processes */;
void P(int i) {
  message msg;
  while (true) {
    receive(box, msg);
    /* critical section */;
    send(box, msg);
    /* remainder */;
  }
}
void main() {
  create_mailbox(box);
  send(box, null);
  parbegin(P(1), P(2), ..., P(n));
}
\end{verbatim}

Bounded-buffer Producer/Consumer using Messages
\begin{verbatim}
const int capacity = /* number of processes */;
int i;
void producer() {
  message pmsg;
  while (true) {
    receive(mayproduce, pmsg);
    pmsg = produce();
    send(mayconsume, pmsg);
  }
}
void consumer() {
  message cmsg;
  while (true) {
    receive(mayconsume, cmsg);
    consume();
    send(mayproduce, null);
  }
}
void main() {
  create_mailbox(mayproduce);
  create_mailbox(mayconsume);
  for (int i = 1; i <= capacity; i++) send(mayproduce, null);
  parbegin(producer, consumer);
}
\end{verbatim}
\section{Readers/Writers Problem}
\label{sec:org4e4e189}
Problem defined as follows: there is a data area shared among processes,
some of which only read and some of which only write.
The conditions on these are:
\begin{enumerate}
\item any number of readers may simultaneously read the file
\item only one writer at a time may write to the file
\item if a writer is writing to the file, no reader may read it
\end{enumerate}
\subsection{Readers Have Priority}
\label{sec:org2ac9899}
\begin{verbatim}
int readcount;
semaphore x = 1, wsem = 1;

void reader() {
  while (true) {
    semWait(x);
    readcount++;
    if (readcount == 1) {
      semWait(wsem);
    }
    semSignal(x);
    READUNIT();
    semWait(x);
    readcount--;
    if (readcount == 0) {
      semSignal(wsem);
    }
    semSignal(x);
  }
}

void writer() {
  while (true) {
    semWait(wsem);
    WRITEUNIT();
    semSignal(wsem);
  }
}

void main() {
  readcount = 0;
  parbegin(reader, writer);
}
\end{verbatim}

When one writer is writing, no other writer or reader can access
the shared data.

The first reader waits on \texttt{wsem} (in case of writing), subsequent
readers need not wait since first one guarantees no writing.

Last reader finishing up signals \texttt{wsem} that writing can occur as
no other readers accessing data.
\subsection{Writers Have Priority}
\label{sec:org9501dbc}
Prevents writer starvation by guaranteeing no new readers once
at least one writer desires to write.

\begin{verbatim}
int readcount, writecount;
semaphore x = 1, y = 1, z = 1, wsem = 1, rsem = 1;

void reader() {
  while (true) {
    semWait(z);
    semWait(rsem);
    semWait(x);
    readcount++;
    if (readcount == 1) {
      semWait(wsem);
    }
    semSignal(x);
    semSignal(rsem);
    semSignal(z);
    READUNIT();
    semWait(x);
    readcount--;
    if (readcount == 0) {
      semSignal(wsem);
    }
    semSignal(x);
  }
}

void writer() {
  while (true) {
    semWait(y);
    writecount++;
    if (writecount == 1) {
        semWait(rsem);
    }
    semSignal(y);
    semWait(wsem);
    WRITEUNIT();
    semSignal(wsem);
    semWait(y);
    writecount--;
    if (writecount == 0) {
        semSignal(rsem);
    }
    semSignal(y);
  }
}

void main() {
  readcount = 0;
  parbegin(reader, writer);
}
\end{verbatim}
\end{document}
