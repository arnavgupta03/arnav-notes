% Created 2024-03-26 Tue 20:34
% Intended LaTeX compiler: pdflatex
\documentclass[11pt]{article}
\usepackage[utf8]{inputenc}
\usepackage[T1]{fontenc}
\usepackage{graphicx}
\usepackage{longtable}
\usepackage{wrapfig}
\usepackage{rotating}
\usepackage[normalem]{ulem}
\usepackage{amsmath}
\usepackage{amssymb}
\usepackage{capt-of}
\usepackage{hyperref}
\usepackage{parskip,darkmode}
\enabledarkmode
\author{Arnav Gupta}
\date{\today}
\title{Concurrency: Deadlock and Starvation}
\hypersetup{
 pdfauthor={Arnav Gupta},
 pdftitle={Concurrency: Deadlock and Starvation},
 pdfkeywords={},
 pdfsubject={},
 pdfcreator={Emacs 29.3 (Org mode 9.7)}, 
 pdflang={English}}
\begin{document}

\maketitle
\tableofcontents

\section{Principles of Deadlock}
\label{sec:orgee056ca}
Deadlock: permanent blocking of a set of processes that either compete
for system resources or communicate with each other.

No efficient solution exists in the general case.

Deadlock can be illustrated with a joint process diagram, which
illustrates the progress of two processes competing for two resources.
\begin{itemize}
\item if an execution path on a joint process diagram enters the fatal region,
deadlock is inevitable
\end{itemize}
\subsection{Consumable Resources}
\label{sec:org8b92b24}
Consumable resource: one that can be created and destroyed (ex. interrupts,
signals, messages, IO buffer info)

When such a resource is acquired by a consuming process, the resource
ceases to exist.

Can lead to deadlock, such as conflicting order of messages.

Common approaches to deal with deadlocks:
\begin{itemize}
\item \textbf{deadlock prevention}: disallow one of the three necessary conditions
for deadlock occurrence (or prevent circular wait condition from
happening)
\item \textbf{deadlock avoidance}: do not grant a resource request if the allocation might lead to deadlock
\item \textbf{deadlock detection}: grant resource requests when possible, but check
for presence of deadlock and take action to recover
\end{itemize}
\subsection{Resource Allocation Grpahs}
\label{sec:org0ba15c3}
Resource allocation graph: directed graph that depicts a state of the system
of resources and processes, with each process and each resource represented
by a node, edges indicating requests for resources not yet granted
\begin{itemize}
\item each dot indicates one instance
\item graph edge directed from a consumable resource node dot to a process
indicates the process is the producer of the resource
\end{itemize}
\subsection{Conditions for Deadlock}
\label{sec:orgf9d7cf9}
Conditions of policy (1-3) must be present for a deadlock to be possible:
\begin{enumerate}
\item \textbf{mutual exclusion}: only one process may use a resource at a time, no process
may access a resource unit that has been allocated by another process
\item \textbf{hold and wait}: process may hold allocated resources while awaiting assignment
of other resources
\item \textbf{no preemption}: no resource can be forcibly removed from a process holding it
\item \textbf{circular wait}: closed chain of processes exists, such that each process
holds at least one resource needed by the next process in the chain
\end{enumerate}

Fatal region in joint progress diagram only exists if all of the first 3 conditions
above are met, otherwise no fatal region or deadlock occur.

Conditions 1-3 represent possibility of deadlock,
conditions 1-4 represent existence of deadlock.
\section{Deadlock Prevention}
\label{sec:org50665ff}
Strategy is to design a system such that the possibility of deadlock is excluded.

Two classes of deadlock prevention methods:
\begin{itemize}
\item \textbf{indirect method}: prevent the occurrence of 1 of the 3 necessary conditions
\item \textbf{direct method}: prevent the occurence of a circular wait
\end{itemize}
\subsection{Mutual Exclusion}
\label{sec:org46cc635}
Mutual exclusion cannot be disallowed, since some resources require it.
\subsection{Hold and Wait}
\label{sec:orgd9f0300}
Can be prevented by requiring a process to request all required resources at once,
which is quite inefficient:
\begin{itemize}
\item a process may be held up with these resource requests even though it does not
need them all
\item allocating resources to this process denies them to other processes
\item a process may not know in advance all resources it requires
\end{itemize}
\subsection{No Preemption}
\label{sec:org3871cc9}
Can be prevented by:
\begin{itemize}
\item if a process holding certain resources is denied more, that process must release
its original resources and request all necessary resources (including those it was
denied) again
\item if a process requests a resource held by another process, the OS may preempt the
other process and require it to release its resources
\end{itemize}

Only practical to prevent when applied to resources whose state can be easily
saved and restored (like the processor).
\subsection{Circular Wait}
\label{sec:orgd964a3d}
Can be prevented by defining a linear ordering of resource types, where resources
can only be requested in the order specified.

Preventing this may be inefficient, slow processes and deny resource access.
\section{Deadlock Avoidance}
\label{sec:orgdd7012a}
Avoidance allows the 3 necessary conditions for deadlock, but makes choices
to assure that deadlock is never reached.
\begin{itemize}
\item allows more concurrency than prevention
\item requires knowledge of future process resource requests
\end{itemize}
\subsection{Process Initiation Denial}
\label{sec:org1ff6b29}
Consider a system of \(n\) processes and \(m\) types of resources.

Let \(R = (R_{1}, R_{2}, \dots, R_{m})\) be the total amount of
each resource in the system.
Let \(V = (V_{1}, V_{2}, \dots, V_{m})\) be the total amount of
each resource not allocated to any process.

Let the claim \(C\) be
$$C = \begin{pmatrix}
       C_{11} & C_{12} & \cdots & C_{1m} \\
       C_{21} & C_{22} & \cdots & C_{2m} \\
       \vdots & \vdots & \vdots & \vdots \\
       C_{n1} & C_{n2} & \cdots & C_{nm}
\end{pmatrix}$$
where \(C_{ij}\) is the requirement of process \(i\) for resource \(j\).

Let the allocation \(A\) be
$$A = \begin{pmatrix}
        A_{11} & A_{12} & \cdots & A_{1m} \\
        A_{21} & A_{22} & \cdots & A_{2m} \\
        \vdots & \vdots & \vdots & \vdots \\
        A_{n1} & A_{n2} & \cdots & A_{nm}
\end{pmatrix}$$
where \(A_{ij}\) is the current allocation to process \(i\) of resource \(j\).

The following relationships hold:
\begin{enumerate}
\item All resources are either available or allocated.
$$R_{j} = V_{j} + \sum_{i = 1}^{n} A_{ij}, \forall j$$
\item No process can claim more than the total amount of resources
in the system.
$$C_{ij} \le R_{j}, \forall i, j$$
\item No process is allocated more resources of any type than the process
originally claimed to need.
$$A_{ij} \le C_{j}, \forall i, j$$
\end{enumerate}

Using these, a deadlock avoidance policy can be defined that refuses to start
a new process if resource requirements might lead to deadlock.
The policy is to only start a new process \(P_{n+1}\) if
$$R_{j} \ge C_{(n+1)j} + \sum_{i = 1}^{n} C_{ij}, \forall j$$
which means to only start a process if the maximum claim of all
current processes plus the new process can be met.
\subsection{Resource Allocation Denial}
\label{sec:org2f8af32}
Strategy of resource allocation denial is called banker's algorithm.

Consider a system with a fixed number of processes and a fixed number of resources.
The state of the system reflects the current allocation of resources to processes,
consisting of vectors \(R\) and \(V\) and matrices \(C\) and \(A\).
Safe state is when there is some sequence of resource allocations to processes
that does not result in deadlock.

To find if a state is safe, check if any process can be run to completion with
the resources available. For some process \(i\), this requires
$$C_{ij} - A_{ij} \le V_{j}, \forall j$$
Once such a process is found, assume it runs to completion and make its resources
available. Begin again.

The deadlock avoidance strategy is to ensure that the system is always in a
safe state. If some resource request will put the system in an unsafe state,
block the process until it is safe to grant the request, otherwise grant it.

Deadlock avoidance strategy does not predict deadlock with certainty,
it just anticipates the possibility and assures this possibility never occurs.

Deadlock avoidance has the advantage that it is not necessary to preempt and
rollback processes and is less restrictive than deadlock prevention.
Restrictions on its use are:
\begin{itemize}
\item maximum resource requirement for each process must be stated in advance
\item processes under consideration must be independent (order of execution
must be unconstrained by synchronization requirements)
\item must be a fixed number of resources to allocate
\item no process may exit while holding resources
\end{itemize}
\section{Deadlock Detection}
\label{sec:org15b8594}
Prevention is conservative, detection finds the circular wait condition.
\subsection{Deadlock Detection Algorithm}
\label{sec:org376429b}
Check for deadlocks can be done on each resource request or less frequently.
Checking at each resource request leads to early detection and simple
algorithm, but uses more processor time.

The Allocation matrix and Available vector are used along with a request
matrix \(Q\) where \(Q_{ij}\) represents the amount of resources of type \(j\)
requested by process \(i\).

Initially all processes are unmarked. The following steps are performed to
mark processes that are not part of a deadlocked set:
\begin{enumerate}
\item Mark each process that has a row in the Allocation matrix of all zeros,
as a process that has no allocated resources cannot participate in a
deadlock.
\item Initialize a temporary vector \(W\) to equal the Available vector.
\item Find an index \(i\) such that process \(i\) is currently unmarked and the
\$i\$th row of \(Q\) is at most \(W\).
That is \(Q_{ik} \le W_{k}\) for \(1 \le k \le m\).
If no such row is found, terminate the algorithm.
\item If such a row is found, mark process \(i\) and add the corresponding row
of the allocation matrix to \(W\).
That is, set \(W_{k} = W_{k} + A_{ik}\) for \(1 \le k \le m\).
Return to step 3.
\end{enumerate}

Deadlock exists if and only if there are unmarked processes at the end
of the algorithm, as these are all deadlocked processes.
\subsection{Recovery}
\label{sec:org17f976a}
Once deadlock is detected, must recover and the following are possible
approaches, in increasing sophistication:
\begin{enumerate}
\item Abort all deadlocked processes.
\item Back up each deadlocked process to some previously defined checkpoint
and restart all processes.
\begin{enumerate}
\item requires rollback and restart mechanisms
\item original deadlock could occur again
\end{enumerate}
\item Successively abort deadlocked process until deadlock no longer exists.
\begin{enumerate}
\item order is on the basis of some criterion of minimum cost
\item detection must be run after each abortion
\end{enumerate}
\item Successively preempt resources until deadlock no longer exists.
\begin{enumerate}
\item order is on the basis of some criterion of minimum cost
\item detection must be run after each abortion
\item process that has resource preempted must be rolled back to a point
prior to resource acquisition
\end{enumerate}
\end{enumerate}

Minimum cost could be:
\begin{itemize}
\item least amount of processor time consumed
\item least amount of output produced
\item most estimated time remaining
\item least total resources allocated so far
\item lowest priority
\end{itemize}
\section{Integrated Deadlock Strategy}
\label{sec:org52606c9}
More efficient to use different deadlock strategies in different situations.
One such approach:
\begin{itemize}
\item group resources into a number of different resource classes
\item use the linear ordering strategy defined previously for the prevention of
circular wait to prevent deadlocks between resource classes
\item within a resource class, use the algorithm that is most appropriate for
that class
\end{itemize}

Some classes could include swappable space (secondary memory), process
resources (devices, tapes, files, etc), main memory, and internal
resources (IO channels).
This order is reasonable due to the sequence of steps that a process may
follow in its lifetime.

Within each class, the following strategies could be used:
\begin{itemize}
\item \textbf{swappable space}: prevention of deadlocks by requiring that all required
resources that may be used be allocated at one time (like hold-and-wait),
which is reasonable if max storage requirements are known; deadlock
avoidance also possible
\item \textbf{process resources}: avoidance often effective because can expect processes
to declare ahead of time the resources they require in this class; prevention
also possible
\item \textbf{main memory}: prevention by preemption most appropriate where preemption
means a process is swapped
\item \textbf{internal resources}: prevention by resource ordering used
\end{itemize}
\section{Dining Philosophers Problem}
\label{sec:orgb84df01}
Round table with spaghetti, 5 plates, 5 forks, and 5 philosophers that each
require 2 forks to eat.

Problem illustrates basic problems in deadlock and starvation, difficulty in
concurrent programming, and coordination of shared resources.
\subsection{Solution Using Semaphores}
\label{sec:org2238ff5}
Each philosopher first picks up the fork on the left, then the fork on the
right. After eating, the forks are replaced on the table. This leads to
deadlock.

Instead have an attendant that only allows 4 philosophers into the room,
then at least 1 philosopher has 2 forks.

\begin{verbatim}
semaphore fork[5] = {1};
semaphore room = {4};
int i;
void philosopher (int i)
{
    while (true) {
        think();
        wait(room);
        wait(fork[i]);
        wait(fork[(i+1) % 5]);
        eat();
        signal(fork[(i+1) % 5]);
        signal(fork[i]);
        signal(room);
    }
}
void main()
{
    parbegin (
       philosopher(0),
       philosopher(1),
       philosopher(2),
       philosopher(3),
       philosopher(4),
        );
}
\end{verbatim}
\subsection{Solution Using a Monitor}
\label{sec:orga71fd57}
Vector of 5 condition variables defined, 1 per fork. Used to enable a philosopher
to wait for the availability of a fork.
Further, there is a Boolean vector that records availability status of each fork.

Monitor consists of 2 procedures.

\texttt{get\_forks} used to seize left and right forks, if either unavailable, philosopher
process queued on the appropriate condition variable so more philosophers
can enter the monitor.

\texttt{release\_forks} used to make 2 forks available.

No deadlocks because only one process at a time may be in the monitor.

\begin{verbatim}
monitor dining_controller;
cond ForkReady[5];
boolean fork[5] = {true};

void get_forks(int pid)
{
    int left = pid;
    int right = (++pid) % 5;

    if (!fork[left])
        cwait(ForkReady[left]);
    fork[left] = false;

    if (!fork[right])
        cwait(ForkReady[right]);
    fork[right] = false;
}
void release_forks(int pid)
{
    int left = pid;
    int right = (++pid) % 5;

    if (empty(ForkReady[left])) // no one waiting for this fork
        fork[left] = true;
    else // awaken a process waiting on this fork
        csignal(ForkReady[left]);

    if (empty(ForkReady[right])) // no one waiting for this fork
        fork[right] = true;
    else // awaken a process waiting on this fork
        csignal(ForkReady[right]);
}

void philosopher(int k)
{
    while (true)
    {
        <think>;
        get_forks(k);
        <eat spaghetti>;
        release_forks(k);
    }
}
\end{verbatim}
\end{document}
