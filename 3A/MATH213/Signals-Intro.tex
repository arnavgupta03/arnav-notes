% Created 2024-03-30 Sat 02:54
% Intended LaTeX compiler: pdflatex
\documentclass[11pt]{article}
\usepackage[utf8]{inputenc}
\usepackage[T1]{fontenc}
\usepackage{graphicx}
\usepackage{longtable}
\usepackage{wrapfig}
\usepackage{rotating}
\usepackage[normalem]{ulem}
\usepackage{amsmath}
\usepackage{amssymb}
\usepackage{capt-of}
\usepackage{hyperref}
\usepackage{parskip,darkmode}
\enabledarkmode
\author{Arnav Gupta}
\date{\today}
\title{Signals Intro}
\hypersetup{
 pdfauthor={Arnav Gupta},
 pdftitle={Signals Intro},
 pdfkeywords={},
 pdfsubject={},
 pdfcreator={Emacs 29.3 (Org mode 9.7)}, 
 pdflang={English}}
\begin{document}

\maketitle
\tableofcontents

\section{Bode Plots}
\label{sec:orge6d95e7}
Bode plots show how a system responds to input signals of the form \(\sin(\omega t)\).
Can be used to quickly tell what type of filtering/amplifying the LTI does to the sine
wave.

A low pass filter is an LTI that removes the high frequency waves, by reducing the
amplitude of all waves with a frequency larger than some cutoff \(\omega\).
All stable systems with a single pole and no zeroes are low pass filters.

For a high pass filter, the amplitude curve must go to 0dB at some cutoff point.
The standard high pass filter transfer function is
$$
T(s) = \frac{as}{s + \omega_{0}}
$$
for \(a, \omega_{0} > 0\).

For a medium band filter, add another pole later
$$
T(s) = \frac{as}{(s+\omega_{0})(s+\omega_{1})}
$$
for \(a, \omega_{0}, \omega_{1} > 0\).
\section{Fourier Series}
\label{sec:org7454600}
Fourier series are useful for working with systems and signals.

\textbf{Taylor's Theorem}: let \(k \ge 1\) be an integer and let \(f\) be a real valued function
that is differentiable at least \(k\) times at some point \(a \in \mathbb{R}\), there then
exists a real valued function \(h_{k}\) such that
$$
f(x) = \left( \sum_{i=0}^{k} \frac{f^{(i)}(a)}{i!} (x-a)^{i} \right) + h_{k}(x) (x-a)^{k+1}
$$
and \(\lim_{x \to a} h_{k}(x) = 0\).

For infinitely differential functions \(f\), this becomes
$$
f(x) = \sum_{i=0}^{\infty} \frac{f^{(i)}(a)}{i!} (x-a)^{i}
$$

Similar to how the polynomials are used here, sinusoidals can be used as a basis instead.
\end{document}
