% Created 2024-04-05 Fri 12:06
% Intended LaTeX compiler: pdflatex
\documentclass[11pt]{article}
\usepackage[utf8]{inputenc}
\usepackage[T1]{fontenc}
\usepackage{graphicx}
\usepackage{longtable}
\usepackage{wrapfig}
\usepackage{rotating}
\usepackage[normalem]{ulem}
\usepackage{amsmath}
\usepackage{amssymb}
\usepackage{capt-of}
\usepackage{hyperref}
\usepackage{parskip,darkmode}
\enabledarkmode
\author{Arnav Gupta}
\date{\today}
\title{Real Fourier Series Convergence}
\hypersetup{
 pdfauthor={Arnav Gupta},
 pdftitle={Real Fourier Series Convergence},
 pdfkeywords={},
 pdfsubject={},
 pdfcreator={Emacs 29.3 (Org mode 9.7)}, 
 pdflang={English}}
\begin{document}

\maketitle
\tableofcontents

\section{Real Fourier Series}
\label{sec:org594112e}
\subsection{Periodic Functions}
\label{sec:org95f9207}
A function \(f\) defined on \(\mathbb{R}\) is \(\tau\) \textbf{periodic} if for all \(t \in \mathbb{R}\)
$$
f(t) = f(t + \tau)
$$
Generally pick the smallest value of \(\tau\) such that the above holds.

The theorem for Fourier Coefficients for Series in Complex Form holds for \(\tau\) periodic
functions, and can integrate over 1 period.

When finding the Fourier series without knowing the domain for \(f\) and knowing that \(f\) is \(\tau\)
periodic, first find the \(\tau\) period of \(f\) and do the computation over a period of \(f\).
\subsection{Fourier Sinusoidal Series}
\label{sec:org272d119}
A function \(f\) is \textbf{even} if \(f(-t) = f(t)\).
A function \(f\) is \textbf{odd} if \(f(-t) = -f(t)\).

For a real valued \(\tau\) periodic function \(f \in L^{2}([-\tau/2,\tau/2])\):
\begin{itemize}
\item if \(f\) is even, then the Fourier series can be simplified to a sum of cosine waves: Fourier
cosine series
\item if \(f\) is odd, then the Fourier series can be simplified to a sum of sine waves: Fourier
sine series
\end{itemize}

If \(f\) is a real valued function that is in \(L^{2}([-\tau/2,\tau/2])\), then
\begin{itemize}
\item if \(f\) is even, then the Fourier cosine series for \(f\) is
$$
  \sum_{n=0}^{\infty}c_{n} \cos\left( \frac{2\pi n}{\tau} t \right)
  $$
where
$$
  c_{n} = \begin{cases}
        \left< f(t), 1 \right> & n = 0 \\
        2\left< f(t), \cos \left( \frac{2\pi n}{\tau} t \right) \right> & n > 0
  \end{cases}
  $$
\item if \(f\) is odd, then the Fourier sine series for \(f\) is
$$
  \sum_{n=1}^{\infty}s_{n} \sin\left( \frac{2\pi n}{\tau} t \right)
  $$
where
$$
  s_{n} = 2\left< f(t), \sin \left( \frac{2\pi n}{\tau} t \right) \right>
  $$
\end{itemize}

If \(f\) is real, then it can be decomposed into even and odd functions as follows:
$$
f_{even}(t) = \frac{f(t) + f(-t)}{2} \quad \text{and} \quad
f_{odd}(t) = \frac{f(t) - f(-t)}{2}
$$
with \(f(t) = f_{even}(t) + f_{odd}(t)\).

Every real valued function in \(L^{2}([-\tau/2,\tau/2])\) admits a real valued Fourier series
with some sin and/or cos terms.
\section{Convergence of Fourier Series}
\label{sec:orgad85872}
\subsection{Types of Convergence}
\label{sec:org93958f7}
If \(f_{1}, f_{2}, \dots, f_{n}, \dots\) is a sequence of \(L^{2}\) functions defined on \([a,b]\),
then:
\begin{itemize}
\item the sequence \textbf{converges in the \(L^{2}([a,b])\) norm}, or \textbf{converges in the mean}, or
\textbf{converges almost everywhere}, to \(f\) if
$$
  \lim_{n \to \infty} \sqrt{\int_{a}^{b} |f_{n}(x) - f(x)|^{2} \, dx} = 0
  $$
which is when the average error goes to 0
\item the sequence \textbf{pointwise converges} to \(f\) if for any \(x \in [a,b]\)
$$
  \lim_{n \to \infty} (f_{n}(x) - f(x)) = 0
  $$
which is when the error at each point goes to 0
\item the sequence \textbf{uniformly converges} to \(f\) if
$$
  \lim_{n \to \infty} \max_{[a,b]} |f_{n}(x) - f(x)| = 0
  $$
which is when the maximum error converges to 0
\begin{itemize}
\item if the maximum does not exist, then replace it with the smallest upper bound (called
the sup)
\end{itemize}
\end{itemize}
\subsection{Fourier Series Convergence}
\label{sec:org038cb5d}
A function \(f\) is \textbf{Piecewise \(C^{1}\)} (PWC1) on the interval \([a,b]\) if there is a finite
partition \(a = t_{0} < t_{1} < \cdots < t_{k} = b\) such that:
\begin{itemize}
\item \(f'\) exists on each interval \((t_{i}, t_{i+1})\)
\item \(f'\) is continuous on each interval \((t_{i}, t_{i+1})\)
\item \(f\) and \(f'\) are bounded on each interval \((t_{i}, t_{i+1})\)
\end{itemize}

The \textbf{periodic extension} of a function \(f\) defined on \([a,b]\) is the \(b-a\) periodic
function \(f_{p}\) such that
\begin{itemize}
\item \(f_{p}(t) = f(t)\) for \(t \in (a,b)\) where \(f(t)\) is continuous
\item \(f_{p}(t) = \frac{f(t^{-}) + f(t^{+})}{2}\) for \(t \in (a,b)\) where \(f(t)\) is not continuous
\item \(f_{p}(a) = \frac{f(a) + f(b)}{2} = f_{p}(b)\)
\end{itemize}

Let \(f_{p}\) be the periodic extension of a function \(f \in L^{2}([-\tau/2, \tau/2])\):
\begin{itemize}
\item the Fourier series of \(f\) converges in the \(L^{2}\) norm to \(f\) and \(f_{p}\) on any finite
subinterval of \([-\tau/2, \tau/2]\)
\item if \(f_{p}\) is piecewise \(C^{1}\), then the Fourier series of \(f\) converges pointwise to
\(f_{p}\) for all \(x \in \mathbb{R}\)
\item if \(f_{p}\) is piecewise \(C^{1}\) and continuous, then the Fourier series of \(f\) converges
uniformly to \(f_{p}\) on any finite interval of \(\mathbb{R}\)
\end{itemize}

\textbf{Gibbs Phenomenon}: for an \(L^{2}([a,b])\) function \(f\) with periodic extension \(f_{p}\),
if \(f_{p}\) is not continuous at some point \(t_{0}\), then \uline{truncated} Fourier series of
\(f\) will have growing oscillations near the point \(t_{0}\)
\begin{itemize}
\item these oscillations do not appear in the infinite sum
\end{itemize}
\end{document}
