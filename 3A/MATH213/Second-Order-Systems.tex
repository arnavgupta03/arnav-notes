% Created 2024-04-05 Fri 11:57
% Intended LaTeX compiler: pdflatex
\documentclass[11pt]{article}
\usepackage[utf8]{inputenc}
\usepackage[T1]{fontenc}
\usepackage{graphicx}
\usepackage{longtable}
\usepackage{wrapfig}
\usepackage{rotating}
\usepackage[normalem]{ulem}
\usepackage{amsmath}
\usepackage{amssymb}
\usepackage{capt-of}
\usepackage{hyperref}
\usepackage{parskip, darkmode}
\enabledarkmode
\author{Arnav Gupta}
\date{\today}
\title{Standard 2nd Order System, PI Controllers, and Extra Poles}
\hypersetup{
 pdfauthor={Arnav Gupta},
 pdftitle={Standard 2nd Order System, PI Controllers, and Extra Poles},
 pdfkeywords={},
 pdfsubject={},
 pdfcreator={Emacs 29.3 (Org mode 9.7)}, 
 pdflang={English}}
\begin{document}

\maketitle
\tableofcontents

\section{Standard Second Order System}
\label{sec:org62c989b}
The standard second order system has a transfer function given by
$$
H(s) = \frac{\omega^{2}}{s^{2} + 2\xi \omega s + \omega^{2}}
$$
where \(\omega > 0\) and \(\xi \ge 0\).

The poles of \(H(s)\) are \(s_{\pm} = -\xi \omega \pm \omega \sqrt{\xi^{2} - 1}\),
and so forms of the solution depend on the poles.
\subsection{Case 1: Overdamped Case}
\label{sec:org93bd4bd}
For 2 real distinct poles, \(\sqrt{\xi^{2} - 1}\) is real and more than 0.
This means \(\xi > 1\) and this means the standard second order system can be
decomposed as a sum of 2 first order systems with poles that have
negative real parts.

This means that for this case any bounded input will give a bounded system
response.

The system's impulse response is \(h(t) = A_{+} e^{s_{ +}t} + A_{-} e^{s_{ -}t}\)
where the \(A\) terms are the coefficients of the corresponding poles in the
PF decomposition.
\subsection{Case 2: Critically Damped Case}
\label{sec:orgdc9fc0f}
For 1 real repeated pole, \(xi = 1\) and the repeated pole is at \(s_{root} = - \omega\),
which has a negative real part.

The system's impulse response is \(h(t) = \omega^{2} t e^{-\omega t}\).
\subsection{Case 3: Underdamped Case}
\label{sec:orgcbf7da7}
For 2 complex conjugate poles with a non-zero real component, \(0 < \xi < 1\) and
poles are at
$$
s_{\pm} = -\xi \omega \pm \omega j \sqrt{1 - \xi^{2}}
$$
which have negative real parts.

The system's impulse response is
$$
h(t) = \frac{\omega}{\sqrt{1 - \xi^{2}}} e^{-\xi \omega t}
\sin((\omega \sqrt{1 - \xi^{2}})t)
$$

If \(\omega\) increases, the poles move further from the origin and the response
is faster.

if \(\xi\) increases, \(\theta\) decreases and there is more damping.
\subsection{Case 4: Undamped Case}
\label{sec:org6339caf}
For 2 complex conjugate poles with a zero real component, \(\xi = 0\) and poles are at
$$
s_{\pm} = \pm \omega j
$$
which have 0 real parts and so the system may not be well behaved.

The system's impulse response is
$$
h(t) = \omega \sin(\omega t)
$$
\subsection{Step Response}
\label{sec:org77ba733}
The step response for \(\xi \ge 0\) other than \(\xi = 1\) is
$$
1 - \frac{1}{\sqrt{1 - \xi^{2}}} e^{-\xi \omega t} \sin \left(
(\omega \sqrt{1 - \xi^{2}}) t + \theta
\right)
$$
where \(\theta = \cos^{-1} \xi\).

At \(\xi = 1\) this is a \(t\) scaled exponential.
\section{PI Controllers}
\label{sec:org97294b8}
P controllers can get to goal quickly but miss the exact value.
I controllers can get to exact velocity but too slowly.

P and I controllers can be used together and give a system with
transfer function
$$
H_{RV}(s) = \frac{\frac{k_{i}}{m} \left( \frac{k_{p}}{k_{i}} s + 1 \right)}{s^{2} +
\left( \frac{b + k_{p}}{m} \right) s + \frac{k_{i}}{m}}
$$
\end{document}
