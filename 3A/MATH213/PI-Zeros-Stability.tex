% Created 2024-04-05 Fri 11:57
% Intended LaTeX compiler: pdflatex
\documentclass[11pt]{article}
\usepackage[utf8]{inputenc}
\usepackage[T1]{fontenc}
\usepackage{graphicx}
\usepackage{longtable}
\usepackage{wrapfig}
\usepackage{rotating}
\usepackage[normalem]{ulem}
\usepackage{amsmath}
\usepackage{amssymb}
\usepackage{capt-of}
\usepackage{hyperref}
\usepackage{parskip, darkmode}
\enabledarkmode
\author{Arnav Gupta}
\date{\today}
\title{PI Controllers, Zeros, Extra Poles, and Stability}
\hypersetup{
 pdfauthor={Arnav Gupta},
 pdftitle={PI Controllers, Zeros, Extra Poles, and Stability},
 pdfkeywords={},
 pdfsubject={},
 pdfcreator={Emacs 29.3 (Org mode 9.7)}, 
 pdflang={English}}
\begin{document}

\maketitle
\tableofcontents

\section{PI Controller Analysis}
\label{sec:org9cdaeef}
The transfer function of the response of a PI controller can be simplified to the form:
$$
H_{z}(s) = \frac{\omega^{2} \left( \frac{s}{\alpha \xi \omega} + 1 \right)}
{s^{2} + 2 \xi \omega s + \omega^{2}}
$$

By analyzing the response of a PI controller to the step function, this can be decomposed
into a standard 2nd order impulse response and a standard 2nd order step response.

Specifically, if \(\alpha \to \infty\), then it goes to the step response, and if
\(\alpha \to 0\), then it goes to \(\infty \cdot\) the impulse response.

Combining these effects can lead to using \(k_{i}\) to adjust the speed of the response
and using \(k_{p}\) to control the dampening of the oscillations.
\section{Adding a Pole}
\label{sec:org10fb33a}
By adding a pole at \(s = -\alpha \xi \omega\) to the transfer function of a system,
it can be decomposed into the linear combination of 1st and 2nd order systems.

For this, if \(\alpha \to 0\), then the impulse response looks like a standard 1st
order system.
If \(\alpha \to \infty\), then the impulse response looks like a standard 2nd order
system.

When a real pole or a complex-conjugate pair of poles are an order of magnitude
closer to the imaginary axis than all other poles, then they are dominant.
\section{Stability}
\label{sec:org360b7c7}
The impulse response of \textbf{all} linear time invariant systems is a linear combination
of the types of functions analyzed.
Further, since \(\{\delta(t - \tau) | \tau \in \mathbb{R}\}\) is a basis for the
set of all functions, the response of any LTI to any function \(f(t)\) can be written
as a convolution of \(f(t)\) with a linear combination of the types of functions
analyzed.

An LTI \(S\) is stable if \(S(\delta(t))\) decays to 0.

An LTI \(S\) is unstable if \(S(\delta(t))\) is unbounded.

An LTI \(S\) is marginally stable if \(S(\delta(t))\) is bounded but does not decay
to 0.

A transfer function is stable if the system it is a transfer function for is stable.

A transfer function is unstable if the system it is a transfer function for is
unstable.

A transfer function is marginally stable if the system it is a transfer function for
is marginally stable.

A transfer function is stable if all poles have a negative real part.

A transfer function is unstable if there is a pole with a positive real part or
there is a 2nd order pole that has real part 0.

A transfer function is marginally stable if there are no poles with postive real parts
or 2nd order poles with real part 0 and in addition at least order 1 pole with real
part 0.

An LTI \(S\) is bounded-input, bounded-output (BIBO) stable if \(S(f)\) is bounded for
all bounded functions \(f\).

An LTI system with a rational transfer function is BIBO stable if and only if its
transfer function is both stable and proper.
\end{document}
