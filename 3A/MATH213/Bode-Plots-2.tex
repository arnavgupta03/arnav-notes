% Created 2024-03-30 Sat 02:29
% Intended LaTeX compiler: pdflatex
\documentclass[11pt]{article}
\usepackage[utf8]{inputenc}
\usepackage[T1]{fontenc}
\usepackage{graphicx}
\usepackage{longtable}
\usepackage{wrapfig}
\usepackage{rotating}
\usepackage[normalem]{ulem}
\usepackage{amsmath}
\usepackage{amssymb}
\usepackage{capt-of}
\usepackage{hyperref}
\usepackage{parskip,darkmode}
\enabledarkmode
\author{Arnav Gupta}
\date{\today}
\title{More Bode Plots}
\hypersetup{
 pdfauthor={Arnav Gupta},
 pdftitle={More Bode Plots},
 pdfkeywords={},
 pdfsubject={},
 pdfcreator={Emacs 29.3 (Org mode 9.7)}, 
 pdflang={English}}
\begin{document}

\maketitle
\tableofcontents

\section{Drawing Bode Plots}
\label{sec:orgaf2ee05}
The starting value of the magnitude curve is \(|H(0)|_{dB}\) and the starting value of the phase curve
is \(\angle H(0)\).
This does not exist if the curve is unbounded, which is if there is a pole with a real value of 0.

If the transfer function has an order 1 stable pole at a point, then the magnitude curve will
experience an extra decrease of -20dB/decade and the phase curve will experience a drop of \(-90^{\circ}\).
These phase adjustment will be rather slow taking an order of magnitude before and after the pole
to adjust.

If the transfer function has complex conjugate poles that are stable, then the magnitude curve
will experience an extra decrease of -40db/decade and the phase curve will experience a drop of
\(-180^{\circ}\).
The location of this adjustment is given by \(\omega\) where \(\omega\) is the term in the second order
system.
The speed of adjustment is related to \(\xi\):
\begin{itemize}
\item \(\xi \approx 0\) causes fast adjustments and overshooting in the amplitude plot for nearby \(\omega\)
\item \(\xi \approx 1\) causes slow adjustments and undershooting in the amplitude plot for nearby \(\omega\)
\end{itemize}

If there are zeros of order 1 or order 2 at some point, then the magnitude and phase curves
experience the opposite effects as listed above.

Does not make sense to have a Bode plot if there is a unstable pole.

If there is a decrease in the slope of asymptotic amplitude curve and an increase in the phase,
the system is unstable (a Bode plot should never have been made).
Cannot conclude anything about stability of the system modelled by the transfer function used to
generate the plot though.
\section{Stability}
\label{sec:org5a83f43}
Consider a closed loop system  where the output is produced by \(P(s)\) and this is controlled by a
controller with transfer function \(C(s)\).
The transfer function for this system is
$$
\frac{P(s)C(s)}{1 + P(s)C(s)}
$$

Hence the closed system will be stable when both \(C(s)\) and \(P(s)\) are stable which is when
$$
1 + P(s)C(s) \ne 0
$$

If \(C(s)\) and \(P(s)\) are both stable, the closed loop control system is unstable when
$$
P(s)C(s) = -1 = 1 \cdot e^{i\pi}
$$
This happens when the amplitude curve is \(|1|_{dB} \approx 0\) while the phase is either
\(\pm 180^{\circ}\) or some other equivalent angle.

To talk about stability, rather than looking for these exact conditions, look at the points
where either the amplitude or phase is critical and look at how far it is from stability.
Generally require some threshold to be far enough from being unstable.
\end{document}
