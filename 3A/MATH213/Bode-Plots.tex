% Created 2024-04-05 Fri 11:56
% Intended LaTeX compiler: pdflatex
\documentclass[11pt]{article}
\usepackage[utf8]{inputenc}
\usepackage[T1]{fontenc}
\usepackage{graphicx}
\usepackage{longtable}
\usepackage{wrapfig}
\usepackage{rotating}
\usepackage[normalem]{ulem}
\usepackage{amsmath}
\usepackage{amssymb}
\usepackage{capt-of}
\usepackage{hyperref}
\usepackage{parskip, darkmode}
\enabledarkmode
\author{Arnav Gupta}
\date{\today}
\title{Bode Plots}
\hypersetup{
 pdfauthor={Arnav Gupta},
 pdftitle={Bode Plots},
 pdfkeywords={},
 pdfsubject={},
 pdfcreator={Emacs 29.3 (Org mode 9.7)}, 
 pdflang={English}}
\begin{document}

\maketitle
\tableofcontents

\section{Frequency Response}
\label{sec:org1af78a1}
If \(S:f \to y\) is an LTI with transfer function \(H(s)\), then for any \(s \in \mathbb{C}\),
$$
e^{st} \xrightarrow{S} H(s) e^{st}
$$

If \(s = j\omega\), then \(H(j\omega)\) can be written in polar form.
\(H(j \omega)\) is then the frequency response.

If \(S\) is an LT with transfer function \(H(s)\), then
$$
\sin(\omega t) \xrightarrow{S} |H(j\omega)| \sin(\omega t + \angle_{H(j\omega)})
$$

This means that the system reponse of an LTI to a sin wave of frequency \(\omega\):
\begin{itemize}
\item has an amplitude scaled by \(H(j\omega)\)
\item has the same frequency
\item has a phase shifted by \(\angle_{H(j\omega)}\)
\end{itemize}

If \(S\) is an LT with transfer function \(H(s)\), then as \(t \to \infty\)
$$
\sin(\omega t)u(t) \xrightarrow{S} |H(j\omega)| \sin(\omega t + \angle_{H(j\omega)})
$$

Fourier series/transforms allow decomposing functions as a sum/integral of sin
and cos waves.
\section{Bode Plots}
\label{sec:orgbb0e82f}
\textbf{Bode plots}: a graphical representation of the frequency response

One plot for magnitude and one for phase.
\subsection{Decibels}
\label{sec:org5b5a24b}
\(|H(j\omega)|\) in decibels is \(20 \log_{10}(|H(j\omega)|)\)

This means magnitude curves for multiplied frequency responses can be found
by adding magnitude curves for each factor.
\subsection{Finding}
\label{sec:orga75cb35}
If given a transfer function \(H(s) = H_{1}(s) \cdot H_{2}(s) \cdots H_{k}(s)\), then
the Bode plot for \(H(j \omega)\) is found by
\begin{itemize}
\item finding the magnitude and phase curves for each \(H_{i}(j\omega)\)
\item adding the magnitude and phase curves
\end{itemize}
\end{document}
