% Created 2024-04-05 Fri 12:01
% Intended LaTeX compiler: pdflatex
\documentclass[11pt]{article}
\usepackage[utf8]{inputenc}
\usepackage[T1]{fontenc}
\usepackage{graphicx}
\usepackage{longtable}
\usepackage{wrapfig}
\usepackage{rotating}
\usepackage[normalem]{ulem}
\usepackage{amsmath}
\usepackage{amssymb}
\usepackage{capt-of}
\usepackage{hyperref}
\usepackage{parskip, darkmode, mathrsfs}
\enabledarkmode
\author{Arnav Gupta}
\date{\today}
\title{Transfer Functions, Responses, and the Standard First-Order System}
\hypersetup{
 pdfauthor={Arnav Gupta},
 pdftitle={Transfer Functions, Responses, and the Standard First-Order System},
 pdfkeywords={},
 pdfsubject={},
 pdfcreator={Emacs 29.3 (Org mode 9.7)}, 
 pdflang={English}}
\begin{document}

\maketitle
\tableofcontents

\section{Transfer Functions}
\label{sec:org3634146}
If a system is able to be modelled by a linear DE with constant coefficients, then the
transfer function can be obtained from the differential equations.

In general, \(Y(s) = H(s) F_{ap}(s)\) so \(y(t) = \mathscr{L}^{-1} \{ H(s) F_{ap}(s) \}\).

In general, \(F_{ap}(s)\) can have poles of its own and thus the system response will
reflect the poles of both the transfer function \(H(s)\) and the Laplace Transform of
the forcing term \(F_{ap}(s)\).

The effects of the transfer function are present for any input so it is particularly
important to understand the effects of the poles (and zeros) of \(H(s)\).
\section{Responses}
\label{sec:org720624d}
The response to the unit impulse is \(Y(s) = H(s)\).

The response to the unit step impulse is \(Y(s) = \frac{1}{s} H(s)\).

The step response is the integral of the impulse response.

In the real world, often easier to physically generate a unit step function than a
unity impulse.

All polynomials with real valued coefficients can be factored into a product of
linear and quadratic terms, so to understand the system response of any system,
it is sufficient to understand how 1st and 2nd order linear systems respond.
\section{Standard 1st Order System}
\label{sec:org47ec93d}
The standard 1st order system has transfer function
$$
        H(s) = \frac{\kappa}{s \tau + 1}
$$
where \(\kappa, \tau > 0\), \(\kappa\) is called the \textbf{DC gain} and \(\tau\) is called
the \textbf{time constant}.

Since \(H(s)\) has a pole at \(s = -\frac{1}{\tau}\), the impulse response of the
standard 1st order system is:
$$
        h(t) = \frac{\kappa}{\tau} e^{-t / \tau} u(t)
$$

The standard 1st order system is causal.

Changing \(\tau\) changes the value of \(h(0)\) inversely and changes the growth rate.

The step response of the standard 1st order system is
$$
        Y(s) = H(s) \frac{1}{s} = \kappa/(s(s\tau + 1)), y(t) = \kappa [ 1 - e^{-t / \tau}] u(t)
$$

Changing \(\kappa\) increases the steady state value of \(y(t)\) and \(H(0) = \kappa\).
\end{document}
