% Created 2024-04-05 Fri 11:36
% Intended LaTeX compiler: pdflatex
\documentclass[11pt]{article}
\usepackage[utf8]{inputenc}
\usepackage[T1]{fontenc}
\usepackage{graphicx}
\usepackage{longtable}
\usepackage{wrapfig}
\usepackage{rotating}
\usepackage[normalem]{ulem}
\usepackage{amsmath}
\usepackage{amssymb}
\usepackage{capt-of}
\usepackage{hyperref}
\usepackage{parskip,darkmode}
\enabledarkmode
\author{Arnav Gupta}
\date{\today}
\title{Fourier Series, Parseval and Dirichlet Theorems, Fourier Transform}
\hypersetup{
 pdfauthor={Arnav Gupta},
 pdftitle={Fourier Series, Parseval and Dirichlet Theorems, Fourier Transform},
 pdfkeywords={},
 pdfsubject={},
 pdfcreator={Emacs 29.3 (Org mode 9.7)}, 
 pdflang={English}}
\begin{document}

\maketitle
\tableofcontents

\section{Integrating and Differentiating Fourier Series}
\label{sec:org465c8c5}
Integrals and derivatives for Fourier series do not follow from linearity.

The Fourier series of a PWC1 \(\tau\) periodic \(L^{2}\) function \(f(t)\) can be term-by-term integrated to give a convergent
series that \textbf{uniformly converges} over any finite interval.
If \(f_{p}\) is in \(L^{2}\) and
$$
f_{p}(t) = \sum_{n = -\infty}^{\infty} c_{n} e^{\frac{2\pi n j t}{\tau}}
$$
then
$$
\int_{a}^{t} f_{p}(t) \, dt = \sum_{n = -\infty}^{\infty} c_{n} \int_{a}^{t} e^{\frac{2\pi n j t}{\tau}} \, dt
$$
and the convergence is at least pointwise.
Here the RHS might not be a Fourier series.

Let \(f\) be a PWC1 \(\tau\) periodic function, that is continuous, and satisfies \(f(-\tau/2) = f(\tau/2)\).
The series for \(f\) can be term-by-term differentiated to give a pointwise convergent series that
\textbf{pointwise converges} to \(f_{p}'(t)\) for all \(t\) such that \(f_{p}''(t)\) exists.
Explicitly, if \(f\) satisfies the above and
$$
f_{p}(t) = \sum_{n = -\infty}^{\infty} c_{n} e^{\frac{2\pi n j t}{\tau}}
$$
then
$$
f_{p}'(t) = \sum_{n = -\infty}^{\infty} c_{n} \frac{d}{dt} e^{\frac{2 \pi n j t}{\tau}}
$$
and the convergence is pointwise for all points where \(f_{p}''(t)\) exists.
\section{Parseval and Dirichlet Theorems}
\label{sec:org44cfda7}
\textbf{Dirichlet's Theorem}: if \(f\) is PWC1 and \(\tau\) periodic, then the Fourier series of \(f(t)\) converges pointwise to
\(f_{p}(t)\)

Dirichlet's Theorem gives that the average error for truncated Fourier series goes to zero in the limit.

\textbf{Parseval's Theorem}: if \(f\) is \(L^{2}[-\tau/2, \tau/2]\) and \(\tau\) periodic, then
$$
\frac{1}{\tau} \int_{-\tau/2}^{\tau/2} |f(t)|^{2} dt = \sum_{n = -\infty}^{\infty} |c_{n}|^{2}
$$

\textbf{Real-Valued Parseval's Theorem}: if \(f\) is a real valued PWC1 and \(\tau\) periodic, then
$$
\frac{1}{\tau} \int_{-\tau/2}^{\tau/2} |f(t)|^{2} dt = c_{0}^{2} +
\frac{1}{2} \sum_{n = 1}^{\infty} \left(c_{n}^{2} + s_{n}^{2}\right)^{2}
$$
where \(c_{n}\) and \(s_{n}\) are the Fourier cosine and sine coefficients.
\section{Fourier Transform}
\label{sec:orgbc73cc3}
Fourier series require \(\tau\) periodic functions.

If \(f \in L^{1}\), then the Fourier transform of \(f\) is
$$
F(\omega) = \mathcal{F}\{ f(t) \} = \int_{-\infty}^{\infty} f(t) e^{-\omega t j} \, dt
$$
and the inverse Fourier transform of \(F(\omega)\) is
$$
f(t) = \mathcal{F}\{ F(\omega) \} = \frac{1}{2\pi} \int_{-\infty}^{\infty} F(\omega) e^{\omega t j}
\, d\omega
$$

Fourier transform is the two-sided Laplace transform where \(s = \omega j\).
\end{document}
