% Created 2024-04-19 Fri 14:26
% Intended LaTeX compiler: pdflatex
\documentclass[11pt]{article}
\usepackage[utf8]{inputenc}
\usepackage[T1]{fontenc}
\usepackage{graphicx}
\usepackage{longtable}
\usepackage{wrapfig}
\usepackage{rotating}
\usepackage[normalem]{ulem}
\usepackage{amsmath}
\usepackage{amssymb}
\usepackage{capt-of}
\usepackage{hyperref}
\usepackage{parskip,darkmode}
\enabledarkmode
\author{Arnav Gupta}
\date{\today}
\title{Accessibility}
\hypersetup{
 pdfauthor={Arnav Gupta},
 pdftitle={Accessibility},
 pdfkeywords={},
 pdfsubject={},
 pdfcreator={Emacs 29.3 (Org mode 9.7)}, 
 pdflang={English}}
\begin{document}

\maketitle
\tableofcontents

\section{Types of Disabilities}
\label{sec:orgb92876a}
UI Accessibility is developing content to be as accessibly as possible, no matter an individual's
physical and cognitive abilities and how they access a UI.

Parallel with physical accessibility, access to UIs is considered a human right.

People vary in physical and mental capabilities (can be permanent, temporary, or situational).
People have a range of ability dimensions like age, gender, cognitive, physical, culture,
language, lived experience, emotional, and spiritual.

Temporary and Situational disabilites are sick/injured, driving a car, underwater diving, using
an ATM, etc.
\section{Accessibility Approaches}
\label{sec:orge66be48}
For those walking, as speed and difficulty increased, time to complete tasks and error rate
increased.
People read more slowly and answer questions more slowly while walking .

To address situational walking impairment, address reduced dexterity and motor control and
reduced cognitive ability.
Increase size of elements and visual cues and widget targets.

With higher age, motor coordination is reduced, visual and hearing impairments, and cognitive
effects like loss of memory.

For age-related impairments, keep infor simple (cognitive), high contrast colours and large
text/icons (vision), and large widget and button sizes (motor).

10 to 20\% of the population is estimated to have a long term disability.

Modern OS support for accessibility is:
\begin{itemize}
\item control cursor from keyboard (motor)
\item adjust acceleration, tracking, precision (motor)
\item speech dictation (visual/motor)
\item magnify portions of the screen, adjust element sizes or fotn size, provide full voice
dictation (visual)
\item captions/subtitles (audio)
\end{itemize}

For visual impairments:
\begin{itemize}
\item zoom screen or specific area, increase font size
\item high contrast colours, dark mode, remove animations
\item screen reader, voice input
\item real world magnifier
\end{itemize}

For hearing impairments:
\begin{itemize}
\item show audio alerts visually (vibrate to flashlight alarm)
\item realtime audio processing to filter background noise and amplify the voice of another
person
\item monitor audio for certain sounds and send alert (ex. baby crying)
\end{itemize}

For motor impairments:
\begin{itemize}
\item sticky keys, slow keys, filter keys
\item reduce key repeat rate
\item eye tracking, voice input
\item physical switched and puffers
\item brain-computer interfaces (BCI)
\end{itemize}

For cognitive impairments:
\begin{itemize}
\item word prediction, grammar and spell check
\item text-to-speech
\item augmenting text with icons and pictures
\item slow down interface
\begin{itemize}
\item avoid sudden state changes
\item reduce or remove unnecessary animations (ex. flickering)
\item eliminate time sensitive actions
\end{itemize}
\end{itemize}

Laws and programs designed to benefit vulnerable groups often end up benefiting
all of society (curbcut effect).

Closed captions benefit more than just those with hearing impairments.
Helps with watching in silence, noisy environments, learning a 2nd language,
watching foreign language content, and data mining video content.

Legal obligations in Canada:
\begin{itemize}
\item Accessible Canada Act
\begin{itemize}
\item government and federally regulated organizations
\item expected to use Web Content Accessibility Guidelines (WCAG)
\item fines up to \$250k
\end{itemize}
\item Accessibility for Ontarians with Disabilities Act (since 2005)
\begin{itemize}
\item applies to all Ontario government websites
\item applies to Ontario public and private entities (50+ employees)
\item must adhere to Web Content Accessibility Guidelines (WCAG) 2.0
\item fines up to \$100k
\end{itemize}
\end{itemize}

Legal obligations elsewhere:
\begin{itemize}
\item US Disabilities and Rehabilitation Act (section 508)
\begin{itemize}
\item any organization doing business with federal agency or receives federal
funding
\end{itemize}
\item Americans with Disabilities Act (ADA)
\begin{itemize}
\item non-profits, business, local and state governments
\end{itemize}
\item European Union Web Accessibility Directive
\begin{itemize}
\item all government websites
\item any organizations financed through public contracts
\end{itemize}
\end{itemize}
\section{Implementing Web Accessibility}
\label{sec:org5c1c293}
\textbf{Accessibilty tree}: the browser generates an accessibility tree from the DOM with
accessibility-related info for most HTML elements like name, description, role, state

Can access accessibility tree from the DevTools accessibility tab.

Web Content Accessibility Guidelines:
\begin{itemize}
\item include alt text for info imags
\item use headings correctly
\item give links unique and descriptive names
\item use colour with care
\item use tables for tabular data, not layout
\item use ARIA roles and landmarks when necessary
\item make dynamic content accessible
\item make all content accessible using keyboard
\item design forms for accessibility
\end{itemize}

Semantic HTML is using semantic elements that clearly describe the content meaning like
article, aside, details, etc. Avoid using div and span for everything.

For Preact, avoid div for component root if semantic element possible.
Use Fragment instead of div if root has no semantic purpose.
Some components could have different semantic roles, so use ``as Element'' pattern to set
component root element.

Add link at the top of the page so screen readers can skip to content.
Can use CSS to hide until focused.

ARIA (Accessible Rich Internet Applications) attributes are a specification to add
semantics to elements.

ARIA roles (use only if not possible to use an HTML element): toolbar, tooltip, feed,
math, presentation, note

ARIA states and properties (use only for attributes not supported on HTML element):
aria-required, aria-checked, aria-disabled

ARIA attributes only change the the accessibility tree.

Colour blindness:
\begin{itemize}
\item dichromacy is 1 type of cone missing
\begin{itemize}
\item protanopia: missing red cones
\item deuteranopia: missing gree cones
\item tritanopia: missing blue cones (and blue sensitive rods)
\end{itemize}
\item monochromacity is 2 or 3 types of cones missing
\end{itemize}

Human ability to discriminate colours depends on context.

Colour contrast is ratio of perceived luminance for 2 colours, such as
white on white (1.0), red on white (4.0), and black on white (21.0).
Can be checked in DevTools.

WCAG colour contrast guidelines are minimum (level AA) at least 4.5 and
enhanced (level AAA) at least 7.0.

Basic a11y testing methods:
\begin{itemize}
\item disconnect mouse and try to use the app
\begin{itemize}
\item tab and shift+tab to focus elements
\item enter to activate element
\item arrow keys when needed (menu dropdown)
\end{itemize}
\item test with a screen reader
\item a11y linters/checkers
\end{itemize}
\end{document}
