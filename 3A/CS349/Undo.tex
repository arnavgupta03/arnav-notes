% Created 2024-04-18 Thu 13:37
% Intended LaTeX compiler: pdflatex
\documentclass[11pt]{article}
\usepackage[utf8]{inputenc}
\usepackage[T1]{fontenc}
\usepackage{graphicx}
\usepackage{longtable}
\usepackage{wrapfig}
\usepackage{rotating}
\usepackage[normalem]{ulem}
\usepackage{amsmath}
\usepackage{amssymb}
\usepackage{capt-of}
\usepackage{hyperref}
\usepackage{parskip,darkmode}
\enabledarkmode
\author{Arnav Gupta}
\date{\today}
\title{Undo}
\hypersetup{
 pdfauthor={Arnav Gupta},
 pdftitle={Undo},
 pdfkeywords={},
 pdfsubject={},
 pdfcreator={Emacs 29.3 (Org mode 9.7)}, 
 pdflang={English}}
\begin{document}

\maketitle
\tableofcontents

\section{Principles and Concepts}
\label{sec:org8c39932}
Benefits:
\begin{itemize}
\item allows recovery from errors (input/human and interpretation/computer)
\begin{itemize}
\item can work quickly and without fear
\end{itemize}
\item enables exploratory learning
\begin{itemize}
\item try things without knowing consequences
\item try alternative solutions
\end{itemize}
\item evaluate modifications
\begin{itemize}
\item fast do-undo-redo cycle to evaluate last change to document
\end{itemize}
\end{itemize}

Checkpointing is a manual undo method, save the current state to rollback later.
\section{Undo Patterns}
\label{sec:org271160b}
Undo design choices:
\begin{enumerate}
\item \textbf{undoable actions}: what actions can/should be undone
\item \textbf{state restoration}: what part of UI is restored after undo
\item \textbf{granularity}: how much should be undone at a time
\item \textbf{scope}: is undo global, local, or somewhere in between
\end{enumerate}
\subsection{Undoable Actions}
\label{sec:orgd30cd03}
Some actions may be omitted, destructive, not easily undone, or cannot be undone.

Suggestions:
\begin{itemize}
\item all changes to the document/model should be undoable
\item changes to the view/interface should only be undoable if tedious or requiring significant effort
\item ask for confirmation before doing a destructive action that cannot be easily undone
\end{itemize}
\subsection{State Restoration}
\label{sec:org06595e6}
User interface state should be meaningful after redo action:
\begin{itemize}
\item restore prior selection of objects restored by redo
\item scroll to show restored objects (if necessary)
\item give focus to the control where the state was restored
\end{itemize}
\subsection{Granularity}
\label{sec:orga5966c8}
A \textbf{chunk} is conceptual change from one state to another, where interaction can be divided into
undoable chunks and undo reverse 1 chunk.

For drawing interactions, undo full lines, not pixels.

For text interactions, rules differ but can be by word, sentence, timing, line, etc.

Suggestions:
\begin{itemize}
\item ignore direct manipulation intermediate states
\item delimit chunks on discrete input breaks
\item chunk all changes resulting from a single interface event
\end{itemize}
\subsection{Scope}
\label{sec:org8af59dc}
One option is to treat all windows as a single model, with a single set of actions that can be undone,
so on undo/redo, change window focus.

Another option is to have multiple undo/redo stacks, one for each window, and use whichever has focus
(more typical).
\section{Implementation}
\label{sec:org5136925}
General approaches are:
\begin{itemize}
\item \textbf{forward undo}: start from base document, maintain list of changes to compute current document,
undo by removing last change from list when computing current document
\item \textbf{reverse undo}: apply change to update document but also save reverse change, undo by applying reverse
change to document
\end{itemize}

A \textbf{change record} defines a single transformation to the document/model.
\subsection{Forward Undo}
\label{sec:org69d6263}
Save baseline document state at some past point: \(S^{*}\)

Save change records to transform baseline document into current document state:
\(S = (c(b(a(S^{*}))))\)

To undo last action, don't apply last change record:
\$S' = (b(a(S\textsuperscript{*})))
\subsection{Reverse Undo}
\label{sec:org77e5abc}
Save complete current document state: \(S\)

Save reverse change records to return to previous state: \(\{c^{-1}, b^{-1}, a^{-1}\}\)

To undo last change, apply last reverse change record: \(S' = c^{-1}(S)\)
\subsection{Implementation with Stacks}
\label{sec:orgfa6ae37}
Both options require stacks:
\begin{itemize}
\item \textbf{undo stack}: all change records, saved as actions performed
\item \textbf{redo stack}: change records that have been undone (must be reapplied with redo)
\end{itemize}
\subsubsection{Reverse Undo Command Pattern}
\label{sec:orgc0199d0}
User issues a command, which is executed to create a new document state, then reverse command is
pushed onto the undo stack, and redo stack is cleared.

Undo pops reverse command from undo stack and executes it to create new document state and
pushes command onto redo stack.

Redo pops command off redo stack and executes it to create new document state and push reverse
command onto undo stack.

Reverse Change Record Implementation Options:
\begin{itemize}
\item \textbf{command} pattern: save command and reverse command to change state
\item \textbf{memento} pattern: save snapshots of each document state, could be complete state or
difference from last state
\begin{itemize}
\item executing undo moves top memento to redo stack, then uses new top of undo stack to set model
\item needs base memento in constructor, so when undo stack is empty, base memento is used
\end{itemize}
\end{itemize}
\subsection{Destructive Commands}
\label{sec:orga377dc3}
One option is to use forward command undo.

Another option is to use reverse command undo, but the un-execute command stores previous state for
destructive commands (a memento).
Could require a lot memory, so some application limit size of undo stack.
\end{document}
