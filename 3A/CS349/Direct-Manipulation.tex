% Created 2024-04-19 Fri 12:46
% Intended LaTeX compiler: pdflatex
\documentclass[11pt]{article}
\usepackage[utf8]{inputenc}
\usepackage[T1]{fontenc}
\usepackage{graphicx}
\usepackage{longtable}
\usepackage{wrapfig}
\usepackage{rotating}
\usepackage[normalem]{ulem}
\usepackage{amsmath}
\usepackage{amssymb}
\usepackage{capt-of}
\usepackage{hyperref}
\usepackage{parskip,darkmode}
\enabledarkmode
\author{Arnav Gupta}
\date{\today}
\title{Direct Manipulation}
\hypersetup{
 pdfauthor={Arnav Gupta},
 pdftitle={Direct Manipulation},
 pdfkeywords={},
 pdfsubject={},
 pdfcreator={Emacs 29.3 (Org mode 9.7)}, 
 pdflang={English}}
\begin{document}

\maketitle
\tableofcontents

\section{Instrumental Interaction}
\label{sec:org12d4cce}
\textbf{Direct manipulation} is when a virtual representation of an object is manipulated in a similar
way to a real world object.
Goal is to make the interaction feel like manipulating a real object instead of working through
and intermediary.

Requires a representation of task objects (something the user can manipulate).
Can be the object of interest or an interface widget.

Characteristics of Direct Manipulation:
\begin{enumerate}
\item continuous representation of task objects and actions
\item task objects manipulated by physical actions, not complex syntax
\item fast, incremental, and reversible actions with effects on task objects immediately
apparent
\item layered, self-revealing approach to learning
\end{enumerate}

Benefit of direct manipulation is the feel as if interacting wih the task object rather than
the interface, so focus is on the task, not the technology.

Feeling of direct involvement with world of task objects rather than communication with an
intermediary.

An \textbf{interaction model} is a set of principles, rules, and properties that guide the design of
an interface. Describes how to combine interaction techniques in a meaningful and consistent
way and defines the look and feel of the interaction from the user's perspective.
Properties of the interaction model can be used to evaluate specific interaction designs.

\textbf{Instrumental interaction} is a model of interaction based on natural use of tools/instruments
to manipulate objects in the physical world.

Interfaces have:
\begin{itemize}
\item \textbf{interaction instruments}: necessary mediators between user and domain objects
\item \textbf{domain objects}: things of interest, data, and associated attributes, which are
manipulated using an interaction instrument
\end{itemize}

\textbf{Instrument activation} is how an instrument is triggered for use.
GUI instruments activated spatially (movement cost) and temporally (time cost).
UI layout and design is concerned with the tradeoff of these costs.

\textbf{Reification}: turning concepts into something concrete

An instrument is the reificiation of a command.

\textbf{Meta-instrument}: an instrument that acts on another instrument, which becomes an object
of interest

Object reficiation involves turning attributes of a primary object into other objects of
interest.

3 properties of instruments:
\begin{itemize}
\item degree of \textbf{indirection}: spatial/temporal offset between instrument and action on object
\item degree of \textbf{integration}: match between input device to instrument degrees of freedom
\item degree of \textbf{compatibility}: similarity of action on device/instrument to action on object
\end{itemize}

Degree of indirection is the 2D measure of distance from instrument to object:
\begin{itemize}
\item spatial dimension can be near (drag to translate, handle to resize), medium
(scrollbar near page), or far (dialog box)
\item temporal dimension can be short (direct drag response), medium (activate tool in toolbar,
start direct manipulation), or long (using dialog, full drag-and-drop operation)
\end{itemize}

Degree of integration is the ratio of the degrees of freedom (DOF) of the instrument over
the DOF captured by the input device, which captures the suitability of the device.

Degree of compatibility is the similarity of physical actions on instrument and response
of the object (more similar is better since it's more natural or intuitive).

A direct manipulation interface allows a user to directly act on a set of objects in the
interface. (low indirection, high integration, high compatibility)

Ideal level of direct manipulation is when instruments are visually indistinguishable from
object they control:
\begin{itemize}
\item actions on instrument/object entities are analogous to actions on similar object in the
real world
\item action on instrument/object entities preserve the conceptual linkage between instrument
and object
\end{itemize}
\section{Dragging}
\label{sec:org13a59c6}
Mousedown on shape starts drag (find offset from position to the shape frame-of-reference),
mousemove to drag, and mouseup to end drag.
\section{Drag-and-Drop}
\label{sec:orgd6f0363}
Most UI tooklits have built-in support for drag-and-drop.

A drag-and-drop interaction is:
\begin{enumerate}
\item press button (mousedown) with cursor on source object/data
\item drag source object onto target object
\item release button (mouseup) to drop object/data on target
\end{enumerate}

HTML drag-and-drop API enables drag-and-drop interactions in browser applications:
\begin{itemize}
\item make element draggable with \texttt{draggable} attribute
\item manage a drag operation by handling drag-and-drop events: dragstart, drag, dragenter,
dragleave, dragover, drop, dragend
\item define what info is manipulated
\end{itemize}

Direct manipulation feedback is important:
\begin{itemize}
\item visual indication of elements that can be dragged
\item feedback showing possible drop targets
\end{itemize}
\end{document}
