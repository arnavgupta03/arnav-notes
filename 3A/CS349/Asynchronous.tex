% Created 2024-04-18 Thu 15:47
% Intended LaTeX compiler: pdflatex
\documentclass[11pt]{article}
\usepackage[utf8]{inputenc}
\usepackage[T1]{fontenc}
\usepackage{graphicx}
\usepackage{longtable}
\usepackage{wrapfig}
\usepackage{rotating}
\usepackage[normalem]{ulem}
\usepackage{amsmath}
\usepackage{amssymb}
\usepackage{capt-of}
\usepackage{hyperref}
\usepackage{parskip,darkmode}
\enabledarkmode
\author{Arnav Gupta}
\date{\today}
\title{Asynchronous}
\hypersetup{
 pdfauthor={Arnav Gupta},
 pdftitle={Asynchronous},
 pdfkeywords={},
 pdfsubject={},
 pdfcreator={Emacs 29.3 (Org mode 9.7)}, 
 pdflang={English}}
\begin{document}

\maketitle
\tableofcontents

\section{UI Responsiveness}
\label{sec:orgbfa1633}
Responsive UIs deliver feedback to the user in a timely manner.
User does not wait longer than necessary and state of long tasks is communicated to the user.

Can make a UI responsive by:
\begin{enumerate}
\item designing to meet human expectations and perceptions
\item loading data efficiently so it is quickly available
\end{enumerate}

Responsiveness is affected by:
\begin{itemize}
\item user exectations: how quickly a system should react to complete some task
\begin{itemize}
\item expectations for technology (ex. web vs desktop)
\end{itemize}
\item application and interface design
\begin{itemize}
\item interface keeps up with user actions
\item interface informs user about application status
\item interface doesn't make user wait unexpectedly
\end{itemize}
\end{itemize}

Responsiveness is the most important factor is the most important factor in user satisfaction, more than
ease of learning/use.

Responsiveness is \textbf{not} just system performance.

If slow performance, to keep the UI responsive:
\begin{itemize}
\item provide feedback to confirm user actions (let user know input received)
\item provide feedback about events (how long an operation will take)
\item allow users to perform other tasks while waiting
\item anticipate the user's most common requests (pre-fetch data below current scroll vew)
\item perform low-priority system tasks in the background
\end{itemize}

Knowing duration of perceptual and cognitive processes can inform design of interactive systems that
feel responsive:
\begin{itemize}
\item minimal time to detect gap of silence in sound: 4 ms
\item minimal time to be affected by visual stimulus: 10 ms
\begin{itemize}
\item continuous input latency should be less than this
\end{itemize}
\item time that vision is supressed during saccade: 100 ms
\item max interval between cause-effect events: 140 ms
\begin{itemize}
\item if UI feedback takes longer, perception of cause-effect is broken
\end{itemize}
\item time to comprehend printed word: 150 ms
\item visual motor reaction time to observed event: 1 s
\begin{itemize}
\item display busy/progress indicators for operations >1 s
\item present a skeleton screen
\end{itemize}
\item time to prepare for conscious cognition task: 10 s
\begin{itemize}
\item display a fake version of application interface or image of document on last save
while real one loads in < 10s
\end{itemize}
\item duration of unbroken attention to a single task: 6 s to 30 s
\end{itemize}

Skelton screens are a minimal version of an interface while the real one loads
(generic layout or minimal version of content).
\begin{itemize}
\item user adjusts to layout
\item loading process seems faster since there is an initial result
\end{itemize}

Progress indicator best practices:
\begin{itemize}
\item show work remaining, not completed
\item show total progress when multiple steps, not only step progress
\item display finished state briefly at the end
\item show smooth progress, not bursts
\item use human precision, not computer precision
\end{itemize}

\textbf{Progressive loading} involves providing user with some data while loading the rest of the data.

\textbf{Predicting next operation} involved using periods of low load to pre-compute resources to high
probability requests, which speeds up subsequent responses.

\textbf{Graceful degradation of feedback} involves simplifying feedback for high-computation tasks.

Goals for handling long tasks in a UI are:
\begin{itemize}
\item to keep the UI responsive
\item provide progress feedback
\item ideally allow long task to be paused or cancelled
\end{itemize}
\section{Worker Thread}
\label{sec:org525c182}
With multi-threading, manage multiple concurrent threads with shared resources, but executing
different instructions.

Threads divide computation and reduce blocking.
Concurrency risks exist, such as 2 threads updating a variable.

Browsers support \textbf{worker threads}: dedicated works, shared works, and service workers.
\section{Fetch API}
\label{sec:orgafdb80e}
The JS runtime environment has a JS Engine, Web API, and message queue.

JS Engine has an execution context stack (ECS), with code to execute next, and a heap, with
function definitions, objects, etc.

Web API has a DOM API for handling DOM events, Fetch API for loading remote resources (cloud API,
server data, etc), and timer functions (\texttt{setTimeout}, \texttt{setInterval}, \texttt{requestAnimationFrame}).

Message queue holds methods for events that have occurred. Methods that generate messages are stored
in the event table.

To execute an asynchronous method from the message queue, it must be moved to the ECS.
An event loop continually checks if the ECS is empty, and if it is, it moves a method from the
message queue to the ECS to the JS Engine can execute it.

Input events are asynchronous methods (callbacks bound to a DOM element).

Fetch API is an interface for fetching resources across the network.
The \texttt{fetch} function starts the process of fetching a resource from the network and returns a
\texttt{Promise} object with 3 possible states:
\begin{itemize}
\item \emph{pending}: when fetch process is happening
\item \emph{resolved}: when process was successful and has a valid response
\item \emph{rejected}: when the process failed and has an error
\end{itemize}

Complex to get progress during fetch.
\end{document}
