% Created 2024-04-19 Fri 02:12
% Intended LaTeX compiler: pdflatex
\documentclass[11pt]{article}
\usepackage[utf8]{inputenc}
\usepackage[T1]{fontenc}
\usepackage{graphicx}
\usepackage{longtable}
\usepackage{wrapfig}
\usepackage{rotating}
\usepackage[normalem]{ulem}
\usepackage{amsmath}
\usepackage{amssymb}
\usepackage{capt-of}
\usepackage{hyperref}
\usepackage{parskip,darkmode}
\enabledarkmode
\author{Arnav Gupta}
\date{\today}
\title{Reactive}
\hypersetup{
 pdfauthor={Arnav Gupta},
 pdftitle={Reactive},
 pdfkeywords={},
 pdfsubject={},
 pdfcreator={Emacs 29.3 (Org mode 9.7)}, 
 pdflang={English}}
\begin{document}

\maketitle
\tableofcontents

\section{Virtual DOM Reconciliation}
\label{sec:orgd87e085}
\textbf{Reactivity} is the automatic update of the UI due to a change in the application's state.

As a developer, focus on the state of the application and let the framework reflect that
state in the UI.

The VDOM is a lightweight representation of the UI in memory.
It is synchronized with the real DOM as follows:
\begin{enumerate}
\item save current VDOM
\item components and/or application state updates the VDOM
\item a re-render is triggered by the framework
\item compare VDOM before update with VDOM after update
\item reconcile the difference, identifying a set of DOM patches
\item perform patch operations on real DOM
\item back to step 1
\end{enumerate}

Reconciliation operations:
\begin{itemize}
\item if node type changes, the whole subtree is rebuilt
\item if node type is the same, attributes are compared and updated
\item if a node is inserted into the list of same node type siblings, all children would be updated
(if no more info provided)
\begin{itemize}
\item key prop should be used when updating children of the same node type, where each key must be
stable and unique, and the key should be assigned when data is created, not rendered
\end{itemize}
\end{itemize}
\section{State with Hooks, Context, Signals}
\label{sec:org966cb39}
Approaches to managing state:
\begin{enumerate}
\item \texttt{useState} hook for local component state, and pass state to children
\item \texttt{useContext} hook to access state, without passing as props
\item Signals
\end{enumerate}

Hook functions are stored in slots associated with each component, and have a slot number
associated.
A function is used to get/create a hook in a slot.

Hooks in Preact are functional methods to compose state and side effects.
\begin{itemize}
\item \texttt{useState} used to get and set state.
\item \texttt{useContext} used to access state context without prop drilling
\item \texttt{useRef} used to get a reference to a DOM node inside a functional component
\item \texttt{useEffect} used to trigger side-effects on state change
\item \texttt{useReducer} used for complex state logic similar to Redux
\end{itemize}

Can also define custom hooks with funtions that return values and callbacks.

With Context, pass state and other values to children without prop drilling:
\begin{enumerate}
\item define context object type
\item create shared context object
\item make context Provider node ancestor of components using context
\item use context in child components
\end{enumerate}

Signals are a state management module introduced by Preact:
\begin{itemize}
\item state acts as a model, and signals and mutations can be exported as well
\end{itemize}

For state with signals:
\begin{itemize}
\item keep state minimal and well-organized, so each signal is focused only on a certain part of
the state
\item use signals only where needed, otherwise performance issues
\begin{itemize}
\item use signals to share state between parts of the application
\item for local app state, use \texttt{useState}
\end{itemize}
\item keep components small and focused, so break UI down into small focused components that each
manage a specific piece of state
\end{itemize}
\end{document}
