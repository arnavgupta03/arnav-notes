% Created 2025-01-17 Fri 00:16
% Intended LaTeX compiler: pdflatex
\documentclass[11pt]{article}
\usepackage[utf8]{inputenc}
\usepackage[T1]{fontenc}
\usepackage{graphicx}
\usepackage{longtable}
\usepackage{wrapfig}
\usepackage{rotating}
\usepackage[normalem]{ulem}
\usepackage{amsmath}
\usepackage{amssymb}
\usepackage{capt-of}
\usepackage{hyperref}
\usepackage{parskip,darkmode}
\enabledarkmode
\author{Arnav Gupta}
\date{\today}
\title{Data Models and Query Languages}
\hypersetup{
 pdfauthor={Arnav Gupta},
 pdftitle={Data Models and Query Languages},
 pdfkeywords={},
 pdfsubject={},
 pdfcreator={Emacs 29.4 (Org mode 9.7.19)}, 
 pdflang={English}}
\begin{document}

\maketitle
\tableofcontents

\section{Introduction}
\label{sec:orgebd648f}
Most applications are built by layering data models, each layer hiding the
complexity of the layers below.

Every data model embodies assumptions about how it will be used.
\section{Relational Model vs Document Model}
\label{sec:orgcf395ab}
Data organized into \textbf{relations}, which are unordered collections of \textbf{tuples}.

Hides implementation detail behind a clean interface.
\subsection{Birth of NoSQL}
\label{sec:orgb44a15e}
Refers to \emph{Not Only SQL}.

NoSQL is adopted due to:
\begin{itemize}
\item need for greater scalability than relational DBs
\item preference for FOSS over commercial DBs
\item specialized query operations not well supported by relational model
\item frustration with restriction of relational schemas
\end{itemize}
\subsection{Object-Relational Mismatch}
\label{sec:orgff679f1}
\textbf{Impedance mismatch}: disconnect between object-oriented model of application
code and relational model of DB

ORM frameworks help hide mismatch.

JSON can reduce impedance mismatch, with better locality (all relevant info
in one place).
This is the \textbf{document} model.
\subsection{Many-to-One and Many-to-Many Relationships}
\label{sec:orge89bd45}
Using standardized lists rather than strings can enforce consistent spelling,
avoid ambiguity, make updates easier, localize support, and allow for better
search.

Using strings causes duplication.
Removing duplication is the idea behind DB \textbf{normalization}.
This is easier to do with the relational model than the document model.
\subsection{Are Document Databases Repeating History?}
\label{sec:org023369d}
\subsubsection{Network Model}
\label{sec:org69ec1ec}
Generalization of the hierarchical model where:
\begin{itemize}
\item each record can have multiple parents
\item links between records are not foreign keys but instead
pointers that can be followed through an \textbf{access path}
\begin{itemize}
\item can have multiple paths to the same record
\end{itemize}
\end{itemize}

Can change access paths, but then must rewrite code to handle
new access paths.
Difficult to change application's data model.
\subsubsection{Relational Model}
\label{sec:orgf8d31bb}
Simpler since no nested structure.

Query optimizer decides which parts of query to execute in which
order, and which indexes are used.
\subsubsection{Comparison to Document Databases}
\label{sec:org802c479}
Document DBs are similar to hierarchical DBs, except for many-to-one
and many-to-many relationships where they are like relational DBs.
\subsection{Relational vs Document Databases Today}
\label{sec:orged2a7cf}
Document data model benefits:
\begin{itemize}
\item schema flexibility
\item better performance due to locality
\item closer to data structures used by the application
\end{itemize}

Relational data model benefits:
\begin{itemize}
\item better support for joins, many-to-one, and many-to-many
relationships
\end{itemize}
\subsubsection{Simpler Application Code}
\label{sec:org8791fd8}
If data is a tree of one-to-many relationships, document
model is better.

If application uses many-to-many relationships, relational mode
is better.
\subsubsection{Schema Flexibility}
\label{sec:org4f1ce9e}
Document DBs do not enforce any schema on the data in documents.

Document DBs have \textbf{schema-on-read} (structure interpreted when data
is read).

Relational DBs have \textbf{schema-on-write} (DB ensures all written data
conforms to schema).

Schema-on-read can be helpful if items in the collection don't all have
the same structure.
\subsubsection{Data Locality}
\label{sec:org749c650}
Documents are usually stored as a string, providing \textbf{storage locality}.

Locality helps if large parts of document are needed at once.

Writes replace the entire document.

Some relational DBs allow for locality.
\subsubsection{Convergence of Document and Relational DBs}
\label{sec:org1e4c5f6}
Most relational DBs support XML, and some support JSON.

Some document DBs support relational-like joins.
\section{Query Languages for Data}
\label{sec:org78c4843}
\textbf{Imperative} languages specify operations to perform.

\textbf{Declarative} languages specify only the pattern of the data
you want.

Declarative languages allow any implementations and lend
themselves to parallel execution.
\subsection{Declarative Queries on the Web}
\label{sec:org3a1aa67}
HTML and CSS are declarative languages used on the web.
\subsection{MapReduce Querying}
\label{sec:org996f17e}
Programming model for processing large amounts of data in
bulk across machines.

Neither fully declarative or imperative.

Steps in MapReduce:
\begin{enumerate}
\item Filter is specified declaratively.
\item \texttt{map} is called once for every document that matches
the query, where \texttt{this} is the document object
\item \texttt{map} emits a key and a value
\item Key-value pairs emitted by \texttt{map} are grouped by the key,
and for all key-value pairs with the same key, \texttt{reduce}
is called once
\item \texttt{reduce} adds up the number of animals from all observations
in a particular month
\item Final output is written to collection specified.
\end{enumerate}

\texttt{map} and \texttt{reduce} only use the data passed as input, cannot
perform additional DB queries, and cannot have any side effects.
\section{Graph-Like Data Models}
\label{sec:org40e0e5a}
Graphs can be used as a data model, especially for many-to-many
relations.
\subsection{Property Graphs}
\label{sec:orga1a4e84}
Each vertex consists of:
\begin{itemize}
\item unique identifier
\item set of outgoing edges
\item set of incoming edges
\item collection of properties (key-value pairs)
\end{itemize}

Each edge consists of:
\begin{itemize}
\item unique identifier
\item \textbf{tail vertex}: vertex at which edge starts
\item \textbf{head vertex}: vertex at which edge ends
\item label to describe relationship between two vertices
\item collection of properties (key-value pairs)
\end{itemize}

Any vertex can have an edge connecting it with any other
vertex. No schema restricts which kinds of things can or cannot
be associated.

Given any vertex, can traverse the graph through incoming and
outgoing edges.

By using different labels for different kinds of relationships,
store several kinds of info in a single graph while still
maintaining a clean data model.

Graphs are good for evolvability: as features are added to an
application, a graph can easily be extended to accommodate
changes in the application's data structures.
\subsection{The Cypher Query Language}
\label{sec:org38f1fc2}
Declarative query language for property graphs.

Query optimizer automatically chooses most efficient strategy.
\subsection{Graph Queries in SQL}
\label{sec:org3e0cb55}
Graph data in a relational structure can be queried with SQL,
though with some difficulty.

Variable-length traversal paths in a query can be expressed
using \textbf{recursive common table expressions}.
\subsection{Triple-Stores and SPARQL}
\label{sec:orge7a13e9}
In a \textbf{triple-store}, all info is stored in the form of simple
3-part statements: (subject, predicate, object).

Subject of the triple is a vertex.
Object can be a value or a vertex.
Predicate is an edge.
\subsubsection{Semantic Web}
\label{sec:orga4d2ab9}
Idea that websites should publish info as machine-readable
data for computers to read.
\subsubsection{RDF Data Model}
\label{sec:orgf36b584}
Used for internet-wide data exchange.

Subject, object, and predicate are URIs.
\subsubsection{SPARQL Query Language}
\label{sec:org395a957}
Query language for triple-stores using RDF data model.
\subsection{The Foundation: Datalog}
\label{sec:org1e9e96f}
Older language used in few data systems.

Define rules that tell that DB about new predicates.

Predicates aren't triples stored in the DB, but are
derived from data or rules.

A rule applies if the system can find a match for all
predicates on the right hand side of the \texttt{:-} operator.
\end{document}
