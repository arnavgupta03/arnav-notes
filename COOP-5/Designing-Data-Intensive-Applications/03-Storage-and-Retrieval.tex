% Created 2025-01-27 Mon 00:32
% Intended LaTeX compiler: pdflatex
\documentclass[11pt]{article}
\usepackage[utf8]{inputenc}
\usepackage[T1]{fontenc}
\usepackage{graphicx}
\usepackage{longtable}
\usepackage{wrapfig}
\usepackage{rotating}
\usepackage[normalem]{ulem}
\usepackage{amsmath}
\usepackage{amssymb}
\usepackage{capt-of}
\usepackage{hyperref}
\usepackage{parskip,darkmode}
\enabledarkmode
\author{Arnav Gupta}
\date{\today}
\title{Storage And Retrieval}
\hypersetup{
 pdfauthor={Arnav Gupta},
 pdftitle={Storage And Retrieval},
 pdfkeywords={},
 pdfsubject={},
 pdfcreator={Emacs 29.4 (Org mode 9.7.19)}, 
 pdflang={English}}
\begin{document}

\maketitle
\tableofcontents

\section{Introduction}
\label{sec:orgb5fc854}
Knowing how a DB handles storage helps when choosing an appropriate
storage engine.
\section{Data Structures that Power your Database}
\label{sec:org95804f7}
\textbf{Log}: an append-only sequence of records

To efficiently find the value for a particular key, an \textbf{index} can be
helpful.
Indexes are derived from the primary data.

Indexes are additional to the primary data, which has overhead.
Indexes speed up reads but slow down writes.

Application developers should choose indexes manually based on the
application's typical query patterns.
\subsection{Hash Indexes}
\label{sec:org5727b43}
Key-value stores are similar to dictionaries, which are usually
implemented as hash maps.

Can index a key-value store using a hash map to store the
byte offset in the data file where the value can be found.

On writes, write to the hash map as well (based on where data
was inserted).

On reads, read from the hash map to get the offset where the record is.

To avoid running out of space, break log into segments and close a segment
file when it reaches a certain size.
Perform \textbf{compaction} on segments, throwing away duplicate keys in the log
and merging segments if possible.
Since segments are never modified after writing, merged segments are
written to a new file.

Each segment has an in-memory hash table, mapping keys to file offsets.
Hash maps are checked in order of recency.

Having logs be append-only helps with speed, concurrency, crash recovery,
and file fragmenting.

Hash table must fit in memory, so cannot have to many keys.
Also, range queries are not efficient (lookups are independent).
\subsection{SSTables and LSM-Trees}
\label{sec:orgbf26492}
In an \textbf{SSTable}, sort the key-value pairs in each segment by key.

This allows simpler merging, removes the need to keep an index for all
keys in memory, and allows the use of blocks of key-value pairs for
compression.
\subsubsection{Constructing and Maintaining SSTables}
\label{sec:org0c31725}
Can use a tree data structure (like red-black tree or AVL tree) so
maintain sorted order.
This structure is maintained in memory, in a \textbf{memtable}.

When the memtable reaches some size threshold, it is written to
disk, becoming the most recent segment.

Only issue is with crashes (since memtable in-memory).
Can solve with a log on disk to which every write is appended,
used to restore the memtable after a crash.
\subsubsection{Making an LSM-Tree out of SSTables}
\label{sec:orgad03099}
This indexing structure is called an \textbf{LSM-Tree}.
\subsubsection{Performance Optimizations}
\label{sec:orgaaef16d}
Use \textbf{Bloom filters} to optimize checking if keys do not exist in
the DB.

Types of compaction include:
\begin{itemize}
\item \textbf{size-tiered}: newer SSTables successively merged into older and
larger SSTables
\item \textbf{leveled}: key range split up into smaller SSTables and older
data moved into separate levels
\end{itemize}

LSM-Tree works well even when data is bigger than available memory.
Also better at range queries since data is sorted.
\subsection{B-Trees}
\label{sec:org0906a2a}
Most widely used indexing structure.

DB broken down into fixed-size blocks/pages.
Design corresponds to underlying hardware.

Root of B-Tree is a page and contains keys and references to child pages.
Keys are a continuous range, indicating where boundaries between ranges lie.

A leaf page contains the values or references to the values.

\textbf{Branching factor}: number of references to child pages in 1 page of the B-tree

If not enough free space in a page to accommodate a new key, it is split
into two half-full pages.
This ensures the tree remains \textbf{balanced}.

A B-Tree with \(n\) keys always has depth of \(O(\log(n))\).
\subsubsection{Making B-Trees Reliable}
\label{sec:org4e46680}
B-Tree write overwrites a page on disk with new data,
but overwrite does not change the location of the page.

To make the DB resilient to crashes, a \textbf{write-ahead log}
is used, which is an append only file to which
every B-tree modification must be written before it can be
applied to the pages of the tree itself.

For concurrency, \textbf{latches} are used to protect the tree's
data structures.
\subsubsection{B-Tree Optimizations}
\label{sec:orgad39b2f}
Some DBs use a copy-on-write scheme where a modified page is rewritten
to a different location and pointers are updated.

The key can be abbreviated to save space.

Some B-tree implementations place leaf pages in sequential
order on disk.

Some B-trees have sibling pointers.

\textbf{Fractal trees} reduce disk seeks.
\subsection{Comparing B-Trees and LSM-Trees}
\label{sec:org951d922}
LSM-trees are faster for writes, B-trees are faster for reads.
\subsubsection{Advantages of LSM-Trees}
\label{sec:orgd8d7c6d}
Less writes required due to \textbf{write amplification}: one write
to the DB results in multiple writes to the disk.

Sequentially write compact SSTable files rather than overwriting
several pages in the tree.

Better compression and less fragmentation due to compaction.
\subsubsection{Downsides of LSM-Trees}
\label{sec:org118ee98}
Compaction can interfere with the performance of ongoing reads
and writes.

Write bandwidth must be shared between the initial write and
compaction threads running in the background.

It is possible that compaction cannot keep up with the rate of
incoming writes, so there are too many segment files.

B-trees are better for transactions.
\subsection{Other Indexing Structures}
\label{sec:org0263037}
Key-value relies on \textbf{primary key}.
\textbf{Secondary indexes} are also possible.

A secondary index can be constructed from a key-value index, but
keys are not unique.
\subsubsection{Storing Values within the Index}
\label{sec:org6763982}
\textbf{Heap file}: where rows are stored, in no particular order
and no duplicated data

Indexes just reference a location in the heap file.

Records can be overwritten in place in the heap file.

\textbf{Clustered index}: store indexed row directly within an
index

\textbf{Covering index}: stores some of a table's columns within the
index

Clustered and covering indexes speed up reads but add overhead
on writes.
\subsubsection{Multi-Column Indexes}
\label{sec:org07f6351}
\textbf{Concatenated index}: combine several fields into one key by
appending one column to another

B-tree and LSM-tree indexes don't answer multi-dimensional queries
efficiently.
R-trees are better for these.
\subsubsection{Full-Text Search and Fuzzy Indexes}
\label{sec:orge5fc2de}
\textbf{Fuzzy querying}: searching for similar keys

Can use a SSTable-like structure or use a trie.

\textbf{Levenshtein automaton}: supports efficient search for words
within a given edit distance
\subsubsection{Keeping Everything in Memory}
\label{sec:org3e387f1}
As RAM gets cheaper, this has led to \textbf{in-memory databases}.

Some used for caching, some more durable.

Writing to disk is still done for durability, but data is loaded
from memory.

In-memory databases are faster because they avoid overheads
of encoding in-memory data structures in a form that can be
written to disk.

\textbf{Anti-caching} allows datasets larger than available memory using a
scheme similar to virtual memory.
\section{Transaction Processing vs Analytics}
\label{sec:orga71622a}
\textbf{Transaction processing}: low latency reads and writes

Data analytics have different access patterns than transactions.

\textbf{Data warehouse}: separate database to run analytics on
\subsection{Data Warehousing}
\label{sec:orgd035722}
Using a data warehouse does not affect OLTP operations
since it is read-only and can be processed through ETL
transformations.

Data warehouse can be optimized for analytic access patterns.
\subsubsection{Divergence Between OLTP DBs and Data Warehouses}
\label{sec:orgef5e370}
OLTP and data warehouse have SQL query interface, but internally
are optimized for different query patterns.
\subsection{Stars and Snowflakes: Schemas for Analytics}
\label{sec:orge6435da}
\textbf{Star schema}: has fact table that represents events at times and
dimension tables that describe events

\textbf{Snowflake schema}: variant of star schema where dimensions are
further broken down into subdimensions (more normalized)
\section{Column-Oriented Storage}
\label{sec:org73b76a8}
Typical data warehouse query only accesses a small fraction of
columns at a time.

With \textbf{column-oriented storage}, all values from a column are stored
together rather than all values from a row.
This can be helpful for fact tables where indexes only exist on a
small number of columns.

Relies on each column file containing rows in the same order.
\subsection{Column Compression}
\label{sec:org1e43554}
Can compress column representation with \textbf{bitmap encoding}, where each distinct
value has a bitmap.
This is well suited for the queries common in data warehouses.
\subsubsection{Memory Bandwidth and Vectorized Processing}
\label{sec:org9460418}
Bottlenecks include getting data from disk into memory, efficiently using
bandwidth from main memory into CPU cache, avoiding branch mispredictions, and
making use of SIMD instructions.

With column compression, more rows from a column fit in the same amount of L1
cache.

\textbf{Vectorized processing}: operators designed to operate on such chunks of
compressed column data directly
\subsection{Sort Order in Column Storage}
\label{sec:orgb35416d}
Order doesn't necessarily matter.

DB admin can choose the columns by which a table should be sorted.

Sorted order can help with column compression, since compression
is strongest on first key.
\subsubsection{Several Different Sort Orders}
\label{sec:org2f92e19}
Can store redundant data in different ways to use the version
best fit to a query.
\subsection{Writing to Column-Oriented Storage}
\label{sec:orgb6df48a}
Writes are more difficult in column-oriented storage.

LSM-trees can be used (with compaction) since update in-place
will not work.

Queries need to examine column data on disk and recent writes in memory.
\subsection{Aggregation: Data Cubes and Materialized Views}
\label{sec:org8e2b263}
\textbf{Materialized view}: copy of query results, written to disk

When underlying data changes, a materialized view needs to be updated.

\textbf{Data cube}: grid of aggregates grouped by different dimensions

With a materialized data cube, certain queries become very fast since
they have been precomputed.

Data cube doesn't have the same flexibility as querying raw data.
\end{document}
