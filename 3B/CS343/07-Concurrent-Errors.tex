% Created 2024-12-08 Sun 12:41
% Intended LaTeX compiler: pdflatex
\documentclass[11pt]{article}
\usepackage[utf8]{inputenc}
\usepackage[T1]{fontenc}
\usepackage{graphicx}
\usepackage{longtable}
\usepackage{wrapfig}
\usepackage{rotating}
\usepackage[normalem]{ulem}
\usepackage{amsmath}
\usepackage{amssymb}
\usepackage{capt-of}
\usepackage{hyperref}
\usepackage{parskip,darkmode}
\enabledarkmode
\author{Arnav Gupta}
\date{\today}
\title{Concurrent Errors}
\hypersetup{
 pdfauthor={Arnav Gupta},
 pdftitle={Concurrent Errors},
 pdfkeywords={},
 pdfsubject={},
 pdfcreator={Emacs 29.4 (Org mode 9.7.11)}, 
 pdflang={English}}
\begin{document}

\maketitle
\tableofcontents

\section{Race Condition}
\label{sec:org1addb70}
\textbf{Race condition} occurs when missing \uline{synchronization} or \uline{mutual exclusion}.

Happens when 2+ tasks race along assuming synchronization or mutual exclusion has occurred.

Can be very difficult to locate.
\section{No Progress}
\label{sec:orgc2e5e05}
\subsection{Live-Lock}
\label{sec:org8a3eda5}
Indefinite postponement, ``You go first'' problem on simultaneous arrival.

Caused by poor scheduling in entry position.

Can always break time on simultaneous arrival by some mechanism that deals effectively with
live-lock.
\subsection{Starvation}
\label{sec:org490fb36}
Selection algorithm ignores 1+ tasks so they are never executed (lack of long-term fairness).

Long-term (infinite) starvation extremely rare, but short-term starvation can occur and is a problem.

Like live-lock, starving task might be ready any time, switching among active, ready, and possibly
blocked states (consuming CPU).
\subsection{Deadlock}
\label{sec:org8567ccf}
State when 1+ processes wait for an event that will not occur.

Unlike live-lock/starvation, deadlocked task is blocked so not consuming CPU.
\subsubsection{Synchronization Deadlock}
\label{sec:orgb9192fd}
Failure in cooperation, so blocked task never unblocked (stuck waiting).

\begin{verbatim}
int main() {
  uSemaphore s(0);  // closed
  s.P();            // wait for lock to open
}
\end{verbatim}
\subsubsection{Mutual Exclusion Deadlock}
\label{sec:org5f187e9}
Failure to acquire resource protected by mutual exclusion.

5 conditions that must occur for a set of processes to deadlock:
\begin{enumerate}
\item \uline{concrete} shared resource requiring mutual exclusion
\item process holding a resource while waiting for access to a resource held by another process
(hold and wait)
\item once a process has gained access to a resource, runtime system cannot get it back (no
preemption)
\item exists a circular wait of processes on resources
\item conditions must occur simultaneously
\end{enumerate}
\section{Deadlock Prevention}
\label{sec:org23cc602}
Eliminate 1+ conditions required for deadlock from an algorithm so deadlock can never occur.
\subsection{Synchronization Prevention}
\label{sec:orgbbb4b85}
Eliminate all synchronization from a program:
\begin{itemize}
\item no communication
\item impossible in most cases
\end{itemize}
\subsection{Mutual Exclusion Prevention}
\label{sec:org84d36c6}
Prevent deadlock by eliminating 1 of 5 conditions:
\begin{enumerate}
\item no mutual exclusion
\begin{enumerate}
\item no shared resources, so impossible in most cases
\end{enumerate}
\item no hold and wait: do not given any resource unless all resources can be given
\begin{enumerate}
\item poor resource utilization and possible starvation
\end{enumerate}
\item allow preemption
\begin{enumerate}
\item preemption is dynamic, so cannot apply statically
\end{enumerate}
\item no circular wait: by controlling order of resource allocations
\begin{enumerate}
\item use an \textbf{ordered resource} policy:
\begin{enumerate}
\item divide all resources into classes \(R_{1}, R_{2}, \dots\)
\item \uline{rule}: can only request a resource from class \(R_{i}\) if holding no resources from any
class \(R_{j}\) for \(j \ge i\)
\item unless each class contains only 1 resource, requires requesting several resources
simultaneously
\item denote highest class number for which \(T\) holds a resource by \(h(T)\)
\item if process \(T_{1}\) is requesting a resource of class \(k\) and is blocked because that
resource is held by process \(T_{2}\), then \(h(T_{1}) < k \le h(T_{2})\)
\item as preceding inequality is strict, circular wait is impossible
\item in some cases, there is natural division of resources into classes that makes
policy work nicely
\item in other cases, some processes are forced to acquire resources in an unnatural sequence,
complicating code and producing poor resource utilization
\end{enumerate}
\end{enumerate}
\item prevent simultaneous occurrence
\begin{enumerate}
\item show previous 4 rules cannot occur simultaneously
\end{enumerate}
\end{enumerate}
\section{Deadlock Avoidance}
\label{sec:org2ed5b82}
Monitor all lock blocking and resource allocation to detect and potential formation of deadlock.

Achieve better resource utilization, but additional overhead to avoid deadlock.
\subsection{Banker's Algorithm}
\label{sec:org449f154}
Demonstrate safe sequence of resource allocations that give no deadlock.
Requires a process state its maximum resource needs.

Check for safe order of execution that avoids deadlock should each process require maximum
resource allocation.

If safe order exists, Banker's algorithm allows resource request.

If there is a choice of processes to choose for execution, it does not matter which path is
taken.

Check for safe order can be performed for every allocation of a resource to a process
(optimizations possible, like same thread asking for another resource).
\subsection{Allocation Graphs}
\label{sec:org589629a}
One method to check for potential deadlock is to graph processes and resource usage at each
moment a resource is allocated.

Multiple instances are put into a resource so that a specific resource does not have to be requested.
Instead, a generic request is made.

If graph contain no cycles, no process in the system is deadlocked.

If any resource has several instances, a cycle does not mean deadlock.

Can also create \textbf{isomorphic graph} without multiple instances (expensive and difficult).

If each resource has one instance, a cycle means a deadlock.

Can use graph reduction to locate deadlocks.

Problems with allocation graphs:
\begin{itemize}
\item when choices for tasks, selection is tricky (like isomorphic graph)
\item for large graphs, detecting cycles is expensive
\item many graphs to examine over time, one for each particular allocation state of the system
\end{itemize}
\section{Detection and Recovery}
\label{sec:org5a6f8eb}
Instead of avoiding deadlock, let it happen and recover.
This requires ability to discover deadlock and preemption.

Discovering deadlock is difficult, since must build and check for cycles in allocation graph.
Not done on each resource allocation, but every interval of seconds or every time a resource cannot
be immediately allocated.

Recovery involves preemption of 1+ processes in a cycle:
\begin{itemize}
\item decision not easy and must prevent starvation
\item preemption victim must be restarted, from beginning or from previous checkpoint, if no guarantee that
all resources have not changed
\item still might not be enough, since victim may have made changes before preemption
\end{itemize}
\section{Method to Choose}
\label{sec:orge825e17}
Might be best to ignore the problem:
\begin{itemize}
\item if some process is blocked for a long time, assume it is deadlocked and abort it
\item do this automatically in transaction processing system, manually elsewhere
\end{itemize}

Of techniques studied, only ordered resource policy has much practical value.
\end{document}
