% Created 2024-12-08 Sun 14:49
% Intended LaTeX compiler: pdflatex
\documentclass[11pt]{article}
\usepackage[utf8]{inputenc}
\usepackage[T1]{fontenc}
\usepackage{graphicx}
\usepackage{longtable}
\usepackage{wrapfig}
\usepackage{rotating}
\usepackage[normalem]{ulem}
\usepackage{amsmath}
\usepackage{amssymb}
\usepackage{capt-of}
\usepackage{hyperref}
\usepackage{parskip,darkmode}
\enabledarkmode
\author{Arnav Gupta}
\date{\today}
\title{Optimization}
\hypersetup{
 pdfauthor={Arnav Gupta},
 pdftitle={Optimization},
 pdfkeywords={},
 pdfsubject={},
 pdfcreator={Emacs 29.4 (Org mode 9.7.11)}, 
 pdflang={English}}
\begin{document}

\maketitle
\tableofcontents

\section{Optimization}
\label{sec:orgcd87bd1}
Computer with infinite memory and speed requires no optimizations to use less memory or run faster
(space and time).

With finite resources, optimization is useful/necessary to conserve resources and for good performance.
Most programs not written in optimal order or in minimal form.

General forms of optimizations:
\begin{itemize}
\item \textbf{reordering}: data and code are reordered to increase performance in certain contexts
\item \textbf{eliding}: removal of unnecessary data, data accesses, and computation
\item \textbf{replication}: processors, memory, data, and code are duplicated because of limitations in
processing and communication speed (speed of light)
\end{itemize}

Optimized program must be isomorphic to original (same result for fixed input).

Kinds of optimizations are restricted by execution environment.
\section{Sequential Optimizations}
\label{sec:org4736795}
Most programs are sequential, even concurrent programs are:
\begin{itemize}
\item large, sections of sequential code per thread connected by
\begin{itemize}
\item small sections of concurrent code where threads interact (protected by synchronization and mutual exclusion --- SME)
\end{itemize}
\end{itemize}

Sequential execution presents simple semantics for optimization, where operations occur in program
order (sequentially).

Dependencies result in partial ordering among a set of statements (precedence graph):
\begin{itemize}
\item \textbf{data dependency}: reordering reads and writes
\item \textbf{control dependency}: statements cannot be reordered if the earlier line determines if the later
line is executed
\end{itemize}

To achieve better performance, compiler/hardware make changes:
\begin{enumerate}
\item reorder disjoint (independent) operations (variables have different addresses)
\item elide unnecesary operations (transformation/dead code)
\begin{enumerate}
\item remove immediate change, no loop body, tail recursion, etc
\end{enumerate}
\item execute in parallel if multiple functional units (adders, floating units, pipelines, cache)
\end{enumerate}

Very complex reordering, reducing, and overlapping of operations allowed.

Overlapping implies micro-parallelism, but limited capacity in sequential execution.
\section{Memory Hierarchy}
\label{sec:org05be5c9}
Memory hierarchy includes registers (on CPU), memory, and possibly cache, where replication
must occur across all layers.

Optimizing data flow along hierarchy defines a computer's speed.

Hardware aggressively optimizes data flow for sequential execution.

Having basic understanding of cache is essential to understanding performance of both sequential
and concurrent programs.
\subsection{Cache}
\label{sec:org6e71138}
Since CPU faster than memory, copy data from memory into registers.
But not many registers, so move highly accessed data within a program from memory to registers
for as long as possible and then back to memory.

Can quickly run out of registers as more data accessed, so must rotate data from memory through
registers dynamically (compiler attempts ot keep highly used variables in regsters).

This does not handle highly accessed data \uline{among} programs (threads), since each context switch
saves and restores most registers to memory (registers are private and cannot be shared).
Solution is to use hardware cache (automatic registers) to stage data without pushing to memory
and allow sharing of data among programs.

Caching transparently hides latency of accessing main memory.
Cache loads in 64/128/256 bytes, called \textbf{cache line}, with addresses multiple of line size.

When cache full, data evicted (remove old cache lines to bring in new based on LRU).
When program ends, addresses are flushed from memory hierarchy.
\subsection{Cache Coherence}
\label{sec:orgb3d3f58}
Multi-level caches used, each larger but with diminishing speed and cost.

Data reads logically percolate variables from memory up memory hierarchy, making cache copies to registers.

Necessary to eagerly move reads up memory hierarchy to take advantage of temporal locality.

Data writes from registers to variables logically percolate down memory hierarchy though
cache copies to memory.

Advantageous to lazily move writes down memory hierarchy since that way later reads see this.

If OS moves program to another process, all caching info is invalid and the program's data
hierarchy reforms.

Unlike registers, all cache values are shared across the computer.
Hence, a variable can be replicated in a large number of locations.

Without cache coherence for shared variable, there is madness.
With cache coherence for shared variable (snooping or directory-based), variables are updated
everywhere.

\textbf{Cache coherence}: hardware protocol ensuring update of duplicate data

\textbf{Cache consistency}: addresses when processor sees update (bidirectional synchronization)

Prevent flickering and scrambling during simultaneous R/W or W/W with update and acknowledge.

\uline{Eager} cache-consistency means data changes appear instantaneous by waiting for acknowledge from
all cores (complex/expensive).

\uline{Lazy} cache-consistency allows reader to see own write before acknowledgment (concurrent programs
read stale data):
\begin{itemize}
\item writes eventually appear in (largely) same order as written
\item critical section works as writes to shared variable appear before write to lock release (so
other threads see write to lock after write to shared variable)
\item otherwise, spin (lock) until write appears
\end{itemize}

\textbf{Cache thrashing}: if threads continually read/write same memory locations, they invalidate
duplicate cache lines, resulting in excessive cache updates (updated value bounces from one
cache to the next)

\textbf{False sharing}: since cache lines contain multiple variables, cache thrashing can occur inadvertently

Fix false sharing by separating variables on same line with sufficient storage (padding) to be
in next cache line.
Difficult for dynamically allocated variables as memory allocator positions storage.
\section{Concurrent optimizations}
\label{sec:orgb41b982}
In sequential execution, \textbf{strong memory ordering}: reading always returns last value written

In concurrent execution, \textbf{weak memory ordering}: reading can return previously written value or
value written in future:
\begin{itemize}
\item happens on multi-processor because of scheduling and buffering
\item notion of current value becomes blurred for shared variables unless everyone can see values
assigned simultaneously
\end{itemize}

SME control order and speed of execution, otherwise non-determinism causes random results or failure.

Sequential sections accessing private variables can be optimized normally, but not across concurrent
boundaries.

Concurrent sections accessing shared variables can be corrupted by sequential optimizations, so restrict
optimizations to ensure correctness.

For correctness and performance, identify concurrent code and only restrict its optimization.
How to restrict depends on what sequential assumptions are implicitly applied by hardware and compiler.
\subsection{Disjoint Reordering}
\label{sec:orgfb0fda4}
\(R_{x} \to R_{y}\) allows \(R_{y} \to R_{x}\), so reordering disjoint reads does not cause problems.

\(W_{x} \to R_{y}\) allows \(R_{y} \to W_{x}\) can cause problems.

\(R_{x} \to W_{y}\) allows \(W_{y} \to R_{x}\) can cause problems.

\(W_{x} \to W_{y}\) allows \(W_{y} \to W_{x}\) can cause problems.

Compiler uses all these reorderings to break mutual exclusion:
\begin{itemize}
\item moves lock entry/exit after/before critical section because entry/exit variables not used
in critical section
\item double-check locking for singleton-pattern can help
\begin{itemize}
\item both checks check for change in storage at different parts of code (inside and outside lock)
\end{itemize}
\item can fail if writes and reads related to malloc are reordered
\end{itemize}
\subsection{Eliding}
\label{sec:org911c0c0}
For high-level language, compiler decides when/which variables are loaded into registers and for how
long.

Elide reads (loads) by copying (replicating) value into a register.
So, variable logically disappears for duration in register (can cause livelock).

Elide meaningless sequential code: can cause task to miss signal by not delaying.
\subsection{Replication}
\label{sec:org6774570}
Benefit to reorder R/W since faster instructions can then happen at once.

Modern processors increase performance by executing multiple instructions in parallel (data flow,
precedence graph) on replicated hardware:
\begin{itemize}
\item internal pool of instructions taken from program order
\item begin simultaneous execution of instructions with inputs
\item collect results from finished instructions
\item feed results back into instruction pool as inputs
\item so, instructions with independent inputs execute out-of-order
\end{itemize}

From sequential perspective, disjoint reordering is unimportant, so hardware starts many instructions
simultaneously.

From concurrent perspective, disjoint reordering is important.
\section{Memory Model}
\label{sec:org2a0b9ff}
Manufacturers define set of optimizations performed implicitly by processor.

\textbf{Memory model} is defined by set of optimizations.

Atomic consistent (AT) has events occur instantaneously, so slow or impossible to optimize.

Sequential consistency (SC) accepts all events cannot occur instantaneously, so may read old values.
Still strong enough for software mutual exclusion (often considered minimum model for concurrency).

No hardware supports just AT/SC.
\section{Preventing Optimization Problems}
\label{sec:org34a2679}
All optimization problems result from races on shared variables.

If shared data is protected by locks (implicit or explicit):
\begin{itemize}
\item locks define sequential/concurrent boundaries
\item boundaries must preclude optimizations that affect concurrency
\end{itemize}

\textbf{Race free}: synchronization and mutual exclusion preclude races

Race free does have races, since races are internal to locks, which lock programmer must deal with.

Two approaches:
\begin{itemize}
\item \uline{ad hoc}: programmer manually augments all data races with pragmas to restrict compiler/hardware
optimizations (not portable but often optimal)
\item \uline{formal}: language has memory model and mechanisms to abstractly define races in program
(portable but often baroque and suboptimal)
\end{itemize}

Data access/compiler: \texttt{volatile} qualifier
\begin{itemize}
\item force variable loads and stores to/from registers (at sequence points)
\item created for \texttt{longjmp} or force access for memory-mapped devices
\item for architectures with few registers, all variables are pretty much implicitly volatile
\item Java \texttt{volatile} and C++11 \texttt{atomic} are stronger, since they prevent eliding and disjoint reordering
\begin{itemize}
\item C++11 \texttt{atomic} automatically fences shared variables, but can be suboptimal
\end{itemize}
\end{itemize}


Program order/compiler (static): disable inlining or use \texttt{asm("" ::: "memory");}

Memory order/runtime (dynamic): \texttt{sfence}, \texttt{lfence}, \texttt{mfence}, which guarantee previous stores
and/or leads are completed before continuing.

Atomic operations test-and-set often imply fencing.

Cache is normally invisible and does not cause issues (except for dynamic memory allocation).

Mechanisms to fix issues are specific to compiler or platform: difficult, low-level, diverse semantics,
and not portable.

Locks built with these features ensure SC for protected shared variables: no user races and strong
locks, so SC memory model.
\end{document}
