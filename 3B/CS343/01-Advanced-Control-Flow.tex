% Created 2024-09-04 Wed 17:57
% Intended LaTeX compiler: pdflatex
\documentclass[11pt]{article}
\usepackage[utf8]{inputenc}
\usepackage[T1]{fontenc}
\usepackage{graphicx}
\usepackage{longtable}
\usepackage{wrapfig}
\usepackage{rotating}
\usepackage[normalem]{ulem}
\usepackage{amsmath}
\usepackage{amssymb}
\usepackage{capt-of}
\usepackage{hyperref}
\usepackage{parskip,darkmode}
\enabledarkmode
\author{Arnav Gupta}
\date{\today}
\title{Advanced Control Flow}
\hypersetup{
 pdfauthor={Arnav Gupta},
 pdftitle={Advanced Control Flow},
 pdfkeywords={},
 pdfsubject={},
 pdfcreator={Emacs 29.4 (Org mode 9.8)}, 
 pdflang={English}}
\begin{document}

\maketitle
\tableofcontents

\section{Basics of Advanced Control Flow}
\label{sec:org38e0b80}
Within a routine, basic and advanced control structures allow any control flow.

\texttt{while} and \texttt{for} are interchangeable for predicate only loops. When using loop indexes, use \texttt{for} only.

\textbf{Multi-exit loops} have 1+ exit locations occurring within the body of the loop (such as \texttt{break}).
With multi-exit loops, exit conditions are reversed from \texttt{while} and outdented. These eliminate priming
(duplicated) code.

Multi-exit loops should not be used to simulate \texttt{while} or \texttt{for} and loop exits never need an \texttt{else} clause.
Multi-exit loops allow multiple exit conditions.

Flag variables that only affect control flow should not be used, since they are equivalent to a \texttt{goto}.
\subsection{Static Multi-Level Exit}
\label{sec:org17a7371}
\textbf{Static multi-level exit} exits multiple control structures where the exit point is known at compile time, done with labeled exit (\texttt{break}, \texttt{continue}).

Labeled exits helps eliminate all flag variables and can help remove duplicated code.

Normal and labeled \texttt{break} are \texttt{goto} with limitations:
\begin{itemize}
\item cannot loop (only forward branch) so only loop constructs branch back
\item cannot branch into a control structure
\end{itemize}

Only use \texttt{goto} to perform static multi-level exit (simulate labeled \texttt{break} and \texttt{continue}).
\subsection{Dynamic Memory Allocation}
\label{sec:orgd5a32ed}
Stack allocation preferred over heap allocation (eliminates storage-management and more efficient).

Situations where dynamic heap allocation is necessary:
\begin{enumerate}
\item when storage must outlive the block in which it is allocated (ownership change)
\item when the amount of data read is unknown
\item when an array of objects must be initialized via the object's constructor and each element has a
different value
\begin{enumerate}
\item use stack with \texttt{uArray} if possible (allocates without constructing and proves subscript checking)
\item calls to \texttt{()} or \texttt{make\_unique<T>()} initialize array elements
\item allocation is \(O(1)\), so better than unique pointer and vector
\end{enumerate}
\item when large local variables are allocated on a small stack
\begin{enumerate}
\item \texttt{uArrayPtr} dynamically allocates the array in heap and implicitly frees it at the end of the block
(like \texttt{unique\_ptr})
\item alternatives are large stacks (waste virtual space) or dynamic stack growth (complex and pauses)
\end{enumerate}
\end{enumerate}
\end{document}
