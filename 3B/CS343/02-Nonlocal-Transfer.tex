% Created 2024-09-05 Thu 00:19
% Intended LaTeX compiler: pdflatex
\documentclass[11pt]{article}
\usepackage[utf8]{inputenc}
\usepackage[T1]{fontenc}
\usepackage{graphicx}
\usepackage{longtable}
\usepackage{wrapfig}
\usepackage{rotating}
\usepackage[normalem]{ulem}
\usepackage{amsmath}
\usepackage{amssymb}
\usepackage{capt-of}
\usepackage{hyperref}
\usepackage{parskip,darkmode}
\enabledarkmode
\author{Arnav Gupta}
\date{\today}
\title{Nonlocal Transfer}
\hypersetup{
 pdfauthor={Arnav Gupta},
 pdftitle={Nonlocal Transfer},
 pdfkeywords={},
 pdfsubject={},
 pdfcreator={Emacs 29.4 (Org mode 9.8)}, 
 pdflang={English}}
\begin{document}

\maketitle
\tableofcontents

\section{Nonlocal Transfer}
\label{sec:org1b4ac61}
Routine \textbf{activation} (call/invocation) introduces complex control flow.
Among routines, control flow is controlled by call/return mechanism.

\textbf{Modularization}: from software engineering, any contiguous code block can be factored into a helper
routine and called in the program

Modularization fails when refactoring exits (multi-level exits).

Software pattern: many routines have multiple outcomes:
\begin{itemize}
\item normal: return normal result and transfer after call
\item exceptional: return alternative result and not transfer after call
\end{itemize}

\textbf{Nonlocal transfer} allows a routine to transfer back to its caller but not after the call
\begin{itemize}
\item generalization of multi-exit loop and multi-level exit
\item control structures end normally or with an exceptional transfer
\end{itemize}

This pattern acknowledges that
\begin{itemize}
\item algorithms can have multiple outcomes
\item separating outcomes makes it easy to read and maintain a program
\end{itemize}

This pattern does not handle multiple levels of nested modularization.
\section{Traditional Approaches}
\label{sec:org857a01f}
\textbf{Return code}: returns value indicating normal or exceptional execution

\textbf{Status flag}: set shared (global) variable indicating normal or exceptional execution which remains
as long as it is not overwritten

\textbf{Fix-up routine}: a global and/or local routine called for an exceptional event to fix-up and return
a corrective result so a computation can continue

\textbf{Return union}: modern approach combining result/return-code and requiring return-code check on result
access, requires that all routines return an appropriate union

Traditional approaches force checking, unless explicitly accessing without \texttt{holds\_alternative}.

Drawbacks of traditional techniques:
\begin{itemize}
\item checking return code or status flag is optional, so can be delayed or omitted
\item return code mixes exceptional and normal values, enlarges type or value range since normal/exceptional
type/values should be independent
\end{itemize}

Testing/handling return code or status flag is often done locally (inline) since info could be lost
otherwise, but local testing/handling:
\begin{itemize}
\item reduces readability, since each call results in multiple statements
\item can be inappropriate since library routines should not terminate a program
\end{itemize}

Nonlocal testing from nested routine calls is difficult since multiple codes are returned for analysis,
compounding the mixing problem.

Status flag can be overwritten before being examined and cannot be used in a concurrent environment
because of sharing issues.

Local fix-up routines increase the number of params since they increase the cost of each call and
must be passed through multiple levels, enlarging param lists even when fix-up routine is unused.

Nonlocal fix-up routines, implemented with global routine pointer, have identical problems with
status flags.
\end{document}
