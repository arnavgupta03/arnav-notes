% Created 2024-10-26 Sat 16:51
% Intended LaTeX compiler: pdflatex
\documentclass[11pt]{article}
\usepackage[utf8]{inputenc}
\usepackage[T1]{fontenc}
\usepackage{graphicx}
\usepackage{longtable}
\usepackage{wrapfig}
\usepackage{rotating}
\usepackage[normalem]{ulem}
\usepackage{amsmath}
\usepackage{amssymb}
\usepackage{capt-of}
\usepackage{hyperref}
\usepackage{parskip,darkmode}
\enabledarkmode
\author{Arnav Gupta}
\date{\today}
\title{Nonlocal Transfer}
\hypersetup{
 pdfauthor={Arnav Gupta},
 pdftitle={Nonlocal Transfer},
 pdfkeywords={},
 pdfsubject={},
 pdfcreator={Emacs 29.4 (Org mode 9.7.11)}, 
 pdflang={English}}
\begin{document}

\maketitle
\tableofcontents

\section{Nonlocal Transfer}
\label{sec:orgac91aa9}
Routine \textbf{activation} (call/invocation) introduces complex control flow.
Among routines, control flow is controlled by call/return mechanism.

\textbf{Modularization}: from software engineering, any contiguous code block can be factored into a helper
routine and called in the program

Modularization fails when refactoring exits (multi-level exits).

Software pattern: many routines have multiple outcomes:
\begin{itemize}
\item normal: return normal result and transfer after call
\item exceptional: return alternative result and not transfer after call
\end{itemize}

\textbf{Nonlocal transfer} allows a routine to transfer back to its caller but not after the call
\begin{itemize}
\item generalization of multi-exit loop and multi-level exit
\item control structures end normally or with an exceptional transfer
\end{itemize}

This pattern acknowledges that
\begin{itemize}
\item algorithms can have multiple outcomes
\item separating outcomes makes it easy to read and maintain a program
\end{itemize}

This pattern does not handle multiple levels of nested modularization.
\section{Traditional Approaches}
\label{sec:org8f7619b}
\textbf{Return code}: returns value indicating normal or exceptional execution

\textbf{Status flag}: set shared (global) variable indicating normal or exceptional execution which remains
as long as it is not overwritten

\textbf{Fix-up routine}: a global and/or local routine called for an exceptional event to fix-up and return
a corrective result so a computation can continue

\textbf{Return union}: modern approach combining result/return-code and requiring return-code check on result
access, requires that all routines return an appropriate union

Traditional approaches force checking, unless explicitly accessing without \texttt{holds\_alternative}.

Drawbacks of traditional techniques:
\begin{itemize}
\item checking return code or status flag is optional, so can be delayed or omitted
\item return code mixes exceptional and normal values, enlarges type or value range since normal/exceptional
type/values should be independent
\end{itemize}

Testing/handling return code or status flag is often done locally (inline) since info could be lost
otherwise, but local testing/handling:
\begin{itemize}
\item reduces readability, since each call results in multiple statements
\item can be inappropriate since library routines should not terminate a program
\end{itemize}

Nonlocal testing from nested routine calls is difficult since multiple codes are returned for analysis,
compounding the mixing problem.

Status flag can be overwritten before being examined and cannot be used in a concurrent environment
because of sharing issues.

Local fix-up routines increase the number of params since they increase the cost of each call and
must be passed through multiple levels, enlarging param lists even when fix-up routine is unused.

Nonlocal fix-up routines, implemented with global routine pointer, have identical problems with
status flags.
\section{Dynamic Multi-Level Exit}
\label{sec:org07ff0a9}
Better than returning one level at a time and allows transfer of control directly to original caller.

\textbf{Dynamic Multi-level Exit}: extends call/return semantics to transfer in the reverse direction to
normal routine calls, requiring nonlocal transfer.

A \textbf{label variable} contains:
\begin{itemize}
\item a pointer to a block activation on the stack
\item a transfer point within the block
\end{itemize}

Nonlocal transfer is a 2 step operation:
\begin{enumerate}
\item direct control flow to the specified activation on the stack
\item go to the transfer point (label constant) within that routine
\end{enumerate}

Label value is not statically determined (hence dynamic MLE).
With recursion, the distance between activations on the stack is unknown and labels can be set
and reset through recursive calls.

Transfer between \texttt{goto} and label causes termination of a stack block.
Termination is implicit for direct transfer of control or requires stack unwinding if activations
contain objects with destructors or finalizers.

DME is possible in C using:
\begin{itemize}
\item \texttt{jmp\_buf} to declare a label variable
\item \texttt{setjmp} to initialize a label variable
\item \texttt{longjmp} to go to a label variable
\end{itemize}

DME allows multiple forms of returns to any level:
\begin{itemize}
\item normal return transfers to statement after the call, implying completion of routine's algorithm
\item exceptional return transfers to statement not after the call, indicating an ancillary completion
\end{itemize}

Nonlocal transfer can be problematic since it is too general.
\section{Exception Handling}
\label{sec:org4edde94}
Refers to DME and nonlocal transfer among routines, and is more than just error handling.

\textbf{Exceptional event}: event that is usually known to exist but is ancillary to an algorithm (occurs
with low frequency)

\textbf{Exception Handling Mechanism} (EHM): provides alternate kinds of control flow

EHM allows for more robustness since exceptions are active, forcing programs to react immediately when
an exceptional event occurs.
\section{Terminology}
\label{sec:org2f467bf}
\textbf{Execution}: language unit in which an exception can be raised, usually any entity with its own
runtime stack

\textbf{Exception}: instance of an exception type, generated by executing an operation indicating an
ancillary (exceptional) situation in execution

\textbf{Raise/Throw}: special operation that creates an exception

\textbf{Source Execution}: execution that raises an exception

\textbf{Faulting Execution}: execution changing control flow due to a raised exception

\textbf{Local Exception}: when an exception is raised and handled by the same execution (source and faulting are
the same)

\textbf{Nonlocal Exception}: when an exception is raised by a source execution but delivered to a different
faulting execution

\textbf{Concurrent Exception}: nonlocal exception where the source and faulting executions are executing
concurrently

\textbf{Propagation}: directs control from a raise in the source execution to a handler in the faulting
execution

\textbf{Propagation Mechanism}: rules used to locate a handler
\begin{itemize}
\item most commonly gives precedence to handlers higher in the call stack and specific handlers, through
linear search during propagation
\end{itemize}

\textbf{Handler}: inline routine responsible for handling raised exception
\begin{itemize}
\item catches exception by matching exception type
\item executes like a normal subroutine and can return (exception handled) or raise exceptions
\item re-raising the current exception continues propagation of caught exception
\end{itemize}

\textbf{Guarded Block}: language block with associated handlers

\textbf{Unguarded Block}: block with no handlers

\textbf{Termination}: control cannot return to the raise point, since \textbf{stack unwinding} occurs

\textbf{Resumption}: control returns to the raise point (no stack unwinding)
\section{Execution Environment}
\label{sec:org9df891e}
An object-oriented concurrent environment requires a more complex EHM than a non object-oriented sequential
environment.

Objects may have destructors that must be executed no matter if the object ends by normal or exceptional
termination. (ex. \texttt{finally} clause)

C++ does not allow direct nonlocal transfer, continuation, coroutine, and tasks with their own execution
stack.

With multiple stacks, an EHM can be more sophisticated (more complex).
\section{Implementation}
\label{sec:org2f95967}
In most programming languages, DME is limited with exception handling.

To implement throw/catch, the \texttt{throw} must know the last guarded block with handler for the raised
exception type. The approach for this is:
\begin{itemize}
\item associate a label variable with each exception type
\item set label variable on entry to each guarded block with handler for the type
\item reset label variable on exit to previous value
\end{itemize}

Setting/resetting label on each \texttt{try} block can have unnecessary cost, so instead \texttt{catch} data is
stored once externally for each block and handler is found by linear search during a stack walk.
This makes \texttt{try} cheap but \texttt{throw} expensive.
\section{Static/Dynamic Call/Return}
\label{sec:org71833c0}
All routine/exceptional control flow can be characterized as:
\begin{enumerate}
\item \textbf{static/dynamic call}: routine/exception name at the call/raise is looked up statically (compile-time)
or dynamically (runtime)
\item \textbf{static/dynamic return}: after a routine/handler completes, it returns to its static (definition) or
dynamic (call) context
\end{enumerate}

\begin{center}
\begin{tabular}{lll}
return/handled & static call & dynamic call\\
static & sequel & termination exception\\
dynamic & routine & routine pointer, virtual routine, resumption\\
\end{tabular}
\end{center}
\subsection{Static Propagation (Sequel)}
\label{sec:orge44c519}
A routine with no return value, where
\begin{itemize}
\item the sequel name is looked up lexically at the call site
\item control returns to the end of the block in which the sequel is declared
\end{itemize}

Allows modularization of code with static exits, and propagation is along the lexical structure.

Sequels help handle termination for non-recoverable events.

\uline{Advantage}: handler is statically known and can be as efficient as a direct transfer

\uline{Disadvantage}: sequel only works for monolithic programs since it must be statically nested at
the point of use
\subsection{Dynamic Propagation}
\label{sec:org7d64108}
Dynamic propagation/static return is just dynamic MLE.

\uline{Advantage}: dynamic propagation works for separately compiled programs

\uline{Disadvantage}: handler is not statically known
\subsubsection{Termination}
\label{sec:orgff680c6}
Control transfers from the start of propagation to a dynamically defined handler and then
performs a static return (like a sequel).

Basic termination forms for a non-recoverable operation:
\begin{itemize}
\item \textbf{terminate}: provides limited mechanism for block transfer on the call stack, like labelled break
and no need to forward alternative outcomes
\begin{itemize}
\item catch can also depend on the object raising the exception (exception object from object used)
\end{itemize}
\item \textbf{retry}: combination of termination with special handler semantics, restart the guarded block
handling the exception
\end{itemize}
\subsubsection{Resumption}
\label{sec:orgdc67b77}
Control transfers from the start of propagation to a dynamically defined handler and then
performs a dynamic return (stack not unwound).

Resumption handler is a corrective action so a computation can continue.
\end{document}
