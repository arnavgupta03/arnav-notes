% Created 2024-10-24 Thu 22:35
% Intended LaTeX compiler: pdflatex
\documentclass[11pt]{article}
\usepackage[utf8]{inputenc}
\usepackage[T1]{fontenc}
\usepackage{graphicx}
\usepackage{longtable}
\usepackage{wrapfig}
\usepackage{rotating}
\usepackage[normalem]{ulem}
\usepackage{amsmath}
\usepackage{amssymb}
\usepackage{capt-of}
\usepackage{hyperref}
\usepackage{parskip,darkmode}
\enabledarkmode
\author{Arnav Gupta}
\date{\today}
\title{Network Layer}
\hypersetup{
 pdfauthor={Arnav Gupta},
 pdftitle={Network Layer},
 pdfkeywords={},
 pdfsubject={},
 pdfcreator={Emacs 29.4 (Org mode 9.7.11)}, 
 pdflang={English}}
\begin{document}

\maketitle
\tableofcontents

\section{Overview}
\label{sec:orga8120ec}
Network layer protocols transport segments from sending to receiving hosts.
The \uline{sender} encapsulates segments into datagrams and passes them to the link
layer. The \uline{receiver} delivers segments to the transport layer protocol.

Every Internet device uses network layer protocols.

\textbf{Routers} examine header fields in all IP datagrams passing through them
and move datagrams from input interfaces to output inferfaces to
transfer datagrams along end-to-end paths.

Key network-layer functions:
\begin{itemize}
\item \textbf{forwarding}: move packets from a router's input link to appropriate
router output links (done for all datagrams very fast)
\item \textbf{routing}: network-wide process that determines route taken by packets
from source to destination to fill forwarding tables (done in the background,
takes longer), uses routing algorithms
\end{itemize}

The \textbf{data plane} is a local, per-router function and determines how a datagram
arriving on router input interface is forwarded to router output interface.

The \textbf{control plane} uses network-wide logic to determine how a datagram is routed
among routers along end-to-end paths from source host to destination host.

Control plane approaches:
\begin{itemize}
\item \textbf{traditional routing algorithms}: implemented in routers (both planes implemented
monolithically within a router)
\begin{itemize}
\item individual routing algorithm components in every router interact in the control
plane
\end{itemize}
\item \textbf{software-defined networking}: explicitly separate the two planes by implementing
the control plane as a service in remote servers
\begin{itemize}
\item remote controller computes and installs forwarding tables in routers
\end{itemize}
\end{itemize}
\subsection{Best Effort Service}
\label{sec:org919607d}
The Internet runs on a ``best effort'' service model, so there are no guarantees on:
\begin{itemize}
\item successful datagram delivery to the destination
\item timing or order of delivery
\item bandwidth available for end-to-end flow
\end{itemize}

Best effort is simple and has allowed Internet to be widely adopted.
Successful provisioning of bandwidth allows performance of real-time
applications to be good enough for most of the time.

Replicated, application-layer distributed services connect close to client networks
which allows services to be provided from multiple locations.
\section{Data Plane: The Internet Protocol}
\label{sec:org28e2ebc}
Data plane is:
\begin{itemize}
\item connectionless (datagram-based)
\item best-effort delivery
\begin{itemize}
\item packets can be lost, delivered out of order, or delayed
\end{itemize}
\item a common packet format for IPv4 and new packet format for IPv6
\item global addressing for identifying all hosts (ARP)
\item sister protocol that performs error reporting and enables signaling between
routers: ICMP (v4, v6)
\end{itemize}

IPv4 datagram has
\begin{itemize}
\item IP version number
\item header length (bytes)
\item type of service
\item total datagram length (bytes)
\item 16-bit identifier
\item flag
\item fragmentation/reassembly and offset info
\item time to live: remaining max hops
\item upper layer protocol (TCP or UDP)
\item header checksum
\item 32-bit source IP address
\item 32-bit destination IP address
\item options (timestamp, record route taken, etc)
\item payload data
\end{itemize}

When no options, overhead is 20 bytes.
Upper layer protocol can be:
\begin{itemize}
\item ICMP: 1
\item TCP: 6
\item UDP: 17
\item IPv4: 4
\item IPv6: 41
\end{itemize}

If header checksum detects an error, datagram is dropped.
Must be recomputed at every hop because of TTL and options.
\subsection{Fragmentation and Assembly}
\label{sec:orgb5bc666}
Link layer protocols have maximum transfer unit size, which is the largest
possible data size in a frame.

Large IP datagrams can be divided (fragmented) within a network, and then
reassembled at the final destination.
IP header bits are used to identify and order related fragments.

A receiver cannot hold fragments forever and fragments can arrive out of order,
so loss of fragments can mean loss of entire datagram.
The receiver starts a timer when the first fragment of a datagram arrives.
If the timer expires before all the fragments are received, those already
received are discarded.

Fragmentation complicated routers and end-systems, which is used by attackers.
\subsection{IPv4 Addressing}
\label{sec:org4612f06}
\textbf{IP address}: 32-bit identifier associated with each host or router interface
(about 4 billion total)

\textbf{Interface}: connection between host/router and physical link
\begin{itemize}
\item routers typically have multiple interfaces
\item hosts typically have 1 or 2 interfaces
\end{itemize}
\subsection{IPv6 Addressing}
\label{sec:org1fd18f7}
Possibly not enough 32-bit IPv4 addresses. Also, IPv4 is slow (variable length header).
IPv6 allows different network-layer treatment of flows and better mobility management.

IPv6 datagram has:
\begin{itemize}
\item IP version
\item priority among datagrams in flow
\item flow label: identify datagrams in same flow
\item payload length
\item next header
\item hop limit
\item 128-bit source address
\item 128-bit destination address
\item data payload
\end{itemize}

Compared to IPv4, has no checksum, fragmentation/reassembly, and options.

Not all routers can be upgraded simultaneously so must operate with mixed IPv4
and IPv6.

\textbf{Tunneling}: IPv6 datagram is carried as payload in IPv4 datagram among IPv4
routers (packet within a packet)

44.5\% of clients access services via IPv6, so takes time to deploy.
\section{What's Inside a Router}
\label{sec:org6df41e9}
High-level view of generic router architecture has \textbf{routing processor} (control plane) and
\textbf{high-speed switching fabric} (data plane).

Input ports have a physical layer, link layer, and decentralized switching (using header
field values, lookup output port using forwarding table in input port memory).

\textbf{Destination-based forwarding}: forward based only on destination IP address

\textbf{Generalized forwarding}: forward based on any set of header field values
\subsection{Switching Fabrics}
\label{sec:org50a226b}
Transfer packets from input link to appropriate output link.

\textbf{Switching rate}: rate at which packets can be transferred from inputs to outputs,
measured as multiple of input/output line rate

For \(N\) inputs, switching rate of \(N\) times the line rate is desirable.

Major types of switching fabrics are:
\begin{itemize}
\item memory
\item bus
\item interconnection network
\end{itemize}
\subsubsection{Switching via Memory}
\label{sec:orgf022c12}
Used traditionally, with switching under direct control of CPU.

The packet is copied into system memory and speed is limited by memory bandwidth
(2 bus crossings per datagram).
\subsubsection{Switching via Bus}
\label{sec:orgd758274}
Datagram from input port memory to output port memory via shared bus.

\textbf{Bus contention}: switching speed is limited by bus bandwidth
\subsubsection{Interconnection Network}
\label{sec:org5963d74}
Initially developed to connect processors in multiprocessor.

\textbf{Multistage switch}: \(n \times n\) switch from multiple stages of smaller switches

With parallelism:
\begin{itemize}
\item fragment datagram into fixed length cells on entry
\item switch cells through the fabric and reassemble datagram at exit
\end{itemize}

Can scale by using multiple switching planes in parallel.

Cisco CRS router:
\begin{itemize}
\item basic unit has 8 switching planes
\item each plane has a 3 stage interconnection network
\item up to 100s of Tbps switching capacity
\end{itemize}
\subsection{Port Queuing}
\label{sec:org0e30395}
\subsubsection{Input Port Queuing}
\label{sec:orge34b70b}
If switch fabric slower than input ports combined, queuing may occur at input queues.
This can lead to queuing delay and loss due to input buffer overflow.

\textbf{Head of the Line (HOL) blocking}: queued datagram at front of queue prevents others
in queue from moving forward
\subsubsection{Output Port Queuing}
\label{sec:org1834fa9}
Buffering occurs when arrival rate via switch exceeds output line speed.

Queuing delay (and loss) due to output port buffer congestion (overflow).

\textbf{Buffering} is required when datagrams arrive from fabric faster than the link
transmission rate.
Must have \textbf{drop policy} to decide which datagrams to drop if no free buffers.

\textbf{Scheduling discipline} chooses among queued datagrams for transmission.
With priority scheduling, this decides who gets best performance.
\section{Control Plane: ICMP, Routing}
\label{sec:org472fb45}
\end{document}
