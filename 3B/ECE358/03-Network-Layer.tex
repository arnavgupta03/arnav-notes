% Created 2024-11-28 Thu 00:09
% Intended LaTeX compiler: pdflatex
\documentclass[11pt]{article}
\usepackage[utf8]{inputenc}
\usepackage[T1]{fontenc}
\usepackage{graphicx}
\usepackage{longtable}
\usepackage{wrapfig}
\usepackage{rotating}
\usepackage[normalem]{ulem}
\usepackage{amsmath}
\usepackage{amssymb}
\usepackage{capt-of}
\usepackage{hyperref}
\usepackage{parskip,darkmode}
\enabledarkmode
\author{Arnav Gupta}
\date{\today}
\title{Network Layer}
\hypersetup{
 pdfauthor={Arnav Gupta},
 pdftitle={Network Layer},
 pdfkeywords={},
 pdfsubject={},
 pdfcreator={Emacs 29.4 (Org mode 9.7.11)}, 
 pdflang={English}}
\begin{document}

\maketitle
\tableofcontents

\section{Overview}
\label{sec:org61e996e}
Network layer protocols transport segments from sending to receiving hosts.
The \uline{sender} encapsulates segments into datagrams and passes them to the link
layer. The \uline{receiver} delivers segments to the transport layer protocol.

Every Internet device uses network layer protocols.

\textbf{Routers} examine header fields in all IP datagrams passing through them
and move datagrams from input interfaces to output inferfaces to
transfer datagrams along end-to-end paths.

Key network-layer functions:
\begin{itemize}
\item \textbf{forwarding}: move packets from a router's input link to appropriate
router output links (done for all datagrams very fast)
\item \textbf{routing}: network-wide process that determines route taken by packets
from source to destination to fill forwarding tables (done in the background,
takes longer), uses routing algorithms
\end{itemize}

The \textbf{data plane} is a local, per-router function and determines how a datagram
arriving on router input interface is forwarded to router output interface.

The \textbf{control plane} uses network-wide logic to determine how a datagram is routed
among routers along end-to-end paths from source host to destination host.

Control plane approaches:
\begin{itemize}
\item \textbf{traditional routing algorithms}: implemented in routers (both planes implemented
monolithically within a router)
\begin{itemize}
\item individual routing algorithm components in every router interact in the control
plane
\end{itemize}
\item \textbf{software-defined networking}: explicitly separate the two planes by implementing
the control plane as a service in remote servers
\begin{itemize}
\item remote controller computes and installs forwarding tables in routers
\end{itemize}
\end{itemize}
\subsection{Best Effort Service}
\label{sec:orgd243604}
The Internet runs on a ``best effort'' service model, so there are no guarantees on:
\begin{itemize}
\item successful datagram delivery to the destination
\item timing or order of delivery
\item bandwidth available for end-to-end flow
\end{itemize}

Best effort is simple and has allowed Internet to be widely adopted.
Successful provisioning of bandwidth allows performance of real-time
applications to be good enough for most of the time.

Replicated, application-layer distributed services connect close to client networks
which allows services to be provided from multiple locations.
\section{Data Plane: The Internet Protocol}
\label{sec:org9a3b8fb}
Data plane is:
\begin{itemize}
\item connectionless (datagram-based)
\item best-effort delivery
\begin{itemize}
\item packets can be lost, delivered out of order, or delayed
\end{itemize}
\item a common packet format for IPv4 and new packet format for IPv6
\item global addressing for identifying all hosts (ARP)
\item sister protocol that performs error reporting and enables signaling between
routers: ICMP (v4, v6)
\end{itemize}

IPv4 datagram has
\begin{itemize}
\item IP version number
\item header length (bytes)
\item type of service
\item total datagram length (bytes)
\item 16-bit identifier
\item flag
\item fragmentation/reassembly and offset info
\item time to live: remaining max hops
\item upper layer protocol (TCP or UDP)
\item header checksum
\item 32-bit source IP address
\item 32-bit destination IP address
\item options (timestamp, record route taken, etc)
\item payload data
\end{itemize}

When no options, overhead is 20 bytes.
Upper layer protocol can be:
\begin{itemize}
\item ICMP: 1
\item TCP: 6
\item UDP: 17
\item IPv4: 4
\item IPv6: 41
\end{itemize}

If header checksum detects an error, datagram is dropped.
Must be recomputed at every hop because of TTL and options.
\subsection{Fragmentation and Assembly}
\label{sec:org2bdbd02}
Link layer protocols have maximum transfer unit size, which is the largest
possible data size in a frame.

Large IP datagrams can be divided (fragmented) within a network, and then
reassembled at the final destination.
IP header bits are used to identify and order related fragments.

A receiver cannot hold fragments forever and fragments can arrive out of order,
so loss of fragments can mean loss of entire datagram.
The receiver starts a timer when the first fragment of a datagram arrives.
If the timer expires before all the fragments are received, those already
received are discarded.

Fragmentation complicated routers and end-systems, which is used by attackers.
\subsection{IPv4 Addressing}
\label{sec:orgef60354}
\textbf{IP address}: 32-bit identifier associated with each host or router interface
(about 4 billion total)

\textbf{Interface}: connection between host/router and physical link
\begin{itemize}
\item routers typically have multiple interfaces
\item hosts typically have 1 or 2 interfaces
\end{itemize}
\subsubsection{Subnets}
\label{sec:orgea5c875}
The addressing scheme is \textbf{Classless InterDomain Routing (CIDR)}:
\begin{itemize}
\item IP addresses have a subnet part and a host part
\item the subnet portion of the address has arbitrary length
\item the address has format a.b.c.d/x where x is the prefix, the number
of bits in the subnet portion
\end{itemize}

\textbf{Subnet}: set of interfaces that have IP addresses with the same
prefix and same subnet portion, and can physically reach each other
without passing through an intervening router

A subnet mask is x 1 bits followed by 0 bits, which is bitwise ANDed
to the IP address.
With masking, for the same subnet, can send a datagram directly to the
destination and for a different subnet, send the datagram to the router.
\subsubsection{DHCP}
\label{sec:org7b337c2}
An IP address can be hard-coded by sysadmin in the config file or using
\textbf{DHCP (Dynamic Host Configuration Protocol)} which dynamically gets
address from a server when the host joins the network.

DHCP allows:
\begin{itemize}
\item addresses to be renewed on use
\item reused since only holding addresses while host connected/on
\end{itemize}

For a client-server protocol using UDP, DHCP is a network function
implemented as an application protocol:
\begin{itemize}
\item host broadcasts DHCP discover message
\item DHCP server responds with DHCP offer message
\item host requests IP address with a DHCP request message
\item DHCP server sends the address with a DHCP ack message
\end{itemize}

Only last 2 steps are needed if the client remembers and wishes
to reuse a previously allocated network address.

A DHCP server is co-located in the router, serving all subnets to which
the router is attached.

DHCP also returns:
\begin{itemize}
\item address of first hop router for client
\item name and IP address of DNS server
\item address prefix (indicating network vs host portion of address)
\end{itemize}

For a network to get the subnet part of the IP address, it gets
allocated a portion of its provider ISP's address space.
\subsubsection{Hierarchical Addressing}
\label{sec:orge136293}
Allows efficient advertisement of routing info.

When the router receives a datagram with some destination address:
\begin{enumerate}
\item mask the destination address with the mask for that row of the
forwarding table
\item check if the results correspond to the value in the table, if so
then remember as candidate for forwarding, and then regardless
continue to the next row
\begin{enumerate}
\item if there are no candidates, datagram sent to otherwise
output interface
\item if there is one candidate, datagram sent to the corresponding
output interface
\item if there are multiple candidates (due to multiple routes to some
host), the most specific one is taken: \textbf{longest prefix rule}
\end{enumerate}
\end{enumerate}

To get a block of addresses, ICANN (Internet Corportation for Assigned
Names and Numbers) allocates IP addresses through 5 regional registries (RRs)
and manages DNS.

DHCP and NAT help with IPv4 address space exhaustion.
IPv6 has 128-bit address space.
\subsubsection{Network Address Translation (NAT)}
\label{sec:org5022a77}
Limiting the number of addresses and re-using IP addresses in a smart way.

Addresses in a private address space are not routable outside it and can
be reused as much as desired.

Advantages:
\begin{itemize}
\item range of addresses not needed from ISP: just one IP address for all devices
\item can change addresses of devices in local network without notifying
outside world
\item can change ISP without changing addresses of devices in local network
\item devices inside local net not explicitly addressable or visible by outside
world
\end{itemize}

Host creates IP datagrams with source and destination IP address that carries
1 transport layer segment. Each segment has source and destination port
numbers (Layer 4 entity).

Client knows the port number at the server for the service it needs and OS
selects a unique source port number.

Receiving host uses IP addresses and port numbers to direct segment to appropriate
process (via sockets).

All devices in local network share just one IPv4 address to the outside world:
\begin{itemize}
\item all datagrams leaving local network share same source NAT IP address but
different source port numbers
\item datagrams with source or destination in local network have subnetted address
for source and destination, as usual
\end{itemize}

The NAT router must transparently:
\begin{itemize}
\item for outgoing datagrams, replace source IP address and port number of every
outgoing datagram to NAT IP address and new port number
\begin{itemize}
\item remote clients and servers will respond using the NAT IP address and
new port number as destination address
\end{itemize}
\item remember in the NAT translation table every pair of source IP address and port
number to NAT IP address and new port number translation mapping
\item for incoming datagrams, replace the NAT IP address and new port number in the
destination fields of every incoming datagram with the corresponding
source IP address and port number stored in the NAT table
\end{itemize}

The 16-bit port number field allows 60000 simultaneous connections with a single
public IP address.

The router keeps NAT entries in the translation table for a configurable length
of time.
For TCP connections, default timeout is 24 hours.
Since UDP is not connection based, default timeout is 5 minutes.

NAT is controversial since:
\begin{itemize}
\item routers should only process up to layer 3
\item address shortage should be solved by IPv6
\item violates end-to-end argument since port number is manipulated by
network-layer device
\item can be tricky if client wants to connect to the server behind NAT
\end{itemize}
\subsubsection{Middleboxes}
\label{sec:org6634835}
Any intermediate box performing functions apart from normal, standard functions
of an IP router on the data path between a source host and destination host.

Includes NAT, application-specific, firewalls, intrusion detection systems,
load balancers, and caches.

The internet has a thin waist, since there is a single network layer protocol: IP
that must be implemented by every Internet-connected device (compared to
many protocols in other layers).

Middleboxes give love handles that operate inside the network.
\subsection{IPv6 Addressing}
\label{sec:org37a52bc}
Possibly not enough 32-bit IPv4 addresses. Also, IPv4 is slow (variable length header).
IPv6 allows different network-layer treatment of flows and better mobility management.

IPv6 datagram has:
\begin{itemize}
\item IP version
\item priority among datagrams in flow
\item flow label: identify datagrams in same flow
\item payload length
\item next header
\item hop limit
\item 128-bit source address
\item 128-bit destination address
\item data payload
\end{itemize}

Compared to IPv4, has no checksum, fragmentation/reassembly, and options.

Not all routers can be upgraded simultaneously so must operate with mixed IPv4
and IPv6.

\textbf{Tunneling}: IPv6 datagram is carried as payload in IPv4 datagram among IPv4
routers (packet within a packet)

44.5\% of clients access services via IPv6, so takes time to deploy.
\section{What's Inside a Router}
\label{sec:org48e3b43}
High-level view of generic router architecture has \textbf{routing processor} (control plane) and
\textbf{high-speed switching fabric} (data plane).

Input ports have a physical layer, link layer, and decentralized switching (using header
field values, lookup output port using forwarding table in input port memory).

\textbf{Destination-based forwarding}: forward based only on destination IP address

\textbf{Generalized forwarding}: forward based on any set of header field values
\subsection{Switching Fabrics}
\label{sec:org79ce811}
Transfer packets from input link to appropriate output link.

\textbf{Switching rate}: rate at which packets can be transferred from inputs to outputs,
measured as multiple of input/output line rate

For \(N\) inputs, switching rate of \(N\) times the line rate is desirable.

Major types of switching fabrics are:
\begin{itemize}
\item memory
\item bus
\item interconnection network
\end{itemize}
\subsubsection{Switching via Memory}
\label{sec:org84b0bc8}
Used traditionally, with switching under direct control of CPU.

The packet is copied into system memory and speed is limited by memory bandwidth
(2 bus crossings per datagram).
\subsubsection{Switching via Bus}
\label{sec:orge2563a2}
Datagram from input port memory to output port memory via shared bus.

\textbf{Bus contention}: switching speed is limited by bus bandwidth
\subsubsection{Interconnection Network}
\label{sec:org12d4619}
Initially developed to connect processors in multiprocessor.

\textbf{Multistage switch}: \(n \times n\) switch from multiple stages of smaller switches

With parallelism:
\begin{itemize}
\item fragment datagram into fixed length cells on entry
\item switch cells through the fabric and reassemble datagram at exit
\end{itemize}

Can scale by using multiple switching planes in parallel.

Cisco CRS router:
\begin{itemize}
\item basic unit has 8 switching planes
\item each plane has a 3 stage interconnection network
\item up to 100s of Tbps switching capacity
\end{itemize}
\subsection{Port Queuing}
\label{sec:orgb539286}
\subsubsection{Input Port Queuing}
\label{sec:org0b84f97}
If switch fabric slower than input ports combined, queuing may occur at input queues.
This can lead to queuing delay and loss due to input buffer overflow.

\textbf{Head of the Line (HOL) blocking}: queued datagram at front of queue prevents others
in queue from moving forward
\subsubsection{Output Port Queuing}
\label{sec:org9cff9ff}
Buffering occurs when arrival rate via switch exceeds output line speed.

Queuing delay (and loss) due to output port buffer congestion (overflow).

\textbf{Buffering} is required when datagrams arrive from fabric faster than the link
transmission rate.
Must have \textbf{drop policy} to decide which datagrams to drop if no free buffers.

\textbf{Scheduling discipline} chooses among queued datagrams for transmission.
With priority scheduling, this decides who gets best performance.
\section{Control Plane: ICMP, Routing}
\label{sec:org9d8bf8d}
Even though best effort, IP attempts to avoid errors and report problems when they
occur.

IP does introduce errors or ignore all errors.

Errors detected can be:
\begin{itemize}
\item corrupted header bits: header checksum
\item illegal addresses: routing tables
\item routing loop: TTL field
\item fragment loss: timeout
\end{itemize}
\subsection{ICMP}
\label{sec:org88dd0eb}
Internet Control Message Protocol is a separate protocol for errors reporting and
information.
Required part of IP (just above IP, layer 3.5) and sends error message to
original source.

Used by hosts and routers to communicate network-level info like error reporting
(unreachable host, network, port, protocol, etc.) and uses echo request/reply (used
by ping).

Network-layer above IP so ICMP messages carried in IP datagrams.

ICMP message has a type, code and first 8 bytes of IP datagram causing error.

IP datagram header contains a bit to specify no fragmentation allowed, which can be
bit 0 \(\to\) must be zero, bit 1 \(\to\) don't fragment, bit 2 \(\to\) more fragments.

ICMP sends an error message when fragmentation required but not permitted.
This is done by probing to find the largest MTU that does not generate an error
message.
This MTU is not guaranteed if routes change.

For traceroute (provides delay measurement from source to router):
\begin{itemize}
\item source sends sets of UDP segments to destination with an unknown port number
where each set has an increasing TTL starting from 1
\item datagram in set \(n\) arrives to router \(n\) where router discards datagram
and sends source ICMP message which possibly includes name of router
and IP address
\item when ICMP message arrives at source, it records RTTs
\end{itemize}

The stopping criteria for traceroute is that the UDP segment eventually arrives
at destination host.
The destination returns ICMP ``port unreachable'' message and so the source stops.
\end{document}
