% Created 2024-12-05 Thu 15:58
% Intended LaTeX compiler: pdflatex
\documentclass[11pt]{article}
\usepackage[utf8]{inputenc}
\usepackage[T1]{fontenc}
\usepackage{graphicx}
\usepackage{longtable}
\usepackage{wrapfig}
\usepackage{rotating}
\usepackage[normalem]{ulem}
\usepackage{amsmath}
\usepackage{amssymb}
\usepackage{capt-of}
\usepackage{hyperref}
\usepackage{parskip,darkmode}
\enabledarkmode
\author{Arnav Gupta}
\date{\today}
\title{Wireless}
\hypersetup{
 pdfauthor={Arnav Gupta},
 pdftitle={Wireless},
 pdfkeywords={},
 pdfsubject={},
 pdfcreator={Emacs 29.4 (Org mode 9.7.11)}, 
 pdflang={English}}
\begin{document}

\maketitle
\tableofcontents

\section{Wireless Networks}
\label{sec:org6b5e385}
Many more wireless phone subscribers than wired, and same for devices.

Challenges arise from wireless (communication over wireless link) and mobility (handling mobile user
who changes point of attachment to network).
Wireless \(\ne\) mobile.

\textbf{Base station} is typically connected to wired network and relays packets between wired network and
wireless hosts in its area.

\textbf{Wireless link} connects mobiles to base station and can be used as a backbone link. Multiple access
protocol coordinates link access, and can have various data rates and transmission distance.

\textbf{Infrastructure mode} is how base station connects mobiles into wired network.

With \textbf{ad hoc mode}, no base stations and so nodes can only transmit to other nodes within link
coverage. With this, nodes organize themselves into a network (route among themselves).

\textbf{Wireless network taxonomy}
\begin{itemize}
\item \uline{infrastructure, single hop}: host connects to base station which connects to larger internet
\item \uline{infrastructure, multiple hops}: host may have to relay through several wireless nodes to connect
to larger Internet: \emph{mesh net}
\item \uline{no infrastructure, single hop}: no base station, no connection to larger internet (Bluetooth,
ad hoc nets)
\item \uline{no infrastructure, multiple hops}: no base station, no connection to larger internet, may have to
replay to reach a given wireless node
\end{itemize}

Long waves mean big antennas and small bandwidth.
Best bands are lower ones, since better propagation characteristics and larger antennas.
Issue is, more bandwidth on higher frequencies than lower ones, so lower ones are oversubscribed.

Spectrum not always used well and re-allocation often necessary.
Spectrum is a scarce resource and hence is heavily reused.
This can create \textbf{interference}.

Interference management and spectrum reuse are critical problems in wireless.

2 types of bands:
\begin{itemize}
\item \uline{licensed}: used by cellular networks
\item \uline{unlicensed}: used by WiFi (and recently cellular)
\end{itemize}

Industrial, Scientific, Medical (ISM) wireless bands and is region of EM spectrum available for use
without license:
\begin{itemize}
\item used for wireless LANs and PANs
\item four separate bands
\end{itemize}

Key words are \textbf{rate}, \textbf{range}, and \textbf{power consumption}.

Lower frequency has longer range.

\textbf{Half-duplex}: a node cannot and receive at the same time

\textbf{Decreased signal strength}: radio signal attenuates as it propagates through matter (path loss), and
some cannot cross walls at all

\textbf{Interference from other sources}: ISM bands shared by other devices and so devices can interfere with
interference occurring in licensed bands due to reuse

\textbf{Multipath propagation}: radio signals reflects off objects ground, arriving at destination at slightly
different times

All these make communication across wireless link difficult, but characteristics depend on the band and
are time-varying.

\textbf{Signal-to-Noise Ratio (SNR)}: larger means easier to extract signal from noise

Given physical layer, increasing power means increasing SNR and therefore decreasing BER.

Given SNR, choosing modulation and coding scheme that meets BER requirement means giving highest
throughput.

SNR may change with mobility, by dynamically adapting physical layer (modulation technique and rate).

By increasing power, interference increases as well.

With multiple wireless senders and receivers, there are additional problems:
\begin{itemize}
\item \uline{hidden terminal problem}: two nodes can talk to some intermediate node but cannot hear each other
\item \uline{signal attenuation}: two nodes have their signal attenuated by some intermediate node
\end{itemize}

Cellular has:
\begin{itemize}
\item wide area coverage, proprietary networks, and more coordination between base stations
\item licensed (expensive) spectrum
\item high mobility
\end{itemize}

Wi-Fi has:
\begin{itemize}
\item local area coverage, last link for Internet, and little coordination between access points
\item unlicensed (free) spectrum (2.4 GHz and 5.3 GHz)
\item low/no mobility
\end{itemize}

In \uline{IEEE standards}, Ethernet is 802.3, Wi-Fi is 802.11, and Wireless PAN (Bluetooth) is 802.15.
\subsection{802.11}
\label{sec:orgf62fbd0}

802.11 Wireless LAN details:
\begin{itemize}
\item \uline{802.11b}: 1999, 11 Mbps, 30 m, 2.4 GHz
\item \uline{802.11g}: 2003, 54 Mbps, 30 m, 2.4 GHz
\item \uline{802.11n} (WiFi 4): 2009, 600 Mbps, 70 m, 2.4/5 GHz
\item \uline{802.11ac} (WiFi 5): 2013, 3.47 Gbps, 70 m, 5 GHz
\item \uline{802.11ax} (WiFi 6): 2020, 14 Gbps, 70 m, 2.4/5 GHz
\item \uline{802.11af}: 2014, 35-560 Mbps, 1 km, unused TV bands
\item \uline{802.11ah}: 2017, 347 Mbps, 1 km, 900 MHz
\end{itemize}

All WiFi use CSMA/CA for multiple access, have access point and ad-hoc network versions, and have
a random access mode (DCF: distributed coordinated function) and polling mode (PCF: point coordination
function).

802.11b's 2.4GHz-2.485 GHz spectrum is divided into 14 channels at different frequencies.
The access point admin chooses a channel for the access point.
Interference is possible since the channel chosen can be the same as the neighbouring access point.

Since channels are at 22MHz bandwidth, channels overlap.
Channels 1, 6, and 11 can operate simultaneously with no interference.

At 5GHz, there are 24 non-overlapping channels.
\subsubsection{Architecture}
\label{sec:org8cc3d4e}
Wireless host communicates with access point.

Basic Service Set (BSS) in infrastructure mode contains wireless hosts and access point.

There is an adhoc mode with no access point, but much less used.

A network administrator allocates a name (SSID) to each access point.

Arriving host must associate with an access point:
\begin{enumerate}
\item scans channels, listening for beacon frames containing SSID and access point's MAC address
\item selects access point to associate with
\item may perform authentication
\item typically run DHCP to get IP address in access point's subnet
\end{enumerate}

\textbf{Passive Scanning}
\begin{enumerate}
\item Beacon frames sent from APs (on different channels)
\item Association Request frame sent: host to selected AP
\item Association Response frame sent: selected AP to host
\end{enumerate}

\textbf{Active Scanning}
\begin{enumerate}
\item Probe Request frame broadcast from host (on all channels)
\item Probe Response frames sent from APs
\item Association Request frame sent: host to selected AP
\item Association Response frame sent: selected AP to host
\end{enumerate}

Passive scan generally takes more time, uses less energy, and if client does not wait long enough
on a channel, it may miss an AP beacon.

802.11 uses CSMA for multiple access (sense before transmitting, but don't collide with ongoing
transmission by other nodes) and no collision detection since difficult to receive when transmitting
due to weak received signals:
\begin{itemize}
\item cannot sense all collisions in any case due to hidden terminal and fading
\item \uline{goal}: avoid collisions with two CA mechanisms
\end{itemize}

For CSMA/CA:
\begin{itemize}
\item 802.11 sender:
\begin{enumerate}
\item if sense channel idle for DIFS, then transmit entire frame (no CD)
\item if sense channel busy, then start random backoff time, timer counts down while channel idle,
then go to 1 when timer expires
\end{enumerate}
\item 802.11 receiver
\begin{enumerate}
\item if frame received OK, return ACK after SIFS<DIFS (ACK needed due to hidden terminal problem and
bad channel conditions)
\item if no ACK, after timeout, increase random backoff interval, try again to transmit starting at
the beginning of the process (max 7 trials)
\end{enumerate}
\end{itemize}

To avoid collisions when hidden terminals, sender reserves channel use for data frames using
small reservation packets:
\begin{itemize}
\item sender first transmits small request-to-send (RTS) packet to AP using CSMA
\begin{itemize}
\item RTSs may still collide with each other (but short)
\end{itemize}
\item AP broadcast clear-to-send CTS (after SIFS) in response to RTS with NAV
\item CTS heard by all nodes (since AP can be heard by everyone)
\begin{itemize}
\item sender transmits data frame
\item other stations defer transmissions
\end{itemize}
\end{itemize}

This approach avoids data frame collisions completely using small reservations packets.

802.11 frame has:
\begin{itemize}
\item \uline{frame control}: 2 bytes
\begin{itemize}
\item protocol version \(\to\) 2 bits
\item type \(\to\) 2 bits, RTS, CTS, ACK, data
\item subtype \(\to\) 4 bits
\item to AP \(\to\) 1 bit
\item from AP \(\to\) 1 bit
\item more frag \(\to\) 1 bit
\item retry \(\to\) 1 bit
\item power management \(\to\) 1 bit
\item more data AP \(\to\) 1 bit
\item WEP \(\to\) 1 bit
\item rsvd \(\to\) 1 bit
\end{itemize}
\item \uline{duration}: 2 bytes, duration of reserved transmission time (NAV in RTS/CTS)
\item \uline{address 1}: 6 bytes, MAC address of wireless host to receive this frame
\item \uline{address 2}: 6 bytes, MAC address of wireless host or AP transmitting this frame
\item \uline{address 3}: 6 bytes, MAC address of router interface to which AP is attached
\item \uline{sequence control}: 2 bytes, frame sequence number for reliable data transfer
\item \uline{address 4}: 6 bytes, used only in ad hoc mode
\item \uline{payload}: 0 to 2312 bytes, datagram or ARP packet, rarely greater than 1500 bytes
\item \uline{CRC}: 4 bytes
\end{itemize}

Issues with mobility are handover and keeping TCP alive (is learning fast enough).

802.11 also has \textbf{power management}:
\begin{itemize}
\item \uline{node-to-AP}: AP knows not to transmit frames to this node, so node wakes up before next beacon
frame (one beacon frame every 100 ms)
\item \uline{beacon frame}: contains list of mobiles with AP-to-mobile frames waiting to be sent
\begin{itemize}
\item node will stay awake if AP-to-mobile frames to be sent, otherwise sleep again until next
beacon frame (wakeup is 250 ms)
\end{itemize}
\end{itemize}

\textbf{Personal Area Network (PAN)}: less than 10 m diameter (short range), low power, low rate, 2.4 GHz,
up to 2 Mbps
\begin{itemize}
\item evolved from Bluetooth specification
\item replacement for cables
\item ad hoc: no infrastructure, 8 active devices at a time, 255 parked
\item master/slaves: a node becomes master
\begin{itemize}
\item master rules, its clock determine time, transmit in odd-numbered slot (625 \(\mu\)s)
\item slaves only transmit to master on even-numbered slot after being talked to
\end{itemize}
\end{itemize}
\end{document}
