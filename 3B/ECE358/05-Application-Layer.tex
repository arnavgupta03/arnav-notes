% Created 2024-12-02 Mon 15:55
% Intended LaTeX compiler: pdflatex
\documentclass[11pt]{article}
\usepackage[utf8]{inputenc}
\usepackage[T1]{fontenc}
\usepackage{graphicx}
\usepackage{longtable}
\usepackage{wrapfig}
\usepackage{rotating}
\usepackage[normalem]{ulem}
\usepackage{amsmath}
\usepackage{amssymb}
\usepackage{capt-of}
\usepackage{hyperref}
\usepackage{parskip,darkmode}
\enabledarkmode
\author{Arnav Gupta}
\date{\today}
\title{Application Layer}
\hypersetup{
 pdfauthor={Arnav Gupta},
 pdftitle={Application Layer},
 pdfkeywords={},
 pdfsubject={},
 pdfcreator={Emacs 29.4 (Org mode 9.7.11)}, 
 pdflang={English}}
\begin{document}

\maketitle
\tableofcontents

\section{Principles of Network Applications}
\label{sec:orgca531dd}
Network applications are programs that run on (different) end systems and communicate over the network.

There is no need to write software for network-core devices since they do not run user applications.
Also, applications on end systems allow for rapid app development and propagation.

In client server paradigm:
\begin{itemize}
\item \uline{server} is an always-on host with a permanent IP address, often in data centers for scaling
\item \uline{clients} (HTTP, IMAP, FTP)
\begin{itemize}
\item contact and communicate with the server
\item may be intermittently connected and have dynamic IP addresses
\item do not communicate directly with each other
\end{itemize}
\end{itemize}

In peer-peer architecture (like P2P file sharing):
\begin{itemize}
\item no always on server so end systems directly communicate
\item peers request service from and provide service to other peers
\begin{itemize}
\item \textbf{self scalability}: new peers bring new service capacity as well as new service demands
\end{itemize}
\item peers are intermittently connected and change IP addresses (so management is complex)
\end{itemize}

\textbf{Process}: program running with a host

With the same host, two processes communicate using inter-process (defined by OS).
In different hosts, processes communicate by exchanging messages.

\textbf{Client process}: process that initiates communication

\textbf{Server process}: process that waits to be contacted

Both are present in P2P architectures.

Application-layer protocol defines:
\begin{itemize}
\item \uline{types of message exchanged}
\item \uline{message syntax}: what fields in messages and how fields are delineated
\item \uline{message semantics}: meaning of info in fields
\item \uline{rules} for when and how processes send and respond to messages
\end{itemize}

Protocols can be:
\begin{itemize}
\item \uline{open protocols}: defined in RFCs, everyone has access to protocol definition, allows for interoperability
\item \uline{proprietary protocols}
\end{itemize}

To choose a transport service, consider:
\begin{itemize}
\item \textbf{data integrity}: can loss be tolerated or is reliability required
\item \textbf{timing}: some apps require low delay to be effectve
\item \textbf{throughput}: does the app require some minimum amount of throughput to be effective
\item \textbf{security}: encryption and data integrity
\end{itemize}
\section{Socket Programming}
\label{sec:orgb1cd792}
\textbf{Socket}: door between application process and end-to-end transport protocol

Sockets can be UDP (unreliable datagram) or TCP (reliable, byte stream-oriented).
\subsection{UDP}
\label{sec:org45c688d}
No connection between client and server (no handshake).

Sender explicitly attaches IP destination address and port number to each packet.
Receiver extracts sender IP address and port number from received packet.

Transmitted data may be lost or received out-of-order.
\subsection{TCP}
\label{sec:org6489d05}
Client must contact running server, which must have created socket to welcome client's contact.

Client contacts server by creating TCP socket with IP address and port number specified.

When contacted by client, server TCP creates new socket for server process to communicate with
that particular client:
\begin{itemize}
\item allows server to talk with multiple clients
\item source port numbers used to distinguish clients
\end{itemize}
\section{Web and HTTP}
\label{sec:orgb5b72df}
Web page consists of objects which can be stored on different web servers,
like base HTML file which includes referenced objects, each addressable by a URL.

\textbf{HTTP: HyperText Transfer Protocol} is the web's application layer protocol:
\begin{itemize}
\item \uline{client}: browser that requests and receives web objects
\item \uline{server}: web server sends objects in response to requests
\end{itemize}

HTTP uses TCP:
\begin{itemize}
\item client initiates TCP connection to server at port 80
\item server accepts TCP connection from client
\item HTTP messages exchanged between client and server
\item TCP connection closed
\end{itemize}

Server maintains no information about past client requests.
(protocols that maintain stat are complex)

\textbf{Non-persistent HTTP}:
\begin{enumerate}
\item TCP connection opened
\item at most 1 object sent over TCP connection
\item TCP connection closed
\end{enumerate}

For non-persistent HTTP,
$$ \text{Response Time} = 2 \text{RTT} + \text{File transmission time} $$
This requires 2 RTTs per object and OS overhead for each TCP connection.
Browsers often open multiple parallel TCP connections to fetch referenced objects in parallel.

\textbf{Persistent HTTP}:
\begin{enumerate}
\item TCP connection opened to a server
\item multiple objects can be sent over a single TCP connection between client and that server
\item TCP connection closed
\end{enumerate}

For persistent HTTP, as little as 1 RTT for all referenced objects (cutting response time in half).

HTTP request message is ASCII (human readable format), with request line (GET, POST, HEAD, etc),
header lines, carriage return and line feed at start of line to indicate end of header lines, and
body.

HTTP request messages can be POST (often form input), GET (query for user data),
HEAD (headers only returned), and PUT (upload new file or replace existing file).

HTTP response message is also ASCII with status line (protocol, status code, status phrase),
header lines, and data (after \texttt{\textbackslash{}r\textbackslash{}n}).

HTTP response status codes include 200 OK, 301 Moved Permanently, 400 Bad Request, 404 Not Found, and
505 HTTP Version Not Supported.
\subsection{QUIC: Quick UDP Internet Connections}
\label{sec:orgdc9e1d3}
Application-layer protocol on top of UDP that increases the performance of HTTP.

Adopts TCP approaches for connection establishment, error control, and congestion control.

Uses multiple application-level streams multiplexed over single QUIC connection, so separate
reliable data transfer and security, as well as common congestion control.

QUIC does a single handshake for TCP and TLS (security/auth) together in one.

With parallelism from QUIC stream, there is no head-of-line blocking (no need to wait for earlier
requests to go through since parallel).
\section{Email, SMTP, IMAP}
\label{sec:org8acc33a}
Major components are user agents, mail servers, and Single Mail Transfer Protocol (SMTP).

\textbf{User agent} is mail reader, for composing, editing, and reading mail messages.

\textbf{Mail server} has mailbox with incoming messages for user and message queue of outgoing mail
messages.

\textbf{SMTP} used between mail servers to send email messages.
\subsection{SMTP}
\label{sec:org96d4860}
Use TCP to reliably transfer email message from client (mail server) to server with port 25.

3 phases of transfer:
\begin{itemize}
\item SMTP handshaking
\item SMTP transfer of messages
\item SMTP closure
\end{itemize}

Command/response interaction with ASCII text commands and status code/phrase response.

As opposed to HTTP, which is client pull SMTP is client push.
In HTTP, each object is encapsulated in its own response message.
In SMTP, multiple objects are sent in multipart message.

SMTP uses persistent connections, requires message to be in 7-bit ASCII, and uses
CRLF.CRLF to determine end of message.

SMTP email message contains header lines (to, from, subject) and body (message).

Internet Mail Access Protocol (IMAP) provides retrieval, deletion, and folders of stored messages
on mail server (retrieval from mail server to user agent).
\section{Domain Name System (DNS)}
\label{sec:org1376803}
Distributed database implemented in hierarchy of many name servers.
An application-layer protocol where hosts and DNS servers communicate to resolve names (translate
from name to address). This is a core Internet function implemented as an application-layer protocol.

Hostname-to-IP address translation to assist with host aliasing (canonical alias names) and mail server
aliasing.
Also helps with load distribution through replicated web servers (many IP addresses correspond to
one name).

Centralizing DNS would cause single point of failure, too high traffic volume, distant centralized DB,
and difficult maintenance. Overall, wouldn't scale.

For DNS, billions of simple records and trillions of queries daily with many more reads than writes.
Must be reliable and secure always, while being organizationally and physically decentralized.

Hierarchy:
\begin{itemize}
\item root server for top level domain DNS server
\begin{itemize}
\item official, contact of last resort by name servers that cannot resolve name
\item managed by ICANN
\item DNSSEC \(\to\) provides security (auth and message integrity)
\item 13 globally
\end{itemize}
\item top level domain DNS server for authoritative DNS servers
\begin{itemize}
\item responsible for TLDs like .com, .org, etc
\end{itemize}
\item authoritative DNS servers for IP address
\begin{itemize}
\item organization's DNS server providing authoritative hostname to IP mappings for organization's named
hosts
\item can be managed by organization or service provider
\end{itemize}
\end{itemize}

DNS query is sent to local DNS server, which returns reply answering from local cache or forwarding
request into DNS hierarchy.

Each ISP has local DNS server (does not strictly belong to hierarchy).

\textbf{Iterated query}: contacted server replies with name of server to contact

\textbf{Recursive query}: puts burden of name resolution on contacted name server, can cause heavy load at
upper levels of hierarchy

Once any name server learns mapping, this is cached and the cached mapping is returned.
Improves response time and cache entries timeout after some TTL.
TLD servers typically cached in local name servers.

Cached entries can be out of date, so DNS is best effort.

DNS records have format (\texttt{name}, \texttt{value}, \texttt{type}, \texttt{ttl}) and can have different types:
\begin{itemize}
\item \uline{type A}: \texttt{name} is hostname, \texttt{value} is IP address
\item \uline{type NS}: \texttt{name} is domain, \texttt{value} is hostname of authoritative name server for domain
\item \uline{type CNAME}: \texttt{name} is alias name, \texttt{value} is canonical name
\item \uline{type MX}: \texttt{value} is name of SMTP mail server associated with \texttt{name}
\end{itemize}

DNS query and reply messages have the same format:
\begin{itemize}
\item \uline{message header}: identification (16 bit number for query and reply) and flags (query or reply,
recursion desired, recursion available, reply is authoritative)
\item \uline{numbers of questions, answer records, authority records, additional records}
\item \uline{questions}
\item \uline{answers}
\item \uline{authority}
\item \uline{additional info}
\end{itemize}

To get info into DNS, register name at DNS registrar by providing names and IP addresses of authoritative
name server (primary and secondary).
Then, create authoritative server locally with IP address.
\end{document}
