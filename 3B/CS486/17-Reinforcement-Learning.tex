% Created 2024-11-27 Wed 15:44
% Intended LaTeX compiler: pdflatex
\documentclass[11pt]{article}
\usepackage[utf8]{inputenc}
\usepackage[T1]{fontenc}
\usepackage{graphicx}
\usepackage{longtable}
\usepackage{wrapfig}
\usepackage{rotating}
\usepackage[normalem]{ulem}
\usepackage{amsmath}
\usepackage{amssymb}
\usepackage{capt-of}
\usepackage{hyperref}
\usepackage{parskip,darkmode}
\usepackage{algorithm}
\usepackage{algpseudocode}
\enabledarkmode
\author{Arnav Gupta}
\date{\today}
\title{Reinforcement Learning}
\hypersetup{
 pdfauthor={Arnav Gupta},
 pdftitle={Reinforcement Learning},
 pdfkeywords={},
 pdfsubject={},
 pdfcreator={Emacs 29.4 (Org mode 9.7.11)}, 
 pdflang={English}}
\begin{document}

\maketitle
\tableofcontents

\section{Reinforcement Learning}
\label{sec:org383568b}
What an agent should do given some:
\begin{itemize}
\item \uline{prior knowledge}: possible states of the worlds and possible actions
\item \uline{observations}: current world state and immediate reward/punishment
\item \uline{goal}: act to maximize accumulated reward
\end{itemize}

Assume some sequence of \textbf{experiences} with a state, action, and reward.
For the agent to decide what to do next, it must decide whether to explore to gain more knowledge
or exploit the knowledge it has already discovered.

Reinforcement learning is difficult due to the \uline{credit assignment problem}, which is what actions
are responsible for the reward but may have occurred a long time before the reward was received
(long-term effect of an action).
This leads to the \uline{explore-exploit dilemma}: when the agent should be greedy vs inquisitive.
\subsection{Main Approaches}
\label{sec:orgb8de291}
Can search through a space of policies.

\textbf{Model Based RL}: learn a model consisting of state transition function \(P(s' \mid a, s)\) and
reward function \(R(s, a, s')\) and solve this as an MDP

\textbf{Model-Free RL}: learn \(Q^{*}(s,a)\) and use to guide action
\section{Q-Learning}
\label{sec:orga31b221}
\subsection{Temporal Differences}
\label{sec:org751606c}
Consider some sequence \(v_{1}, v_{2}, v_{3}, \dots\) and take a running estimate of the first \(k\)
values:
$$ A_{k} = \frac{v_{1} + \cdots + v_{k}}{k} $$
when a new value arrives
$$ A_{k} = \frac{k-1}{k} A_{k-1} + \frac{1}{k} v_{k} $$
The \textbf{TD formula} is that, for \(\alpha = \frac{1}{k}\)
$$ A_{k} = A_{k-1} + \alpha (v_{k} - A_{k-1}) $$
\subsection{Q-Learning}
\label{sec:org0463543}
Store \(Q\){[}State, Action] and update this as in asynchronous value iteration, but using
experience (empirical probabilities and rewards).

Suppose the agent has an experience \(\left< s, a, r, s' \right>\), which provides a datum to
update \(Q[s, a]\).
\(\left< s, a, r, s' \right>\) provides the data point
$$ r + \gamma \max_{a'} Q[s', a'] $$
which can be used in the TD formula giving
$$ Q[s, a] \gets Q[s, a] + \alpha \left( r + \gamma \max_{a'} Q[s', a'] - Q[s, a] \right) $$

\begin{algorithm}
\caption{Q-learning Algorithm}
\begin{algorithmic}[1]
\State Initialize \( Q[S, A] \) arbitrarily
\State Observe current state \( s \)
\Repeat
    \State Select and carry out an action \( a \)
    \State Observe reward \( r \) and state \( s' \)
    \State \( Q[s, a] \gets Q[s, a] + \alpha \left( r + \gamma \max_{a'} Q[s', a'] - Q[s, a] \right) \)
    \State \( s \gets s' \)
\Until{forever}
\end{algorithmic}
\end{algorithm}
\subsection{Properties}
\label{sec:orgc6fddc4}
Q-learning converges to the optimal policy, no matter what the agent does, as long as it tries
each action in each state enough (infinitely often).

The agent can either exploit (select the action that maximizes \(Q[s, a]\)) or
explore (select another action).
\section{Strategies for Reinforcement Learning}
\label{sec:org6e12644}
\subsection{Exploration Strategies}
\label{sec:org3ecf1d9}
The \textbf{\(\epsilon\) greedy strategy} is to choose a random action with probability \(\epsilon\) and choose
a best action with probability \(1-\epsilon\).

In \textbf{softmax action selection}, starting from state \(s\), choose an action \(a\) with probability
$$ \frac{e^{Q[s, a] / \tau}}{\sum_{a} e^{Q[s, a]} / \tau} $$
where \(\tau > 0\) is the temperature.
Good actions are chosen more often than bad actions and for \(\tau \to \infty\) all actions are
equally probably and for \(\tau \to 0\), only the best action is chosen.

For optimism in the face of uncertainty, initialize \(Q\) to values that encourage exploration.

The \textbf{upper confidence bound (UCB)}, store \(N[s, a]\), which is th number of times that the
state-action pair has been tried) and use
$$ \text{arg} \max_{a} \left[ \right] $$
where \(N[s] = \sum_{a} = N[s, a]\).
\subsection{Off/On-Policy Learning}
\label{sec:orgeaea476}
Q-learning does \textbf{off-policy} learning, where it learns the value of the optimal policy, no matter
what it does, but this could be bad if the exploration policy is dangerous.

\textbf{On-policy} learning learns the value of the policy being followed.

If the agent will explore, it may be better to optimize the actual policy it is going to do.
\subsection{SARSA}
\label{sec:org3905e30}
Uses the experience \(\left< s, a, r, s', a' \right>\) to update \(Q[s, a]\).
\textbf{On-policy} since it uses empirical values for \(s'\) and \(a'\).

\begin{algorithm}
\caption{SARSA Algorithm}
\begin{algorithmic}[1]
\State Initialize $Q[s, a]$ arbitrarily
\State Observe current state $s$
\State Select action $a$ using a policy based on $Q$ (includes exploration)
\Repeat
    \State Carry out action $a$
    \State Observe reward $r$ and state $s'$
    \State Select action $a'$ using a policy based on $Q$
    \State $Q[s,a] \gets Q[s,a] + \alpha (r + \gamma Q[s',a'] - Q[s,a])$
    \State $s \gets s'$
    \State $a \gets a'$
\Until{forever}
\end{algorithmic}
\end{algorithm}
\subsection{Model-Based}
\label{sec:org7771393}
Uses the experiences more effectively.
Used when collecting experiences is expensive and can do lots of computation between
each experience.

Idea is to learn the MDP and interleave acting and planning.

After each experience, update probabilities and the reward, then do some steps of
asynchronous value iteration.

\begin{algorithm}
\caption{Model-based Learner}
\begin{algorithmic}[1]
\State \textbf{Data Structures:}
\State \quad $Q[S,A]$  \Comment{Action-value function}
\State \quad $T[S,A,S]$  \Comment{State transition counts}
\State \quad $R[S,A]$  \Comment{Reward function}
\State Assign $Q$, $R$ arbitrarily, $T \gets$ prior counts
\State $\alpha \gets$ learning rate
\State Observe current state $s$
\Repeat
    \State Select and carry out action $a$
    \State Observe reward $r$ and state $s'$
    \State $T[s, a, s'] \gets T[s, a, s'] + 1$
    \State $R[s, a] \gets \alpha \times r + (1 - \alpha) \times R[s, a]$
\Until{done}
\State \textbf{repeat for a while (asynchronous VI):}
\State Select state $s_1$, action $a_1$
\State Let $P = \sum_{s_2} T[s_1, a_1, s_2]$
\State $Q[s_1, a_1] \gets \frac{\sum_{s_2} T[s_1, a_1, s_2]}{P} \times (R[s_1, a_1] + \gamma \max_{a_2} Q[s_2, a_2])$
\State $s \gets s'$
\State \textbf{end repeat}
\end{algorithmic}
\end{algorithm}

Let \(s = (x_{1}, \dots, x_{N})^{T}\) where \(x\) are features,
the \(Q\) function can be approximated linearly as
$$ Q_{w}(s,a) \approx \sum_{i} w_{ai}x_{i} $$
and non-linearly as
$$ Q_{w} (s,a) \approx g(x ; w) $$

For experience \(\left< s, a, r, s', a' \right>\),
the target Q-function is
$$ R(s) + \gamma \max_{a} Q_{w}(s', a) = R(s) + \gamma Q_{w}(s', a') $$
and the current Q-function is
$$ Q_{\bar{w}}(s, a) $$

The squared error is
$$ \text{Err}(w) = \frac{1}{2} \left[ Q_{w}(s, a) - R(s) - \gamma \max_{a'} Q_{\bar{w}} (s', a') \right]^{2} $$

The gradient is
$$ \frac{\partial \text{Err}}{\partial w} = \left[ Q_{w}(s, a) - R(s) - \gamma \max_{a'} Q_{\bar{w}} (s', a') \right] \frac{\partial Q_{w(s, a)}}{\partial w} $$

For SARSA with linear function approximation
\begin{algorithm}
\caption{SARSA with Linear Function Approximation}
\begin{algorithmic}[1]
\State \textbf{Input:} Discount factor $\gamma$, Learning rate $\alpha$
\State Assign weights $w = \langle w_0, \dots, w_n \rangle$ arbitrarily
\State \textbf{begin}
\State \quad Observe current state $s$
\State \quad Select action $a$
\Repeat
    \State \quad Carry out action $a$
    \State \quad Observe reward $r$ and state $s'$
    \State \quad Select action $a'$ (using a policy based on $Q_w$)
    \State \quad Let $\delta = r + \gamma Q_w(s', a') - Q_w(s, a)$
    \For{$i = 0$ \textbf{to} $n$}
        \State \quad $w_i \gets w_i + \alpha \times \delta \times \frac{\partial Q_w(s, a)}{\partial w_i}$
    \EndFor
    \State \quad $s \gets s'$; $a \gets a'$
\Until{forever}
\State \textbf{end}
\end{algorithmic}
\end{algorithm}
\section{Convergence and Divergence}
\label{sec:org8fe349d}
Linear Q-learning converges under the same conditions as Q-learning
$$ w_{i} \gets w_{i} + \alpha [Q_{w}(s, a) - R(s) - \gamma Q_{w}(s', a')] x_{i} $$

Nonlinear Q-learning may diverge, as adjusting \(w\) to increase \(Q\) at \((s, a)\) might
introduce errors at nearby state-action pairs.

To mitigate divergence, can use \textbf{experience relay} or \textbf{two Q functions} (Q network or
target network).
\subsection{Experience Relay}
\label{sec:org6a78022}
Store previous experiences \((s, a, r, s', a')\) in a buffer and sample a mini-batch
of previous experiences at each step to learn by Q-learning.

This breaks correlations between successive updates, so learning is more stable.

A few interactions with the environment are needed to converge (greater data
efficiency).
\subsection{Target Network}
\label{sec:orged2c8f0}
Use a separate target network that is updated only periodically.
The target network has weights \(\bar{w}\) and computes \(Q_{\bar{w}}(s, a)\).

This should be repeated for each \((s, a, r, s', a')\) in a mini-batch:
$$ w \gets w + \alpha [Q_{w}(s, a) - R(s) - \gamma Q_{\bar{w}}(s', a')] \frac{\partial Q_{w}(s, a)}{\partial w} $$
Then \(\bar{w} = w\).
\subsection{Deep Q Network}
\label{sec:org2bc9089}
\begin{algorithm}
\caption{Q-learning with Experience Replay}
\begin{algorithmic}[1]
\State Assign weights \( w = \langle w_0, \dots, w_n \rangle \) at random in \([-1, 1]\)
\State \textbf{begin}
\State \quad Observe current state \( s \)
\State \quad Select action \( a \)
\Repeat
    \State \quad Carry out action \( a \)
    \State \quad Observe reward \( r \) and state \( s' \)
    \State \quad Select action \( a' \) using a policy based on \( Q_w \)
    \State \quad Add \( (s, a, r, s', a') \) to experience buffer
    \State \quad Sample mini-batch of experiences from buffer
    \For{each experience \( (\hat{s}, \hat{a}, \hat{r}, \hat{s'} , \hat{a'}) \) in mini-batch}
        \State \quad Let \( \delta = \hat{r} + \gamma Q_w(\hat{s'}, \hat{a'}) - Q_w(\hat{s}, \hat{a}) \)
        \State \quad \( w \gets w + \alpha \times \delta \times \frac{\partial Q_w(\hat{s}, \hat{a})}{\partial w} \)
    \EndFor
    \State \quad \( s \gets s' \); \( a \gets a' \)
    \If{every \( c \) steps}
        \State \quad Update target: \( w \gets w \)
    \EndIf
\Until{forever}
\State \textbf{end}
\end{algorithmic}
\end{algorithm}
\subsection{Bayesian Reinforcement Learning}
\label{sec:orgc07415e}
Include the parameters (transition function and observation function) in the state space.

Model-based learning is done through inference (belief state).

The state space becomes continuous but the belief space is a space of continuous functions.

Can mitigate complexity by modeling reachable beliefs.

This gives optimal exploration-exploitation tradeoff.
\end{document}
