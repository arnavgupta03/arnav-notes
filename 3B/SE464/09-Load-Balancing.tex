% Created 2024-12-07 Sat 19:05
% Intended LaTeX compiler: pdflatex
\documentclass[11pt]{article}
\usepackage[utf8]{inputenc}
\usepackage[T1]{fontenc}
\usepackage{graphicx}
\usepackage{longtable}
\usepackage{wrapfig}
\usepackage{rotating}
\usepackage[normalem]{ulem}
\usepackage{amsmath}
\usepackage{amssymb}
\usepackage{capt-of}
\usepackage{hyperref}
\usepackage{parskip,darkmode}
\enabledarkmode
\author{Arnav Gupta}
\date{\today}
\title{Load Balancing}
\hypersetup{
 pdfauthor={Arnav Gupta},
 pdftitle={Load Balancing},
 pdfkeywords={},
 pdfsubject={},
 pdfcreator={Emacs 29.4 (Org mode 9.7.11)}, 
 pdflang={English}}
\begin{document}

\maketitle
\tableofcontents

\section{Load Balancers}
\label{sec:orgb83b1c3}
Single point of contact for a service that forwards requests to a cluster of workers and remembers
request source.
Load balancer can relay requests for 10s-100s of applications servers.

One IP address appears like a big machine, but is actually a cluster.

Load balancer provides the same interface as a single server.

Individual servers can be replaced without affecting overall service.
Proxy can monitor health of servers by periodically sending health check, and if
request fails, server must be crashed (stop relaying requests there).

2 types of local load balancers:
\begin{itemize}
\item \textbf{Network Address Translation}
\begin{itemize}
\item works at TCP/IP layer (layer 4 load balancer)
\item forwards packets one-by-one but remembers which server was assigned to each client
\item changes IP addresses and ports of packets in both directions (maintains mappings)
\begin{itemize}
\item more efficient since need not implement TCP, HTTP, or store full requests/responses
\end{itemize}
\item compatible with any type of service, not just HTTP
\item more scalable
\end{itemize}
\item \textbf{Reverse Proxy}
\begin{itemize}
\item works at HTTP layer (application layer load balancer)
\item stores full requests/response before forwarding
\item can also do SSL termination, caching, and compression
\begin{itemize}
\item TLS/SSL certificates stored just on proxy (internal communication unencrypted)
\item proxy might cache responses, but limits scalability
\end{itemize}
\item less scalable
\end{itemize}
\end{itemize}

Cloud-based load balancers can do either type of load balancing.

Load balancer is single point of failure and has limited throughput.
\section{Domain Name Service}
\label{sec:org99e9e3d}
Distributed directory that maps hostnames to IP addresses.
Uses hierarchical caching architecture for scalability.

Local DNS resolver has cached copies of recent answers.
If no answer, ask up the hierarchy (go to nameserver).

DNS allows multiple answers to be given for a query (client chooses or DNS server can give different
responses to different users).
Each IP address is a different reverse-proxying load balancer in front of many app servers (scaling
by DNS).

DNS can also connect user to closes replica of service (geographically) using IP address geolocation
of requester.
\section{Content Delivery Network (CDN)}
\label{sec:orge76361a}
Globally distributed web servers that cache responses for local clients.

Distributed caching HTTP reverse proxy that uses DNS to geographically load-balance.

\textbf{Origin server}: original, central web server (sets \texttt{Cache Control} HTTP header in response)

Edge servers are \textbf{caching proxies} and ask origin server if no cached response.
This is where CDNs are.

8.8.8.8 is where Google's DNS is.

IP Anycast load balancing implemented with BGP.
Traffic destined for 8.8.8.8 is sent to whichever of these Google routers are closest to the customer.
Violates principle that IP address is a particular destination, but for DNS this doesn't matter since
UDP and stateless.

DNS vs IP Anycast:
\begin{itemize}
\item DNS TTL has slow changes (minutes to hours) but IP Anycast uses BGP convergence
\item DNS required advanced DNS software/config but IP Anycast requires operating own Autonomous System (ISP)
\end{itemize}

For global load balancing, DNS is the most common choice.

Tech giants can use IP Anycast.

For local load balancing:
\begin{itemize}
\item routing done by NAT or HTTP proxy
\item scale limited by speed of one machine
\item changes can be made in milliseconds
\item simple deployment using off-the-shelf software and hardware
\end{itemize}

For global load balancing:
\begin{itemize}
\item routing done by DNS or BGP
\item scale limited by Internet itself
\item changes can be made in minutes to hours
\item deployment requires advanced DNS software/config
\end{itemize}

Most large services load balance locally and at DNS:
\begin{itemize}
\item local provides continuous operation and scale within a data center (health checks, rolling updates)
\item DNS for global scalability and low latency
\end{itemize}
\end{document}
