% Created 2024-12-05 Thu 09:29
% Intended LaTeX compiler: pdflatex
\documentclass[11pt]{article}
\usepackage[utf8]{inputenc}
\usepackage[T1]{fontenc}
\usepackage{graphicx}
\usepackage{longtable}
\usepackage{wrapfig}
\usepackage{rotating}
\usepackage[normalem]{ulem}
\usepackage{amsmath}
\usepackage{amssymb}
\usepackage{capt-of}
\usepackage{hyperref}
\usepackage{parskip,darkmode}
\enabledarkmode
\author{Arnav Gupta}
\date{\today}
\title{Non Functional Requirements}
\hypersetup{
 pdfauthor={Arnav Gupta},
 pdftitle={Non Functional Requirements},
 pdfkeywords={},
 pdfsubject={},
 pdfcreator={Emacs 29.4 (Org mode 9.7.11)}, 
 pdflang={English}}
\begin{document}

\maketitle
\tableofcontents

\section{Requirements Analysis}
\label{sec:orgd181184}
Architecture analysis involves requirement engineering.

Requirements come from users and stakeholders who have demands/needs.

A requirement engineer gathers raw requirements, analysis for consistency,
feasibility, and completeness, writes a specification, and validates that
gathered requirements reflect stakeholder needs.

Considerations during requirement gathering:
\begin{itemize}
\item business need vs requirement
\item nice-to-have vs must-have
\item system goal vs contractual requirement
\end{itemize}

Types of Requirements
\begin{itemize}
\item \textbf{Functional Requirements}
\begin{itemize}
\item specify function of system
\item record, compute, transform, transmit
\end{itemize}
\item \textbf{Non-Functional Requirements}
\begin{itemize}
\item \uline{Quality Requirements}
\begin{itemize}
\item Specify how well system performs intended functions
\item performance, usability, maintenance, reliability, portability
\end{itemize}
\item \uline{Managerial Requirements}
\begin{itemize}
\item when to deliver
\item verification
\item legal responsibilities if things go wrong
\end{itemize}
\item \uline{Context/Environment Requirements}
\begin{itemize}
\item range of conditions in which system should operate
\end{itemize}
\end{itemize}
\end{itemize}

Also data requirements handle system state and IO.

\textbf{Software Specification}: bridge between real-world environment (demands
and needs of stakeholders) and software system

\textbf{System Perspective}: block diagram that describes system boundaries, users,
and other interfaces

A good requirement spec is:
\begin{itemize}
\item \uline{correct}: requirement reflects need
\item \uline{complete}: necessary requirements included
\item \uline{unambiguous}: all parties agree on meaning
\item \uline{consistent}: all parties match
\item \uline{ranked for importance and stability}: priority and expected changes per
requirement
\item \uline{modifiable}: easy to change, maintaining consistency
\item \uline{verifiable}: possible to see whether requirements met
\item \uline{traceable}: to goals/purposes and to design/code
\item \uline{necessary and feasible}
\end{itemize}
\section{Non-Functional Requirements}
\label{sec:org925df67}
Often called quality attributes, specify how well system performs functions like
speed, ease of use, security, and maintainability.

Non-functional requirements can be mandatory or not mandatory.
Become more mandatory as market matures.

NFRs should be clear, concise, and measurable.
NFR design requires customers and developers.

NFRs affect many high level subsystems, so must be considered during software
architecture design.

Difficult to modify NFR once architecture complete.
\end{document}
