% Created 2024-12-07 Sat 01:42
% Intended LaTeX compiler: pdflatex
\documentclass[11pt]{article}
\usepackage[utf8]{inputenc}
\usepackage[T1]{fontenc}
\usepackage{graphicx}
\usepackage{longtable}
\usepackage{wrapfig}
\usepackage{rotating}
\usepackage[normalem]{ulem}
\usepackage{amsmath}
\usepackage{amssymb}
\usepackage{capt-of}
\usepackage{hyperref}
\usepackage{parskip,darkmode}
\enabledarkmode
\author{Arnav Gupta}
\date{\today}
\title{Stateless, Proxies, and Caching}
\hypersetup{
 pdfauthor={Arnav Gupta},
 pdftitle={Stateless, Proxies, and Caching},
 pdfkeywords={},
 pdfsubject={},
 pdfcreator={Emacs 29.4 (Org mode 9.7.11)}, 
 pdflang={English}}
\begin{document}

\maketitle
\tableofcontents

\section{Stateless}
\label{sec:orge27eae5}
The worker that is chosen to handle request can depend on application code.

A \textbf{stateless} thread/process/service remembers nothing from past requests:
\begin{itemize}
\item behaviour determined entirely by input request and request handling code
\item different copies of service are running same code, so they give exact same
response for a given request
\item has no \uline{local} state
\item no long term memory (pure function)
\item not affected by previous inputs
\end{itemize}

A \textbf{stateful} thread/process/service changes over time as side effect of
handling requests:
\begin{itemize}
\item persistent, global variables are modified by the request processing code
\end{itemize}

OOP purposes:
\begin{itemize}
\item \uline{inheritance}: allows strong typing without losing abstraction, creates
generic, abstract interfaces, enabling abstraction
\item \uline{modeling}: real world concepts
\item \uline{grouping} sets of related state (memory/variable)
\item well-defined, \uline{limited side effects}
\begin{itemize}
\item a class defines a set of member functions whose side-effects are limited
to a small set of variables (object's data members)
\end{itemize}
\end{itemize}

Parallelism is easy with stateless code, but difficult with stateful code (
related request from the same client must go to the same handler).

DB and cookies can help make stateless applications keep some state.

With cookies, a server returns a cookie (for example, after login), the browser
includes the cookie in all future HTTP requests, and the server uses the cookie
to determine which user it came from.

Statelessness allows for load balancing, since coordination only happens from
shared DBs.

Push state \uline{up} to client or \uline{down} to DB.
\section{Proxies and Caching}
\label{sec:orgc859192}
\textbf{Proxy server}: intermediate router for requests

Proxy does not know how to answer requests but knows who to ask.
Request relayed to another server and response relayed back.

Proxies can be transparently added to any stateless service.

\textbf{Load balancer}: type of proxy that creates a single point of contact
for a large cluster of app servers

\textbf{Caching proxy}: stores recently retrieved items for reuse

\textbf{Cache}: small data storage structure designed to improve performance when
accessing a large data store
\begin{itemize}
\item stores \uline{most recently} or \uline{most frequently} accessed data
\end{itemize}

When reading data:
\begin{enumerate}
\item check cache, if it has data, \uline{cache hit}
\begin{enumerate}
\item record in cache that entry was accessed
\end{enumerate}
\item if data not in cache, \uline{cache miss}
\begin{enumerate}
\item get data from main data store
\item make room by evicting another data element
\item store data in cache and repeat step 1
\end{enumerate}
\end{enumerate}

Most common eviction policy is LRU.

\textbf{Managed cache}:
\begin{itemize}
\item client has direct access to small and large data store
\item client responsible for implementing caching logic
\end{itemize}

\textbf{Transparent cache}:
\begin{itemize}
\item client connects to one data store
\item caching implemented inside storage black box
\end{itemize}

Data writes cause cache to be out of date.
Solutions to this are:
\begin{itemize}
\item expire cache entries after some TTL (time to live)
\item after writes, send new data or invalidation message to all caches,
creating \uline{coherent cache} (adds performance overhead)
\item use \uline{versioned data}: create new filename every time new data
added
\end{itemize}

HTTP is stateless, so same response can be saved and reused for
repeat requests:
\begin{itemize}
\item GET requests easy to cache (no modification)
\item Cache-Control header for both client and server to enable/disable
caching and control expiration time
\item HTTP caching proxy is compatible with any web server and can be
transparently added
\end{itemize}
\end{document}
