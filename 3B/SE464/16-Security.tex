% Created 2024-12-10 Tue 23:49
% Intended LaTeX compiler: pdflatex
\documentclass[11pt]{article}
\usepackage[utf8]{inputenc}
\usepackage[T1]{fontenc}
\usepackage{graphicx}
\usepackage{longtable}
\usepackage{wrapfig}
\usepackage{rotating}
\usepackage[normalem]{ulem}
\usepackage{amsmath}
\usepackage{amssymb}
\usepackage{capt-of}
\usepackage{hyperref}
\usepackage{parskip,darkmode}
\enabledarkmode
\author{Arnav Gupta}
\date{\today}
\title{Security}
\hypersetup{
 pdfauthor={Arnav Gupta},
 pdftitle={Security},
 pdfkeywords={},
 pdfsubject={},
 pdfcreator={Emacs 29.4 (Org mode 9.7.11)}, 
 pdflang={English}}
\begin{document}

\maketitle
\tableofcontents

\section{Security}
\label{sec:orgb64db43}
The protection afforded to an automated information system in order to attain applicable
objectives of preserving integrity, availability, and confidentiality of info
system resources.

\textbf{Confidentiality}: preventing unauthorized parties from accessing info or even being
aware of the existence of info

\textbf{Integrity}: only authorized parties can manipulate info in authorized ways

\textbf{Availability}: resources accessible by authorized parties on all appropriate
occasions

Must check before performing requested action:
\begin{itemize}
\item \uline{authenticity}: is agent's claimed identity authentic or is someone
masquerading as agent
\item \uline{integrity}: is the request the one the agent made or did someone tamper
with it
\item \uline{authorization}: has proper authority granted permission to this agent to
perform this action
\end{itemize}
\subsection{Guard Model}
\label{sec:org79c668e}
\textbf{Complete mediation}: every request for the resource goes through the guard

Guard performs authentication (is principal who they claim to be) and
authorization (does principal have access to performance request on
resource).

Guard model can suffer from adversarial (targeted) attacks, where one
successful attack can bring down a system.

Securing a system starts by specifying goals (\uline{policy}) and
assumptions (\uline{threat model}).

Even though things can go wrong, systems that use guard model avoid
common pitfalls.
\subsection{Threats}
\label{sec:org28d9622}
Potential security violations caused either by a planned attack by an
adversary or unintended mistakes by legitimate users.

3 broad categories of threats:
\begin{itemize}
\item \textbf{unauthorized info release}: unauthorized person can read and take
advantage of info stored in computer or being transmitted over networks
\begin{itemize}
\item extends to traffic analysis, where adversary observes only patterns
of info use and from those patterns can infer info content
\end{itemize}
\item \textbf{unauthorized info modification}: unauthorized person can make
changes in stored info or modify messages that cross a network
\begin{itemize}
\item adversary might engage in behaviour to sabotage system or trick
receiver of message to divulge useful info or take unintended action
\item does not require that adversary be able to see info it changes
\end{itemize}
\item \textbf{unauthorized denial of use}: adversary prevents authorized user
from reading or modifying info, even though adversary cannot either
\end{itemize}
\subsection{Goals and Principles}
\label{sec:org59ac928}
Narrow view of security can miss preventing some unauthorized actions.

A system's security can be insecure, secure, or unknown.
\begin{itemize}
\item to prove insecure, find security hole
\item to prove secure means to show there can be no security hole
\item typical outcome is unknown security holes exist
\end{itemize}
\subsubsection{Design Principles}
\label{sec:org37c38a5}
\textbf{Least Privilege}: give each component only the privileges it
requires

\textbf{Fail-safe Defaults}: deny access if explicit permission is absent

\textbf{Economy of Mechanism}: adopt simple security mechanisms

\textbf{Complete Mediation}: ensure every access is permitted

\textbf{Open Design}: do not rely on secrecy for security

\textbf{Separation of Privilege}: introduce multiple parties to avoid
exploitation of privileges

\textbf{Least Common Mechanism}: limit critical resource sharing to
only a few mechanisms

\textbf{Psychological Acceptability}: make security mechanisms usable

\textbf{Defense in Depth}: have multiple layers of countermeasures
\section{Authentication}
\label{sec:org8f305df}
Establish the following:
\begin{itemize}
\item who is the principal making the request?
\begin{itemize}
\item guard must establish origin of the message
\end{itemize}
\item is the request the one that the principal made
\begin{itemize}
\item guard must establish integrity of the message
\end{itemize}
\end{itemize}

Steps to authenticate:
\begin{enumerate}
\item real-world person configures the guard
\begin{enumerate}
\item agreeing on a method by which the guard can later
identify the principal identifier for the person
\end{enumerate}
\item identity verification, occurs at later times using the
agreed upon method
\end{enumerate}

Verification methods:
\begin{itemize}
\item unique physical property of the user (faces, voices, fingerprints)
\item unique thing the user has (ID)
\item unique thing the user knows (password)
\end{itemize}

Most apps require users to set a password to reconnect to an account:
\begin{itemize}
\item \uline{salted hash} of password is stored in DB
\item passwords in future login can be compared to stored hash
\end{itemize}

Email address or phone number can also be used to prove identity.

Clients make requests to access user data and access must be
protected.

HTTP requests should be sent using HTTPS (TLS): encrypts data in
transit.
Request/response data cannot be intercepted:
\begin{itemize}
\item TLS authenticates server using certificates
\item TLS does not authenticate client
\end{itemize}

Can send password in each request, but storing the password locally
is a security risk since all backend apps see the password (potentially
in logs too).
Adversary with access to server can see it.

Hash should be easy to compute, but it should be
\begin{itemize}
\item difficult to compute unhashed value from hashed value
\item difficult to find another input with the same hashed value
\item hashed value as short as possible, but long enough for low
collision probability
\end{itemize}

Risk of \textbf{dictionary attack}: adversary compiles list of passwords
and computes cryptographic has of these strings and compares result
to value stored in computer system or uses computer program that
tries login with these strings.

Using salt means that adversary would need separate rainbow table
for every possible salt.

To avoid transmitting passwords repeatedly, server sends cookie
which can be used to authenticate for some period of time.

Adversaries can create their own cookies, even with hash or
server key.

Best to use session key:
\begin{itemize}
\item when user logs in, backend generates random session key, stores
it in user account DB, and returns it to client
\item client includes session key in all future requests, often in a header
\item every request handler then checks auth token
\end{itemize}

Client device might send sign in request directly to auth service.
Other microservices ask auth service to check auth tokens.
\subsection{API Keys}
\label{sec:orgc0dba23}
Auth token that's valid for a long time.
Used for 3rd parties to gain programmatic access to system.
Can also be used for backend microservices to authenticate with each other,
using local cache to check quickly without reading from DB.

API Key can be in query params, headers, or body.
Headers are especially good since they are separate from request-specific
params.
\subsection{Phishing}
\label{sec:org757622b}
To protect against phishing attacks, must avoid sending password to server
directly but still allow valid servers to authenticate users.

\textbf{Challenge response protocol}: password never sent directly, instead it is
hashed with random number
\begin{itemize}
\item adversary-owned servers will only ever learn hashed password, which the
password cannot be recovered from
\end{itemize}
\subsubsection{Two Factor Authentication}
\label{sec:org12fe8fc}
For added security, some services require more than a password, like a
random challenge code (to email or SMS).

Passwords can be misused, so some services use email exclusively for
login.
\section{Authorization}
\label{sec:orgde00cf1}
\textbf{Architecture Access Control Models}: decide whether access to a protected resource should be
granted or denied

\textbf{Discretionary access control}: based on requestor's identity, resource, and whether
requestor has permission to access

Mandatory access control is policy based.
\subsection{Ticket System}
\label{sec:orgf83a122}
Each guard holds a separate ticket for each object it is guarding.

Principal holds a separate ticket for each different object it is authorized to use.

To authorize principal for access, authority gives principal a matching ticket for object.
Principal can pass ticket to other principals.

To revoke principal's permissions, authority must hunt down principal and take ticket back
or change guard's ticket and reissue tickets to any other authorized principals.
\subsection{List System}
\label{sec:org1d99523}
Principal has token identifying principal and guard holds  list of tokens that correspond
to set of principals that authority has authorized.

To revoke access, authority removes principal's token from guard's list.
\subsection{Mandatory Access Control}
\label{sec:orgadbe78f}
Every user has some access level and can access info at their level and below.
\subsection{Known-Key Attacks}
\label{sec:org2d88ed8}
Adversary steals key to communicate or get info.
\subsection{Replay Attacks}
\label{sec:orgb0cd52c}
Same as known-key but with replaying the same session.
\subsection{Misrepresentation}
\label{sec:org12e78e0}
Make an authorized user seem unreliable.
\subsection{Collusion}
\label{sec:orgdfc9696}
Same as misrepresentation, except done by multiple adversaries.
\subsection{Fraudulent Actions}
\label{sec:orge6d7df7}
Adversary does not fulfill their part of a contract.
\subsection{Addition of Unknowns}
\label{sec:orgbf9a090}
Unsure about safety of interacting with new entrants into the system.
\subsection{Trust Management}
\label{sec:org2d41dc4}
\textbf{Trust}: level of subjective probability with which agent assesses
that another agent will perform a particular action in a context that affects their
actions

\textbf{Reputation}: expectation about entity's behaviour based on past behaviour, can be
used to determine trust

2 types of trust management systems:
\begin{itemize}
\item credential and policy-based
\item reputation-base
\end{itemize}
\end{document}
