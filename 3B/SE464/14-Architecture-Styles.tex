% Created 2024-12-10 Tue 19:32
% Intended LaTeX compiler: pdflatex
\documentclass[11pt]{article}
\usepackage[utf8]{inputenc}
\usepackage[T1]{fontenc}
\usepackage{graphicx}
\usepackage{longtable}
\usepackage{wrapfig}
\usepackage{rotating}
\usepackage[normalem]{ulem}
\usepackage{amsmath}
\usepackage{amssymb}
\usepackage{capt-of}
\usepackage{hyperref}
\usepackage{parskip,darkmode}
\enabledarkmode
\author{Arnav Gupta}
\date{\today}
\title{Architecture Styles}
\hypersetup{
 pdfauthor={Arnav Gupta},
 pdftitle={Architecture Styles},
 pdfkeywords={},
 pdfsubject={},
 pdfcreator={Emacs 29.4 (Org mode 9.7.11)}, 
 pdflang={English}}
\begin{document}

\maketitle
\tableofcontents

\section{Architectures}
\label{sec:org396ec11}
No good way to characterize architectural structures, but best to make common use of architectural
principles when designing software.
Principles represent rules of thumb or emerging patterns, or industry standards.
\section{Architecture Styles of Software Systems}
\label{sec:org7901bfc}
\textbf{Architectural Style}: family of systems in terms of a pattern of structural organization

Architectural style determines:
\begin{itemize}
\item \textbf{vocabulary} of components and connectors that can be used in instances of that style
\item \textbf{constraints} on how they can be combined
\begin{itemize}
\item constraints of topology of descriptions or execution semantics
\end{itemize}
\end{itemize}

To determine an architectural style, determine:
\begin{itemize}
\item \uline{structural pattern} (components, connectors, constraints)
\item underlying \uline{computational model}
\item essential \uline{invariants}
\item \uline{examples} of use
\item \uline{advantages and disadvantages} of use
\item common \uline{specializations} of style
\end{itemize}

System architecture is collection of computational components and description of interactions between
components (connectors).

Software architectures are represented as graphs where \uline{nodes} represent components: procedures, modules,
processes, tools, and DBs; and \uline{edges} represent connectors: procedure calls, event broadcasts.
DB queries, and pipes.
\subsection{Repository Style}
\label{sec:org80b6ca7}
Suitable for applications in which central issue is establishing, augmenting, and maintaining a
complex \uline{central body of info}.

Typically, info must be manipulated in a variety of ways, so long-term persistence is required.

\uline{Components}:
\begin{itemize}
\item central data structure representing current system state
\item collection of independent components that operate on central data structure
\end{itemize}

\uline{Connectors}:
\begin{itemize}
\item procedure calls or direct memory access
\end{itemize}

Changes to data structure trigger computation.

Data structure in memory and on disk.

Concurrent computations and data accesses.

Multiple clients collaborate, accessing data, processing it, and updating it for others to process
further.

Examples include programming environments, info systems, graphical editors, and AI knowledge
bases.

\uline{Advantages}:
\begin{itemize}
\item efficient way to store large amounts of data
\item sharing model published as the repository schema
\item centralized management: backup, security, concurrency control
\end{itemize}

\uline{Disadvantages}:
\begin{itemize}
\item must agree on data model beforehand
\item difficult to distribute data
\item data evolution is expensive
\end{itemize}
\subsection{Pipe and Filter Style}
\label{sec:org3b943f8}
Suitable for applications that require defined series of independent computations to be performed on
data.

Component reads \uline{streams} of data as input and produces streams of data as output.

\uline{Components}:
\begin{itemize}
\item \textbf{filters}: apply local transformations to input streams and compute incrementally so output begins
before all input consumed
\end{itemize}

\uline{Connectors}:
\begin{itemize}
\item \textbf{pipes}: conduits for streams, transmitting outputs of 1 filter to inputs of another filter
\end{itemize}

Filters do not share state with other filters.
Filters do not know identity of their upstream or downstream filters.

\uline{Advantages}:
\begin{itemize}
\item easy to understand overall input/output behaviour of system as composition of behaviours of
individual filters
\item support reuse since any 2 filters can be hooked together provided they agree on data transmitted
between them
\item easy to maintain and enhance (add new filters and replace old filters easily)
\item permit specialized analysis (throughput and deadlock analysis)
\item support concurrent execution
\end{itemize}

\uline{Disadvantages}:
\begin{itemize}
\item bad for interactive systems due to transformational character
\item excessive parsing and unparsing leads to loss of performance and increased complexity in
writing filters
\end{itemize}
\subsubsection{Specializations}
\label{sec:org6bf1201}
\textbf{Pipelines}: restricts topologies to linear sequences of filters

\textbf{Batch Sequential}: degenerate case of pipeline architecture where each filter processes all
input data before producing any output

Compilers started with sequential process and moved towards repository as intermediate
representation became more important.
\subsection{Object-Oriented}
\label{sec:orga86c1dc}
Suitable for applications in which a central issue is identifying and protecting related bodies
of info (data).

Data representations and associated operations are encapsulated in abstract data types.

\uline{Components}: objects

\uline{Connectors}: function and procedure invocations (methods)

Objects are responsible for preserving integrity (invariants) of data representation.

Data representation is hidden from other objects.

Specializations include distributed objects and objects with multiple interfaces.

\uline{Advantages}:
\begin{itemize}
\item object hides data representation from clients, so possible to change implementation without
affecting clients
\item can design systems as collections of autonomous interacting agents
\end{itemize}

\uline{Disadvantages}:
\begin{itemize}
\item for 1 object to interact with another object (via method invocation), first object must know
\uline{identity} of second object
\begin{itemize}
\item when identity changes, must modify all objects that invoke it
\end{itemize}
\item objects cause side effect problems, especially when object used by multiple different objects
\end{itemize}
\subsection{Layered}
\label{sec:orgbc1537d}
Suitable for applications that involve distinct classes of services that can be
\uline{organized hierarchically}.

Each layer provides service to the layer above and serves as client to layer below.

Only carefully selected procedures from inner layers are made available (exported) to adjacent
outer layer.

\uline{Components}: collections of procedures

\uline{Connectors}: procedure calls under restricted visibility

Tier is \uline{physical} partition of system across servers, layer is \uline{logical} partition of related
functionality, typically at different layers of abstraction.

\uline{Advantages}:
\begin{itemize}
\item design based on increasing levels of abstraction
\item enhancement easy since changes to one layer affects at most 2 other layers
\item allows reuse since different implementations of same layer can be used interchangeably
\end{itemize}

\uline{Disadvantages}:
\begin{itemize}
\item not all systems are easily structured in layered fashion
\item performance requirements may force coupling of high-level functions to lower-level implementations
\end{itemize}
\subsubsection{Specializations}
\label{sec:org6864d88}
Exceptions are made to permit non-adjacent layers to communicate directly, done for
efficiency reasons.
\subsection{Interpreter}
\label{sec:orgc180ed1}
Suitable for applications in which the most appropriate language or machine.

\uline{Components}: include one state machine for the execution engine and 3 memories:
\begin{itemize}
\item current state of the execution engine
\item program being interpreted
\item current state of the program being interpreted
\end{itemize}

\uline{Connectors}:
\begin{itemize}
\item procedure calls
\item direct memory accesses
\end{itemize}

\uline{Advantages}:
\begin{itemize}
\item simulation of non-implemented hardware
\item facilitates portability of application or languages across
a variety of platforms
\end{itemize}

\uline{Disadvantages}:
\begin{itemize}
\item defining, implementing, and testing interpreter components
is non-trivial
\item extra level of indirection slows down execution:
\end{itemize}
\subsection{Process-Control}
\label{sec:orgc7bb5eb}
Suitable for applications whose purpose is to maintain
specified properties of the outputs of the process sufficiently
near given reference values.

\uline{Components}:
\begin{itemize}
\item process definition includes mechanisms for manipulating process
variables
\item control algorithm for deciding how to manipulate process
variables
\end{itemize}

\uline{Connectors}: data flow relations for
\begin{itemize}
\item \textbf{process variables}: controlled variables whose value the system
is intended to control
\begin{itemize}
\item input variable that measures process input
\item manipulated variable whose value can be changed by the controller
\end{itemize}
\item \textbf{set point}: desired value for a controlled variable
\item \textbf{sensors}: obtain values of process variables to control
\end{itemize}
\subsubsection{Feedback Control System}
\label{sec:org37201d0}
Controlled variable is measured and result is used to manipulate
1+ process variables.
\subsubsection{Open-Loop Control System}
\label{sec:orgeb20d20}
Given pure input materials, fully defined process, and completely
repeatable process, process can run without surveillance (no
feedback).

Info about process variables is not used to adjust system.
\subsection{Client-Server}
\label{sec:org7398212}
Suitable for applications that involve \uline{distributed data} and processing
across a range of components.

\uline{Components}:
\begin{itemize}
\item \textbf{server}: standalone component that provides specific services
\item \textbf{client}: component that call on the services provided by servers
\end{itemize}

\uline{Connector}: network, which allows clients to access remote servers

\uline{Advantages}:
\begin{itemize}
\item straightforward distribution of data
\item transparency of location
\item mix/match heterogeneous platforms
\item easy to add new servers or upgrade existing servers
\end{itemize}

\uline{Disadvantages}:
\begin{itemize}
\item system performance depends on network performance
\item tricky to design and implement client/server systems
\item hard to find out what services are available
\end{itemize}
\subsection{Peer-to-Peer}
\label{sec:org0247f2d}
Suitable for applications that partition tasks/workloads between peers
\uline{without centralized control}.

\uline{Components}:
\begin{itemize}
\item peers: standalone components that provide specific services to each
other (act as both client and server)
\end{itemize}

\uline{Connector}: some kind of network that allows peers to access each other
directly without a central server

Connections can be transient/ad hoc.

\uline{Advantages}:
\begin{itemize}
\item no single point of failure (since no central server)
\item resilient against network issues or failing peer by establishing
new links to other peers
\item scalable (since no central server)
\item more privacy (in theory)
\item easy to add new peers or upgrade existing peers
\end{itemize}

\uline{Disadvantages}:
\begin{itemize}
\item cliques of peers could be disconnected from each other if no overlapping
peer
\item requires continuous exchange of lists of known peers between each peer
\item could require central directory server to be able to locate each peer
\item no guaranteed response from peers (could be down, delayed, hostile)
\end{itemize}
\end{document}
