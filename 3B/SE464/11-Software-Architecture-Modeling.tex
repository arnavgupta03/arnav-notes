% Created 2024-12-08 Sun 00:10
% Intended LaTeX compiler: pdflatex
\documentclass[11pt]{article}
\usepackage[utf8]{inputenc}
\usepackage[T1]{fontenc}
\usepackage{graphicx}
\usepackage{longtable}
\usepackage{wrapfig}
\usepackage{rotating}
\usepackage[normalem]{ulem}
\usepackage{amsmath}
\usepackage{amssymb}
\usepackage{capt-of}
\usepackage{hyperref}
\usepackage{parskip,darkmode}
\enabledarkmode
\author{Arnav Gupta}
\date{\today}
\title{Software Architecture Modeling}
\hypersetup{
 pdfauthor={Arnav Gupta},
 pdftitle={Software Architecture Modeling},
 pdfkeywords={},
 pdfsubject={},
 pdfcreator={Emacs 29.4 (Org mode 9.7.11)}, 
 pdflang={English}}
\begin{document}

\maketitle
\tableofcontents

\section{Architectural Modeling}
\label{sec:orgd8495da}
\textbf{Architectural Model}: artifact that captures some or all design decisions that
comprise a system's architecture

\textbf{Architectural Modeling}: reification and documentation of design decisions

\textbf{Architectural Modeling Notation}: language or means of capturing design decisions

Architects and other stakeholders must make critical modeling decisions (what, how
detailed, formality), weighing cost and benefit.

Model basic architectural elements like components, connectors, interfaces,
configurations, and rationale (reasoning behind decisions).
Also model constraints on interactions, behaviour, and concurrency.

Aspects:
\begin{itemize}
\item \textbf{static}: do not change as system runs
\item \textbf{dynamic}: change as system runs
\item \textbf{functional}: something the system does directly
\item \textbf{non-functional}: something about the qualities of the system
\end{itemize}
\section{Software Performance Models}
\label{sec:org2919fd1}
Includes \uline{Execution Graphs}, \uline{Queueing Networks}, and
\uline{Machine learning-based performance needs}.

\textbf{Formal representations}: capture aspects of software performance,
help understand non-functional requirements

Software Performance Models allow:
\begin{itemize}
\item estimating \uline{performance} of software
\item estimating \uline{resource needs}
\item identify \uline{performance issues} as early as possible
\item \uline{simulate} execution of software under certain conditions (number of users
and size of infrastructure)
\item establish performance for average, best, and worst-case scenarios
\end{itemize}

Design models of system capture static aspects, performance models capture
dynamic aspects.
\section{Software Execution Models}
\label{sec:org3ce5478}
Constructed early in development process to ensure that chosen software
architecture can achieve required performance objectives.

Provides static analysis of mean, best, and worst case response times.

Generally sufficient for identifying serious performance problems at
architectural and early design phases.

Absence of problems in software model does not mean there are none.
\subsection{Execution Graphs}
\label{sec:orgc6c15d1}
Execution graphs represent sequence of operations of system, and are constructed
for each performance scenario.

Not sufficient for a complete analysis of software performance but work well
for understanding software and non-functional requirements.
Special annotation can give idea of performance.
Combine graphs with other models to complement analysis.
\subsubsection{Graph Notation}
\label{sec:orgfb1c861}
\uline{Basic nodes}: represent processing steps at lowest level of detail
that is appropriate for current development stage

\uline{Expanded nodes}: represent processing steps elaborated in another subgraph

\uline{Repetition nodes}: subsequent nodes repeated \(n\) times, last node has an edge
to the repeat node

\uline{Case node}: represent conditional execution of processing steps, each
attached node has a probability of execution

\uline{Pardo node}: attached nodes run in parallel: all nodes must complete (join)
before proceeding

\uline{Division node}: attached nodes represent new processing threads, need not all
complete before proceeding
\subsection{Execution Model Analysis}
\label{sec:orgd8f9ef1}
Software execution model analysis helps:
\begin{itemize}
\item make quick check of best-case response time to ensure architecture
and design will lead to satisfactory performance
\item assess performance impact of alternatives
\item identify critical parts of system for performance management
\item derive params for system execution model
\end{itemize}

Algorithms are formulated for evaluating graphs.

Basic solution algorithms are easy to understand by examining graphs,
identifying basic structure, computing time of basic structure and
reducing basic structure to a computed node, and continuing until only
one node left.

Basic structures are sequences, loops, and cases.

Computation for case nodes differs for:
\begin{itemize}
\item \uline{shortest path}: time for case node is minimum of times for
conditionally executed nodes
\item \uline{longest path}: time for case node is maximum of times for
conditionally executed nodes
\item \uline{average analysis}: time is multiplying each node's time by
execution probability
\end{itemize}
\subsection{Software Resource Requirements}
\label{sec:org5fb967a}
Each basic node has specified software resource requirements \(A_{j}\)
for each service unit \(j\).

\textbf{Processing Overhead Matrix}: chart of computer (hardware) resource requirements for each
of the software resource requirements

To compute total execution time:
\begin{enumerate}
\item use processing overhead matrix to calculate total computer resources required per
software resource for each node in the graph
\item compute total computer resource requirements for the graph
\item compute best-case elapsed time
\end{enumerate}
\section{Queuing Network Models}
\label{sec:orgb6a1a22}
Software execution models provide a static analysis of the mean, best, and worst case
response times for software and characterize resource requirements of proposed software alone.

\textbf{Queuing Network Models} characterize software's performance in the presence of
dynamic factors (like other workloads or multiple users) and aims to solve resource
contention.

If software execution model indicates no problems, ready to construct and solve
queuing networks to account for contention efforts.

Benefits of QNM:
\begin{itemize}
\item more precise metrics that account for \uline{resource contention}
\item sensitivity of performance metrics to \uline{variations in workload} composition
\item \uline{scalability} of hardware and software to meet future demands
\item effect of new software on \uline{service level objectives} of other systems
\item identification of \uline{bottleneck} resources
\item comparative data on \uline{performance improvement options}
\end{itemize}

Resource contention can come from:
\begin{itemize}
\item multiple users of an application or transaction executing at once
\item multiple applications or systems executing on the same hardware at once
\item concurrent processes
\item multi-threaded application
\end{itemize}

Key computer system resources in QNM are server and queue:
\begin{itemize}
\item \textbf{server}: component of the environment that provides some service
to the software
\item \textbf{queue}: jobs waiting for service, consists of waiting line and server
\end{itemize}

Kendall notation for queues is A/S/m/B/K/SD:
\begin{itemize}
\item \(A\) is interarrival time distribution
\item \(S\) is service time distribution
\item \(m\) is number of servers
\item \(B\) is number of buffers (service capacity)
\item \(K\) is population size
\item \(SD\) is service discipline
\end{itemize}

Distributions can be:
\begin{itemize}
\item M: exponential
\item E: Erlang with param \(k\)
\item H: hyperexponential with param \(k\)
\item D: deterministic
\item G: general
\end{itemize}

Performance metrics of interest:
\begin{itemize}
\item \textbf{residence time (RT)}: average time jobs spend in server, in service
and waiting
\item \textbf{utilization (U)}: average percentage of time server is busy
\item \textbf{throughput (X)}: average rate at which jobs complete service
\item \textbf{queue length (N)}: average numbers of jobs at the server (receiving
service and waiting)
\end{itemize}

Value of metrics depends on:
\begin{itemize}
\item number of jobs
\item amount of service they need
\item time required for server to process individual jobs
\item policy used to select next job from queue
\end{itemize}

Can know performance from Kendall notation values.

For busy time \(B\), completed jobs \(C\), and total period \(T\) (and the
above performance metrics):
\begin{itemize}
\item utilization: \(U = B/T\)
\item throughput: \(X = C/T\)
\item mean service time: \(S = B/C\)
\item area under graph: \(W = \sum_{time}(\text{number of jobs})\)
\item residence time: \(RT = W/C\)
\item queue length: \(N = W/T\)
\end{itemize}

In early phases of development, can't make measurement of software
to derive execution profile.

\textbf{Workload intensity}: measure of number of requests made by a workload in given
time interval

\textbf{Service requirement}: amount of time that workload requires from each device in
processing facility

\textbf{Jobs-flow balance}: assume that system is fast enough to handle arrivals and thus
completion rate or throughput equals arrival rate

Let \(\lambda\) be arrival rate (workload intensity) and \(S\) be mean service time (service
requirements).
Then:
\begin{itemize}
\item throughput: \(X = \lambda\)
\item utilization law: \(U = XS\)
\item residence time: \(RT = \frac{S}{1 - U}\)
\item queue length: \(N = X * RT\) (Little's Law)
\end{itemize}

To find \textbf{steady-state probabilities} of birth-death process, define the rate
at which events occur (birth rate and death rate):
\begin{itemize}
\item let \(\lambda_{n}\) be the birth rate when the system is in state \(n\)
\item let \(\mu_{n}\) be the death rate when the system is in state \(n\)
\item let \(\rho_{n}\) be the steady-state probability of being in state \(n\)
\item then rate of flow into state \(n\) is rate of flow out of state \(n\) (since
steady-state)
\$\$\(\rho\)\textsubscript{n} \(\lambda\)\textsubscript{n} = \(\rho\)\textsubscript{n+1} \(\mu\)\textsubscript{n+1}\$
\end{itemize}

For general \(\rho_{n}\):
$$ \rho_{n} = \frac{\prod_{i=0}^{n-1} \lambda_{i}}{\prod_{i=1}^{n}} \rho_{0} $$

Probability of having \(n\) jobs in the system for M/M/1 queue:
$$ \rho_{n} = \left( \frac{\lambda}{\mu} \right)^{n} \rho_{0} $$
where \(\frac{\lambda}{\mu}\) is the traffic intensity, average number of customers
arriving compared to the number being served.

Probability of having \(n\) or more jobs in the system for M/M/1 queue:
$$ \left( \frac{\lambda}{\mu} \right)^{n} $$

Mean number of jobs in the system:
$$ \frac{\frac{\lambda}{\mu}}{1 - \frac{\lambda}{\mu}} $$

For M/M/m queue, birth rate is arrival rate which is \(\lambda\) and
death rate is service rate which is
\(\mu_{n} = n \mu\) for \(n = 1, \dots, m-1\) and \(\mu_{n} = m \mu\) for \(n \ge m\):
\begin{itemize}
\item utility is \(\rho = \lambda / m \mu\)
\item probability of having \(n\) jobs in the system is
$$ \frac{(m \rho)^{n}}{n!} p_{0} $$ for \(n = 1, \dots, m-1\)
and $$ \frac{\rho^{n} m^{m}}{m!} p_{0} $$ for \(n \ge m\)
\item probability that an arrivign job has to wait in the queue is
$$ p_{0} \frac{(m \rho)^{m}}{m!} \sum_{n=m}^{\infty} \rho^{n-m} $$
\end{itemize}

\textbf{Queuing network} consists of 2+ queues connected together and serve requests
sent by clients.

Request routing in queuing network is specified by \textbf{probability matrix}.

Types of QNM:
\begin{itemize}
\item \textbf{open models}: requests come from a source external of queuing network and leave
network after service completion
\begin{itemize}
\item appropriate for systems with external arrivals and departures
\item specify workload intensity and service requirements
\begin{itemize}
\item workload is arrival rate
\item service requirements are number of visits for each device and average
service time per visit (or total demand for that device)
\end{itemize}
\end{itemize}
\item \textbf{closed models}: no external source of requests and no departing requests (population
of requests in queuing network remains constant)
\begin{itemize}
\item needs number of users and think time (average delay between receipt of response
and submission of next)
\end{itemize}
\item \textbf{mixed models}: open for some workload classes and closed for others
\end{itemize}

To derive system model params from software model results:
\begin{enumerate}
\item use queue-servers to represent key computer resources or devices specified in
the software execution model and add connections between queues to complete model
topology
\item decide whether system is best modeled as an open or closed QNM
\item determine workload intensities for each scenario
\item specify service requirements
\end{enumerate}

Modeling hints:
\begin{itemize}
\item multiple users and workload
\item average vs peak performance (basic QNMs calculate average values)
\item sensitivity: if small change in one param causes large change in
computed metrics, model is sensitive to that quantity
\item scalability: improves response times for anticipated future loads
\item bottlenecks: bottleneck device is one with highest utilization
\end{itemize}
\end{document}
