% Created 2024-12-05 Thu 09:49
% Intended LaTeX compiler: pdflatex
\documentclass[11pt]{article}
\usepackage[utf8]{inputenc}
\usepackage[T1]{fontenc}
\usepackage{graphicx}
\usepackage{longtable}
\usepackage{wrapfig}
\usepackage{rotating}
\usepackage[normalem]{ulem}
\usepackage{amsmath}
\usepackage{amssymb}
\usepackage{capt-of}
\usepackage{hyperref}
\usepackage{parskip,darkmode}
\enabledarkmode
\author{Arnav Gupta}
\date{\today}
\title{Scalability}
\hypersetup{
 pdfauthor={Arnav Gupta},
 pdftitle={Scalability},
 pdfkeywords={},
 pdfsubject={},
 pdfcreator={Emacs 29.4 (Org mode 9.7.11)}, 
 pdflang={English}}
\begin{document}

\maketitle
\tableofcontents

\section{Scalability}
\label{sec:orgcdbeadf}
\textbf{Service}: different from simple program since it listens for requests from
clients/users and may handle multiple requests concurrently

Requests usually delivered as messages that arrive over a network.
Service runs constantly, waiting for requests and processing them.

A service is scalable if it can easily handle growth in the number of
concurrent users/requests.

Measured through \textbf{work throughput}:
\begin{itemize}
\item requests/queries per second
\item concurrent users
\item monthly active users
\end{itemize}

Ignore cost, just consider scale achieved.

Scaling challenges:
\begin{itemize}
\item limited speed from one machine
\item coordinating multiple machines
\item sharing data among machines
\item failure probability
\item high latency from worldwide users
\item authentication of service components
\item software updates without downtime
\end{itemize}

\textbf{Vertical scaling}: make machine bigger and stronger

\textbf{Horizontal scaling}: add more machines

Computer performance affected by number/speed of CPU cores, RAM, disk type,
number of disks, type of network connectivity, GPUs, TPUs, etc.

\textbf{Shared Memory Parallelism}: a single process can have multiple threads which execute
concurrently while sharing the same memory

Cloud computing resources are elastic since size and quantity of resources
and can be quickly changed.

Vertical scaling is not scalable:
\begin{itemize}
\item easy to write programs since most OSs have multithreading
\item less communication required with other machines
\item cannot handle big loads
\item cannot be scaled easily (must replace entire machine)
\item single point of failure
\item bad price/performance ratio
\end{itemize}
\section{HTTP and Web Servers}
\label{sec:org51da123}
To convert a program to a service:
\begin{enumerate}
\item listen to requests on the network
\item run many copies of program concurrently
\item use queues to store unhandled requests and unsent responses
\begin{enumerate}
\item allows competing threads to share single network socket
\end{enumerate}
\end{enumerate}

With parallel threads, even if only a single CPU core, multiple threads can run
since the app may block to request data from disk or network (IO), and in the
same time, another thread can run.

\textbf{HTTP}: client-server data exchange protocol

\textbf{Request} specifies:
\begin{itemize}
\item human readable header with URL, method, and more
\item optional body with raw data
\end{itemize}

\textbf{Response} includes:
\begin{itemize}
\item human readable header with response code and more
\item optional body
\end{itemize}
\end{document}
