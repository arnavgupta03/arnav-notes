% Created 2024-10-19 Sat 00:32
% Intended LaTeX compiler: pdflatex
\documentclass[11pt]{article}
\usepackage[utf8]{inputenc}
\usepackage[T1]{fontenc}
\usepackage{graphicx}
\usepackage{longtable}
\usepackage{wrapfig}
\usepackage{rotating}
\usepackage[normalem]{ulem}
\usepackage{amsmath}
\usepackage{amssymb}
\usepackage{capt-of}
\usepackage{hyperref}
\usepackage{parskip,darkmode}
\enabledarkmode
\usepackage{tikz,xcolor}
\usetikzlibrary{arrows, positioning}
\author{Arnav Gupta}
\date{\today}
\title{Intro To Control Systems}
\hypersetup{
 pdfauthor={Arnav Gupta},
 pdftitle={Intro To Control Systems},
 pdfkeywords={},
 pdfsubject={},
 pdfcreator={Emacs 29.4 (Org mode 9.7.11)}, 
 pdflang={English}}
\begin{document}

\maketitle
\tableofcontents

\section{Basics of Control Systems}
\label{sec:org490279a}
Control systems have:
\begin{itemize}
\item some reference/desired output
\item a controller that provides an input into a system
\item possibly some disturbance that influences the input
\item a system that takes in input and produces some output
\item a sensor that reads the output and feeds this to the controller
\item some noise that influences the output read by the sensor
\end{itemize}

If the measured output is subtracted from the desired output, it is
negative feedback control. If it is added, it is positive feedback
control.

\tikzstyle{block} = [draw, rectangle, minimum height=3em, minimum width=6em]
\tikzstyle{sum} = [draw, circle, node distance=1cm]
\tikzstyle{input} = [coordinate]
\tikzstyle{output} = [coordinate]
\tikzstyle{controller} = [block, fill=cyan!30!black!60]
\tikzstyle{plant} = [block, fill=lime!30!black!60]
\tikzstyle{sensor} = [block, fill=red!30!black!60]
\begin{center}
\begin{tikzpicture}[auto, node distance=2cm,>=latex']
    % Nodes for the block diagram
    \node [input, name=input] {};
    \node [sum, right of=input, node distance=2cm] (sum) {};
    \node [controller, right of=sum, node distance=3cm] (controller) {controller};
    \node [plant, right of=controller, node distance=3cm] (plant) {plant};
    \node [sensor, below of=plant] (sensor) {sensor};
    \node [output, right of=plant, node distance=3cm] (output) {};

    % Connections between nodes
    \draw [->] (input) -- node {reference} (sum);
    \draw [->] (sum) -- node {error} (controller);
    \draw [->] (controller) -- node[name=u] {} (plant);
    \draw [->] (plant) -- node [name=y] {output} (output);
    \draw [->] (y) |- (sensor);
    \draw [->] (sensor) -| node[pos=0.99] {$-$} node [near end] {measured output} (sum);
\end{tikzpicture}
\end{center}
\section{Control System Design}
\label{sec:orge198d2f}
Consists of the following steps:
\begin{enumerate}
\item Study the system to be controlled and establish control goals.
\item Identify variables to be controlled.
\item Define control specifications (performance metrics).
\item Obtain a model for the system (first principles or data).
\item Simplify the model as much as possible.
\begin{enumerate}
\item No extra info
\item Linearize to simplify
\end{enumerate}
\item Design the controller and select key parameters to be adjusted.
\item Simulate the controlled system (unsimplified).
\item Optimize the parameters and analyze performance through iterating
if necessary.
\end{enumerate}
\end{document}
