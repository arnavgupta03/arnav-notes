% Created 2024-10-19 Sat 11:47
% Intended LaTeX compiler: pdflatex
\documentclass[11pt]{article}
\usepackage[utf8]{inputenc}
\usepackage[T1]{fontenc}
\usepackage{graphicx}
\usepackage{longtable}
\usepackage{wrapfig}
\usepackage{rotating}
\usepackage[normalem]{ulem}
\usepackage{amsmath}
\usepackage{amssymb}
\usepackage{capt-of}
\usepackage{hyperref}
\usepackage{parskip,darkmode}
\enabledarkmode
\usepackage{mathtools}
\usepackage{tikz,xcolor}
\usetikzlibrary{arrows, positioning, calc}
\author{Arnav Gupta}
\date{\today}
\title{Mathematical Models Of Control Systems}
\hypersetup{
 pdfauthor={Arnav Gupta},
 pdftitle={Mathematical Models Of Control Systems},
 pdfkeywords={},
 pdfsubject={},
 pdfcreator={Emacs 29.4 (Org mode 9.7.11)}, 
 pdflang={English}}
\begin{document}

\maketitle
\tableofcontents

\section{From Differential Equations to State Space}
\label{sec:org4399ed6}
\textbf{Time-varying control system}: system in which one or more system parameters
may vary as a function of time

\textbf{State of a system}: set of variables whose values, together with input,
will provide future state and output of the system

\textbf{State vector}: column vector consisting of the state variables
$$
\mathbf{x}(t) = \begin{pmatrix} x_{1}(t) \\ x_{2}(t) \\ \vdots \\ x_{n}(t) \end{pmatrix}
$$

\textbf{Input vector}: column vector consisting of the input variables
$$
\mathbf{u}(t) = \begin{pmatrix} u_{1}(t) \\ u_{2}(t) \\ \vdots \\ u_{m}(t) \end{pmatrix}
$$

\textbf{State differential equation}: representation of the system
$$
\dot{\mathbf{x}}(t) = \mathbf{Ax}(t) + \mathbf{Bu}(t)
$$
where \(\mathbf{A}\) is an \(n \times n\) matrix and \(\mathbf{B}\) is a \(n \times m\) matrix.

\textbf{Output equation}: relation of outputs to state variables and input signals
$$
\mathbf{y}(t) = \mathbf{Cx}(t) + \mathbf{Du}(t)
$$

\tikzstyle{block} = [draw, rectangle, minimum height=3em, minimum width=6em, fill=cyan!50!black!80]
\tikzstyle{input} = [coordinate]
\tikzstyle{output} = [coordinate]
\begin{center}
\begin{tikzpicture}[auto, node distance=2cm,>=latex']
    % Nodes for the block diagram
    \node [input, name=input_u] {};
    \node [block, right of=input_u, node distance=4cm] (block) {Block};
    \node [input, name=input_x] [above of=block] {};
    \node [output, right of=block, node distance=4cm] (output) {};

    % Connections
    \draw [->] (input_u) -- node[above] {Input: $u(t)$} (block);
    \draw [->] (input_x) -- node[right] {ICs: $x(0)$} (block);
    \draw [->] (block) -- node[above] {Output: $y(t)$} (output);
\end{tikzpicture}
\end{center}

\textbf{State space representation} uses the state differential equation and output equation.
\begin{center}
$$
\begin{cases}
        \dot{\mathbf{x}}(t) = f(x(t), u(t)) \\
        \mathbf{y}(t) = h(x(t), u(t))
\end{cases}
$$
$\mathbf{x}(t) \in \mathbb{R}^{n}, \mathbf{u}(t) \in \mathbb{R}^{m},
\mathbf{y}(t) \in \mathbb{R}^{p}, f : \mathbb{R}^{n} \times \mathbb{R}^{m} \to \mathbb{R}^{m},
h: \mathbb{R}^{n} \times \mathbb{R}^{m} \to \mathbb{R}^{p}$
\end{center}
\section{Linearization}
\label{sec:orge61ed74}
A pair \((\bar{x}, \bar{u}) \in \mathbb{R}^{n} \times \mathbb{R}^{m}\) is an \textbf{equilibrium point}
if \(f(\bar{x}, \bar{u}) = 0\).

Note that \(\mathbf{x}(t) \equiv \bar{x}, \mathbf{u}(t) \equiv \bar{u}, \mathbf{y}(t) \equiv \bar{y} = h(\bar{x}, \bar{u})\)
is a solution to the state space representation.

Consider some small deviation from an equilibrium point \(\delta \mathbf{x}\) and \(\delta \mathbf{u}\).
By defining these variations of state and output as \(\delta \mathbf{x}(t) = \mathbf{x}(t) - \bar{x}\)
and \(\delta \mathbf{y}(t) = \mathbf{y}(t) - \bar{y}\), the state space form becomes (with Taylor expansion)
\begin{center}
$$
\begin{cases}
        \dot{\delta \mathbf{x}}(t) = f(x, u)
        = \frac{\partial f}{\partial x} ( \bar{x}, \bar{u} ) \delta x(t) + \frac{\partial f}{\partial u} ( \bar{x}, \bar{u} ) \delta u(t)
        + \mathcal{O} ( \| \delta x(t) \|^{2} ) + \mathcal{O} ( \| \delta u(t) \|^{2} )\\
        \mathbf{y}(t) = h(x, u) - \bar{y}
        = \frac{\partial h}{\partial x} ( \bar{x}, \bar{u} ) \delta x(t) + \frac{\partial h}{\partial u} ( \bar{x}, \bar{u} ) \delta u(t)
        + \mathcal{O} ( \| \delta x(t) \|^{2} ) + \mathcal{O} ( \| \delta u(t) \|^{2} )
\end{cases}
$$
\end{center}

This can be represented with matrices as (dropping the deltas):
\begin{center}
$$
\begin{cases}
        \dot{\mathbf{x}}(t) = \mathbf{Ax}(t) + \mathbf{Bu}(t) \\
        \mathbf{y}(t) = \mathbf{Cx}(t) + \mathbf{Du}(t)
\end{cases}
$$
\end{center}

which is the linearization around \((\bar{x}, \bar{u})\).

With the definition from the Taylor expansion, the matrices can be found from the Jacobian matrix.
For example:
$$
\mathbf{A} = \frac{\partial f}{\partial x} ( \bar{x}, \bar{u} )
= \begin{pmatrix} \frac{\partial f_{1}}{\partial x_{1}} (\bar{x}, \bar{u}) & \dots & \frac{\partial f_{1}}{\partial x_{n}} (\bar{x}, \bar{u}) \\
\vdots & \ddots & \vdots \\
\frac{\partial f_{n}}{\partial x_{1}} (\bar{x}, \bar{u}) & \dots & \frac{\partial f_{n}}{\partial x_{n}} (\bar{x}, \bar{u}) \end{pmatrix}
$$
\section{Laplace Transform}
\label{sec:orgf018a5d}
The \textbf{Laplace Transform} for a function \(f(t)\) is
$$
F(s) = \int_{0^{-}}^{\infty} f(t) e^{-st} \, dt = \mathcal{L} \{ f(t) \}
$$

Some Laplace Tranform properties:
\begin{itemize}
\item Laplace Transform is linear
\item \(\mathcal{L}\{f(t - \tau)\} = e^{-\tau s}F(s)\)
\item \(\mathcal{L}\{e^{\alpha t} f(t)\} = F(s - \alpha)\)
\item \(\mathcal{L} \left\{ \frac{d}{dt} f(t) \right\} = sF(s) - f(0^{-})\)
\item \(\mathcal{L} \left\{ t f(t) \right\} = -\frac{d}{ds} F(s)\)
\item \(\mathcal{L} \left\{ f(t) * g(t) \right\} = F(s) G(s)\)
\end{itemize}

The \textbf{Final Value Theorem} states that
$$
\lim_{t \to \infty} y(t) = \lim_{s \to 0} sY(s)
$$
where all poles of \(Y(s)\) lie in the left-hand side of the plane and
\(Y(s)\) must not have \(>1\) pole at the origin.

The \textbf{Initial Value Theorem} states that
$$
\lim_{t \to 0^{+}} f(t) = \lim_{s \to \infty} sF(s)
$$
\section{Transfer Function}
\label{sec:orga041901}
For a single-input-single-output (SISO) system, the \textbf{transfer function} is
$$
G(s) = \frac{\mathcal{L}\{y(t)\}}{\mathcal{L}\{u(t)\}} = \frac{Y(s)}{U(s)}
$$
where Laplace transforms are taken assuming zero initial conditions of the
state.

For a general LTI system, the transfer function \(G(s)\) is given by
$$
G(s) = C(sI - A)^{-1} B + D
$$

With the transfer function, the output to an input signal can be found by:
\begin{enumerate}
\item Compute \(U(s)\).
\item Compute \(Y(s) = G(s)U(s)\).
\item Compute \(y(t) = \mathcal{L}^{-1} \{ Y(s) \}\).
\end{enumerate}

A transfer function \(G(s)\) is \textbf{real rational} if it can be expressed as
$$
G(s) = \frac{b_{m}s^{m} + \cdots + b_{1}s + b_{0}}{s^{n} + a_{n-1}s^{n-1} + \cdots + a_{1} s + a_{0}}
$$
for real coefficients \(a_{i} \in \mathbb{R}, \forall i \in [0,n-1]\), and \(b_{i} \in \mathbb{R}, \forall i \in [0,m]\).

\(G(s)\) is \textbf{proper} if \(\lim_{s \to \infty}G(s)\) exists in \(\mathbb{C}\) (such that \(n \ge m\)).
\(G(s)\) is \textbf{strictly proper} if \(\lim_{s \to \infty}G(s) = 0\) (such that \(n > m\)).

\(p \in \mathbb{C}\) is a \textbf{pole} of \(G(s)\) if \(\lim_{s \to p} |G(s)| = \infty\).
\(z \in \mathbb{C}\) is a \textbf{zero} of \(G(s)\) if \(\lim_{s \to z} |G(s)| = 0\).

For rational transfer function, poles and zeros are the roots of the denominator and numerator respectively.

\begin{center}
\begin{tikzpicture}[scale=1]

% Draw axes
\draw[->] (-3.5, 0) -- (3.5, 0) node[right] {\textbf{Real}};
\draw[->] (0, -2.5) -- (0, 3.5) node[above] {\textbf{Imaginary}};

% Draw the poles (X's)
\filldraw[red!80!black!60] (-2, 0) node[anchor=north] {} circle (2pt);
\filldraw[red!80!black!60] (2, 0) node[anchor=south] {} circle (2pt);
\filldraw[red!80!black!60] (-1, 2) node[anchor=south east] {} circle (2pt);
\filldraw[red!80!black!60] (-1, -2) node[anchor=north east] {} circle (2pt);

% Draw the zero (O)
\filldraw[cyan!50!black!80] (-3, 0) node[anchor=north] {} circle (2pt);

% Draw dashed lines to axes from complex poles
\draw[dashed] (-1, 2) -- (-1, 0); % Line from -1 + i to real axis
\draw[dashed] (-1, -2) -- (-1, 0); % Line from -1 - i to real axis
\draw[dashed] (-1, 2) -- (0, 2);   % Line from -1 + i to imaginary axis
\draw[dashed] (-1, -2) -- (0, -2); % Line from -1 - i to imaginary axis

% Add a caption using a node
\node at (0, -3) {Poles in Red, Zero in Blue};

\end{tikzpicture}
\end{center}

First order systems have transfer functions of the form:
$$
G_{1}(s) = \frac{K}{1 + \tau s}
$$
where \(K\) is the \textbf{steady-state gain} of the system and \(\tau\) is the
\textbf{time constant} of the system.

\tikzstyle{block} = [draw, rectangle, minimum height=3em, minimum width=6em]
\tikzstyle{sum} = [draw, circle, node distance=1cm]
\tikzstyle{input} = [coordinate]
\tikzstyle{output} = [coordinate]
\tikzstyle{controller} = [block, fill=cyan!50!black!80]
\begin{center}
\begin{tikzpicture}[auto, node distance=2cm,>=latex']
    % Nodes for the block diagram
    \node [input, name=input] {};
    \node [sum, right of=input, node distance=2cm] (sum) {};
    \node [controller, right of=sum, node distance=3cm] (controller) {$\frac{K}{\tau s}$};
    \node [output, right of=controller, node distance=3cm] (output) {};

    % Connections between nodes
    \draw [->] (input) -- node {$U(s)$} (sum);
    \draw [->] (sum) -- node {$E(s)$} (controller);
    \draw [->] (controller) -- node [name=y] {$Y(s)$} (output);
    \draw [->] (y) -- ++(0, -2) |- ++(-5, 0) -| node[pos=0.99] {$-$} node [near end] {} (sum);
\end{tikzpicture}
\end{center}
\vspace{10pt}
\begin{center}
\begin{tikzpicture}[auto, node distance=2cm,>=latex']
    % Nodes for the block diagram
    \node [input, name=input] {$U(s)$};
    \node [controller, right of=input, node distance=4cm] (controller) {$G(s) = \frac{K}{\tau s + 1}$};
    \node [output, right of=controller, node distance=3cm] (output) {};

    % Connections between nodes
    \draw [->] (input) -- node {} (controller);
    \draw [->] (controller) -- node [name=y] {$Y(s)$} (output);
\end{tikzpicture}
\end{center}

Second order systems have transfer functions of the form:
$$
G_{2}(s) = \frac{\omega_{n}^{2}}{s^{2} + 2 \zeta \omega_{n} + \omega_{n}^{2}}
$$
where \(\omega_{n}\) is the \textbf{natural frequency} of the system and \(\zeta\) is the
\textbf{damping ratio} of the system.
This has roots at \(-\zeta \omega_{n} \pm \omega_{n} \sqrt{\zeta^{2} - 1}\).

When \(\zeta > 1\), the roots are real and the system is \textbf{overdamped}.
When \(\zeta < 1\), the roots are real and the system is \textbf{underdamped}.
When \(\zeta = 1\), the roots are real and the system is \textbf{critically damped}.

\tikzstyle{block} = [draw, rectangle, minimum height=3em, minimum width=6em]
\tikzstyle{sum} = [draw, circle, node distance=1cm]
\tikzstyle{input} = [coordinate]
\tikzstyle{output} = [coordinate]
\tikzstyle{controller} = [block, fill=cyan!50!black!80]
\tikzstyle{plant} = [block, fill=lime!60!black!100]
\tikzstyle{sensor} = [block, fill=red!50!black!80]
\begin{center}
\begin{tikzpicture}[auto, node distance=2cm,>=latex']
    % Nodes for the block diagram
    \node [input, name=input] {};
    \node [sum, right of=input, node distance=2cm] (sum) {};
    \node [plant, right of=sum, node distance=3cm] (plant) {$P(s)$};
    \node [sensor, below of=plant] (sensor) {$Q(s)$};
    \node [output, right of=plant, node distance=3cm] (output) {$Y(s)$};

    % Connections between nodes
    \draw [->] (input) -- node {$U(s)$} (sum);
    \draw [->] (sum) -- node {$E(s)$} (plant);
    \draw [->] (plant) -- node [name=y] {$Y(s)$} (output);
    \draw [->] (y) |- (sensor);
    \draw [->] (sensor) -| node[pos=0.99] {$-$} node [near end] {} (sum);
\end{tikzpicture}
\end{center}
\vspace{10pt}
\begin{center}
\begin{tikzpicture}[auto, node distance=2cm,>=latex']
    % Nodes for the block diagram
    \node [input, name=input] {};
    \node [controller, right of=input, node distance=3cm] (controller) {$G(s) = \frac{\omega_{n}^{2}}{s^{2} + 2 \zeta \omega_{n} + \omega_{n}^{2}}$};
    \node [output, right of=controller, node distance=3cm] (output) {};

    % Connections between nodes
    \draw [->] (input) -- node {$U(s)$} (controller);
    \draw [->] (controller) -- node [name=y] {$Y(s)$} (output);
\end{tikzpicture}
\end{center}

The \textbf{Laplacian} matrix is the degree matrix minus the adjacency matrix.
\begin{center}
$$
L = \begin{pmatrix}
2 & 0 & 0 & 0 & 0 \\
0 & 3 & 0 & 0 & 0 \\
0 & 0 & 2 & 0 & 0 \\
0 & 0 & 0 & 2 & 0 \\
0 & 0 & 0 & 0 & 3
\end{pmatrix} - \begin{pmatrix}
0 & 1 & 0 & 0 & 1 \\
1 & 0 & 1 & 0 & 1 \\
0 & 1 & 0 & 1 & 0 \\
0 & 0 & 1 & 0 & 1 \\
1 & 1 & 0 & 1 & 0
\end{pmatrix}
$$
\end{center}

A system where \(\delta \dot{x} = \delta u\) is a \textbf{single integrator} system, since the
system output is simply an integral of the input signal.
\end{document}
