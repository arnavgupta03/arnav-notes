% Created 2024-11-14 Thu 13:59
% Intended LaTeX compiler: pdflatex
\documentclass[11pt]{article}
\usepackage[utf8]{inputenc}
\usepackage[T1]{fontenc}
\usepackage{graphicx}
\usepackage{longtable}
\usepackage{wrapfig}
\usepackage{rotating}
\usepackage[normalem]{ulem}
\usepackage{amsmath}
\usepackage{amssymb}
\usepackage{capt-of}
\usepackage{hyperref}
\usepackage{parskip,darkmode}
\enabledarkmode
\author{DESKTOP-H800RKQ}
\date{\today}
\title{Controller Synthesis}
\hypersetup{
 pdfauthor={DESKTOP-H800RKQ},
 pdftitle={Controller Synthesis},
 pdfkeywords={},
 pdfsubject={},
 pdfcreator={Emacs 29.4 (Org mode 9.7.11)}, 
 pdflang={English}}
\begin{document}

\maketitle
\tableofcontents

\section{Loop Shaping}
\label{sec:org92714ee}
Translating control specifications of the closed loop system into constraints of the
Bode plots of the open loop system is loop shaping.

The steps involved in loop shaping are:
\begin{enumerate}
\item turn specifications from the time domain behaviour of \(F(s)\) to the frequency domain
behaviour of \(L(s)\)
\item design the transfer function of the controller \(C(s)\) so that \(L(s) = C(s)G(s)\) has the desired
frequency domain behaviour
\end{enumerate}

For robust stability, a high \(K_{m}\) and high \(\phi_{m}\) are desirable.

\textbf{Delay} is negative phase shift, which can decrease the phase margin and so
\(\phi_{m} > 0\) may no longer hold.

Integral action in a controller, that is some \(L(s)\) where
$$ L(s) = \frac{\mu}{s^{\rho}} $$
gives a steady state gain of 1 in the closed-loop system.

Low bandwidth gives a slow response, high bandwidth gives a fast response.

The crossover frequency \(\omega_{c}\) helps find which frequencies pass through and which get attenuated.
Frequencies within the crossover frequency are within the bandwidth and frequencies above the crossover
frequency are not within the bandwidth.
This is because the bandwidth of \(F(s)\) (the closed-loop transfer function) is approximated by
the crossover frequency of \(L(s)\) (the open-loop transfer function).

The closed-loop transfer function \(F(s)\) is approximated by 1 for \(\omega \ll \omega_{c}\) of \(L(s)\)
and \(L(s)\) for \(\omega \gg \omega_{c}\) of \(L(s)\).

The \textbf{lower bound} on \(\omega_{c}\) of \(L(s)\) is used to filter out noise, not the reference.
The disturbance signals contain low frequencies and noise signals contain high frequencies.

Consider a system with open-loop transfer function \(L(s) = C(s)G(s)\) and corresponding closed loop
transfer function \(F(s)\).
To \uline{reject disturbance}:
$$ \left| \frac{G(j \omega)}{1 + C(j\omega)G(j \omega)} \right| $$
which is the lower bound on \(\omega_{c}\) for \(L(s)\), should be kept low for \(\omega\) contained in \(D\),
specifically low frequencies.

To \uline{attenuate noise}:
$$ \left| \frac{C(j\omega)G(j \omega)}{1 + C(j\omega)G(j\omega)} \right| $$
which is the upper bound on \(\omega_{c}\) for \(L(s)\), should be kept low for \(\omega\) contained in \(N\),
specifically high frequencies.

For \(L(s) = C(s)G(s)\), where \(C(s)\) is proper (in order to be realizable in state-space form),
the slope of \(|L(j\omega)|\) should be at most the slope of \(|G(j\omega)|\) as \(\omega \to \infty\).
\section{Integral Control}
\label{sec:orgd62f359}
For a closed-loop system with open loop transfer function \(L(s) = C(s)G(s)\),
for \(C(s) = \frac{\mu}{s^{\rho}}\):
\begin{itemize}
\item if \(\rho > 0\), there is an integrator in the loop, so the closed-loop transfer function \(F(s)\) has
\(F(0) = 1\)
\item if \(\rho = 0\), there is no integrator in the loop, so \(\mu\) must be high for \(F(0)\) to approach 1
\end{itemize}
\section{Lead-Lag Compensators}
\label{sec:org28e00d0}
\subsection{Lead Controller}
\label{sec:org6b6e83d}
Consider a controller with transfer function
$$ C(s) = \mu\frac{1 + Ts}{1 + \alpha Ts} $$
for \(\mu > 0, T > 0, 0 < \alpha < 1\).
This controller has a zero at \(-\frac{1}{T}\) and a pole at \(-\frac{1}{\alpha T}\),
which adds a phase lead between the zero and the pole, which is an increase in the phase margin.

This controller has high frequency gain \(\frac{\mu}{\alpha}\).
A good value of \(\alpha\) is 0.1, since this gives a \(55^{\circ}\) phase lead.

As \(\alpha \to 0\), \(C(s)\) becomes \(\mu(1 + Ts)\).
When \(C(s) = \frac{E(s)}{U(s)}\), then this gives that the control action \(u(t)\) is made of 2 terms:
$$ u(t) = \mu e(t) + \mu T \frac{de}{dt}(t) $$
where one term is proportional to the error and the other is proportional to the time derivative of the
error, so this is a PD controller.
For a non-continuous space, the control action can be written as
$$ u(t_{k}) = \mu e(t_{k}) + \mu T \frac{e(t_{k}) - e(t_{k-1})}{t_{k} - t_{k-1}} $$
Adding poles/zeros makes the controller realizable to get this approximation of the derivative until
the frequency of the pole of error.
\subsection{Lag Controller}
\label{sec:org50a3161}
Consider a controller with transfer function
$$ C(s) = \mu\frac{1 + Ts}{1 + \alpha Ts} $$
for \(\mu > 0, T > 0, \alpha > 1\).
This controller has a zero at \(-\frac{1}{T}\) and a pole at \(-\frac{1}{\alpha T}\),
which adds a phase lag between the zero and the pole, which is an decrease in the phase margin.
The lag controller helps to improve static performance without losing stability.

A good value of \(\alpha\) is \(10\).
For \(\frac{10}{T} < \omega_{c}\), this gives \(-6^{\circ}\) at \(\omega_{c}\).
At \(\mu = \alpha\), the high frequency behaviour is unchanged.
For \(\mu = 1\), there is a lower \(\omega_{c}\) and which increases \(\phi_{m}\).
As \(\alpha \to \infty\), the corresponding increase of the steady-state gain \(\mu\) such that
\(\frac{\mu}{\alpha} = c\) is constant, gives the lag controller
$$ C(s) = c + \frac{c}{Ts} $$
which is a PI controller.
\section{PID Controller}
\label{sec:org905b6af}
Placing the zero and pole of a lead controller \(1/\alpha\) decade away from \(\omega_{c}\) requires
$$ \omega_{c} = \frac{1}{\sqrt{\alpha} T} $$

$$ \zeta \approx \frac{\phi_{m}}{100} $$
so if \(\phi_{m}\) increases, there are less oscillations and less overshoot for the closed loop system.

For a lag controller, as \(\alpha \to \infty\), the pole \(-\frac{1}{\alpha T} \to 0\).
\end{document}
