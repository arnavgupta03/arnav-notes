% Created 2024-12-11 Wed 15:14
% Intended LaTeX compiler: pdflatex
\documentclass[11pt]{article}
\usepackage[utf8]{inputenc}
\usepackage[T1]{fontenc}
\usepackage{graphicx}
\usepackage{longtable}
\usepackage{wrapfig}
\usepackage{rotating}
\usepackage[normalem]{ulem}
\usepackage{amsmath}
\usepackage{amssymb}
\usepackage{capt-of}
\usepackage{hyperref}
\usepackage{parskip,darkmode}
\enabledarkmode
\author{Arnav Gupta}
\date{\today}
\title{Controller Synthesis}
\hypersetup{
 pdfauthor={Arnav Gupta},
 pdftitle={Controller Synthesis},
 pdfkeywords={},
 pdfsubject={},
 pdfcreator={Emacs 29.4 (Org mode 9.7.11)}, 
 pdflang={English}}
\begin{document}

\maketitle
\tableofcontents

\section{Loop Shaping}
\label{sec:orgc6e739b}
Translating control specifications of the closed loop system into constraints of the
Bode plots of the open loop system is loop shaping.

The steps involved in loop shaping are:
\begin{enumerate}
\item turn specifications from the time domain behaviour of \(F(s)\) to the frequency domain
behaviour of \(L(s)\)
\item design the transfer function of the controller \(C(s)\) so that \(L(s) = C(s)G(s)\) has the desired
frequency domain behaviour
\end{enumerate}

For robust stability, a high \(K_{m}\) and high \(\phi_{m}\) are desirable.

\textbf{Delay} is negative phase shift, which can decrease the phase margin and so
\(\phi_{m} > 0\) may no longer hold.

Integral action in a controller, that is some \(L(s)\) where
$$ L(s) = \frac{\mu}{s^{\rho}} $$
gives a steady state gain of 1 in the closed-loop system.

Low bandwidth gives a slow response, high bandwidth gives a fast response.

The crossover frequency \(\omega_{c}\) helps find which frequencies pass through and which get attenuated.
Frequencies within the crossover frequency are within the bandwidth and frequencies above the crossover
frequency are not within the bandwidth.
This is because the bandwidth of \(F(s)\) (the closed-loop transfer function) is approximated by
the crossover frequency of \(L(s)\) (the open-loop transfer function).

The closed-loop transfer function \(F(s)\) is approximated by 1 for \(\omega \ll \omega_{c}\) of \(L(s)\)
and \(L(s)\) for \(\omega \gg \omega_{c}\) of \(L(s)\).

The \textbf{lower bound} on \(\omega_{c}\) of \(L(s)\) is used to filter out noise, not the reference.
The disturbance signals contain low frequencies and noise signals contain high frequencies.

Consider a system with open-loop transfer function \(L(s) = C(s)G(s)\) and corresponding closed loop
transfer function \(F(s)\).
To \uline{reject disturbance}:
$$ \left| \frac{G(j \omega)}{1 + C(j\omega)G(j \omega)} \right| $$
which is the lower bound on \(\omega_{c}\) for \(L(s)\), should be kept low for \(\omega\) contained in \(D\),
specifically low frequencies.

To \uline{attenuate noise}:
$$ \left| \frac{C(j\omega)G(j \omega)}{1 + C(j\omega)G(j\omega)} \right| $$
which is the upper bound on \(\omega_{c}\) for \(L(s)\), should be kept low for \(\omega\) contained in \(N\),
specifically high frequencies.

For \(L(s) = C(s)G(s)\), where \(C(s)\) is proper (in order to be realizable in state-space form),
the slope of \(|L(j\omega)|\) should be at most the slope of \(|G(j\omega)|\) as \(\omega \to \infty\).
\section{Integral Control}
\label{sec:orga288686}
For a closed-loop system with open loop transfer function \(L(s) = C(s)G(s)\),
for \(C(s) = \frac{\mu}{s^{\rho}}\):
\begin{itemize}
\item if \(\rho > 0\), there is an integrator in the loop, so the closed-loop transfer function \(F(s)\) has
\(F(0) = 1\)
\item if \(\rho = 0\), there is no integrator in the loop, so \(\mu\) must be high for \(F(0)\) to approach 1
\end{itemize}
\section{Lead-Lag Compensators}
\label{sec:orgdcbd3fb}
\subsection{Lead Controller}
\label{sec:org82568a7}
Consider a controller with transfer function
$$ C(s) = \mu\frac{1 + Ts}{1 + \alpha Ts} $$
for \(\mu > 0, T > 0, 0 < \alpha < 1\).
This controller has a zero at \(-\frac{1}{T}\) and a pole at \(-\frac{1}{\alpha T}\),
which adds a phase lead between the zero and the pole, which is an increase in the phase margin.

This controller has high frequency gain \(\frac{\mu}{\alpha}\).
A good value of \(\alpha\) is 0.1, since this gives a \(55^{\circ}\) phase lead.

As \(\alpha \to 0\), \(C(s)\) becomes \(\mu(1 + Ts)\).
When \(C(s) = \frac{E(s)}{U(s)}\), then this gives that the control action \(u(t)\) is made of 2 terms:
$$ u(t) = \mu e(t) + \mu T \frac{de}{dt}(t) $$
where one term is proportional to the error and the other is proportional to the time derivative of the
error, so this is a PD controller.
For a non-continuous space, the control action can be written as
$$ u(t_{k}) = \mu e(t_{k}) + \mu T \frac{e(t_{k}) - e(t_{k-1})}{t_{k} - t_{k-1}} $$
Adding poles/zeros makes the controller realizable to get this approximation of the derivative until
the frequency of the pole of error.
\subsection{Lag Controller}
\label{sec:org9cacbff}
Consider a controller with transfer function
$$ C(s) = \mu\frac{1 + Ts}{1 + \alpha Ts} $$
for \(\mu > 0, T > 0, \alpha > 1\).
This controller has a zero at \(-\frac{1}{T}\) and a pole at \(-\frac{1}{\alpha T}\),
which adds a phase lag between the zero and the pole, which is an decrease in the phase margin.
The lag controller helps to improve static performance without losing stability.

A good value of \(\alpha\) is \(10\).
For \(\frac{10}{T} < \omega_{c}\), this gives \(-6^{\circ}\) at \(\omega_{c}\).
At \(\mu = \alpha\), the high frequency behaviour is unchanged.
For \(\mu = 1\), there is a lower \(\omega_{c}\) and which increases \(\phi_{m}\).
As \(\alpha \to \infty\), the corresponding increase of the steady-state gain \(\mu\) such that
\(\frac{\mu}{\alpha} = c\) is constant, gives the lag controller
$$ C(s) = c + \frac{c}{Ts} $$
which is a PI controller.
\section{PID Controller}
\label{sec:orga74f4e9}
Placing the zero and pole of a lead controller \(1/\alpha\) decade away from \(\omega_{c}\) requires
$$ \omega_{c} = \frac{1}{\sqrt{\alpha} T} $$

$$ \zeta \approx \frac{\phi_{m}}{100} $$
so if \(\phi_{m}\) increases, there are less oscillations and less overshoot for the closed loop system.

For a lag controller, as \(\alpha \to \infty\), the pole \(-\frac{1}{\alpha T} \to 0\).

A \textbf{Proportional-Integral-Derivative (PID)} controller produces a control action \(u(t)\) proportional
to the error \(e(t)\), as well as to its integral and derivative:
$$ u(t) = K_{P}e(t) + K_{I} \int_{0}^{t}e(\tau) d \tau + K_{D} \frac{de}{dt} $$
where \(K_{P}, K_{I}, K_{D} > 0\).

This has transfer function
$$ C_{PID}(s) = K_{P} + \frac{K_{I}}{s} + K_{D} s = K_{P} \left( 1 + \frac{1}{T_{I}s} + T_{D}s \right) $$
where \(T_{I} = \frac{K_{P}}{K_{I}}\) is the \textbf{integration time}
and \(T_{D} = \frac{K_{D}}{K_{P}}\) is the \textbf{derivative time}.

The integral action gives 0 steady-state error and perfect disturbance rejection
since infinite gain for \(\omega \to 0\).
The derivative action gives faster response due to phase lead, since infinite gain for \(\omega \to \infty\).

With derivative action, \(C_{PID}(s)\) is not realizable since it is improper.
To make this realizable, add a pole at high frequency:
$$ C_{PID}(s) = K_{P} \left( 1 + \frac{1}{T_{I}s} + \frac{T_{D}s}{1 + \frac{T_{D}}{N}s} \right) $$
where \(N\) is chosen so that high-frequency pole at \(-N/T_{D}\) is beyond the bandwidth of the
control task.

Practical considerations of PID controllers include:
\begin{itemize}
\item limitation of derivative action (avoid spiking response to step reference)
\item anti-windup of integral action for input saturation
\end{itemize}

A realizable controller is one that has a state-space representation.

A lead controller is a realizable PD and a lag controller is a realizable PI, which makes
a lead-lag controller a realizable PID (pole at high frequency on lag).
\section{Root Locus}
\label{sec:orge33fdd5}
Root refers to the roots of the denominator of the closed loop \(F(s)\) and the locus is of the
geometric roots.

The goal is to find the poles of \(F(s)\) as a function of \(K_{P} \in (0, \infty)\),
which gives the root locus for positive controller gain.

Consider the open loop transfer function
$$ L(s) = \mu \frac{N_{0}(s)}{D(s)} $$
The \textbf{root locus} for loop gain is the set of all points of the complex plane which are roots of
\(1 + L(s)\) for some \(\mu \in (-\infty, 0) \cup (0, \infty)\).

Using a proportionally controlled simple system, it can be found that using the phase
$$ \sum_{i}^{m} \angle (s - z_{i}) - \sum_{i}^{n} \angle (s - p_{i}) = (2k + 1) \pi $$
where there are \(m\) open loop zeros and \(n\) open loop poles.

The rules to plot a root locus for \(\mu > 0\) are:
\begin{enumerate}
\item the locus has \(n\) branches, one per pole
\item the locus is symmetric with respect to the real axis
\item the branches of the locus start at the poles of \(L(s)\)
\item for \(\mu \to \infty\)
$$ \begin{cases} m \text{ branches } &\to \text{ zeros of } L(s) \\ n - m \text{ branches } &\to \infty \end{cases} $$
\item the asymptotes of the root locus meet on the real axis at
$$ x_{a} = \frac{1}{n-m} \left( \sum_{i=1}^{n} p_{i} - \sum_{i=1}^{m} z_{i} \right) $$
and have slope
$$ \psi_{a} = \frac{2k+1}{n-m} \pi, k = 0, \dots, n-m-1 $$
\item the locus includes all points of the real axis on the left of an odd number of poles/zeros
\end{enumerate}
\section{Pole Placement}
\label{sec:org0857e26}
The root locus for loop gain shows how the poles of the closed-loop system move on the complex plane as
a function of the gain of the open-loop transfer function.
The position of the dominant poles determine performance:
\begin{itemize}
\item OS\% = \(100e^{-\xi \pi / \sqrt{1 - \xi^{2}}}\)
\item \(T_{p} = \frac{1}{\omega_{n} \sqrt{1 - \xi^{2}}}\)
\item Oscillation frequency = \(\omega_{n} \sqrt{1 - \xi^{2}}\)
\item Settling time = \(\frac{\ln(0.01 \varepsilon)}{\xi \omega_{n}} = \frac{\ln(0.01 \varepsilon)}{\sigma}\)
\item Phase margin \(\approx 100 \xi\)
\item Crossover frequency \(\approx \omega_{n}\)
\end{itemize}
where \(\xi, \omega_{n}, \sigma = -\xi\omega_{n}\) are the damping, natural frequency, and real part of the
pair of complex conjugate poles.

The damping of dominant poles \(\xi\) controls the overshoot, peak time, frequency of oscillations, settling
time, and phase margin.
This creates a cone on the complex plane at the angle \(\omega \cos(\bar{\xi})\) from the negative
real axis.

The natural frequency of the dominant poles \(\omega_{n}\) controls the peak time, frequency of
oscillations, settling time, and crossover frequency.
This occurs on the left side of the complex plane, with a hole going through \$j\(\omega\)\textsubscript{n}, -\(\omega\)\textsubscript{n},\$
and \(-j\omega_{n}\).

The real part of the dominant poles \(\sigma\) controls the settling time.
This occurs to the left of \(\sigma\) on the complex plane.

Using the root locus, try to keep the parameters in the regions specified by the preferred
performance metrics.
\section{State Feedback Control}
\label{sec:org3d0a16f}
Sometimes the static feedback of the output is not sufficient to stabilize a system.
For such cases, static state feedback might be better to stabilize a system.

Consider a transfer function
$$ G(s) = \frac{b_{n-1}s^{n-1} + \cdots + b_{1}s + b_{0}}{s^{n} + a_{n-1}s^{n-1} + \cdots + a_{1} s + a_{0}} $$
This has \textbf{controllable canonical state space form}
$$ \dot{x} = \begin{bmatrix} 0 & 1 & 0 & \cdots & 0 \\ \vdots & 0 & 1 & & \vdots \\ \vdots & & \ddots & \ddots & 0 \\ 0 & \cdots & \cdots & 0 & 1 \\ -a_{0} & -a_{1} & -a_{2} & \cdots & a_{n-1} \end{bmatrix}x + \begin{bmatrix} 0 \\ 0 \\ \vdots \\ 0 \\ 1 \end{bmatrix} u $$
$$ y = \begin{bmatrix} b_{0} & b_{1} & \cdots & b_{n-1} \end{bmatrix} x $$

A static state feedback controller is
$$ u = -Kx $$
for \(K^{T} \in \mathbb{R}^{n}\).
Controlling a system
$$ \dot{x} = Ax + Bu $$
$$y = Cx $$
with the controller yields
$$ \dot{x} = (A - BK) x $$
$$y = Cx $$
so the state-feedback controlled closed-loop system behaves according to the eigenvalues of
the matrix \(A - BK\).

\uline{Pole placement via state feedback}: Given a system in controllable canonical form and \(n\) complex
numbers \(\lambda^{\*}_{1}, \dots, \lambda^{\*}_{n}\) (either real or pairs of complex conjugate
numbers), there exists \(K^{T} \in \mathbb{R}^{n}\) such that the eigenvalues of \(A - BK\) are
\(\lambda^{\*}_{1}, \dots, \lambda_{n}^{\*}\).

For some \(\lambda_{1}^{\*}, \dots, \lambda_{n}^{\*}\), the desired characteristic polynomial of \(A-BK\) is
$$ P(\lambda) = \lambda^{n} + c_{n-1} \lambda^{n-1} + \cdots + c_{1} \lambda + c_{0} $$
for some \(c_{0}, c_{1} ,\dots, c_{n-1}\).

\(K\) should be chosen so that \(a_{i-1} + K_{i} = c_{i-1}\) for \(i \in [1, n]\)
to result in a closed-loop system with eigenvalues at \(\lambda_{1}^{\*}, \dots, \lambda_{n}^{\*}\).
\end{document}
