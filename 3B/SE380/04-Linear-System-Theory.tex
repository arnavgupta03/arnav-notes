% Created 2024-10-19 Sat 16:08
% Intended LaTeX compiler: pdflatex
\documentclass[11pt]{article}
\usepackage[utf8]{inputenc}
\usepackage[T1]{fontenc}
\usepackage{graphicx}
\usepackage{longtable}
\usepackage{wrapfig}
\usepackage{rotating}
\usepackage[normalem]{ulem}
\usepackage{amsmath}
\usepackage{amssymb}
\usepackage{capt-of}
\usepackage{hyperref}
\usepackage{parskip,darkmode}
\enabledarkmode
\usepackage{tikz}
\usepackage{pgfplots}
\author{Arnav Gupta}
\date{\today}
\title{Linear System Theory}
\hypersetup{
 pdfauthor={Arnav Gupta},
 pdftitle={Linear System Theory},
 pdfkeywords={},
 pdfsubject={},
 pdfcreator={Emacs 29.4 (Org mode 9.7.11)}, 
 pdflang={English}}
\begin{document}

\maketitle
\tableofcontents

\section{Stability}
\label{sec:org985bd11}
For the initial value problem with \(\dot{\mathbf{x}} = \mathbf{Ax}\) and
\(\mathbf{x}(t_{0}) = x_{0}\),
the unique solution is
$$
\mathbf{x}(t) = e^{A(t-t_{0})} x_{0}
$$

Note that the matrix exponential comes from the Taylor series
$$
e^{M} = \sum_{k = 0}^{\infty} \frac{M^{k}}{k!}
$$
where \(M^{0} = I\).

A system in state-space form is:
\begin{itemize}
\item \uline{stable} if, for every initial condition \(\mathbf{x}(t_{0}) = x_{0} \in \mathbb{R}^{n}\),
\(\mathbf{x}(t) = e^{A(t - t_{0})} x_{0}\) is uniformly bounded
\item \uline{asymptotically stable} if, in addition, for every initial condition \(\mathbf{x}(t_{0}) = x_{0} \in \mathbb{R}^{n}\),
\(\mathbf{x}(t) \to 0\) as \(t \to \infty\)
\item \uline{unstable} otherwise
\end{itemize}

A system is asymptotically stable if and only if all eigenvalues of \(A\) have negative real part.

A \textbf{Bounded-Input-Bounded-Output (BIBO) stable} system is one with bounded response to a bounded input.

A system is BIBO stable if and only if all poles of all entries of its transfer function have negative real
part.

Asymptotic stability implies BIBO stability.
\section{Performance}
\label{sec:org8a0071d}
Important performance metrics to describe the behaviour of a system with a unit step signal are:
\begin{itemize}
\item \textbf{steady-state gain}: \(\lim_{t \to \infty} y(t)\)
\item \textbf{rise time}: time to go from 10\% to 90\% of steady-state
\item \textbf{peak time}: time at which peak is achieved
\item \textbf{overshoot percentage}: \(OS\% = \frac{|C_{\text{max}| - C_{ss}}}{C_{ss}} \times 100 \%\)
\item \textbf{settling time}: smallest time such that \(\forall t \ge T, \frac{|C_{ss}-y(t)|}{|C_{ss}|} \le \varepsilon\)
\end{itemize}

\begin{center}
\begin{tikzpicture}
\begin{axis}[
        axis lines=middle,
        xmin=0, xmax=15,
        ymin=0, ymax=1.5,
        xlabel=$t$,
        ylabel={$y(t)$},
        xlabel style={at=(current axis.right of origin), anchor=west},
        ylabel style={at=(current axis.above origin), anchor=south},
        xtick={0, 0.4726, 1.79398, 1.96605, 3.2236, 11.0855},
        xticklabels={$0$, {}, {}, $t_r$, $t_p$, $t_s$},
        ytick={0, 0.1, 0.9, 1, 1.3714},
        yticklabels={$0$, $0.1$, $0.9$, $1$, $M_p$}
    ]
    \addplot[smooth,
             red!60!black!80,
             thick,
             mark=none,
             domain=0:12.4,
             samples=100]
    {1-exp(-0.3*x)*(cos(deg(sqrt(1-0.3^2)*x))+0.3/(sqrt(1-0.3^2))*sin(deg(sqrt(1-0.3^2)*x)))};
    %
    \addplot[cyan, dotted] coordinates{(0.4726,0.1)} -- (axis cs:0,0.1);
    \addplot[cyan, dotted] coordinates{(0.4726,0.1)} -- (axis cs:0.4726,0);
    %
    \addplot[cyan, dotted] coordinates{(1.79398,0.9)} -- (axis cs:0,0.9);
    \addplot[cyan, dotted] coordinates{(1.79398,0.9)} -- (axis cs:1.79398,0);
    %
    \addplot[cyan, dotted] coordinates{(1.96605,1)} -- (axis cs:1.96605,0);
    %
    \addplot[cyan, dotted] coordinates{(3.2236,1.3714)} -- (axis cs:0,1.3714);
    \addplot[cyan, dotted] coordinates{(3.2236,1.3714)} -- (axis cs:3.2236,0);
    %
    \addplot[cyan, dotted] coordinates{(11.0855,1.025)} -- (axis cs:11.0855,0);
    %
    \addplot[red!60!black!80, thick] coordinates{(15,0.9872)} -- (axis cs:12.4,0.9872);
    %
    \addplot[cyan, dashed] coordinates{(15,1)} -- (axis cs:0,1);
    %
    \addplot[lime, dashed] coordinates{(15,0.975)} -- (axis cs:0,0.975);
    \addplot[lime, dashed] coordinates{(15,1.025)} -- (axis cs:0,1.025);

    \coordinate (a) at (axis cs:0,0);
    \coordinate (b) at (axis cs:1.79398,0);

    \end{axis}

    % \draw [shorten <=1mm,shorten >=1mm] (a) -- ++(0,-1cm) coordinate (aa);
    % \draw (b) -- ++(0,-1cm) coordinate (bb);
    % \draw [|<->|] (aa) -- (bb) node [midway,fill=black] {$t_r$};

    \end{tikzpicture}
\end{center}
\subsection{First Order Systems}
\label{sec:org61cc685}
Steady-state gain is \(\mu\).

Rise-time is \(2 \tau\).

Peak time and overshoot would not make sense for a first order system.

Settling time is \(-\tau \ln(\varepsilon)\).
For \(\varepsilon = 0.05, T_{s} \approx 3\tau\).
For \(\varepsilon = 0.03, T_{s} \approx 4\tau\).
For \(\varepsilon = 0.01, T_{s} \approx 5\tau\).
\subsection{Second Order Systems}
\label{sec:org544a734}
Steady-state gain is \(\mu\).

Rise-time is
$$
\frac{2.16 \zeta + 0.16}{\omega_{n}}
$$

Peak time is
$$
\frac{\pi}{\sqrt{1 - \zeta^{2} \omega_{n}}}
$$

Overshoot percentage is
$$
e^{\frac{-\zeta \pi}{1 - \zeta^{2}}}
$$

Settling time is
$$
-\frac{\ln(\varepsilon)}{\zeta \omega_{n}}
$$
\section{Lower Order Approximations}
\label{sec:org3ea6cba}
Higher order systems often have a few poles that are much closer to the imaginary
axis than others, these are \textbf{dominant poles}.

For a system with transfer function:
$$
G(s) = G_{fast}(s) G_{slow}(s)
$$
where the poles of \(G_{slow}(s)\) are more dominant, the fast components can be
approximated by a static transfer function whose value is equal to steady-state
gain, \(G_{fast}(0)\).
\section{System Identification}
\label{sec:org3dcf00f}
Assume a structure of the model
$$G(s) = \frac{Y(s)}{U(s)} = \frac{b_{m} s^{m} + \cdots + b_{1} s + b_{0}}{s^{n} + a_{n-1} s^{n-1} + \cdots + a_{1} s + a_{0}}$$
which gives
$$y^{(n)}(t) + a_{n-1} y^{(n-1)}(t) + \cdots + a_{1} \dot{y}(t) + a_{0} y(t) = b_{m} u^{(m)}(t) + \cdots + b_{1} u(t) + b_{0} u(t)$$

For any time \(t_{k}\), set
$$
y^{(n)}(t_{k}) = b_{m}u^{(m)}(t_{k}) + \cdots + b_{0} u(t_{k})+ a_{n-1} \left( -y^{(n-1)} (t_{k}) \right) + \cdots + a_{0} (-y(t_{k}))
$$
which can be represented as:
$$ d^{T} (t_{k}) \theta = \begin{pmatrix} u^{(m)} (t_{k}) & \cdots & u(t_{k}) & -y^{(n-1)} (t_{k}) & \cdots & -y(t_{k}) \end{pmatrix} \begin{pmatrix} b_{m} \\ \vdots \\ b_{0} \\ a_{n-1} \\ \vdots \\ a_{0} \end{pmatrix} $$

Consider times \(t_{1}, \dots, t_{N}\), where \(N \gg n + m + 1\).
$$ Y = D \theta \implies \begin{pmatrix} y^{(n)} (t_{1}) \\ \vdots \\ y^{(n)} (t_{N}) \end{pmatrix} = \begin{pmatrix} d^{T} (t_{1}) \\ \vdots \\ d^{T} (t_{N}) \end{pmatrix} \theta $$

By minimizing \(\| Y - D \theta\|^{2}\), \(\theta\) can be found that best approximates the system:
$$
\theta^{*} = (D^{T}D)^{-1} D^{T} Y = D^{\dagger} Y
$$
\end{document}
