% Created 2024-11-14 Thu 01:24
% Intended LaTeX compiler: pdflatex
\documentclass[11pt]{article}
\usepackage[utf8]{inputenc}
\usepackage[T1]{fontenc}
\usepackage{graphicx}
\usepackage{longtable}
\usepackage{wrapfig}
\usepackage{rotating}
\usepackage[normalem]{ulem}
\usepackage{amsmath}
\usepackage{amssymb}
\usepackage{capt-of}
\usepackage{hyperref}
\usepackage{parskip,darkmode}
\enabledarkmode
\usepackage{tikz}
\usepackage{pgfplots}
\author{DESKTOP-H800RKQ}
\date{\today}
\title{Analysis Of Feedback Control Systems}
\hypersetup{
 pdfauthor={DESKTOP-H800RKQ},
 pdftitle={Analysis Of Feedback Control Systems},
 pdfkeywords={},
 pdfsubject={},
 pdfcreator={Emacs 29.4 (Org mode 9.7.11)}, 
 pdflang={English}}
\begin{document}

\maketitle
\tableofcontents

\section{Block Diagrams}
\label{sec:org279d2db}

\section{Stability of Interconnected Systems}
\label{sec:orgc36cdd0}

\section{Routh-Hurwitz Criterion}
\label{sec:orgf7c1ec0}
For a closed-loop system, the transfer function tends to be of the form
$$ F(s) = \frac{L(s)}{1 + L(s)} $$
To determine if such a system is stable, it must be known whether
the roots of \(1+L(s)\) are in the left half of the complex plane
\(\mathbb{C}^{-}\) (preferably without direct computation).

For any polynomial \(\pi(s) = s^{n} + \cdots + a_{1}s + a_{0}\) for
\(a_{i} \in \mathbb{R}\),
\(\pi(s)\) is \textbf{Hurwitz} if all its roots are in \(\mathbb{C}^{-}\).

A closed loop system is stable if \(1 + L(s)\) is Hurwitz.

Note that a polynomial \(\pi(s)\) can be factored into
$$ \pi(s) = (s - \lambda_{1})\cdots (s - \lambda_{r}) (s - \mu_{1})(s - \bar{\mu}_{1}) \cdots (s - \mu_{p})(s - \bar{\mu}_{p}) $$
where \(\lambda_{1}, \dots, \lambda_{r}\) are real roots and \(\mu_{1}, \bar{\mu}_{1}, \cdots \mu_{p}, \bar{\mu}_{p}\) are complex conjugate pairs of roots.

If \(\pi(s)\) is Hurwitz, \(\pi(s)\) has strictly positive coefficients.
\subsection{Routh's Algorithm}
\label{sec:org42918be}
The first step is to build the following table:
$$ \begin{tabular}{c|c|c|c|c}
$s^{n}$ & 1 & $a_{n-2}$ & $a_{n-4}$ & $\cdots$ \\
$s^{n - 1}$ & $a_{n-1}$ & $a_{n-3}$ & $a_{n-5}$ & $\cdots$ \\
$s^{n - 2}$ & $r_{2,0}$ & $r_{2,1}$ & $r_{2,2}$ & $\cdots$ \\
$\vdots$ & $\vdots$ & $\vdots$ & $\vdots$ & $\ddots$ \\
$s^{2}$ & $r_{n-2,0}$ & $r_{n-2,1}$ & $r_{n-2,2}$ & $\cdots$ \\
$s^{1}$ & $r_{n-1,0}$ & $r_{n-1,1}$ & $r_{n-1,2}$ & $\cdots$ \\
$s^{0}$ & $r_{n,0}$ & & &
\end{tabular} $$
where each \(r\) is the negative inverse of the value above it, multiplied by the determinant of the
\(2 \times 2\) square above it.
For examples,
$$ r_{2, 0} = -\frac{1}{a_{n-1}} \begin{vmatrix} 1 & a_{n-2} \\ a_{n-1} & a_{n-3} \end{vmatrix} $$
This table stops along each row once 0 is reached.
The process is terminated when a 0 is reached in the first column.

The next step is the use the \textbf{Routh-Hurwitz Criterion}:
\begin{enumerate}
\item \(\pi(s)\) is Hurwitz \(\iff\) all elements in the Routh array (first column of table) have the same sign
\item If the Routh array has no zeros, then
\begin{enumerate}
\item the number of sign changes is the number of bad roots (non-negative real parts)
\item no roots exist on the imaginary axis
\end{enumerate}
\end{enumerate}
\section{Nyquist Criterion}
\label{sec:org718a74c}
The transfer function \(G(s)\) is a mapping from \(s\) to \(G(s)\), which is \(\mathbb{C} \to \mathbb{C}\).

The \textbf{Nyquist contour} goes along the real axis and then in a circular fashion on the postive real side.
The \textbf{Nyquist plot} is the image of the Nyquist contour through \(L(s)\), the open loop transfer function
of the closed loop system.

If \(L(s)\) has poles with 0 real, part, the Nyquist plot goes around them.

The Nyquist plot has symmetry with the real line, that is
$$ L(-j\omega) = \overline{L(j \omega)} $$

If \(L(s)\) is strictly proper \(L(j\omega) \to 0\) as \(| \omega | \to \infty\).

Let \(p\) be the number of poles of \(L(s)\) with positive real part.
Let \(N\) be the number of loops the Nyquist plot makes around the point \(-1 \in \mathbb{C}\),
with \(>0\) if counter-clockwise and \(<0\) if clockwise.
\(N\) is undefined if the Nyquist plot goes through -1.

By the \textbf{Nyquist Criterion}, the closed loop system is stable if and only if \(N = P\).

If \(N\) is undefined, the closed loop system may be stable or unstable.

If \(N\) is well defined and \(N \ne P\), the closed loop system is unstable.

\(P - N\) is the number of poles of the closed loop system with positive real part.

\(L(s)\) is stable for \(P = 0\).
Further, if:
\begin{itemize}
\item \(|L(j\omega) < 1 \; \forall \omega \implies\) the closed loop system is stable
\item \(|\angle L(j\omega) < 1 \; \forall \omega \implies\) the closed loop system is stable
\end{itemize}
\section{Bode Criterion}
\label{sec:org8060e12}
Let \(L(s)\) be the open loop transfer function of a closed-loop system.
Assume \(L(s)\) is stable and the Nyquist plot of \(L(s)\) intersects the negative real axis
only once.
The distance from -1 to \(L(j \omega \pi)\) gives the gain margin.

On a Bode plot, the \textbf{gain margin} \(K_{m}\) occurs when the phase hits \(-180^{\circ}\) and is the
distance from the frequency axis to \(|L(j \omega \pi)|\).
The gain margin is positive if \(|L(j \omega \pi)|_{dB} < 0_{dB}\) (which indicates a stable
system) and is otherwise negative.
The gain margin represents the maximum multiplicative factor on the gain of \(L(s)\) at
\(\omega_{\pi}\) that the system can tolerate before becoming unstable.
$$ K_{m} = \frac{1}{|L(j \omega_{\pi})|} $$
where \(\omega_{\pi}\) is the frequency such that \(\angle L (j \omega_{\pi}) = -180^{\circ}\).

Let \(L(s)\) be the open loop transfer function of a closed-loop system.
Assume \(L(s)\) is stable and the Nyquist plot of \(L(s)\) intersects the unit circle once once
from outside to inside.
The frequency at which \(|L(j \omega)| = 1, |L(j \omega)|_{dB} = 0_{dB}\) is
the \textbf{crossover frequency}.

The \textbf{phase margin} \(\phi_{m}\) is the distance between \(\angle L(j \omega)\) and \(-180^{\circ}\).
Specifically, this is the frequency at which the magnitude \(|L(j\omega)|\) goes to 0.
The phase margin is positive if \(180^{\circ} - | \angle L(j \omega_{c}) | > 0\) and
negative otherwise.

The \textbf{Bode Criterion} states that if \(L(s)\) has no poles with positive real parts and
\(|L(j \omega)|_{dB}\) crosses the \(0_{dB}\) axis only once from above to below, then
$$ \mu > 0, \phi_{m} > 0 \iff F(s) = \frac{L(s)}{1 + L(s)} \text{ stable} $$

The closed loop system \(F(s)\) is stable for \(K_{m} > 0, \phi_{m} > 0\).
\end{document}
