% Created 2024-10-19 Sat 14:04
% Intended LaTeX compiler: pdflatex
\documentclass[11pt]{article}
\usepackage[utf8]{inputenc}
\usepackage[T1]{fontenc}
\usepackage{graphicx}
\usepackage{longtable}
\usepackage{wrapfig}
\usepackage{rotating}
\usepackage[normalem]{ulem}
\usepackage{amsmath}
\usepackage{amssymb}
\usepackage{capt-of}
\usepackage{hyperref}
\usepackage{parskip,darkmode}
\enabledarkmode
\usepackage{tikz,xcolor}
\usepackage{bodegraph}
\author{Arnav Gupta}
\date{\today}
\title{Frequency Response And Bode Plots}
\hypersetup{
 pdfauthor={Arnav Gupta},
 pdftitle={Frequency Response And Bode Plots},
 pdfkeywords={},
 pdfsubject={},
 pdfcreator={Emacs 29.4 (Org mode 9.7.11)}, 
 pdflang={English}}
\begin{document}

\maketitle
\tableofcontents

\section{Frequency Response}
\label{sec:orgd248e28}
The steady state response of the system to a sinusoidal input signal.
The resulting output signal for a linear system is sinusoidal in the steady
state, and differs from the input in the amplitude and phase signal.

Specifically, the \textbf{frequency response} of a system is the complex-valued function
$$
G : \omega \in \mathbb{R} \mapsto G(j\omega) \in \mathbb{C}
$$
which can be expressed as:
$$ |G| : \omega \in \mathbb{R} \mapsto |G(j\omega)| \in \mathbb{R} $$
$$ \angle G : \omega \in \mathbb{R} \mapsto \angle G(j\omega) \in \mathbb{R} $$

The transfer function in the frequency domain becomes
$$ G(j \omega) = |G(j \omega)| e^{j \phi(\omega)} $$
where \(\phi(\omega) = \angle G(j\omega)\) comes from the zeros and poles of the transfer function.

For an system \(Y(s) = G(s) U(s)\), where \(u(t) = \mathcal{U} \sin(\omega t)\):
$$ y(t) = \text{transient} + \mathcal{U} |G(j \omega)| \sin(\omega t + \angle G(j \omega)) $$
where the transient part goes to 0 and the sinusoidal part is the steady state response.
\section{Bode Plots}
\label{sec:org1017130}
\textbf{Bode plots} are a graphical representation of a system's frequency response with plots for the
magnitude and phase along a logarithmic scale for \(\omega\).

The magnitude is representation with
$$ |G(j \omega)|_{dB} = 20 \log |G(j \omega)| $$

Consider the general transfer function:
$$
G(s) = \frac{\mu \prod_{i} ( 1 + T_{i}s ) \prod_{i} \left( 1 + \frac{2 \xi_{i}}{\alpha_{n, i}}s + \frac{s^{2}}{\alpha^{2}_{n,i}} \right)}{s^{\rho} \prod_{i} (1 + \tau_{i} s) \prod_{i} \left( 1 + \frac{2\zeta_{i}}{\omega_{n, i}}s + \frac{s^{2}}{\omega^{2}_{n, i}} \right)}
$$
where \(\mu\) is the gain, \(\rho\) is the number of poles if \(\rho > 0\) or zeros if \(\rho < 0\) at the origin,
\(T_{i},\tau_{i}\) are the time constants of the real zeros and poles respectively,
\(\xi_{i}, \zeta_{i}\) are the damping constants of the complex conjugate zeros and poles,
and \(\alpha_{n, i}, \omega_{n, i}\) are the natural frequencies of the complex conjugate zeros and poles.
\subsection{Constant Gain}
\label{sec:orga057d1a}
The constant gain \(\mu\) has
$$ |G(j\omega)|_{dB} = 20 \log |\mu|, \angle G(j \omega) = \angle \mu $$
\begin{tikzpicture}

% Define the axes
\draw[->] (0,0) -- (10,0) node[right] {Frequency (rad/s)};
\draw[->] (0,-4) -- (0,3) node[above] {Magnitude (dB)};
\draw[->] (10,-4) -- (10,3) node[above] {Phase (degrees)};

% Draw magnitude plot
\draw[thick, cyan!50] (0,2) -- (10,2); % Magnitude line

% Draw phase plot
\draw[dashed, lime] (0,-3) -- (10,-3); % Phase line

% Labels
\node at (5,-0.5) {Bode Plot for $G(s) = \mu$};
\node at (5,-3.5) {Phase = $\angle \mu$};
\node at (5,1.5) {Magnitude = $20 \log|\mu|$};

% Draw the phase plot
% \draw[thick] (1,-3) -- (9,-3); % Constant phase line
\end{tikzpicture}
\subsection{Poles/Zeros at Origin}
\label{sec:orgf5a0edb}
Poles/Zeros at the origin \(j\omega\) have:
$$ |G(j\omega)|_{dB} = -20\rho \log|j \omega|, \angle G(j \omega) = \rho \angle (j \omega) $$
\begin{tikzpicture}

% Define the axes
\draw[->] (0,0) -- (10,0) node[right] {Frequency (rad/s)};
\draw[->] (0,-4) -- (0,3) node[above] {Magnitude (dB)};
\draw[->] (10,-4) -- (10,3) node[above] {Phase (degrees)};

% Draw magnitude plot
\draw[thick, cyan!50] (0,2) -- (10,-2); % Magnitude line

% Draw phase plot
\draw[dashed, lime] (0,-3) -- (10,-3); % Phase line

% Labels
\node at (3,-0.5) {Bode Plot for $G(s) = \frac{1}{s^{\rho}}$};
\node at (5,-3.5) {Phase = $-\rho \angle (j\omega)$};
\node at (5,1.5) {Magnitude = $-20\rho \log|j\omega|$};

% Draw the phase plot
% \draw[thick] (1,-3) -- (9,-3); % Constant phase line
\end{tikzpicture}
\subsection{Poles/Zeros on Real Axis}
\label{sec:org7e14189}
Zeros on the real axis have:
$$ |G(j \omega)|_{dB} = \sum_{i} 20 \log |1 + j\omega T_{i} |, \; \angle G(j\omega) = \sum_{i} \angle ( 1 + j \omega T_{i} ) $$

Poles on the real axis have:
$$ |G(j \omega)|_{dB} = -\sum_{i} 20 \log |1 + j\omega \tau_{i} |, \; \angle G(j\omega) = -\sum_{i} \angle ( 1 + j \omega \tau_{i} ) $$
\begin{tikzpicture}

% Define the axes
\draw[->] (0,0) -- (10,0) node[right] {Frequency (rad/s)};
\draw[->] (0,-4) -- (0,3) node[above] {Magnitude (dB)};
\draw[->] (10,-4) -- (10,3) node[above] {Phase (degrees)};

% Draw magnitude plot
\draw[thick, cyan!50] (0,1) -- (3,1); % Magnitude line
\draw[thick, cyan!50] (3,1) -- (6,3); % Magnitude line
\draw[thick, cyan!50] (6,3) -- (10,0.67); % Magnitude line

% Draw phase plot
\draw[dashed, lime] (0,-1.5) -- (3,-1.5); % Phase line
\draw[dashed, lime] (3,-1.5) -- (3,0); % Phase line
\draw[dashed, lime] (3,0) -- (6,0); % Phase line
\draw[dashed, lime] (6,0) -- (6,-1.5); % Phase line
\draw[dashed, lime] (6,-1.5) -- (10,-1.5); % Phase line

% Labels
\node at (5,-5) {Bode Plot for Zero and Pole on Real Axis};
\node at (7,-2) {Phase};
\node at (2,1.5) {Magnitude};

% Draw the phase plot
% \draw[thick] (1,-3) -- (9,-3); % Constant phase line
\end{tikzpicture}
\subsection{Complex Conjugate Poles/Zeros}
\label{sec:org0537fe3}
Complex conjugate zeros have:
$$ |G(j \omega)|_{dB} = \sum_{i} 20 \log \left|1 + \frac{2j\xi_{i} \omega}{\alpha_{n,i}} - \frac{\omega^{2}}{\alpha^{2}_{n,i}} \right|, \; \angle G(j\omega) = \sum_{i} \angle \left( 1 + \frac{2j\xi_{i} \omega}{\alpha_{n,i}} - \frac{\omega^{2}}{\alpha^{2}_{n,i}} \right) $$

Complex conjugate poles have:
$$ |G(j \omega)|_{dB} = -\sum_{i} 20 \log \left|1 + \frac{2j\xi_{i} \omega}{\alpha_{n,i}} - \frac{\omega^{2}}{\alpha^{2}_{n,i}} \right|, \; \angle G(j\omega) = -\sum_{i} \angle \left( 1 + \frac{2j\xi_{i} \omega}{\alpha_{n,i}} - \frac{\omega^{2}}{\alpha^{2}_{n,i}} \right) $$
\begin{tikzpicture}

% Define the axes
\draw[->] (0,0) -- (10,0) node[right] {Frequency (rad/s)};
\draw[->] (0,-4) -- (0,3) node[above] {Magnitude (dB)};
\draw[->] (10,-4) -- (10,3) node[above] {Phase (degrees)};

% Draw magnitude plot
\draw[thick, cyan!50] (0,-1) -- (3,-1); % Magnitude line
\draw[thick, cyan!50] (3,-1) -- (6,3); % Magnitude line
\draw[thick, cyan!50] (6,3) -- (10,-3.33); % Magnitude line

% Draw phase plot
\draw[dashed, lime] (0,-3) -- (3,-3); % Phase line
\draw[dashed, lime] (3,-3) -- (3,0); % Phase line
\draw[dashed, lime] (3,0) -- (6,0); % Phase line
\draw[dashed, lime] (6,0) -- (6,-3); % Phase line
\draw[dashed, lime] (6,-3) -- (10,-3); % Phase line

% Labels
\node at (5,-5) {Bode Plot for Conjugate Complex Zeros and Poles};
\node at (7,-3.5) {Phase};
\node at (3,1.5) {Magnitude};

% Draw the phase plot
% \draw[thick] (1,-3) -- (9,-3); % Constant phase line
\end{tikzpicture}

The \textbf{bandwidth} frequency occurs when the logarithmic gain is -3 dB, which is at \(\omega = 1/tau\)
for poles and zeros on the real axis.

As \(\zeta\) goes from 1 to 0, overshoot increases for the magnitude plot and the steepness increases
for the phase plot.

The \textbf{resonant frequency} \(\omega_{r}\) occurs at \(\omega_{r} = \omega_{n} \sqrt{1 - 2\zeta^{2}}\).
\end{document}
